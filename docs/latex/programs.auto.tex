% auto generated by GROOPS
\section{Programs: Covariance}
\subsection{AutoregressiveModel2CovarianceMatrix}\label{AutoregressiveModel2CovarianceMatrix}
This program computes the covariance structure of a random process represented by an AR model sequence.
The covariance matrix is determined by accumulating the normal equations of all AR models in \config{autoregressiveModelSequence}
and inverting the combined normal equation matrix.
For each output file in \configFile{outputfileCovarianceMatrix}{matrix},
the covariance matrix of appropriate time lag is saved (the first file contains the auto-covariance,
second file cross covariance and so on). The matrix for lag $h$ describes the covariance between $x_{t-h}$ and $x_{t}$, i.e. $\Sigma(t-h, t)$.


\keepXColumns
\begin{tabularx}{\textwidth}{N T A}
\hline
Name & Type & Annotation\\
\hline
\hfuzz=500pt\includegraphics[width=1em]{element-mustset-unbounded.pdf}~outputfileCovarianceMatrix & \hfuzz=500pt filename & \hfuzz=500pt covariance matrix for each lag\\
\hfuzz=500pt\includegraphics[width=1em]{element-mustset.pdf}~autoregressiveModelSequence & \hfuzz=500pt \hyperref[autoregressiveModelSequenceType]{autoregressiveModelSequence} & \hfuzz=500pt AR model sequence\\
\hline
\end{tabularx}

\clearpage
%==================================
\subsection{CovarianceFunction2DigitalFilter}\label{CovarianceFunction2DigitalFilter}
Computes digital filter coefficients for a \configClass{digital filter}{digitalFilterType} of given degree and
order. The filter coefficients are computed by fitting them to an approximated
impulse response represented by the cholesky factor of the covariance matrix.

The parameter \config{warmup} determines from which element of the cholesky matrix the
coefficients (default: half the covariance length) are fitted.

Per default, the program computes filter coefficients which generate colored noise
when applied to a white noise sequence. When \config{decorrelationFilter} is set,
a decorrelation filter is computed which yields white noise when applied to colored noise.


\keepXColumns
\begin{tabularx}{\textwidth}{N T A}
\hline
Name & Type & Annotation\\
\hline
\hfuzz=500pt\includegraphics[width=1em]{element-mustset.pdf}~outputfileFilter & \hfuzz=500pt filename & \hfuzz=500pt filter coefficients\\
\hfuzz=500pt\includegraphics[width=1em]{element-mustset.pdf}~inputfileCovariance & \hfuzz=500pt filename & \hfuzz=500pt first column: time steps, following columns: covariance functions\\
\hfuzz=500pt\includegraphics[width=1em]{element.pdf}~column & \hfuzz=500pt uint & \hfuzz=500pt Column with covariance function to be fitted\\
\hfuzz=500pt\includegraphics[width=1em]{element.pdf}~warmup & \hfuzz=500pt uint & \hfuzz=500pt number of samples until diagonal of Cholesky factor is flat (default: half covariance length)\\
\hfuzz=500pt\includegraphics[width=1em]{element.pdf}~numeratorDegree & \hfuzz=500pt uint & \hfuzz=500pt Maximum degree of numerator polynomial (MA constituent)\\
\hfuzz=500pt\includegraphics[width=1em]{element.pdf}~denominatorDegree & \hfuzz=500pt uint & \hfuzz=500pt Maximum degree of denominator polynomial (AR constitutent)\\
\hfuzz=500pt\includegraphics[width=1em]{element.pdf}~decorrelationFilter & \hfuzz=500pt boolean & \hfuzz=500pt compute a decorrelation filter\\
\hline
\end{tabularx}

\clearpage
%==================================
\subsection{CovarianceFunction2PowerSpectralDensity}\label{CovarianceFunction2PowerSpectralDensity}
One sided Power Spectral Density (PSD) from a covariance function. The first column of \configFile{inputfileCovarianceFunction}{matrix}
should contain the time lag in seconds.
Multiple covariance functions (in the following column)s are supported.
The output is a \file{matrix}{matrix} with first column contains the frequency $[Hz]$ and the other columns the PSD $[unit^2/Hz]$.

Conversion between covariance function $c_j$ and PSD $p_k$ is performed by discrete cosine transformation:
\begin{equation}
p_k = 2\Delta t\left(c_0 + c_{n-1} (-1)^k + \sum_{j=1}^{n-2} 2 c_j \cos(\pi jk/(n-1))\right).
\end{equation}

See also \program{PowerSpectralDensity2CovarianceFunction}.


\keepXColumns
\begin{tabularx}{\textwidth}{N T A}
\hline
Name & Type & Annotation\\
\hline
\hfuzz=500pt\includegraphics[width=1em]{element-mustset.pdf}~outputfilePSD & \hfuzz=500pt filename & \hfuzz=500pt first column: frequency [Hz], other columns PSD [unit\textasciicircum{}2/Hz]\\
\hfuzz=500pt\includegraphics[width=1em]{element-mustset.pdf}~inputfileCovarianceFunction & \hfuzz=500pt filename & \hfuzz=500pt first column: time steps, following columns: covariance functions\\
\hline
\end{tabularx}

\clearpage
%==================================
\subsection{CovarianceMatrix2AutoregressiveModel}\label{CovarianceMatrix2AutoregressiveModel}
This program computes a VAR(p) model from empirical covariance matrices.
The \configFile{inputfileCovarianceMatrix}{matrix} represent the covariance structure of the process:
the first file should contain the auto-covariance, the second the cross-covariance of lag one,
the next cross-covariance of lag two and so on.

Cross-covariance matrices $\Sigma_{\Delta_k}$ are defined as the cross-covariance between epoch $t-k$ and $t$.
If the process realizations $x_{t}$ are arrange by ascending time stamps
($\{\dots, x_{t-2}, x_{t-1}, x_{t}, x_{t+1}, x_{t+2},\dots\}$),
the covariance structure of the (stationary) process is therefore given by
\begin{equation}
\begin{bmatrix}
\Sigma & \Sigma_{\Delta_1} & \Sigma_{\Delta_2} & \cdots \\
\Sigma_{\Delta_1}^T & \Sigma & \Sigma_{\Delta_1} &  \cdots \\
\Sigma_{\Delta_2}^T & \Sigma_{\Delta_1}^T & \Sigma & \cdots \\
\vdots & \vdots & \vdots & \ddots \\
\end{bmatrix}.
\end{equation}

The estimate AR model is saved as single matrix \config{outputfileAutoregressiveModel} according to the GROOPS AR model conventions.


\keepXColumns
\begin{tabularx}{\textwidth}{N T A}
\hline
Name & Type & Annotation\\
\hline
\hfuzz=500pt\includegraphics[width=1em]{element-mustset.pdf}~outputfileAutoregressiveModel & \hfuzz=500pt filename & \hfuzz=500pt coefficients and white noise covariance of AR(p) model\\
\hfuzz=500pt\includegraphics[width=1em]{element-mustset-unbounded.pdf}~inputfileCovarianceMatrix & \hfuzz=500pt filename & \hfuzz=500pt file name of covariance matrix\\
\hline
\end{tabularx}

\clearpage
%==================================
\subsection{CovarianceMatrix2Correlation}\label{CovarianceMatrix2Correlation}
This program computes the pearson correlation coefficient
\begin{equation}
  \rho_{ij} = \frac{\sigma_{ij}}{\sigma_i \sigma_j}
\end{equation}
from a given covariance matrix stored in \configFile{inputfileCovarianceMatrix}{matrix}.
The result is stored in \configFile{outputfileCorrelationMatrix}{matrix}.


\keepXColumns
\begin{tabularx}{\textwidth}{N T A}
\hline
Name & Type & Annotation\\
\hline
\hfuzz=500pt\includegraphics[width=1em]{element-mustset.pdf}~outputfileCorrelationMatrix & \hfuzz=500pt filename & \hfuzz=500pt correlation matrix\\
\hfuzz=500pt\includegraphics[width=1em]{element-mustset.pdf}~inputfileCovarianceMatrix & \hfuzz=500pt filename & \hfuzz=500pt covariance matrix\\
\hline
\end{tabularx}

This program is \reference{parallelized}{general.parallelization}.
\clearpage
%==================================
\subsection{PowerSpectralDensity2CovarianceFunction}\label{PowerSpectralDensity2CovarianceFunction}
Covariance function from Power Spectral Density (PSD).
The \configFile{inputfilePSD}{matrix} contains in the first column the frequency $[Hz]$, followed by (possibly multiple) PSDs $[unit^2/Hz]$.
The output is a \file{matrix}{matrix}, the first column containing time lag $[s]$ and the other columns the covariance functions $[unit^2]$.
Conversion between PSD $p_j$ and covariance function $c_k$ is performed by discrete cosine transformation:
\begin{equation}
c_k = \frac{1}{4\Delta t (n-1)}\left(p_0 + p_{n-1} (-1)^k + \sum_{j=1}^{n-2} 2 p_j \cos(\pi jk/(n-1))\right).
\end{equation}

See also \program{CovarianceFunction2PowerSpectralDensity}.


\keepXColumns
\begin{tabularx}{\textwidth}{N T A}
\hline
Name & Type & Annotation\\
\hline
\hfuzz=500pt\includegraphics[width=1em]{element-mustset.pdf}~outputfileCovarianceFunction & \hfuzz=500pt filename & \hfuzz=500pt first column: time steps [seconds], following columns: covariance functions\\
\hfuzz=500pt\includegraphics[width=1em]{element-mustset.pdf}~inputfilePSD & \hfuzz=500pt filename & \hfuzz=500pt first column: frequency [Hz], following columns PSD [unit\textasciicircum{}2/Hz]\\
\hline
\end{tabularx}

\clearpage
%==================================
\section{Programs: DoodsonHarmonics}
\subsection{DoodsonAdmittanceInterpolation}\label{DoodsonAdmittanceInterpolation}
To visualize the interpolation of the minor tides.
The output is a \file{matrix}{matrix} with the first column containing the tidal frequency,
the second column is the tide generating amplitude (from \configFile{inputfileTideGeneratingPotential}{tideGeneratingPotential}), and the following
columns the contribution of the major tides to the this tidal frequency as defined in in \configFile{inputfileAdmittance}{admittance}.

\fig{!hb}{0.8}{doodsonAdmittanceInterpolation}{fig:doodsonAdmittanceInterpolation}{Linear interpolation of minor tides in the diurnal band.}


\keepXColumns
\begin{tabularx}{\textwidth}{N T A}
\hline
Name & Type & Annotation\\
\hline
\hfuzz=500pt\includegraphics[width=1em]{element-mustset.pdf}~outputfile & \hfuzz=500pt filename & \hfuzz=500pt \\
\hfuzz=500pt\includegraphics[width=1em]{element-mustset.pdf}~inputfileAdmittance & \hfuzz=500pt filename & \hfuzz=500pt interpolation of minor constituents\\
\hfuzz=500pt\includegraphics[width=1em]{element.pdf}~inputfileTideGeneratingPotential & \hfuzz=500pt filename & \hfuzz=500pt \\
\hline
\end{tabularx}

\clearpage
%==================================
\subsection{DoodsonAdmittanceTimeSeries}\label{DoodsonAdmittanceTimeSeries}
To visualize the interpolation of the minor tides it computes cosine multipliers of all major tides.
Without admittance this would be a simple cos oscillation.
The \config{outputfileTimeSeries} is an \file{instrument file}{instrument} (MISCVALUES) containining the cos of all the major tides.

\fig{!hb}{0.8}{doodsonAdmittanceTimeSeries}{fig:doodsonAdmittanceTimeSeries}{Cosine of the Mf tidal frequency with modulation from the interpolated minor tides.}


\keepXColumns
\begin{tabularx}{\textwidth}{N T A}
\hline
Name & Type & Annotation\\
\hline
\hfuzz=500pt\includegraphics[width=1em]{element-mustset.pdf}~outputfileTimeSeries & \hfuzz=500pt filename & \hfuzz=500pt MISCVALUES (cos of major tides, ...)\\
\hfuzz=500pt\includegraphics[width=1em]{element-mustset.pdf}~inputfileAdmittance & \hfuzz=500pt filename & \hfuzz=500pt cos/sin multipliers of the major tides\\
\hfuzz=500pt\includegraphics[width=1em]{element-mustset-unbounded.pdf}~timeSeries & \hfuzz=500pt \hyperref[timeSeriesType]{timeSeries} & \hfuzz=500pt \\
\hline
\end{tabularx}

\clearpage
%==================================
\subsection{DoodsonArguments2TimeSeries}\label{DoodsonArguments2TimeSeries}
Time series of doodson/fundamental arguments.
The \configFile{outputfileTimeSeries}{instrument} contains the six Doodson arguments,
followed by the five fundamental arguments in radians.


\keepXColumns
\begin{tabularx}{\textwidth}{N T A}
\hline
Name & Type & Annotation\\
\hline
\hfuzz=500pt\includegraphics[width=1em]{element-mustset.pdf}~outputfileTimeSeries & \hfuzz=500pt filename & \hfuzz=500pt each epoch: 6 doodson args, 5 fundamental args [rad]\\
\hfuzz=500pt\includegraphics[width=1em]{element-mustset-unbounded.pdf}~timeSeries & \hfuzz=500pt \hyperref[timeSeriesType]{timeSeries} & \hfuzz=500pt \\
\hline
\end{tabularx}

\clearpage
%==================================
\subsection{DoodsonHarmonics2GriddedAmplitudeAndPhase}\label{DoodsonHarmonics2GriddedAmplitudeAndPhase}
This program reads a \configFile{inputfileDoodsonHarmonics}{doodsonHarmonic} and evaluates a single tidal
constituent selected by \config{dooddson} (Doodson number or Darwin´s name, e.g. 255.555 or M2).
This program computes the amplitude and phase from the cos and sin coefficients on
a given \configClass{grid}{gridType}. The type of functional (e.g gravity anomalies or geoid heights)
can be choosen with \configClass{kernel}{kernelType}.
The values will be saved together with points expressed as ellipsoidal coordinates (longitude, latitude, height)
based on a reference ellipsoid with parameters \config{R} and \config{inverseFlattening}.
To visualize the results use \program{PlotMap}.

\fig{!hb}{1.}{doodsonHarmonics2GriddedAmplitudeAndPhase}{fig:doodsonHarmonics2GriddedAmplitudeAndPhase}{M2 amplitude and phase of FES2014b.}


\keepXColumns
\begin{tabularx}{\textwidth}{N T A}
\hline
Name & Type & Annotation\\
\hline
\hfuzz=500pt\includegraphics[width=1em]{element-mustset.pdf}~outputfileGrid & \hfuzz=500pt filename & \hfuzz=500pt ampl, phase [-pi,pi], cos, sin\\
\hfuzz=500pt\includegraphics[width=1em]{element-mustset.pdf}~inputfileDoodsonHarmonics & \hfuzz=500pt filename & \hfuzz=500pt \\
\hfuzz=500pt\includegraphics[width=1em]{element-mustset.pdf}~doodson & \hfuzz=500pt \hyperref[doodson]{doodson} & \hfuzz=500pt tidal constituent\\
\hfuzz=500pt\includegraphics[width=1em]{element-unbounded.pdf}~filter & \hfuzz=500pt \hyperref[sphericalHarmonicsFilterType]{sphericalHarmonicsFilter} & \hfuzz=500pt \\
\hfuzz=500pt\includegraphics[width=1em]{element-mustset-unbounded.pdf}~grid & \hfuzz=500pt \hyperref[gridType]{grid} & \hfuzz=500pt \\
\hfuzz=500pt\includegraphics[width=1em]{element-mustset.pdf}~kernel & \hfuzz=500pt \hyperref[kernelType]{kernel} & \hfuzz=500pt \\
\hfuzz=500pt\includegraphics[width=1em]{element.pdf}~minDegree & \hfuzz=500pt uint & \hfuzz=500pt \\
\hfuzz=500pt\includegraphics[width=1em]{element.pdf}~maxDegree & \hfuzz=500pt uint & \hfuzz=500pt \\
\hfuzz=500pt\includegraphics[width=1em]{element.pdf}~factor & \hfuzz=500pt double & \hfuzz=500pt the values on grid are multiplied by this factor\\
\hfuzz=500pt\includegraphics[width=1em]{element.pdf}~R & \hfuzz=500pt double & \hfuzz=500pt reference radius for ellipsoidal coordinates on output\\
\hfuzz=500pt\includegraphics[width=1em]{element.pdf}~inverseFlattening & \hfuzz=500pt double & \hfuzz=500pt reference flattening for ellipsoidal coordinates on output, 0: spherical coordinates\\
\hline
\end{tabularx}

This program is \reference{parallelized}{general.parallelization}.
\clearpage
%==================================
\subsection{DoodsonHarmonics2PotentialCoefficients}\label{DoodsonHarmonics2PotentialCoefficients}
The \configFile{inputfileDoodsonHarmonics}{doodsonHarmonic} contains a Fourier series of a time variable
gravitational potential at specific tidal frequencies (tides)
\begin{equation}
V(\M x,t) = \sum_{f} V_f^c(\M x)\cos(\theta_f(t)) + V_f^s(\M x)\sin(\theta_f(t)),
\end{equation}
where $V_f^c(\M x)$ and $V_f^s(\M x)$ are spherical harmonics expansions.
If set the expansions are limited in the range between \config{minDegree}
and \config{maxDegree} inclusivly. The coefficients are related to the reference radius~\config{R}
and the Earth gravitational constant \config{GM}.

The \configFile{outputfilePotentialCoefficients}{potentialCoefficients} is not a single file but a series of files.
For each spherical harmonics expansion $V_f^c(\M x)$ and $V_f^s(\M x)$ a separate file is created
where the variables \config{variableLoopName}, \config{variableLoopDoodson}, \config{variableLoopCosSin} are set accordingly.
The file name should contain these variables, e.g. \verb|coeff.{name}.{doodson}.{cossin}.gfc|.

If \config{applyXi} the Doodson-Warburg phase correction (see IERS conventions) is applied to the cos/sin
potentialCoefficients before.


\keepXColumns
\begin{tabularx}{\textwidth}{N T A}
\hline
Name & Type & Annotation\\
\hline
\hfuzz=500pt\includegraphics[width=1em]{element-mustset.pdf}~outputfilePotentialCoefficients & \hfuzz=500pt filename & \hfuzz=500pt \\
\hfuzz=500pt\includegraphics[width=1em]{element.pdf}~variableLoopName & \hfuzz=500pt string & \hfuzz=500pt variable with darwins's name of each constituent\\
\hfuzz=500pt\includegraphics[width=1em]{element.pdf}~variableLoopDoodson & \hfuzz=500pt string & \hfuzz=500pt variable with doodson code of each constituent\\
\hfuzz=500pt\includegraphics[width=1em]{element.pdf}~variableLoopCosSin & \hfuzz=500pt string & \hfuzz=500pt variable with 'cos' or 'sin' of each constituent\\
\hfuzz=500pt\includegraphics[width=1em]{element.pdf}~variableLoopIndex & \hfuzz=500pt string & \hfuzz=500pt variable with index of each constituent (starts with zero)\\
\hfuzz=500pt\includegraphics[width=1em]{element.pdf}~variableLoopCount & \hfuzz=500pt string & \hfuzz=500pt variable with total number of constituents\\
\hfuzz=500pt\includegraphics[width=1em]{element-mustset.pdf}~inputfileDoodsonHarmonics & \hfuzz=500pt filename & \hfuzz=500pt \\
\hfuzz=500pt\includegraphics[width=1em]{element.pdf}~inputfileTideGeneratingPotential & \hfuzz=500pt filename & \hfuzz=500pt to compute Xi phase correction\\
\hfuzz=500pt\includegraphics[width=1em]{element.pdf}~minDegree & \hfuzz=500pt uint & \hfuzz=500pt \\
\hfuzz=500pt\includegraphics[width=1em]{element.pdf}~maxDegree & \hfuzz=500pt uint & \hfuzz=500pt \\
\hfuzz=500pt\includegraphics[width=1em]{element.pdf}~GM & \hfuzz=500pt double & \hfuzz=500pt Geocentric gravitational constant\\
\hfuzz=500pt\includegraphics[width=1em]{element.pdf}~R & \hfuzz=500pt double & \hfuzz=500pt reference radius\\
\hfuzz=500pt\includegraphics[width=1em]{element.pdf}~applyXi & \hfuzz=500pt boolean & \hfuzz=500pt apply Doodson-Warburg phase correction (see IERS conventions)\\
\hline
\end{tabularx}

\clearpage
%==================================
\subsection{DoodsonHarmonicsCalculateAdmittance}\label{DoodsonHarmonicsCalculateAdmittance}
Computes the admittance function to interpolate minor tides from
tides given in \configFile{inputfileDoodsonHarmonics}{doodsonHarmonic}
using \configFile{inputfileTideGeneratingPotential}{tideGeneratingPotential}.


\keepXColumns
\begin{tabularx}{\textwidth}{N T A}
\hline
Name & Type & Annotation\\
\hline
\hfuzz=500pt\includegraphics[width=1em]{element-mustset.pdf}~outputfileAdmittance & \hfuzz=500pt filename & \hfuzz=500pt \\
\hfuzz=500pt\includegraphics[width=1em]{element-mustset.pdf}~inputfileDoodsonHarmonics & \hfuzz=500pt filename & \hfuzz=500pt \\
\hfuzz=500pt\includegraphics[width=1em]{element.pdf}~inputfileTideGeneratingPotential & \hfuzz=500pt filename & \hfuzz=500pt TGP\\
\hfuzz=500pt\includegraphics[width=1em]{element.pdf}~threshold & \hfuzz=500pt double & \hfuzz=500pt [m\textasciicircum{}2/s\textasciicircum{}2] only interpolate tides with TGP greater than threshold\\
\hfuzz=500pt\includegraphics[width=1em]{element.pdf}~degreeInterpolation & \hfuzz=500pt uint & \hfuzz=500pt polynomial degree for interpolation\\
\hfuzz=500pt\includegraphics[width=1em]{element.pdf}~degreeExtrapolation & \hfuzz=500pt uint & \hfuzz=500pt polynomial degree for extrapolation\\
\hfuzz=500pt\includegraphics[width=1em]{element-unbounded.pdf}~excludeDoodsonForInterpolation & \hfuzz=500pt \hyperref[doodson]{doodson} & \hfuzz=500pt major tides not used for interpolation\\
\hline
\end{tabularx}

\clearpage
%==================================
\subsection{ModelEquilibriumTide}\label{ModelEquilibriumTide}
Computes the equilibrium ocean tide of the long periodic \config{tideGeneratingPotential}.

The ocean surface is represented by \configClass{gridOcean}{gridType} and the gravitational
effect is numerical integrated to spherical harmonics using \config{maxDegree}, \config{GM},
and \config{R}.

It takes self attraction and loading into account using the Love numbers
\configFile{inputfilePotentialLoadLoveNumber}{matrix} and
\configFile{inputfileDeformationLoadLoveNumber}{matrix}.

Additionally the effects of the solid Earth tide are considered,
both the gravitational (Love numbers \config{k20}, \config{k20plus})
and the geometrical (Love numbers \config{h20,0}, \config{h20,2}) effect.

See also \program{PotentialCoefficients2DoodsonHarmonics}.

\fig{!hb}{0.8}{modelEquilibriumTide}{fig:modelEquilibriumTide}{Equilibrium tide of SA constituent}


\keepXColumns
\begin{tabularx}{\textwidth}{N T A}
\hline
Name & Type & Annotation\\
\hline
\hfuzz=500pt\includegraphics[width=1em]{element-mustset.pdf}~outputfilePotentialCoefficients & \hfuzz=500pt filename & \hfuzz=500pt includes the loading\\
\hfuzz=500pt\includegraphics[width=1em]{element-mustset-unbounded.pdf}~gridOcean & \hfuzz=500pt \hyperref[gridType]{grid} & \hfuzz=500pt \\
\hfuzz=500pt\includegraphics[width=1em]{element.pdf}~maxDegree & \hfuzz=500pt uint & \hfuzz=500pt \\
\hfuzz=500pt\includegraphics[width=1em]{element.pdf}~GM & \hfuzz=500pt double & \hfuzz=500pt Geocentric gravitational constant\\
\hfuzz=500pt\includegraphics[width=1em]{element.pdf}~R & \hfuzz=500pt double & \hfuzz=500pt reference radius\\
\hfuzz=500pt\includegraphics[width=1em]{element.pdf}~density & \hfuzz=500pt double & \hfuzz=500pt [kg/m**3] density of sea water\\
\hfuzz=500pt\includegraphics[width=1em]{element-mustset.pdf}~tideGeneratingPotential & \hfuzz=500pt double & \hfuzz=500pt [m**2/s**2]\\
\hfuzz=500pt\includegraphics[width=1em]{element.pdf}~k20 & \hfuzz=500pt double & \hfuzz=500pt earth tide love number\\
\hfuzz=500pt\includegraphics[width=1em]{element.pdf}~k20plus & \hfuzz=500pt double & \hfuzz=500pt earth tide love number\\
\hfuzz=500pt\includegraphics[width=1em]{element.pdf}~h20\_0 & \hfuzz=500pt double & \hfuzz=500pt earth tide love number\\
\hfuzz=500pt\includegraphics[width=1em]{element.pdf}~h20\_2 & \hfuzz=500pt double & \hfuzz=500pt earth tide love number\\
\hfuzz=500pt\includegraphics[width=1em]{element-mustset.pdf}~inputfilePotentialLoadLoveNumber & \hfuzz=500pt filename & \hfuzz=500pt \\
\hfuzz=500pt\includegraphics[width=1em]{element-mustset.pdf}~inputfileDeformationLoadLoveNumber & \hfuzz=500pt filename & \hfuzz=500pt \\
\hline
\end{tabularx}

This program is \reference{parallelized}{general.parallelization}.
\clearpage
%==================================
\subsection{PotentialCoefficients2DoodsonHarmonics}\label{PotentialCoefficients2DoodsonHarmonics}
Create a \file{DoodsonHarmonic file}{doodsonHarmonic} from a list of
cos/sin \file{potentialCoefficients}{potentialCoefficients} for given \config{doodson}
(Doodson number or Darwin´s name, e.g. 255.555 or M2) tidal constituents.
If \config{applyXi} the Doodson-Warburg phase correction (see IERS conventions) is applied before.


\keepXColumns
\begin{tabularx}{\textwidth}{N T A}
\hline
Name & Type & Annotation\\
\hline
\hfuzz=500pt\includegraphics[width=1em]{element-mustset.pdf}~outputfileDoodsonHarmonics & \hfuzz=500pt filename & \hfuzz=500pt \\
\hfuzz=500pt\includegraphics[width=1em]{element.pdf}~inputfileTideGeneratingPotential & \hfuzz=500pt filename & \hfuzz=500pt to compute Xi phase correction\\
\hfuzz=500pt\includegraphics[width=1em]{element-mustset-unbounded.pdf}~constituent & \hfuzz=500pt sequence & \hfuzz=500pt \\
\hfuzz=500pt\includegraphics[width=1em]{connector.pdf}\includegraphics[width=1em]{element-mustset.pdf}~doodson & \hfuzz=500pt \hyperref[doodson]{doodson} & \hfuzz=500pt \\
\hfuzz=500pt\includegraphics[width=1em]{connector.pdf}\includegraphics[width=1em]{element-mustset.pdf}~inputfileCosPotentialCoefficients & \hfuzz=500pt filename & \hfuzz=500pt \\
\hfuzz=500pt\includegraphics[width=1em]{connector.pdf}\includegraphics[width=1em]{element-mustset.pdf}~inputfileSinPotentialCoefficients & \hfuzz=500pt filename & \hfuzz=500pt \\
\hfuzz=500pt\includegraphics[width=1em]{element.pdf}~minDegree & \hfuzz=500pt uint & \hfuzz=500pt \\
\hfuzz=500pt\includegraphics[width=1em]{element.pdf}~maxDegree & \hfuzz=500pt uint & \hfuzz=500pt \\
\hfuzz=500pt\includegraphics[width=1em]{element.pdf}~GM & \hfuzz=500pt double & \hfuzz=500pt Geocentric gravitational constant\\
\hfuzz=500pt\includegraphics[width=1em]{element.pdf}~R & \hfuzz=500pt double & \hfuzz=500pt reference radius\\
\hfuzz=500pt\includegraphics[width=1em]{element.pdf}~applyXi & \hfuzz=500pt boolean & \hfuzz=500pt apply Doodson-Warburg phase correction (see IERS conventions)\\
\hline
\end{tabularx}

\clearpage
%==================================
\section{Programs: Gnss}
\subsection{GnssAntennaDefinition2ParameterVector}\label{GnssAntennaDefinition2ParameterVector}
Estimates parameters of a parametrization of \configClass{antennaCenterVariations}{parametrizationGnssAntennaType},
which represents all antennas from \configFile{inputfileAntennaDefinition}{gnssAntennaDefinition}
matching the wildcard patterns of \config{name}, \config{serial}, \config{radome}.

The provided values at the grid points of the pattern of each gnssType are used as pseudo-observations.
A subset of patterns can be selected with \configClass{types}{gnssType}.

The \file{GnssAntennaDefinition file}{gnssAntennaDefinition} can be modified to the demands before with
\program{GnssAntennaDefinitionCreate}.

See also \program{ParameterVector2GnssAntennaDefinition}.


\keepXColumns
\begin{tabularx}{\textwidth}{N T A}
\hline
Name & Type & Annotation\\
\hline
\hfuzz=500pt\includegraphics[width=1em]{element-mustset.pdf}~outputfileSolution & \hfuzz=500pt filename & \hfuzz=500pt \\
\hfuzz=500pt\includegraphics[width=1em]{element.pdf}~outputfileParameterNames & \hfuzz=500pt filename & \hfuzz=500pt \\
\hfuzz=500pt\includegraphics[width=1em]{element-mustset-unbounded.pdf}~antennaCenterVariations & \hfuzz=500pt \hyperref[parametrizationGnssAntennaType]{parametrizationGnssAntenna} & \hfuzz=500pt \\
\hfuzz=500pt\includegraphics[width=1em]{element-mustset.pdf}~inputfileAntennaDefinition & \hfuzz=500pt filename & \hfuzz=500pt \\
\hfuzz=500pt\includegraphics[width=1em]{element.pdf}~name & \hfuzz=500pt string & \hfuzz=500pt \\
\hfuzz=500pt\includegraphics[width=1em]{element.pdf}~serial & \hfuzz=500pt string & \hfuzz=500pt \\
\hfuzz=500pt\includegraphics[width=1em]{element.pdf}~radome & \hfuzz=500pt string & \hfuzz=500pt \\
\hfuzz=500pt\includegraphics[width=1em]{element-unbounded.pdf}~types & \hfuzz=500pt \hyperref[gnssType]{gnssType} & \hfuzz=500pt if not set, all types in the file are used\\
\hline
\end{tabularx}

\clearpage
%==================================
\subsection{GnssAntennaDefinition2Skyplot}\label{GnssAntennaDefinition2Skyplot}
Produce a \file{skyplot}{griddedData} of antenna center variations
which can be plotted with \program{PlotMap}.

The first antenna from \configFile{inputfileAntennaDefinition}{gnssAntennaDefinition}
matching the wildcard patterns of \config{name}, \config{serial}, \config{radome} is used.

For each antenna pattern (gnssType) a separate data column is computed.
A subset of patterns can be selected with \configClass{types}{gnssType}.

Azimuth and elevation are written as ellipsoidal longitude and latitude in a \file{griddedData file}{griddedData}.
The choosen ellipsoid parameters \config{R} and \config{inverseFlattening} are arbitrary but should be the same
as in \configClass{grid}{gridType} and \program{PlotMap}.

\fig{!hb}{1.0}{fileFormatGnssAntennaDefinition}{fig:gnssAntennaDefinition2Skyplot}{Antenna Center Variations of ASH701945D\_M for two frequencies of GPS and GLONASS}


\keepXColumns
\begin{tabularx}{\textwidth}{N T A}
\hline
Name & Type & Annotation\\
\hline
\hfuzz=500pt\includegraphics[width=1em]{element-mustset.pdf}~outputfileGriddedData & \hfuzz=500pt filename & \hfuzz=500pt data column for each gnssType\\
\hfuzz=500pt\includegraphics[width=1em]{element-mustset.pdf}~inputfileAntennaDefinition & \hfuzz=500pt filename & \hfuzz=500pt \\
\hfuzz=500pt\includegraphics[width=1em]{element-mustset-unbounded.pdf}~grid & \hfuzz=500pt \hyperref[gridType]{grid} & \hfuzz=500pt \\
\hfuzz=500pt\includegraphics[width=1em]{element.pdf}~name & \hfuzz=500pt string & \hfuzz=500pt \\
\hfuzz=500pt\includegraphics[width=1em]{element.pdf}~serial & \hfuzz=500pt string & \hfuzz=500pt \\
\hfuzz=500pt\includegraphics[width=1em]{element.pdf}~radome & \hfuzz=500pt string & \hfuzz=500pt \\
\hfuzz=500pt\includegraphics[width=1em]{element-unbounded.pdf}~types & \hfuzz=500pt \hyperref[gnssType]{gnssType} & \hfuzz=500pt if not set, all types in the file are used\\
\hfuzz=500pt\includegraphics[width=1em]{element.pdf}~R & \hfuzz=500pt double & \hfuzz=500pt reference radius for ellipsoidal coordinates\\
\hfuzz=500pt\includegraphics[width=1em]{element.pdf}~inverseFlattening & \hfuzz=500pt double & \hfuzz=500pt reference flattening for ellipsoidal coordinates\\
\hline
\end{tabularx}

\clearpage
%==================================
\subsection{GnssAntennaDefinitionCreate}\label{GnssAntennaDefinitionCreate}
Create a \file{GNSS antenna definition file}{gnssAntennaDefinition} (Antenna Center Variations, ACV) consisting of multiple antennas.
The antennas can be created from scratch or can be selected from existing files.
This program can also be used to modify existing files.

Furthermore it can be used to create accuracy definition files containing azimuth and elevation dependent accuracy values for antennas.
To create an accuracy pattern for phase observations with \verb|1 mm| accuracy at zenith and no azimuth dependency, define a
pattern with \config{type}=\verb|L|, \config{values}=\verb|0.001/cos(zenith/rho)|.

The antennas in \configFile{outputfileAntennaDefinition}{gnssAntennaDefinition}
are sorted by names and duplicates are removed (first one is kept).


\keepXColumns
\begin{tabularx}{\textwidth}{N T A}
\hline
Name & Type & Annotation\\
\hline
\hfuzz=500pt\includegraphics[width=1em]{element-mustset.pdf}~outputfileAntennaDefinition & \hfuzz=500pt filename & \hfuzz=500pt \\
\hfuzz=500pt\includegraphics[width=1em]{element-mustset-unbounded.pdf}~antenna & \hfuzz=500pt \hyperref[gnssAntennaDefintionListType]{gnssAntennaDefintionList} & \hfuzz=500pt \\
\hline
\end{tabularx}

\clearpage
%==================================
\subsection{GnssAntennaNormalsConstraint}\label{GnssAntennaNormalsConstraint}
Apply constraints to \file{normal equations}{normalEquation}
containing \configClass{antennaCenterVariations}{parametrizationGnssAntennaType}.
Usually the antenna center variations are estimated together with other parameters
like station coordinates, signal biases and slant TEC in \program{GnssProcessing}.
This results in a rank deficient matrix as not all parameters can be separated.
The deficient can be solved by adding pseudo observation equations as constraints.

To separate antenna center variations and signal biases
apply \config{constraint:mean} for each GNSS \configClass{type}{gnssType}.
The observation equation for the integral mean of antenna center variations (ACV)
in all azimuth~$A$ and elevation~$E$ dependent directions
\begin{equation}
  0 = \iint ACV(A,E)\, d\Phi \approx \sum_i ACV(A_i,E_i)\, \Delta\Phi_i
\end{equation}
is approximated by a grid defined by
\config{deltaAzimuth}, \config{deltaZenith}, and \config{maxZenith}.

To separate from station coordinates use \config{constraint:centerMean}
and from slant TEC parameters use \config{constraint:TEC}.

The constraints are applied separately to all antennas matching
the wildcard patterns of \config{name}, \config{serial}, \config{radome}.

See also \program{ParameterVector2GnssAntennaDefinition}.


\keepXColumns
\begin{tabularx}{\textwidth}{N T A}
\hline
Name & Type & Annotation\\
\hline
\hfuzz=500pt\includegraphics[width=1em]{element-mustset.pdf}~outputfileNormalEquation & \hfuzz=500pt filename & \hfuzz=500pt with applied constraints\\
\hfuzz=500pt\includegraphics[width=1em]{element-mustset.pdf}~inputfileNormalEquation & \hfuzz=500pt filename & \hfuzz=500pt \\
\hfuzz=500pt\includegraphics[width=1em]{element-mustset-unbounded.pdf}~constraint & \hfuzz=500pt choice & \hfuzz=500pt \\
\hfuzz=500pt\includegraphics[width=1em]{connector.pdf}\includegraphics[width=1em]{element-mustset.pdf}~center & \hfuzz=500pt sequence & \hfuzz=500pt zero center (x,y,z) of a single pattern\\
\hfuzz=500pt\quad\includegraphics[width=1em]{connector.pdf}\includegraphics[width=1em]{element-mustset.pdf}~type & \hfuzz=500pt \hyperref[gnssType]{gnssType} & \hfuzz=500pt applied for each matching types\\
\hfuzz=500pt\quad\includegraphics[width=1em]{connector.pdf}\includegraphics[width=1em]{element.pdf}~applyWeight & \hfuzz=500pt boolean & \hfuzz=500pt from normal equations\\
\hfuzz=500pt\quad\includegraphics[width=1em]{connector.pdf}\includegraphics[width=1em]{element.pdf}~sigma & \hfuzz=500pt double & \hfuzz=500pt [m]\\
\hfuzz=500pt\includegraphics[width=1em]{connector.pdf}\includegraphics[width=1em]{element-mustset.pdf}~centerMean & \hfuzz=500pt sequence & \hfuzz=500pt zero center (x,y,z) as (weighted) mean of all patterns\\
\hfuzz=500pt\quad\includegraphics[width=1em]{connector.pdf}\includegraphics[width=1em]{element.pdf}~applyWeight & \hfuzz=500pt boolean & \hfuzz=500pt from normal equations\\
\hfuzz=500pt\quad\includegraphics[width=1em]{connector.pdf}\includegraphics[width=1em]{element.pdf}~sigma & \hfuzz=500pt double & \hfuzz=500pt [m]\\
\hfuzz=500pt\includegraphics[width=1em]{connector.pdf}\includegraphics[width=1em]{element-mustset.pdf}~constant & \hfuzz=500pt sequence & \hfuzz=500pt zero constant (mean of all directions) of a single pattern\\
\hfuzz=500pt\quad\includegraphics[width=1em]{connector.pdf}\includegraphics[width=1em]{element-mustset.pdf}~type & \hfuzz=500pt \hyperref[gnssType]{gnssType} & \hfuzz=500pt applied for each matching types\\
\hfuzz=500pt\quad\includegraphics[width=1em]{connector.pdf}\includegraphics[width=1em]{element.pdf}~applyWeight & \hfuzz=500pt boolean & \hfuzz=500pt from normal equations\\
\hfuzz=500pt\quad\includegraphics[width=1em]{connector.pdf}\includegraphics[width=1em]{element.pdf}~sigma & \hfuzz=500pt double & \hfuzz=500pt [m]\\
\hfuzz=500pt\includegraphics[width=1em]{connector.pdf}\includegraphics[width=1em]{element-mustset.pdf}~constantMean & \hfuzz=500pt sequence & \hfuzz=500pt zero constant (mean of all directions) as (weighted) mean of all patterns\\
\hfuzz=500pt\quad\includegraphics[width=1em]{connector.pdf}\includegraphics[width=1em]{element.pdf}~applyWeight & \hfuzz=500pt boolean & \hfuzz=500pt from normal equations\\
\hfuzz=500pt\quad\includegraphics[width=1em]{connector.pdf}\includegraphics[width=1em]{element.pdf}~sigma & \hfuzz=500pt double & \hfuzz=500pt [m]\\
\hfuzz=500pt\includegraphics[width=1em]{connector.pdf}\includegraphics[width=1em]{element-mustset.pdf}~TEC & \hfuzz=500pt sequence & \hfuzz=500pt zero TEC computed as (weighetd) least squares from all types\\
\hfuzz=500pt\quad\includegraphics[width=1em]{connector.pdf}\includegraphics[width=1em]{element.pdf}~applyWeight & \hfuzz=500pt boolean & \hfuzz=500pt from normal equations\\
\hfuzz=500pt\quad\includegraphics[width=1em]{connector.pdf}\includegraphics[width=1em]{element.pdf}~sigma & \hfuzz=500pt double & \hfuzz=500pt [TECU]\\
\hfuzz=500pt\includegraphics[width=1em]{element-mustset-unbounded.pdf}~antennaCenterVariations & \hfuzz=500pt \hyperref[parametrizationGnssAntennaType]{parametrizationGnssAntenna} & \hfuzz=500pt \\
\hfuzz=500pt\includegraphics[width=1em]{element.pdf}~antennaName & \hfuzz=500pt string & \hfuzz=500pt apply constraints to all machting antennas\\
\hfuzz=500pt\includegraphics[width=1em]{element.pdf}~antennaSerial & \hfuzz=500pt string & \hfuzz=500pt apply constraints to all machting antennas\\
\hfuzz=500pt\includegraphics[width=1em]{element.pdf}~antennaRadome & \hfuzz=500pt string & \hfuzz=500pt apply constraints to all machting antennas\\
\hfuzz=500pt\includegraphics[width=1em]{element.pdf}~deltaAzimuth & \hfuzz=500pt angle & \hfuzz=500pt [degree] sampling of pattern to estimate center/constant\\
\hfuzz=500pt\includegraphics[width=1em]{element.pdf}~deltaZenith & \hfuzz=500pt angle & \hfuzz=500pt [degree] sampling of pattern to estimate center/constant\\
\hfuzz=500pt\includegraphics[width=1em]{element.pdf}~maxZenith & \hfuzz=500pt angle & \hfuzz=500pt [degree] sampling of pattern to estimate center/constant\\
\hline
\end{tabularx}

\clearpage
%==================================
\subsection{GnssAttitudeInfoCreate}\label{GnssAttitudeInfoCreate}
Creates attitude info file (\file{Instrument(MISCVALUES)}{instrument})
used by \program{SimulateStarCameraGnss}. One or more \config{attitudeInfo}s can be specified.
They are valid from \config{timeStart} until the start of the subsequent \config{attitudeInfo}.
\config{maxManeuverTime} is used by \program{SimulateStarCameraGnss} to look
for ongoing orbit maneuvers before/after the given orbit that might affect the attitude at
the beginning or end of a given orbit.

\fig{!hb}{0.9}{gnssAttitudeModes}{fig:gnssAttitudeModes2}{Overview of attitude modes used by GNSS satellites}

Here is a list of GNSS satellite types for which the attitude behavior is known and their
respective attitude modes and required parameters:
\begin{itemize}
\item \textbf{GPS-II/IIA} [1]
\begin{itemize}
  \item \config{defaultMode}: nominalYawSteering
  \item \config{midnightMode}: shadowMaxYawSteeringAndRecovery
  \item \config{noonMode}: catchUpYawSteering
  \item \config{maxYawRate}: 0.12~deg/s
  \item \config{yawBias}: 0.5~deg
  \item \config{maxManeuverTime}: 2~h
\end{itemize}
\item \textbf{GPS-IIR/IIR-M} [1]
\begin{itemize}
  \item \config{defaultMode}: nominalYawSteering
  \item \config{midnightMode}: catchUpYawSteering
  \item \config{noonMode}: catchUpYawSteering
  \item \config{maxYawRate}: 0.2~deg/s
  \item \config{maxManeuverTime}: 30~min
\end{itemize}
\item \textbf{GPS-IIF} [2]
\begin{itemize}
  \item \config{defaultMode}: nominalYawSteering
  \item \config{midnightMode}: shadowConstantYawSteering
  \item \config{noonMode}: catchUpYawSteering
  \item \config{maxYawRate}: 0.11~deg/s
  \item \config{yawBias}: -0.7~deg
  \item \config{maxManeuverTime}: 1.5~h
\end{itemize}
\item \textbf{GLO-M} [3]
\begin{itemize}
  \item \config{defaultMode}: nominalYawSteering
  \item \config{midnightMode}: shadowMaxYawSteeringAndStop
  \item \config{noonMode}: centeredMaxYawSteering
  \item \config{maxYawRate}: 0.25~deg/s
  \item \config{noonBetaThreshold}: 2~deg
  \item \config{maxManeuverTime}: 1.5~h
\end{itemize}
\item \textbf{GAL-1} [4]
\begin{itemize}
  \item \config{defaultMode}: nominalYawSteering
  \item \config{midnightMode}: smoothedYawSteering1
  \item \config{noonMode}: smoothedYawSteering1
  \item \config{maxManeuverTime}: 1.5~h
\end{itemize}
\item \textbf{GAL-2} [4]
\begin{itemize}
  \item \config{defaultMode}: nominalYawSteering
  \item \config{midnightMode}: smoothedYawSteering2
  \item \config{noonMode}: smoothedYawSteering2
  \item \config{midnightBetaThreshold}: 4.1~deg
  \item \config{noonBetaThreshold}: 4.1~deg
  \item \config{activationThreshold}: 10~deg
  \item \config{maxManeuverTime}: 5656~s
\end{itemize}
\item \textbf{BDS-2G/3G} [5, 6]
\begin{itemize}
  \item \config{defaultMode}: orbitNormal
  \item \config{midnightMode}: orbitNormal
  \item \config{noonMode}: orbitNormal
\end{itemize}
\item \textbf{BDS-2I} [5]
\begin{itemize}
  \item \config{defaultMode}: nominalYawSteering
  \item \config{midnightMode}: betaDependentOrbitNormal
  \item \config{noonMode}: betaDependentOrbitNormal
  \item \config{maxYawRate}: 0.085~deg/s
  \item \config{midnightBetaThreshold}: 4~deg
  \item \config{noonBetaThreshold}: 4~deg
  \item \config{activationThreshold}: 5~deg
  \item \config{maxManeuverTime}: 24~h
\end{itemize}
\item \textbf{BDS-2M} [5]
\begin{itemize}
  \item \config{defaultMode}: nominalYawSteering
  \item \config{midnightMode}: betaDependentOrbitNormal
  \item \config{noonMode}: betaDependentOrbitNormal
  \item \config{maxYawRate}: 0.159~deg/s
  \item \config{midnightBetaThreshold}: 4~deg
  \item \config{noonBetaThreshold}: 4~deg
  \item \config{activationThreshold}: 5~deg
  \item \config{maxManeuverTime}: 13~h
\end{itemize}
\item \textbf{BDS-3I/3SI} [6]
\begin{itemize}
  \item \config{defaultMode}: nominalYawSteering
  \item \config{midnightMode}: smoothedYawSteering2
  \item \config{noonMode}: smoothedYawSteering2
  \item \config{midnightBetaThreshold}: 3~deg
  \item \config{noonBetaThreshold}: 3~deg
  \item \config{activationThreshold}: 6~deg
  \item \config{maxManeuverTime}: 5740~s
\end{itemize}
\item \textbf{BDS-3M/3SM} [6]
\begin{itemize}
  \item \config{defaultMode}: nominalYawSteering
  \item \config{midnightMode}: smoothedYawSteering2
  \item \config{noonMode}: smoothedYawSteering2
  \item \config{midnightBetaThreshold}: 3~deg
  \item \config{noonBetaThreshold}: 3~deg
  \item \config{activationThreshold}: 6~deg
  \item \config{maxManeuverTime}: 3090~s
\end{itemize}
\item \textbf{QZS-1} [7]
\begin{itemize}
  \item \config{defaultMode}: nominalYawSteering
  \item \config{midnightMode}: betaDependentOrbitNormal
  \item \config{noonMode}: betaDependentOrbitNormal
  \item \config{maxYawRate}: 0.01~deg/s
  \item \config{yawBias}: 180~deg
  \item \config{midnightBetaThreshold}: 20~deg
  \item \config{noonBetaThreshold}: 20~deg
  \item \config{activationThreshold}: 18.5~deg
  \item \config{maxManeuverTime}: 24~h
\end{itemize}
\item \textbf{QZS-2G} [7]
\begin{itemize}
  \item \config{defaultMode}: orbitNormal
  \item \config{midnightMode}: orbitNormal
  \item \config{noonMode}: orbitNormal
  \item \config{yawBias}: 180~deg
\end{itemize}
\item \textbf{QZS-2I} [7]
\begin{itemize}
  \item \config{defaultMode}: nominalYawSteering
  \item \config{midnightMode}: centeredMaxYawSteering
  \item \config{noonMode}: centeredMaxYawSteering
  \item \config{maxYawRate}: 0.055~deg/s
  \item \config{midnightBetaThreshold}: 5~deg
  \item \config{noonBetaThreshold}: 5~deg
  \item \config{maxManeuverTime}: 1.5~h
\end{itemize}
\end{itemize}

Some specific satellites may deviate in their attitude behavior or parameters
(e.g. G013-G040, R713, C005, C015, C017, J001).

References for the attitude behavior information:
\begin{enumerate}
\item \href{https://doi.org/10.1007/s10291-008-0092-1}{Kouba (2009)}
\item \href{https://doi.org/10.1007/s10291-016-0562-9}{Kuang et al. (2017)}
\item \href{https://doi.org/10.1016/j.asr.2010.09.007}{Dilssner et al. (2011)}
\item \url{https://www.gsc-europa.eu/support-to-developers/galileo-satellite-metadata#3}
\item \href{https://doi.org/10.1007/s10291-018-0783-1}{Wang et al. (2018)}
\item \href{https://doi.org/10.1017/S0373463318000103}{Li et al. (2018)}
\item \url{https://qzss.go.jp/en/technical/qzssinfo/index.html}
\end{enumerate}


\keepXColumns
\begin{tabularx}{\textwidth}{N T A}
\hline
Name & Type & Annotation\\
\hline
\hfuzz=500pt\includegraphics[width=1em]{element-mustset.pdf}~outputfileAttitudeInfo & \hfuzz=500pt filename & \hfuzz=500pt \\
\hfuzz=500pt\includegraphics[width=1em]{element-mustset-unbounded.pdf}~attitudeInfo & \hfuzz=500pt sequence & \hfuzz=500pt \\
\hfuzz=500pt\includegraphics[width=1em]{connector.pdf}\includegraphics[width=1em]{element-mustset.pdf}~timeStart & \hfuzz=500pt time & \hfuzz=500pt \\
\hfuzz=500pt\includegraphics[width=1em]{connector.pdf}\includegraphics[width=1em]{element-mustset.pdf}~defaultMode & \hfuzz=500pt choice & \hfuzz=500pt default attitude mode\\
\hfuzz=500pt\quad\includegraphics[width=1em]{connector.pdf}\includegraphics[width=1em]{element-mustset.pdf}~nominalYawSteering & \hfuzz=500pt  & \hfuzz=500pt yaw to keep solar panels aligned to Sun (e.g. most GNSS satellites outside eclipse)\\
\hfuzz=500pt\quad\includegraphics[width=1em]{connector.pdf}\includegraphics[width=1em]{element-mustset.pdf}~orbitNormal & \hfuzz=500pt  & \hfuzz=500pt keep fixed yaw angle, for example point X-axis in flight direction (e.g. BDS-2G, BDS-3G, QZS-2G)\\
\hfuzz=500pt\includegraphics[width=1em]{connector.pdf}\includegraphics[width=1em]{element-mustset.pdf}~midnightMode & \hfuzz=500pt choice & \hfuzz=500pt attitude mode for maneuvers around orbit midnight\\
\hfuzz=500pt\quad\includegraphics[width=1em]{connector.pdf}\includegraphics[width=1em]{element-mustset.pdf}~nominalYawSteering & \hfuzz=500pt  & \hfuzz=500pt yaw to keep solar panels aligned to Sun (e.g. most GNSS satellites outside eclipse)\\
\hfuzz=500pt\quad\includegraphics[width=1em]{connector.pdf}\includegraphics[width=1em]{element-mustset.pdf}~orbitNormal & \hfuzz=500pt  & \hfuzz=500pt keep fixed yaw angle, for example point X-axis in flight direction (e.g. BDS-2G, BDS-3G, QZS-2G)\\
\hfuzz=500pt\quad\includegraphics[width=1em]{connector.pdf}\includegraphics[width=1em]{element-mustset.pdf}~catchUpYawSteering & \hfuzz=500pt  & \hfuzz=500pt yaw at maximum yaw rate to catch up to nominal yaw angle (e.g. GPS-* (noon), GPS-IIR (midnight))\\
\hfuzz=500pt\quad\includegraphics[width=1em]{connector.pdf}\includegraphics[width=1em]{element-mustset.pdf}~shadowMaxYawSteeringAndRecovery & \hfuzz=500pt  & \hfuzz=500pt yaw at maximum yaw rate from shadow start to end, recover after shadow (e.g. GPS-IIA (midnight))\\
\hfuzz=500pt\quad\includegraphics[width=1em]{connector.pdf}\includegraphics[width=1em]{element-mustset.pdf}~shadowMaxYawSteeringAndStop & \hfuzz=500pt  & \hfuzz=500pt yaw at maximum yaw rate from shadow start until nominal yaw angle at shadow end is reached, then stop (e.g. GLO-M (midnight))\\
\hfuzz=500pt\quad\includegraphics[width=1em]{connector.pdf}\includegraphics[width=1em]{element-mustset.pdf}~shadowConstantYawSteering & \hfuzz=500pt  & \hfuzz=500pt yaw at constant yaw rate from shadow start to end (e.g. GPS-IIF (midnight))\\
\hfuzz=500pt\quad\includegraphics[width=1em]{connector.pdf}\includegraphics[width=1em]{element-mustset.pdf}~centeredMaxYawSteering & \hfuzz=500pt  & \hfuzz=500pt yaw at maximum yaw rate centered around noon/midnight (e.g. QZS-2I, GLO-M (noon))\\
\hfuzz=500pt\quad\includegraphics[width=1em]{connector.pdf}\includegraphics[width=1em]{element-mustset.pdf}~smoothedYawSteering1 & \hfuzz=500pt  & \hfuzz=500pt yaw based on an auxiliary Sun vector for a smooth yaw maneuver (e.g. GAL-1)\\
\hfuzz=500pt\quad\includegraphics[width=1em]{connector.pdf}\includegraphics[width=1em]{element-mustset.pdf}~smoothedYawSteering2 & \hfuzz=500pt  & \hfuzz=500pt yaw based on a modified yaw-steering law for a smooth yaw maneuver (e.g. GAL-2, BDS-3M, BDS-3I)\\
\hfuzz=500pt\quad\includegraphics[width=1em]{connector.pdf}\includegraphics[width=1em]{element-mustset.pdf}~betaDependentOrbitNormal & \hfuzz=500pt  & \hfuzz=500pt switch to orbit normal mode if below beta angle threshold (e.g. BDS-2M, BDS-2I, QZS-1)\\
\hfuzz=500pt\includegraphics[width=1em]{connector.pdf}\includegraphics[width=1em]{element-mustset.pdf}~noonMode & \hfuzz=500pt choice & \hfuzz=500pt attitude mode for maneuvers around orbit noon\\
\hfuzz=500pt\quad\includegraphics[width=1em]{connector.pdf}\includegraphics[width=1em]{element-mustset.pdf}~nominalYawSteering & \hfuzz=500pt  & \hfuzz=500pt yaw to keep solar panels aligned to Sun (e.g. most GNSS satellites outside eclipse)\\
\hfuzz=500pt\quad\includegraphics[width=1em]{connector.pdf}\includegraphics[width=1em]{element-mustset.pdf}~orbitNormal & \hfuzz=500pt  & \hfuzz=500pt keep fixed yaw angle, for example point X-axis in flight direction (e.g. BDS-2G, BDS-3G, QZS-2G)\\
\hfuzz=500pt\quad\includegraphics[width=1em]{connector.pdf}\includegraphics[width=1em]{element-mustset.pdf}~catchUpYawSteering & \hfuzz=500pt  & \hfuzz=500pt yaw at maximum yaw rate to catch up to nominal yaw angle (e.g. GPS-* (noon), GPS-IIR (midnight))\\
\hfuzz=500pt\quad\includegraphics[width=1em]{connector.pdf}\includegraphics[width=1em]{element-mustset.pdf}~centeredMaxYawSteering & \hfuzz=500pt  & \hfuzz=500pt yaw at maximum yaw rate centered around noon/midnight (e.g. QZS-2I, GLO-M (noon))\\
\hfuzz=500pt\quad\includegraphics[width=1em]{connector.pdf}\includegraphics[width=1em]{element-mustset.pdf}~smoothedYawSteering1 & \hfuzz=500pt  & \hfuzz=500pt yaw based on an auxiliary Sun vector for a smooth yaw maneuver (e.g. GAL-1)\\
\hfuzz=500pt\quad\includegraphics[width=1em]{connector.pdf}\includegraphics[width=1em]{element-mustset.pdf}~smoothedYawSteering2 & \hfuzz=500pt  & \hfuzz=500pt yaw based on a modified yaw-steering law for a smooth yaw maneuver (e.g. GAL-2, BDS-3M, BDS-3I)\\
\hfuzz=500pt\quad\includegraphics[width=1em]{connector.pdf}\includegraphics[width=1em]{element-mustset.pdf}~betaDependentOrbitNormal & \hfuzz=500pt  & \hfuzz=500pt switch to orbit normal mode if below beta angle threshold (e.g. BDS-2M, BDS-2I, QZS-1)\\
\hfuzz=500pt\includegraphics[width=1em]{connector.pdf}\includegraphics[width=1em]{element.pdf}~maxYawRate & \hfuzz=500pt double & \hfuzz=500pt [degree/s] maximum yaw rate of the satellite\\
\hfuzz=500pt\includegraphics[width=1em]{connector.pdf}\includegraphics[width=1em]{element.pdf}~yawBias & \hfuzz=500pt double & \hfuzz=500pt [degree] yaw bias applied in satellite attitude control system\\
\hfuzz=500pt\includegraphics[width=1em]{connector.pdf}\includegraphics[width=1em]{element.pdf}~midnightBetaThreshold & \hfuzz=500pt double & \hfuzz=500pt [degree] limit midnight maneuver to this absolute angle of the Sun above/below the satellite orbital plane\\
\hfuzz=500pt\includegraphics[width=1em]{connector.pdf}\includegraphics[width=1em]{element.pdf}~noonBetaThreshold & \hfuzz=500pt double & \hfuzz=500pt [degree] limit noon maneuver to this absolute angle of the Sun above/below the satellite orbital plane\\
\hfuzz=500pt\includegraphics[width=1em]{connector.pdf}\includegraphics[width=1em]{element.pdf}~activationThreshold & \hfuzz=500pt double & \hfuzz=500pt [degree] limit maneuver to this yaw/Earth-spacecraft-Sun angle (depending on mode)\\
\hfuzz=500pt\includegraphics[width=1em]{connector.pdf}\includegraphics[width=1em]{element.pdf}~maxManeuverTime & \hfuzz=500pt double & \hfuzz=500pt [s] maximum duration of maneuver or maximum maneuver lookup time before/after orbit start/end\\
\hline
\end{tabularx}

\clearpage
%==================================
\subsection{GnssBiasClockAlignment}\label{GnssBiasClockAlignment}
This program can be used to absolutely align GNSS transmitter clocks to reference clocks (i.e. broadcast clocks).
Each 'group' of \config{transmitter}s, usually a system like GPS or Galileo, is aligned individually by a constant shift over all transmitters.
If \config{alignClocksByFreqNo} is set, GLONASS transmitters will be divided by frequency number into groups of nominally two transmitters.
The offset between clocks and reference clocks will be shifted into receiver code biases, if \config{receiver} is provided."

By setting \config{alignFreqNoBiasesAtReceiver} and providing \config{receiver}, this program can further align GLONASS transmitter signal
biases so that the differences between frequency number-dependent receiver signal biases are minimal, which helps if PPP users don't set
up individual signal biases per frequency number at the receiver. Alignment is done by computing signal bias residuals to the mean over all
frequency numbers of a signal type at each receiver and then computing the means over all receivers for each frequency number and shifting
those from the receiver signal biases to the transmitter signal biases. Internal consistency of the biases is not affected by this.

If you only want to align GLONASS frequency numbers, provide the same clocks in
\configFile{inputfileClock}{instrument} and \configFile{inputfileReferenceClock}{instrument}.


\keepXColumns
\begin{tabularx}{\textwidth}{N T A}
\hline
Name & Type & Annotation\\
\hline
\hfuzz=500pt\includegraphics[width=1em]{element-mustset-unbounded.pdf}~transmitter & \hfuzz=500pt sequence & \hfuzz=500pt one element per satellite\\
\hfuzz=500pt\includegraphics[width=1em]{connector.pdf}\includegraphics[width=1em]{element-mustset.pdf}~outputfileClock & \hfuzz=500pt filename & \hfuzz=500pt aligned clock instrument file\\
\hfuzz=500pt\includegraphics[width=1em]{connector.pdf}\includegraphics[width=1em]{element.pdf}~outputfileSignalBias & \hfuzz=500pt filename & \hfuzz=500pt (GLONASS only) aligned signal bias file\\
\hfuzz=500pt\includegraphics[width=1em]{connector.pdf}\includegraphics[width=1em]{element-mustset.pdf}~inputfileClock & \hfuzz=500pt filename & \hfuzz=500pt clock instrument file\\
\hfuzz=500pt\includegraphics[width=1em]{connector.pdf}\includegraphics[width=1em]{element-mustset.pdf}~inputfileReferenceClock & \hfuzz=500pt filename & \hfuzz=500pt reference clock instrument file\\
\hfuzz=500pt\includegraphics[width=1em]{connector.pdf}\includegraphics[width=1em]{element.pdf}~inputfileSignalBias & \hfuzz=500pt filename & \hfuzz=500pt (GLONASS only) signal bias file\\
\hfuzz=500pt\includegraphics[width=1em]{connector.pdf}\includegraphics[width=1em]{element-mustset.pdf}~inputfileTransmitterInfo & \hfuzz=500pt filename & \hfuzz=500pt transmitter info file\\
\hfuzz=500pt\includegraphics[width=1em]{element-unbounded.pdf}~receiver & \hfuzz=500pt sequence & \hfuzz=500pt one element per station\\
\hfuzz=500pt\includegraphics[width=1em]{connector.pdf}\includegraphics[width=1em]{element-mustset.pdf}~outputfileSignalBias & \hfuzz=500pt filename & \hfuzz=500pt aligned signal bias file\\
\hfuzz=500pt\includegraphics[width=1em]{connector.pdf}\includegraphics[width=1em]{element-mustset.pdf}~inputfileSignalBias & \hfuzz=500pt filename & \hfuzz=500pt signal bias file\\
\hfuzz=500pt\includegraphics[width=1em]{element.pdf}~alignClocksByFreqNo & \hfuzz=500pt boolean & \hfuzz=500pt align clocks for each GLONASS frequency number separately\\
\hfuzz=500pt\includegraphics[width=1em]{element.pdf}~alignFreqNoBiasesAtReceiver & \hfuzz=500pt boolean & \hfuzz=500pt align frequency number-dependent code biases for each receiver\\
\hline
\end{tabularx}

\clearpage
%==================================
\subsection{GnssEstimateClockShift}\label{GnssEstimateClockShift}
This program estimates an epoch-wise clock shift in a constellation of GNSS satellites.
Each separate \config{data} represents a satellite... (e.g. 32 GPS satellites).
The shift to reference clocks can be estimated by providing \configFile{inputfileInstrumentRef}{instrument}.
Clock shifts are estimated for each epoch given by \configClass{timeSeries}{timeSeriesType}.


\keepXColumns
\begin{tabularx}{\textwidth}{N T A}
\hline
Name & Type & Annotation\\
\hline
\hfuzz=500pt\includegraphics[width=1em]{element.pdf}~outputfileShiftTimeSeries & \hfuzz=500pt filename & \hfuzz=500pt columns: mjd, clock shift\\
\hfuzz=500pt\includegraphics[width=1em]{element-mustset-unbounded.pdf}~data & \hfuzz=500pt sequence & \hfuzz=500pt e.g. satellite\\
\hfuzz=500pt\includegraphics[width=1em]{connector.pdf}\includegraphics[width=1em]{element.pdf}~outputfileInstrument & \hfuzz=500pt filename & \hfuzz=500pt corrected clocks\\
\hfuzz=500pt\includegraphics[width=1em]{connector.pdf}\includegraphics[width=1em]{element.pdf}~outputfileInstrumentDiff & \hfuzz=500pt filename & \hfuzz=500pt clock difference after correction\\
\hfuzz=500pt\includegraphics[width=1em]{connector.pdf}\includegraphics[width=1em]{element-mustset.pdf}~inputfileInstrument & \hfuzz=500pt filename & \hfuzz=500pt input clocks\\
\hfuzz=500pt\includegraphics[width=1em]{connector.pdf}\includegraphics[width=1em]{element.pdf}~inputfileInstrumentRef & \hfuzz=500pt filename & \hfuzz=500pt reference clocks (subtracted from input clocks)\\
\hfuzz=500pt\includegraphics[width=1em]{element-mustset-unbounded.pdf}~timeSeries & \hfuzz=500pt \hyperref[timeSeriesType]{timeSeries} & \hfuzz=500pt clock epochs\\
\hfuzz=500pt\includegraphics[width=1em]{element.pdf}~margin & \hfuzz=500pt double & \hfuzz=500pt [s] margin for time comparison\\
\hline
\end{tabularx}

\clearpage
%==================================
\subsection{GnssGlonassFrequencyNumberUpdate}\label{GnssGlonassFrequencyNumberUpdate}
Update/set GLONASS frequency number in \configFile{inputfileTransmitterInfo}{gnssStationInfo} files.

PRN/SVN to frequency number source: \url{http://semisys.gfz-potsdam.de/semisys/api/?symname=2002&format=json&satellite=GLO}.

See also \program{GnssAntex2AntennaDefinition}.


\keepXColumns
\begin{tabularx}{\textwidth}{N T A}
\hline
Name & Type & Annotation\\
\hline
\hfuzz=500pt\includegraphics[width=1em]{element.pdf}~outputfileTransmitterInfo & \hfuzz=500pt filename & \hfuzz=500pt templated for PRN list (variableNamePrn)\\
\hfuzz=500pt\includegraphics[width=1em]{element-mustset.pdf}~inputfileTransmitterInfo & \hfuzz=500pt filename & \hfuzz=500pt templated for PRN list (variableNamePrn)\\
\hfuzz=500pt\includegraphics[width=1em]{element-mustset.pdf}~inputfilePrn2FrequencyNumber & \hfuzz=500pt filename & \hfuzz=500pt GROOPS matrix with columns: GLONASS PRN, SVN, mjdStart, mjdEnd, frequencyNumber\\
\hfuzz=500pt\includegraphics[width=1em]{element-unbounded.pdf}~prn & \hfuzz=500pt string & \hfuzz=500pt PRN (e.g. G01) for transmitter info files\\
\hfuzz=500pt\includegraphics[width=1em]{element.pdf}~variableNamePrn & \hfuzz=500pt string & \hfuzz=500pt variable name for PRN in transmitter info files\\
\hline
\end{tabularx}

\clearpage
%==================================
\subsection{GnssPrn2SvnBlockVariables}\label{GnssPrn2SvnBlockVariables}
Create \reference{variables}{general.parser} containing SVN and block based on an
\configFile{inputfileTransmitterInfo}{gnssStationInfo} of a GNSS satellite/PRN and
a specified \config{time}.


\keepXColumns
\begin{tabularx}{\textwidth}{N T A}
\hline
Name & Type & Annotation\\
\hline
\hfuzz=500pt\includegraphics[width=1em]{element.pdf}~variableSVN & \hfuzz=500pt string & \hfuzz=500pt name of the SVN variable\\
\hfuzz=500pt\includegraphics[width=1em]{element.pdf}~variableBlock & \hfuzz=500pt string & \hfuzz=500pt name of the satellites block variable\\
\hfuzz=500pt\includegraphics[width=1em]{element-mustset.pdf}~inputfileTransmitterInfo & \hfuzz=500pt filename & \hfuzz=500pt used for GNSS PRN-to-SVN/model relation\\
\hfuzz=500pt\includegraphics[width=1em]{element-mustset.pdf}~time & \hfuzz=500pt time & \hfuzz=500pt used for GNSS PRN-to-SVN/model relation\\
\hline
\end{tabularx}

This program is \reference{parallelized}{general.parallelization}.
\clearpage
%==================================
\subsection{GnssProcessing}\label{GnssProcessing}
This program processes GNSS observations. It calculates the linearized observation equations,
accumulates them into a system of normal equations and solves it.

The primary use cases of this program are:
\begin{itemize}
  \item \reference{GNSS satellite orbit determination and station network analysis}{cookbook.gnssNetwork}
  \item \reference{Kinematic orbit determination of LEO satellites}{cookbook.kinematicOrbit}
  \item \reference{GNSS precise point positioning (PPP)}{cookbook.gnssPpp}
\end{itemize}

The observation epochs are defined by \configClass{timeSeries}{timeSeriesType}
and only observations at these epochs (within a \config{timeMargin}) are considered.

To calculate observation equations from the tracks, the model parameters or unknown parameters need to be
defined beforehand. These unknown parameters can be chosen arbitrarily by the user with an adequate list of defined
\configClass{parametrization}{gnssParametrizationType}.
Some of the \configClass{parametrization}{gnssParametrizationType} also include a priori models.

Lastly it is required to define the process flow of the gnssProcessing. This is accomplished
with a list of \configClass{processingSteps}{gnssProcessingStepType}.
Each step is processed consecutively. Some steps allow the selection of parameters, epochs,
or the normal equation structure, which affects all subsequent steps.
A minimal example consists of following steps:
\begin{itemize}
  \item \configClass{estimate}{gnssProcessingStepType:estimate}: iterative float solution with outlier downeighting
  \item \configClass{resolveAmbiguities}{gnssProcessingStepType:resolveAmbiguities}:
        fix ambiguities to integer and remove them from the normals
  \item \configClass{estimate}{gnssProcessingStepType:estimate}: few iteration for final outlier downweighting
  \item \configClass{writeResults}{gnssProcessingStepType:writeResults}:
        write the output files defined in \configClass{parametrization}{gnssParametrizationType}
\end{itemize}

If the program is run on multiple processes the \configClass{receiver}{gnssReceiverGeneratorType}s
(stations or LEO satellites) are distributed over the processes.

See also \program{GnssSimulateReceiver}.


\keepXColumns
\begin{tabularx}{\textwidth}{N T A}
\hline
Name & Type & Annotation\\
\hline
\hfuzz=500pt\includegraphics[width=1em]{element-mustset-unbounded.pdf}~timeSeries & \hfuzz=500pt \hyperref[timeSeriesType]{timeSeries} & \hfuzz=500pt defines observation epochs\\
\hfuzz=500pt\includegraphics[width=1em]{element.pdf}~timeMargin & \hfuzz=500pt double & \hfuzz=500pt [seconds] margin to consider two times identical\\
\hfuzz=500pt\includegraphics[width=1em]{element-mustset-unbounded.pdf}~transmitter & \hfuzz=500pt \hyperref[gnssTransmitterGeneratorType]{gnssTransmitterGenerator} & \hfuzz=500pt constellation of GNSS satellites\\
\hfuzz=500pt\includegraphics[width=1em]{element-mustset-unbounded.pdf}~receiver & \hfuzz=500pt \hyperref[gnssReceiverGeneratorType]{gnssReceiverGenerator} & \hfuzz=500pt ground station network or LEO satellite\\
\hfuzz=500pt\includegraphics[width=1em]{element-mustset.pdf}~earthRotation & \hfuzz=500pt \hyperref[earthRotationType]{earthRotation} & \hfuzz=500pt apriori earth rotation\\
\hfuzz=500pt\includegraphics[width=1em]{element-mustset-unbounded.pdf}~parametrization & \hfuzz=500pt \hyperref[gnssParametrizationType]{gnssParametrization} & \hfuzz=500pt models and parameters\\
\hfuzz=500pt\includegraphics[width=1em]{element-mustset-unbounded.pdf}~processingStep & \hfuzz=500pt \hyperref[gnssProcessingStepType]{gnssProcessingStep} & \hfuzz=500pt steps are processed consecutively\\
\hline
\end{tabularx}

This program is \reference{parallelized}{general.parallelization}.
\clearpage
%==================================
\subsection{GnssReceiverDefinitionCreate}\label{GnssReceiverDefinitionCreate}
Create a \file{GNSS receiver definition file}{gnssReceiverDefinition}.


\keepXColumns
\begin{tabularx}{\textwidth}{N T A}
\hline
Name & Type & Annotation\\
\hline
\hfuzz=500pt\includegraphics[width=1em]{element-mustset.pdf}~outputfileGnssReceiverDefinition & \hfuzz=500pt filename & \hfuzz=500pt \\
\hfuzz=500pt\includegraphics[width=1em]{element-mustset-unbounded.pdf}~receiverDefinition & \hfuzz=500pt sequence & \hfuzz=500pt \\
\hfuzz=500pt\includegraphics[width=1em]{connector.pdf}\includegraphics[width=1em]{element-mustset.pdf}~name & \hfuzz=500pt string & \hfuzz=500pt \\
\hfuzz=500pt\includegraphics[width=1em]{connector.pdf}\includegraphics[width=1em]{element.pdf}~serial & \hfuzz=500pt string & \hfuzz=500pt \\
\hfuzz=500pt\includegraphics[width=1em]{connector.pdf}\includegraphics[width=1em]{element.pdf}~version & \hfuzz=500pt string & \hfuzz=500pt \\
\hfuzz=500pt\includegraphics[width=1em]{connector.pdf}\includegraphics[width=1em]{element.pdf}~comment & \hfuzz=500pt string & \hfuzz=500pt \\
\hfuzz=500pt\includegraphics[width=1em]{connector.pdf}\includegraphics[width=1em]{element-unbounded.pdf}~gnssType & \hfuzz=500pt \hyperref[gnssType]{gnssType} & \hfuzz=500pt \\
\hline
\end{tabularx}

\clearpage
%==================================
\subsection{GnssResiduals2AccuracyDefinition}\label{GnssResiduals2AccuracyDefinition}
Compute antenna accuracies from observation \configFile{inputfileResiduals}{instrument}.
The \configFile{inputfileStationInfo}{gnssStationInfo} is needed to assign
the residuals to the equipped antenna at observation times.

The \configFile{outputfileAccuracyDefinition}{gnssAntennaDefinition} contains
at first step the same accuracy information for all antennas as the input file.
Only the azimuth~$A$ and elevation~$E$ dependent grid points of the patterns
where enough residuals are available ($>$ \config{minRedundancy})
are replaced by estimated accuracy
\begin{equation}
 \sigma(A,E) = \sqrt{\frac{\sum_i e_i^2(A,E)}{\sum_i r_i(A,E)}},
\end{equation}
where $e_i$ are the azimuth and elevation dependent residuals and $r_i$ the
corresponding redundancies (number of observations minus the contribution to
the estimated parameters).

The \configFile{inputfileAccuracyDefinition}{gnssAntennaDefinition} can be modified
to the demands before with \program{GnssAntennaDefinitionCreate}
(e.g. with \config{antenna:resample}).

To verify the results the \configFile{outputfileAntennaMean}{gnssAntennaDefinition}
and the accumulated \configFile{outputfileAntennaRedundancy}{gnssAntennaDefinition}
of the computed pattern grid points can be written.

Example: Analysis of TerraSAR-X residuals of one month shows that low elevation
GPS satellites are not tracked by the onboard receiver. An estimation of accuracies
for these directions is not possible from the residuals and the apriori accuracies
are left untouched. The other directions show very low phase noise hardly elevation
and azimuth dependent for L2W. A nearly zero mean indicates the use of adequate antennca
center variations in the processing.

\fig{!hb}{0.8}{gnssResiduals2AccuracyDefinition}{fig:gnssResiduals2AccuracyDefinition}{L2W accuracies of TerraSAR-X determined from residuals of one month}


\keepXColumns
\begin{tabularx}{\textwidth}{N T A}
\hline
Name & Type & Annotation\\
\hline
\hfuzz=500pt\includegraphics[width=1em]{element.pdf}~outputfileAccuracyDefinition & \hfuzz=500pt filename & \hfuzz=500pt elevation and azimuth dependent accuracy\\
\hfuzz=500pt\includegraphics[width=1em]{element.pdf}~outputfileAntennaMean & \hfuzz=500pt filename & \hfuzz=500pt weighted mean of the residuals\\
\hfuzz=500pt\includegraphics[width=1em]{element.pdf}~outputfileAntennaRedundancy & \hfuzz=500pt filename & \hfuzz=500pt redundancy of adjustment\\
\hfuzz=500pt\includegraphics[width=1em]{element-mustset.pdf}~inputfileAccuracyDefinition & \hfuzz=500pt filename & \hfuzz=500pt apriori accuracies\\
\hfuzz=500pt\includegraphics[width=1em]{element-mustset.pdf}~inputfileStationInfo & \hfuzz=500pt filename & \hfuzz=500pt to assign residuals to antennas\\
\hfuzz=500pt\includegraphics[width=1em]{element.pdf}~isTransmitter & \hfuzz=500pt boolean & \hfuzz=500pt stationInfo is of a transmitter\\
\hfuzz=500pt\includegraphics[width=1em]{element.pdf}~thresholdOutlier & \hfuzz=500pt double & \hfuzz=500pt ignore residuals with sigma/sigma0 greater than threshold\\
\hfuzz=500pt\includegraphics[width=1em]{element.pdf}~minRedundancy & \hfuzz=500pt double & \hfuzz=500pt min number of residuals. to estimate sigma\\
\hfuzz=500pt\includegraphics[width=1em]{element-mustset-unbounded.pdf}~inputfileResiduals & \hfuzz=500pt filename & \hfuzz=500pt GNSS receiver residuals\\
\hline
\end{tabularx}

\clearpage
%==================================
\subsection{GnssResiduals2Skyplot}\label{GnssResiduals2Skyplot}
Write GNSS residuals together with azimuth and elevation to be plotted with \program{PlotMap}.
Azimuth and elevation are written as ellipsoidal longitude and latitude in a \file{griddedData file}{griddedData}.
The choosen ellipsoid parameters \config{R} and \config{inverseFlattening} are arbitrary but should be the same
as in \program{PlotMap}. If with \configClass{typeTransmitter}{gnssType} (e.g. '\verb|***G18|')
a single transmitter is selected the azimuth and elevation are computed from the transmitter point of view.

For each GNSS \configClass{type}{gnssType} an extra data column is created.

A \file{GNSS residual file}{instrument} includes additional information
besides the residuals, which can also be selected with \configClass{type}{gnssType}
\begin{itemize}
\item \verb|A1*|, \verb|E1*|: azimuth and elevation at receiver
\item \verb|A2*|, \verb|E2*|: azimuth and elevation at transmitter
\item \verb|I**|: Estimated slant total electron content (STEC)
\end{itemize}

Furthermore these files may include for each residual \configClass{type}{gnssType}
information about the redundancy and the accuracy relation $\sigma/\sigma_0$
of the estimated $\sigma$ versus the apriori $\sigma_0$ from the least squares adjustment.
The 3 values (residuals, redundancy, $\sigma/\sigma_0$) are coded with the same type.
To get access to all values the corresponding type must be repeated in \configClass{type}{gnssType}.

\fig{!hb}{0.5}{gnssResiduals2Skyplot}{fig:gnssResiduals2Skyplot}{GPS C2W residuals of GRAZ station at 2012-01-01}


\keepXColumns
\begin{tabularx}{\textwidth}{N T A}
\hline
Name & Type & Annotation\\
\hline
\hfuzz=500pt\includegraphics[width=1em]{element-mustset.pdf}~outputfileGriddedData & \hfuzz=500pt filename & \hfuzz=500pt \\
\hfuzz=500pt\includegraphics[width=1em]{element-mustset-unbounded.pdf}~type & \hfuzz=500pt \hyperref[gnssType]{gnssType} & \hfuzz=500pt \\
\hfuzz=500pt\includegraphics[width=1em]{element.pdf}~typeTransmitter & \hfuzz=500pt \hyperref[gnssType]{gnssType} & \hfuzz=500pt choose transmitter view, e.g. '***G18'\\
\hfuzz=500pt\includegraphics[width=1em]{element-mustset-unbounded.pdf}~inputfileResiduals & \hfuzz=500pt filename & \hfuzz=500pt GNSS receiver residuals\\
\hfuzz=500pt\includegraphics[width=1em]{element.pdf}~R & \hfuzz=500pt double & \hfuzz=500pt reference radius for ellipsoidal coordinates\\
\hfuzz=500pt\includegraphics[width=1em]{element.pdf}~inverseFlattening & \hfuzz=500pt double & \hfuzz=500pt reference flattening for ellipsoidal coordinates\\
\hline
\end{tabularx}

\clearpage
%==================================
\subsection{GnssSignalBias2Matrix}\label{GnssSignalBias2Matrix}
Computes signal biases for a given list of \configClass{types}{gnssType}.
If the type list is empty, all types contained in \configFile{inputfileSignalBias}{gnssSignalBias} are used.
The resulting \configFile{outputfileMatrix}{matrix} contains a vector with an entry for each type.


\keepXColumns
\begin{tabularx}{\textwidth}{N T A}
\hline
Name & Type & Annotation\\
\hline
\hfuzz=500pt\includegraphics[width=1em]{element-mustset.pdf}~outputfileMatrix & \hfuzz=500pt filename & \hfuzz=500pt \\
\hfuzz=500pt\includegraphics[width=1em]{element.pdf}~outputfileTypes & \hfuzz=500pt filename & \hfuzz=500pt ASCII list of types\\
\hfuzz=500pt\includegraphics[width=1em]{element-mustset.pdf}~inputfileSignalBias & \hfuzz=500pt filename & \hfuzz=500pt \\
\hfuzz=500pt\includegraphics[width=1em]{element-unbounded.pdf}~types & \hfuzz=500pt \hyperref[gnssType]{gnssType} & \hfuzz=500pt if not set, all types in the file are used\\
\hline
\end{tabularx}

\clearpage
%==================================
\subsection{GnssSimulateReceiver}\label{GnssSimulateReceiver}
This program simulates observations from receivers to GNSS satellites.
These simulated observations can then be used in \program{GnssProcessing}, for example to conduct closed-loop simulations.

One or more GNSS constellations must be defined via \configClass{transmitter}{gnssTransmitterGeneratorType}.
Receivers such as ground station networks or Low Earth Orbit (LEO) satellites can be defined via \configClass{receiver}{gnssReceiverGeneratorType}.

If multiple receivers defined an \configFile{outputfileGnssReceiver}{instrument} and \configFile{outputfileClock}{instrument}
are written for each single receiver with the \reference{variable}{general.parser} \verb|{station}| being replaced by the receiver name.

A list of simulated observation types can be defined via \configClass{observationType}{gnssType}. Noise can be added to both observations and clock errors
via \configClass{noiseObervation}{noiseGeneratorType} and \configClass{noiseClockReceiver}{noiseGeneratorType}, respectively. Observation noise is
interpreted as a factor that is multiplied to the accuracy derived from the accuracy pattern of the respective observation type
(see \configFile{inputfileAccuracyDefinition}{gnssAntennaDefinition} in \configClass{receiver}{gnssReceiverGeneratorType}).

The \configClass{parametrization}{gnssParametrizationType} are used to simulate a priori models (e.g. troposphere, signal biases).
Parameter settings and outputfiles are irgnored.

If the program is run on multiple processes the \configClass{receiver}{gnssReceiverGeneratorType}s
(stations or LEO satellites) are distributed over the processes.


\keepXColumns
\begin{tabularx}{\textwidth}{N T A}
\hline
Name & Type & Annotation\\
\hline
\hfuzz=500pt\includegraphics[width=1em]{element-mustset.pdf}~outputfileGnssReceiver & \hfuzz=500pt filename & \hfuzz=500pt variable \{station\} available, simulated observations\\
\hfuzz=500pt\includegraphics[width=1em]{element.pdf}~outputfileClock & \hfuzz=500pt filename & \hfuzz=500pt variable \{station\} available, simulated receiver clock errors\\
\hfuzz=500pt\includegraphics[width=1em]{element-mustset-unbounded.pdf}~timeSeries & \hfuzz=500pt \hyperref[timeSeriesType]{timeSeries} & \hfuzz=500pt defines observation epochs\\
\hfuzz=500pt\includegraphics[width=1em]{element.pdf}~timeMargin & \hfuzz=500pt double & \hfuzz=500pt [seconds] margin to consider two times identical\\
\hfuzz=500pt\includegraphics[width=1em]{element-mustset-unbounded.pdf}~transmitter & \hfuzz=500pt \hyperref[gnssTransmitterGeneratorType]{gnssTransmitterGenerator} & \hfuzz=500pt constellation of GNSS satellites\\
\hfuzz=500pt\includegraphics[width=1em]{element-mustset-unbounded.pdf}~receiver & \hfuzz=500pt \hyperref[gnssReceiverGeneratorType]{gnssReceiverGenerator} & \hfuzz=500pt ground station network or LEO satellite\\
\hfuzz=500pt\includegraphics[width=1em]{element-mustset.pdf}~earthRotation & \hfuzz=500pt \hyperref[earthRotationType]{earthRotation} & \hfuzz=500pt apriori earth rotation\\
\hfuzz=500pt\includegraphics[width=1em]{element-unbounded.pdf}~parametrization & \hfuzz=500pt \hyperref[gnssParametrizationType]{gnssParametrization} & \hfuzz=500pt models and parameters\\
\hfuzz=500pt\includegraphics[width=1em]{element-mustset-unbounded.pdf}~observationType & \hfuzz=500pt \hyperref[gnssType]{gnssType} & \hfuzz=500pt simulated observation types\\
\hfuzz=500pt\includegraphics[width=1em]{element-unbounded.pdf}~noiseObservation & \hfuzz=500pt \hyperref[noiseGeneratorType]{noiseGenerator} & \hfuzz=500pt [-] noise is multiplied with type accuracy pattern of receiver\\
\hfuzz=500pt\includegraphics[width=1em]{element-unbounded.pdf}~noiseClockReceiver & \hfuzz=500pt \hyperref[noiseGeneratorType]{noiseGenerator} & \hfuzz=500pt [m] noise added to the simulated receiver clock\\
\hline
\end{tabularx}

This program is \reference{parallelized}{general.parallelization}.
\clearpage
%==================================
\subsection{GnssStationInfoCreate}\label{GnssStationInfoCreate}
Create a \file{GnssStationInfo file}{gnssStationInfo} from scratch by defining attributes such as
\config{markerName}, \config{markerNumber}, \config{comment}, \config{approxPosition},
\config{antenna} and \config{receiver}.

See also \program{GnssAntex2AntennaDefinition} and \program{GnssStationLog2StationInfo}.


\keepXColumns
\begin{tabularx}{\textwidth}{N T A}
\hline
Name & Type & Annotation\\
\hline
\hfuzz=500pt\includegraphics[width=1em]{element-mustset.pdf}~outputfileStationInfo & \hfuzz=500pt filename & \hfuzz=500pt \\
\hfuzz=500pt\includegraphics[width=1em]{element-mustset.pdf}~markerName & \hfuzz=500pt string & \hfuzz=500pt \\
\hfuzz=500pt\includegraphics[width=1em]{element.pdf}~markerNumber & \hfuzz=500pt string & \hfuzz=500pt \\
\hfuzz=500pt\includegraphics[width=1em]{element.pdf}~comment & \hfuzz=500pt string & \hfuzz=500pt \\
\hfuzz=500pt\includegraphics[width=1em]{element.pdf}~approxPositionX & \hfuzz=500pt double & \hfuzz=500pt [m] in TRF\\
\hfuzz=500pt\includegraphics[width=1em]{element.pdf}~approxPositionY & \hfuzz=500pt double & \hfuzz=500pt [m] in TRF\\
\hfuzz=500pt\includegraphics[width=1em]{element.pdf}~approxPositionZ & \hfuzz=500pt double & \hfuzz=500pt [m] in TRF\\
\hfuzz=500pt\includegraphics[width=1em]{element-mustset-unbounded.pdf}~antenna & \hfuzz=500pt sequence & \hfuzz=500pt \\
\hfuzz=500pt\includegraphics[width=1em]{connector.pdf}\includegraphics[width=1em]{element-mustset.pdf}~name & \hfuzz=500pt string & \hfuzz=500pt \\
\hfuzz=500pt\includegraphics[width=1em]{connector.pdf}\includegraphics[width=1em]{element.pdf}~serial & \hfuzz=500pt string & \hfuzz=500pt \\
\hfuzz=500pt\includegraphics[width=1em]{connector.pdf}\includegraphics[width=1em]{element.pdf}~radome & \hfuzz=500pt string & \hfuzz=500pt \\
\hfuzz=500pt\includegraphics[width=1em]{connector.pdf}\includegraphics[width=1em]{element.pdf}~comment & \hfuzz=500pt string & \hfuzz=500pt \\
\hfuzz=500pt\includegraphics[width=1em]{connector.pdf}\includegraphics[width=1em]{element.pdf}~timeStart & \hfuzz=500pt time & \hfuzz=500pt \\
\hfuzz=500pt\includegraphics[width=1em]{connector.pdf}\includegraphics[width=1em]{element.pdf}~timeEnd & \hfuzz=500pt time & \hfuzz=500pt \\
\hfuzz=500pt\includegraphics[width=1em]{connector.pdf}\includegraphics[width=1em]{element-mustset.pdf}~positionX & \hfuzz=500pt double & \hfuzz=500pt [m] ARP in north, east, up or vehicle system\\
\hfuzz=500pt\includegraphics[width=1em]{connector.pdf}\includegraphics[width=1em]{element-mustset.pdf}~positionY & \hfuzz=500pt double & \hfuzz=500pt [m] ARP in north, east, up or vehicle system\\
\hfuzz=500pt\includegraphics[width=1em]{connector.pdf}\includegraphics[width=1em]{element-mustset.pdf}~positionZ & \hfuzz=500pt double & \hfuzz=500pt [m] ARP in north, east, up or vehicle system\\
\hfuzz=500pt\includegraphics[width=1em]{connector.pdf}\includegraphics[width=1em]{element.pdf}~rotationX & \hfuzz=500pt angle & \hfuzz=500pt [degree] from local/vehicle to left-handed antenna system\\
\hfuzz=500pt\includegraphics[width=1em]{connector.pdf}\includegraphics[width=1em]{element.pdf}~rotationY & \hfuzz=500pt angle & \hfuzz=500pt [degree] from local/vehicle to left-handed antenna system\\
\hfuzz=500pt\includegraphics[width=1em]{connector.pdf}\includegraphics[width=1em]{element.pdf}~rotationZ & \hfuzz=500pt angle & \hfuzz=500pt [degree] from local/vehicle to left-handed antenna system\\
\hfuzz=500pt\includegraphics[width=1em]{connector.pdf}\includegraphics[width=1em]{element.pdf}~flipX & \hfuzz=500pt boolean & \hfuzz=500pt flip x-axis (after rotation)\\
\hfuzz=500pt\includegraphics[width=1em]{connector.pdf}\includegraphics[width=1em]{element.pdf}~flipY & \hfuzz=500pt boolean & \hfuzz=500pt flip y-axis (after rotation)\\
\hfuzz=500pt\includegraphics[width=1em]{connector.pdf}\includegraphics[width=1em]{element.pdf}~flipZ & \hfuzz=500pt boolean & \hfuzz=500pt flip z-axis (after rotation)\\
\hfuzz=500pt\includegraphics[width=1em]{element-unbounded.pdf}~receiver & \hfuzz=500pt sequence & \hfuzz=500pt \\
\hfuzz=500pt\includegraphics[width=1em]{connector.pdf}\includegraphics[width=1em]{element-mustset.pdf}~name & \hfuzz=500pt string & \hfuzz=500pt \\
\hfuzz=500pt\includegraphics[width=1em]{connector.pdf}\includegraphics[width=1em]{element.pdf}~serial & \hfuzz=500pt string & \hfuzz=500pt \\
\hfuzz=500pt\includegraphics[width=1em]{connector.pdf}\includegraphics[width=1em]{element.pdf}~version & \hfuzz=500pt string & \hfuzz=500pt \\
\hfuzz=500pt\includegraphics[width=1em]{connector.pdf}\includegraphics[width=1em]{element.pdf}~comment & \hfuzz=500pt string & \hfuzz=500pt \\
\hfuzz=500pt\includegraphics[width=1em]{connector.pdf}\includegraphics[width=1em]{element.pdf}~timeStart & \hfuzz=500pt time & \hfuzz=500pt \\
\hfuzz=500pt\includegraphics[width=1em]{connector.pdf}\includegraphics[width=1em]{element.pdf}~timeEnd & \hfuzz=500pt time & \hfuzz=500pt \\
\hfuzz=500pt\includegraphics[width=1em]{element-unbounded.pdf}~referencePoint & \hfuzz=500pt sequence & \hfuzz=500pt e.g. center of mass in satellite frame\\
\hfuzz=500pt\includegraphics[width=1em]{connector.pdf}\includegraphics[width=1em]{element.pdf}~comment & \hfuzz=500pt string & \hfuzz=500pt \\
\hfuzz=500pt\includegraphics[width=1em]{connector.pdf}\includegraphics[width=1em]{element-mustset.pdf}~xStart & \hfuzz=500pt double & \hfuzz=500pt [m] in north, east, up or vehicle system\\
\hfuzz=500pt\includegraphics[width=1em]{connector.pdf}\includegraphics[width=1em]{element-mustset.pdf}~yStart & \hfuzz=500pt double & \hfuzz=500pt linear motion between start and end\\
\hfuzz=500pt\includegraphics[width=1em]{connector.pdf}\includegraphics[width=1em]{element-mustset.pdf}~zStart & \hfuzz=500pt double & \hfuzz=500pt \\
\hfuzz=500pt\includegraphics[width=1em]{connector.pdf}\includegraphics[width=1em]{element-mustset.pdf}~xEnd & \hfuzz=500pt double & \hfuzz=500pt [m] in north, east, up or vehicle system\\
\hfuzz=500pt\includegraphics[width=1em]{connector.pdf}\includegraphics[width=1em]{element-mustset.pdf}~yEnd & \hfuzz=500pt double & \hfuzz=500pt linear motion between start and end\\
\hfuzz=500pt\includegraphics[width=1em]{connector.pdf}\includegraphics[width=1em]{element-mustset.pdf}~zEnd & \hfuzz=500pt double & \hfuzz=500pt \\
\hfuzz=500pt\includegraphics[width=1em]{connector.pdf}\includegraphics[width=1em]{element.pdf}~timeStart & \hfuzz=500pt time & \hfuzz=500pt \\
\hfuzz=500pt\includegraphics[width=1em]{connector.pdf}\includegraphics[width=1em]{element.pdf}~timeEnd & \hfuzz=500pt time & \hfuzz=500pt \\
\hline
\end{tabularx}

\clearpage
%==================================
\subsection{InstrumentGnssReceiver2TimeSeries}\label{InstrumentGnssReceiver2TimeSeries}
Convert selected GNSS observations or residuals into a simpler time series format.
The \config{outputfileTimeSeries} is an \file{instrument file}{instrument} (MISCVALUES).
For each epoch the first data column contains the PRN, the second the satellite system,
followed by a column for each GNSS \configClass{type}{gnssType}.
As normally more than one GNSS transmitter is tracked per epoch, the output file
has several lines per observed epoch (epochs with the same time, one for each transmitter).

The second data column of the output contains a number representating the system
\begin{itemize}
\item 71: 'G', GPS
\item 82: 'R', GLONASS
\item 69: 'E', GALILEO
\item 67: 'C', BDS
\item 83: 'S', SBAS
\item 74: 'J', QZSS
\item 73: 'I', IRNSS .
\end{itemize}

A \file{GNSS residual file}{instrument} includes additional information
besides the residuals, which can also be selected with \configClass{type}{gnssType}
\begin{itemize}
\item \verb|A1*|, \verb|E1*|: azimuth and elevation at receiver
\item \verb|A2*|, \verb|E2*|: azimuth and elevation at transmitter
\item \verb|I**|: Estimated slant total electron content (STEC)
\end{itemize}

Furthermore these files may include for each residual \configClass{type}{gnssType}
information about the redundancy and the accuracy relation $\sigma/\sigma_0$
of the estimated $\sigma$ versus the apriori $\sigma_0$ from the least squares adjustment.
The three values (residuals, redundancy, $\sigma/\sigma_0$) are coded with the same type.
To get access to all values the corresponding type must be repeated in \configClass{type}{gnssType}.

Example: Selected GPS phase residuals (\configClass{type}{gnssType}='\verb|L1*G|' and \configClass{type}{gnssType}='\verb|L2*G|').
Plotted with \program{PlotGraph} with two \configClass{layer:linesAndPoints}{plotGraphLayerType}
(\config{valueX}='\verb|data0|',  \config{valueY}='\verb|100*data3+data1|' and \config{valueY}='\verb|100*data4+data1|' respectively).
\fig{!hb}{0.8}{instrumentGnssReceiver2TimeSeries}{fig:instrumentGnssReceiver2TimeSeries}{GPS residuals in cm, shifted by PRN}


\keepXColumns
\begin{tabularx}{\textwidth}{N T A}
\hline
Name & Type & Annotation\\
\hline
\hfuzz=500pt\includegraphics[width=1em]{element-mustset.pdf}~outputfileTimeSeries & \hfuzz=500pt filename & \hfuzz=500pt Instrument (MISCVALUES): prn, system, values for each type\\
\hfuzz=500pt\includegraphics[width=1em]{element-mustset-unbounded.pdf}~inputfileGnssReceiver & \hfuzz=500pt filename & \hfuzz=500pt GNSS receiver observations or residuals\\
\hfuzz=500pt\includegraphics[width=1em]{element-mustset-unbounded.pdf}~type & \hfuzz=500pt \hyperref[gnssType]{gnssType} & \hfuzz=500pt \\
\hline
\end{tabularx}

\clearpage
%==================================
\subsection{ParameterVector2GnssAntennaDefinition}\label{ParameterVector2GnssAntennaDefinition}
Updates an \file{GnssAntennaDefinition file}{gnssAntennaDefinition} with estimated parameters which belongs
to the parametrization \configClass{antennaCenterVariations}{parametrizationGnssAntennaType}.
The \configFile{outfileAntennaDefinition}{gnssAntennaDefinition} contains all antennas
from \configFile{inputfileAntennaDefinition}{gnssAntennaDefinition}.
The antenna center variations representend by the \configFile{inputfileSolution}{matrix} are added
to the matching antennas.

The \file{GnssAntennaDefinition file}{gnssAntennaDefinition} can be modified to the demands before with
\program{GnssAntennaDefinitionCreate}.

The following steps are used to estimate antenna center variations:
\begin{itemize}
\item \program{GnssAntennaDefinitionCreate} or \program{GnssAntex2AntennaDefinition}
\item \program{GnssProcessing} with \config{inputfileAntennaDefinition} as apriori
      and writing \file{normal equations}{normalEquation} with
      parametrization of \configClass{antennaCenterVariations}{parametrizationGnssAntennaType}
\item \program{NormalsEliminate}: eliminate all other than antenna parameters
\item \program{NormalsAccumulate}: accumulate normals over a sufficient long period
\item \program{GnssAntennaNormalsConstraint}: constrain unsolvable parameter linear combinations
\item \program{NormalsSolverVCE}: estimate the parameter vector
\item \program{ParameterVector2GnssAntennaDefinition}: update \config{inputfileAntennaDefinition}
\end{itemize}

See also \program{ParameterVector2GnssAntennaDefinition}, \program{GnssAntennaNormalsConstraint}.


\keepXColumns
\begin{tabularx}{\textwidth}{N T A}
\hline
Name & Type & Annotation\\
\hline
\hfuzz=500pt\includegraphics[width=1em]{element-mustset.pdf}~outfileAntennaDefinition & \hfuzz=500pt filename & \hfuzz=500pt all apriori antennas\\
\hfuzz=500pt\includegraphics[width=1em]{element-mustset.pdf}~inputfileAntennaDefinition & \hfuzz=500pt filename & \hfuzz=500pt apriori antennas\\
\hfuzz=500pt\includegraphics[width=1em]{element-mustset-unbounded.pdf}~antennaCenterVariations & \hfuzz=500pt \hyperref[parametrizationGnssAntennaType]{parametrizationGnssAntenna} & \hfuzz=500pt \\
\hfuzz=500pt\includegraphics[width=1em]{element-mustset.pdf}~inputfileSolution & \hfuzz=500pt filename & \hfuzz=500pt \\
\hfuzz=500pt\includegraphics[width=1em]{element-mustset.pdf}~inputfileParameterNames & \hfuzz=500pt filename & \hfuzz=500pt \\
\hline
\end{tabularx}

\clearpage
%==================================
\section{Programs: Grace}
\subsection{EnsembleAveragingScaleModel}\label{EnsembleAveragingScaleModel}
This programs estimate satellite-to-satellite-tracking (SST) deterministic signals due to eclipse transits from residuals.
The ensemble averaging method is used to characterize the average properties of signal shapes across all transit events.
Each shape is assigned to one arc of 3 hours (default). This can be modefied by enabling \config{averagingInterval}.


\keepXColumns
\begin{tabularx}{\textwidth}{N T A}
\hline
Name & Type & Annotation\\
\hline
\hfuzz=500pt\includegraphics[width=1em]{element-mustset.pdf}~outputfileScaleModel & \hfuzz=500pt filename & \hfuzz=500pt \\
\hfuzz=500pt\includegraphics[width=1em]{element-mustset.pdf}~inputfileGrace1EclipseFactor & \hfuzz=500pt filename & \hfuzz=500pt GRACE-A eclipse factors computed with integrated orbit\\
\hfuzz=500pt\includegraphics[width=1em]{element-mustset.pdf}~inputfileGrace2EclipseFactor & \hfuzz=500pt filename & \hfuzz=500pt GRACE-B eclipse factors computed with integrated orbit\\
\hfuzz=500pt\includegraphics[width=1em]{element-mustset.pdf}~inputfileGraceResiduals & \hfuzz=500pt filename & \hfuzz=500pt SST Residuals\\
\hfuzz=500pt\includegraphics[width=1em]{element-mustset.pdf}~timeMargin & \hfuzz=500pt uint & \hfuzz=500pt epochs before eclipse mode\\
\hfuzz=500pt\includegraphics[width=1em]{element-mustset.pdf}~waveLength & \hfuzz=500pt uint & \hfuzz=500pt length of the sample wave\\
\hfuzz=500pt\includegraphics[width=1em]{element.pdf}~averagingInterval & \hfuzz=500pt sequence & \hfuzz=500pt \\
\hfuzz=500pt\includegraphics[width=1em]{connector.pdf}\includegraphics[width=1em]{element.pdf}~nearestNeighborNumber & \hfuzz=500pt uint & \hfuzz=500pt \\
\hline
\end{tabularx}

\clearpage
%==================================
\subsection{GraceAntennaCenterCorrectionArcCovariance}\label{GraceAntennaCenterCorrectionArcCovariance}
This program computes covariance information for the non­-stationary noise of the KBR antenna offset correction (AOC)
from the orientation covariance matrices provided in Level-1B products via variance propagation.
By using the output \configFile{outputfileSatelliteTrackingCovariance}{matrix} in \program{PreprocessingSst},
noise model distinguishes between the stationary noise of ranging observations and the non­stationary AOC noise.

The covariances are derived from the partial derivative of the AOC w.r.t. the roll/pitch/yaw rotations
and star camera covariances \configFile{inputfileScaCovariance1}{matrix} and \configFile{inputfileScaCovariance2}{matrix}.

The covariances for the range-rates and range-acceleration are computed by differentiating
an interpolation polynomial of degree \config{interpolationDegree}.


\keepXColumns
\begin{tabularx}{\textwidth}{N T A}
\hline
Name & Type & Annotation\\
\hline
\hfuzz=500pt\includegraphics[width=1em]{element.pdf}~outputfileSatelliteTrackingCovariance & \hfuzz=500pt filename & \hfuzz=500pt corrections for range, range-rate, and range-accelerations\\
\hfuzz=500pt\includegraphics[width=1em]{element-mustset.pdf}~sstType & \hfuzz=500pt choice & \hfuzz=500pt \\
\hfuzz=500pt\includegraphics[width=1em]{connector.pdf}\includegraphics[width=1em]{element-mustset.pdf}~range & \hfuzz=500pt  & \hfuzz=500pt \\
\hfuzz=500pt\includegraphics[width=1em]{connector.pdf}\includegraphics[width=1em]{element-mustset.pdf}~rangeRate & \hfuzz=500pt  & \hfuzz=500pt \\
\hfuzz=500pt\includegraphics[width=1em]{connector.pdf}\includegraphics[width=1em]{element-mustset.pdf}~rangeAcceleration & \hfuzz=500pt  & \hfuzz=500pt \\
\hfuzz=500pt\includegraphics[width=1em]{element-mustset.pdf}~inputfileOrbit1 & \hfuzz=500pt filename & \hfuzz=500pt \\
\hfuzz=500pt\includegraphics[width=1em]{element-mustset.pdf}~inputfileOrbit2 & \hfuzz=500pt filename & \hfuzz=500pt \\
\hfuzz=500pt\includegraphics[width=1em]{element-mustset.pdf}~inputfileStarCamera1 & \hfuzz=500pt filename & \hfuzz=500pt \\
\hfuzz=500pt\includegraphics[width=1em]{element-mustset.pdf}~inputfileStarCamera2 & \hfuzz=500pt filename & \hfuzz=500pt \\
\hfuzz=500pt\includegraphics[width=1em]{element-mustset.pdf}~inputfileScaCovariance1 & \hfuzz=500pt filename & \hfuzz=500pt \\
\hfuzz=500pt\includegraphics[width=1em]{element-mustset.pdf}~inputfileScaCovariance2 & \hfuzz=500pt filename & \hfuzz=500pt \\
\hfuzz=500pt\includegraphics[width=1em]{element.pdf}~sigmaAccelerometerX & \hfuzz=500pt double & \hfuzz=500pt [rad/s\textasciicircum{}2]\\
\hfuzz=500pt\includegraphics[width=1em]{element.pdf}~sigmaAccelerometerY & \hfuzz=500pt double & \hfuzz=500pt [rad/s\textasciicircum{}2]\\
\hfuzz=500pt\includegraphics[width=1em]{element.pdf}~sigmaAccelerometerZ & \hfuzz=500pt double & \hfuzz=500pt [rad/s\textasciicircum{}2]\\
\hfuzz=500pt\includegraphics[width=1em]{element-mustset.pdf}~antennaCenters & \hfuzz=500pt choice & \hfuzz=500pt KBR antenna phase center\\
\hfuzz=500pt\includegraphics[width=1em]{connector.pdf}\includegraphics[width=1em]{element-mustset.pdf}~value & \hfuzz=500pt sequence & \hfuzz=500pt \\
\hfuzz=500pt\quad\includegraphics[width=1em]{connector.pdf}\includegraphics[width=1em]{element.pdf}~center1X & \hfuzz=500pt double & \hfuzz=500pt x-coordinate of antenna position in SRF [m] for GRACEA\\
\hfuzz=500pt\quad\includegraphics[width=1em]{connector.pdf}\includegraphics[width=1em]{element.pdf}~center1Y & \hfuzz=500pt double & \hfuzz=500pt y-coordinate of antenna position in SRF [m] for GRACEA\\
\hfuzz=500pt\quad\includegraphics[width=1em]{connector.pdf}\includegraphics[width=1em]{element.pdf}~center1Z & \hfuzz=500pt double & \hfuzz=500pt z-coordinate of antenna position in SRF [m] for GRACEA\\
\hfuzz=500pt\quad\includegraphics[width=1em]{connector.pdf}\includegraphics[width=1em]{element.pdf}~center2X & \hfuzz=500pt double & \hfuzz=500pt x-coordinate of antenna position in SRF [m] for GRACEB\\
\hfuzz=500pt\quad\includegraphics[width=1em]{connector.pdf}\includegraphics[width=1em]{element.pdf}~center2Y & \hfuzz=500pt double & \hfuzz=500pt y-coordinate of antenna position in SRF [m] for GRACEB\\
\hfuzz=500pt\quad\includegraphics[width=1em]{connector.pdf}\includegraphics[width=1em]{element.pdf}~center2Z & \hfuzz=500pt double & \hfuzz=500pt z-coordinate of antenna position in SRF [m] for GRACEB\\
\hfuzz=500pt\includegraphics[width=1em]{connector.pdf}\includegraphics[width=1em]{element-mustset.pdf}~file & \hfuzz=500pt sequence & \hfuzz=500pt \\
\hfuzz=500pt\quad\includegraphics[width=1em]{connector.pdf}\includegraphics[width=1em]{element-mustset.pdf}~inputAntennaCenters & \hfuzz=500pt filename & \hfuzz=500pt \\
\hfuzz=500pt\includegraphics[width=1em]{element.pdf}~interpolationDegree & \hfuzz=500pt uint & \hfuzz=500pt differentiation by polynomial approximation of degree n\\
\hline
\end{tabularx}

This program is \reference{parallelized}{general.parallelization}.
\clearpage
%==================================
\subsection{GraceOrbit2TransplantTimeOffset}\label{GraceOrbit2TransplantTimeOffset}
This program computes the time shift between two co-orbiting satellites based on dynamic orbit data.
When applied to data of the first satellite, the computed time shift virtually shifts data of first satellite into the location of the second satellite.
Note that \config{inputfileOrbit1} and \config{inputfileOrbit2} need velocity and acceleration data, which
can be computed with \program{OrbitAddVelocityAndAcceleration}.
The program tries to find a minimum of the objective function
\begin{equation}
  f(\Delta t) = \| r_1(t) - r_2(t + \Delta t) \|^2,
\end{equation}
by applying Newton's method to the first derivative, thus iteratively computing
\begin{equation}
  \Delta t_{k+1} = \Delta t_k + \frac{f'(\Delta t_k)}{f''(\Delta t_k)}.
\end{equation}
This iteration is stopped when the difference between to consecutive time shift values falls below \config{threshold} or
\config{maximumIterations} is reached. An \config{initialGuess} of the time shift can speed up convergence.

See also \program{OrbitAddVelocityAndAcceleration} and \program{InstrumentApplyTimeOffset}.


\keepXColumns
\begin{tabularx}{\textwidth}{N T A}
\hline
Name & Type & Annotation\\
\hline
\hfuzz=500pt\includegraphics[width=1em]{element-mustset.pdf}~outputfileTimeOffset & \hfuzz=500pt filename & \hfuzz=500pt estimated time offset in seconds (MISCVALUE)\\
\hfuzz=500pt\includegraphics[width=1em]{element-mustset.pdf}~inputfileOrbit1 & \hfuzz=500pt filename & \hfuzz=500pt orbit data of satellite 1\\
\hfuzz=500pt\includegraphics[width=1em]{element-mustset.pdf}~inputfileOrbit2 & \hfuzz=500pt filename & \hfuzz=500pt orbit data of satellite 2\\
\hfuzz=500pt\includegraphics[width=1em]{element.pdf}~interpolationDegree & \hfuzz=500pt uint & \hfuzz=500pt polynomial degree for the interpolation of position, velocity and acceleration\\
\hfuzz=500pt\includegraphics[width=1em]{element.pdf}~initialGuess & \hfuzz=500pt double & \hfuzz=500pt initial guess for the time shift [seconds]\\
\hfuzz=500pt\includegraphics[width=1em]{element.pdf}~maximumIterations & \hfuzz=500pt uint & \hfuzz=500pt maximum number of iterations\\
\hfuzz=500pt\includegraphics[width=1em]{element.pdf}~threshold & \hfuzz=500pt double & \hfuzz=500pt when the maximum difference between two iterations is below this value, stop [seconds]\\
\hline
\end{tabularx}

This program is \reference{parallelized}{general.parallelization}.
\clearpage
%==================================
\subsection{GraceSstResidualAnalysis}\label{GraceSstResidualAnalysis}
This program applies the Multi-Resolution Analysis (MRA) using
Discrete Wavelet Transform (DWT) to the monthly GRACE SST post-fit residuals.
First, the residuals are transferred into wavelet domain by applying an 8 level
Daubechies wavelet transform (default).
In the next step, detail coefficients are merged into three major groups
due to their approximate frequency subbands:
\begin{itemize}
\item Low scale details, corresponding to the frequency band above 10 mHz;
\item Intermediate scale details, corresponding to the approximate frequency
      range above 3 mHz up to 10 mHz;
\item High scale details, corresponding to the approximate frequency range
above 0.5 mHz up to 10 mHz.
\end{itemize}
In the last step, each group is reconstructed back into time domain.


\keepXColumns
\begin{tabularx}{\textwidth}{N T A}
\hline
Name & Type & Annotation\\
\hline
\hfuzz=500pt\includegraphics[width=1em]{element-mustset.pdf}~outputfileInstrumentHighScale & \hfuzz=500pt filename & \hfuzz=500pt High scale details\\
\hfuzz=500pt\includegraphics[width=1em]{element-mustset.pdf}~outputfileInstrumentMidScale & \hfuzz=500pt filename & \hfuzz=500pt Intermediate scale details\\
\hfuzz=500pt\includegraphics[width=1em]{element-mustset.pdf}~outputfileInstrumentLowScale & \hfuzz=500pt filename & \hfuzz=500pt Low scale details\\
\hfuzz=500pt\includegraphics[width=1em]{element-mustset.pdf}~inputfileInstrument & \hfuzz=500pt filename & \hfuzz=500pt GRACE SST Residuals\\
\hfuzz=500pt\includegraphics[width=1em]{element-mustset.pdf}~inputfileWavelet & \hfuzz=500pt filename & \hfuzz=500pt wavelet coefficients\\
\hline
\end{tabularx}

\clearpage
%==================================
\subsection{GraceSstScaleModel}\label{GraceSstScaleModel}
This programs estimate satellite-to-satellite-tracking (SST) deterministic signals
due to eclipse transits and low-SNR values from post-fit residuals.
The low-SNR effects are estimated by directly using the residual values.
The ensemble averaging method is used to characterize the average properties of eclipse transit signal shapes across all transit events.
Each shape is assigned to one arc of 3 hours (default). This can be modefied by enabling \config{averagingInterval}.


\keepXColumns
\begin{tabularx}{\textwidth}{N T A}
\hline
Name & Type & Annotation\\
\hline
\hfuzz=500pt\includegraphics[width=1em]{element-mustset.pdf}~inputfileGraceResiduals & \hfuzz=500pt filename & \hfuzz=500pt SST Residuals\\
\hfuzz=500pt\includegraphics[width=1em]{element-mustset.pdf}~timeMargin & \hfuzz=500pt uint & \hfuzz=500pt epochs before instrumental events\\
\hfuzz=500pt\includegraphics[width=1em]{element-mustset.pdf}~waveLength & \hfuzz=500pt uint & \hfuzz=500pt length of the sample wave\\
\hfuzz=500pt\includegraphics[width=1em]{element.pdf}~estimateEclipseTransitScale & \hfuzz=500pt sequence & \hfuzz=500pt \\
\hfuzz=500pt\includegraphics[width=1em]{connector.pdf}\includegraphics[width=1em]{element-mustset.pdf}~outputfileScaleModel & \hfuzz=500pt filename & \hfuzz=500pt \\
\hfuzz=500pt\includegraphics[width=1em]{connector.pdf}\includegraphics[width=1em]{element-mustset.pdf}~inputfileGrace1EclipseFactor & \hfuzz=500pt filename & \hfuzz=500pt GRACE-A eclipse factors computed with integrated orbit\\
\hfuzz=500pt\includegraphics[width=1em]{connector.pdf}\includegraphics[width=1em]{element-mustset.pdf}~inputfileGrace2EclipseFactor & \hfuzz=500pt filename & \hfuzz=500pt GRACE-B eclipse factors computed with integrated orbit\\
\hfuzz=500pt\includegraphics[width=1em]{connector.pdf}\includegraphics[width=1em]{element.pdf}~averagingInterval & \hfuzz=500pt sequence & \hfuzz=500pt \\
\hfuzz=500pt\quad\includegraphics[width=1em]{connector.pdf}\includegraphics[width=1em]{element.pdf}~nearestNeighborNumber & \hfuzz=500pt uint & \hfuzz=500pt \\
\hfuzz=500pt\includegraphics[width=1em]{element.pdf}~estimateLowSnrScale & \hfuzz=500pt sequence & \hfuzz=500pt \\
\hfuzz=500pt\includegraphics[width=1em]{connector.pdf}\includegraphics[width=1em]{element-mustset.pdf}~outputfileScaleModel & \hfuzz=500pt filename & \hfuzz=500pt \\
\hfuzz=500pt\includegraphics[width=1em]{connector.pdf}\includegraphics[width=1em]{element-mustset.pdf}~inputfileGraceSstSNR & \hfuzz=500pt filename & \hfuzz=500pt GRACE SNR values\\
\hline
\end{tabularx}

\clearpage
%==================================
\subsection{GraceSstSpecialEvents}\label{GraceSstSpecialEvents}
Time-indexing deterministic signals in the GRACE K-Band measurements caused by Sun intrusions
into the star camera baffles of GRACE-A and eclipse transits of the satellites.
The events are determined by satellites' position (\configFile{inputfileOrbit1/2}{instrument})
and orientation (\configFile{inputfileStarCamera1/2}{instrument}). Each type of event is represented
by its mid-interval point per orbit revolution and is reported in \configFile{outputfileEvents}{instrument}.

The waveform of each event is nearly constant within one month and can be approximated by a polynomial.
For the purpose of gravity field recovery, each waveform is parameterized by a polynomial and the coefficients
of this polynomial are estimated as additional instrument calibration parameters in a common adjustment
with all other instrument, satellite, and gravity field parameters,
see \configClass{parametrizationSatelliteTracking:specialEffect}{parametrizationSatelliteTrackingType:specialEffect}.


\keepXColumns
\begin{tabularx}{\textwidth}{N T A}
\hline
Name & Type & Annotation\\
\hline
\hfuzz=500pt\includegraphics[width=1em]{element-mustset.pdf}~outputfileEvents & \hfuzz=500pt filename & \hfuzz=500pt \\
\hfuzz=500pt\includegraphics[width=1em]{element.pdf}~outputfileIntervals & \hfuzz=500pt filename & \hfuzz=500pt \\
\hfuzz=500pt\includegraphics[width=1em]{element-mustset.pdf}~inputfileOrbit1 & \hfuzz=500pt filename & \hfuzz=500pt \\
\hfuzz=500pt\includegraphics[width=1em]{element-mustset.pdf}~inputfileOrbit2 & \hfuzz=500pt filename & \hfuzz=500pt \\
\hfuzz=500pt\includegraphics[width=1em]{element-mustset.pdf}~inputfileStarCamera1 & \hfuzz=500pt filename & \hfuzz=500pt \\
\hfuzz=500pt\includegraphics[width=1em]{element-mustset.pdf}~inputfileStarCamera2 & \hfuzz=500pt filename & \hfuzz=500pt \\
\hfuzz=500pt\includegraphics[width=1em]{element-mustset.pdf}~ephemerides & \hfuzz=500pt \hyperref[ephemeridesType]{ephemerides} & \hfuzz=500pt \\
\hfuzz=500pt\includegraphics[width=1em]{element-mustset.pdf}~eclipse & \hfuzz=500pt \hyperref[eclipseType]{eclipse} & \hfuzz=500pt \\
\hfuzz=500pt\includegraphics[width=1em]{element.pdf}~marginLeft & \hfuzz=500pt double & \hfuzz=500pt margin size (on both sides) [seconds]\\
\hfuzz=500pt\includegraphics[width=1em]{element.pdf}~marginRight & \hfuzz=500pt double & \hfuzz=500pt margin size (on both sides) [seconds]\\
\hline
\end{tabularx}

\clearpage
%==================================
\subsection{GraceThrusterResponse2Accelerometer}\label{GraceThrusterResponse2Accelerometer}
Add modeled thruster responses to accelerometer data.
The epochs and durations are given in the \configFile{inputfileThruster}{instrument} (THRUSTER).

The \configFile{inputfileThrusterResponse}{matrix} is a $(6\times 3)$ matrix with
the linear accelerations in the SRF ($x, y, z$) in one line per pair:
\begin{enumerate}
\item Negative Yaw,
\item Positive Pitch,
\item Positive Yaw,
\item Negative Pitch,
\item Negative Roll,
\item Positive Roll.
\end{enumerate}


\keepXColumns
\begin{tabularx}{\textwidth}{N T A}
\hline
Name & Type & Annotation\\
\hline
\hfuzz=500pt\includegraphics[width=1em]{element-mustset.pdf}~outputfileAccelerometer & \hfuzz=500pt filename & \hfuzz=500pt ACCELEROMETER\\
\hfuzz=500pt\includegraphics[width=1em]{element-mustset.pdf}~inputfileAccelerometer & \hfuzz=500pt filename & \hfuzz=500pt ACCELEROMETER\\
\hfuzz=500pt\includegraphics[width=1em]{element-mustset.pdf}~inputfileThruster & \hfuzz=500pt filename & \hfuzz=500pt THRUSTER\\
\hfuzz=500pt\includegraphics[width=1em]{element-mustset.pdf}~inputfileThrusterResponse & \hfuzz=500pt filename & \hfuzz=500pt thruster model (matrix with one line per pair)\\
\hline
\end{tabularx}

\clearpage
%==================================
\subsection{InstrumentSatelliteTrackingAntennaCenterCorrection}\label{InstrumentSatelliteTrackingAntennaCenterCorrection}
This program computes the correction due to offset of the antenna center relative the center of mass.
The offsets $\M c_A$ and $\M c_B$ in \configFile{inputfileAntennaCenters}{matrix} are given in the satellite
reference frame. These offsets are rotated into the the inertial frame with $\M D_A$ and $\M D_B$ from
\configFile{inputfileStarCamera}{instrument} and projected onto the line of sight (LOS)
\begin{equation}
  \rho_{AOC} = \M e_{AB}\cdot(\M D_A\,\M c_A - \M D_B\,\M c_B),
\end{equation}
with the unit vector in line of sight direction
\begin{equation}
  \M e_{AB} = \frac{\M r_B - \M r_A}{\left\lVert{\M r_B - \M r_A}\right\rVert}.
\end{equation}
The corrections for the range-rates and range-acceleration are computed by differentiating
an interpolation polynomial of degree \config{interpolationDegree}.


\keepXColumns
\begin{tabularx}{\textwidth}{N T A}
\hline
Name & Type & Annotation\\
\hline
\hfuzz=500pt\includegraphics[width=1em]{element.pdf}~outputfileSatelliteTracking & \hfuzz=500pt filename & \hfuzz=500pt corrections for range, range-rate, and range-accelerations\\
\hfuzz=500pt\includegraphics[width=1em]{element-mustset.pdf}~inputfileOrbit1 & \hfuzz=500pt filename & \hfuzz=500pt \\
\hfuzz=500pt\includegraphics[width=1em]{element-mustset.pdf}~inputfileOrbit2 & \hfuzz=500pt filename & \hfuzz=500pt \\
\hfuzz=500pt\includegraphics[width=1em]{element-mustset.pdf}~inputfileStarCamera1 & \hfuzz=500pt filename & \hfuzz=500pt \\
\hfuzz=500pt\includegraphics[width=1em]{element-mustset.pdf}~inputfileStarCamera2 & \hfuzz=500pt filename & \hfuzz=500pt \\
\hfuzz=500pt\includegraphics[width=1em]{element-mustset.pdf}~antennaCenters & \hfuzz=500pt choice & \hfuzz=500pt KBR antenna phase center\\
\hfuzz=500pt\includegraphics[width=1em]{connector.pdf}\includegraphics[width=1em]{element-mustset.pdf}~value & \hfuzz=500pt sequence & \hfuzz=500pt \\
\hfuzz=500pt\quad\includegraphics[width=1em]{connector.pdf}\includegraphics[width=1em]{element.pdf}~center1X & \hfuzz=500pt double & \hfuzz=500pt x-coordinate of antenna position in SRF [m] for GRACEA\\
\hfuzz=500pt\quad\includegraphics[width=1em]{connector.pdf}\includegraphics[width=1em]{element.pdf}~center1Y & \hfuzz=500pt double & \hfuzz=500pt y-coordinate of antenna position in SRF [m] for GRACEA\\
\hfuzz=500pt\quad\includegraphics[width=1em]{connector.pdf}\includegraphics[width=1em]{element.pdf}~center1Z & \hfuzz=500pt double & \hfuzz=500pt z-coordinate of antenna position in SRF [m] for GRACEA\\
\hfuzz=500pt\quad\includegraphics[width=1em]{connector.pdf}\includegraphics[width=1em]{element.pdf}~center2X & \hfuzz=500pt double & \hfuzz=500pt x-coordinate of antenna position in SRF [m] for GRACEB\\
\hfuzz=500pt\quad\includegraphics[width=1em]{connector.pdf}\includegraphics[width=1em]{element.pdf}~center2Y & \hfuzz=500pt double & \hfuzz=500pt y-coordinate of antenna position in SRF [m] for GRACEB\\
\hfuzz=500pt\quad\includegraphics[width=1em]{connector.pdf}\includegraphics[width=1em]{element.pdf}~center2Z & \hfuzz=500pt double & \hfuzz=500pt z-coordinate of antenna position in SRF [m] for GRACEB\\
\hfuzz=500pt\includegraphics[width=1em]{connector.pdf}\includegraphics[width=1em]{element-mustset.pdf}~file & \hfuzz=500pt sequence & \hfuzz=500pt \\
\hfuzz=500pt\quad\includegraphics[width=1em]{connector.pdf}\includegraphics[width=1em]{element-mustset.pdf}~inputAntennaCenters & \hfuzz=500pt filename & \hfuzz=500pt \\
\hfuzz=500pt\includegraphics[width=1em]{element.pdf}~interpolationDegree & \hfuzz=500pt uint & \hfuzz=500pt differentiation by polynomial approximation of degree n\\
\hline
\end{tabularx}

This program is \reference{parallelized}{general.parallelization}.
\clearpage
%==================================
\subsection{InstrumentStarCameraAngularAccelerometerFusion}\label{InstrumentStarCameraAngularAccelerometerFusion}
This program estimates the satellites orientation from star camera data
\configFile{inputfileStarCamera}{instrument} and angular accelerometer data
\configFile{inputfileAngularAcc}{instrument}. The combination of both observation types
is achieved in a least square adjustment. The optimal weighting between the two different
observation groups is achieved by means of VCE in combination with a robust estimator.
The system of linearized observation equations within the sensor fusion approach can be formulated as:
\begin{equation}
  \begin{bmatrix}
  \M l_{ACC1B}\\
  \M l_{SCA1B}
  \end{bmatrix}
  =
  \begin{bmatrix}
  \M A_{ACC1B} & \M B_{ACC1B}\\
  \M A_{SCA1B} & \M 0
  \end{bmatrix}
  \begin{bmatrix}
  \M q\\
  \M b
  \end{bmatrix}
  =
  \begin{bmatrix}
  \frac{\partial \dot{\boldsymbol{\omega}}}{\partial \M q} & \frac{\partial \dot{\boldsymbol{\omega}}}{\partial \M b}\\
  \M I & \M 0
  \end{bmatrix}
  \begin{bmatrix}
  \M q\\
  \M b
  \end{bmatrix}
\end{equation}
with
\begin{equation}\begin{split}
  \M l_{ACC1B}  &= \dot{\boldsymbol{\omega}}_{ACC1B} - \dot{\boldsymbol{\omega}}_{0}, \\
  \M l_{SCA1B}  &= \M q_{SCA1B} - \M q_{0}, \\
  \M q_{Fusion} &= \M q + \M q_{0}.
\end{split}\end{equation}
The reference values $\M q_{0}$ and $\dot{\boldsymbol{\omega}}_{0}$ are derived
from \configFile{inputfileStarCameraReference}{instrument}. In the course of the estimation,
the accelerometer data is calibrated, by setting a bias factor $\M b$ with \config{accBias}.


\keepXColumns
\begin{tabularx}{\textwidth}{N T A}
\hline
Name & Type & Annotation\\
\hline
\hfuzz=500pt\includegraphics[width=1em]{element-mustset.pdf}~outputfileStarCamera & \hfuzz=500pt filename & \hfuzz=500pt combined quaternions\\
\hfuzz=500pt\includegraphics[width=1em]{element.pdf}~outputfileCovariance & \hfuzz=500pt filename & \hfuzz=500pt epoch-wise covariance matrix\\
\hfuzz=500pt\includegraphics[width=1em]{element.pdf}~outputfileCovarianceMatrix & \hfuzz=500pt filename & \hfuzz=500pt full arc-wise covariance matrix per arc. arc number is appended to filename\\
\hfuzz=500pt\includegraphics[width=1em]{element.pdf}~outputfileEpochSigmaStarCamera & \hfuzz=500pt filename & \hfuzz=500pt from vce and outlier detection\\
\hfuzz=500pt\includegraphics[width=1em]{element.pdf}~outputfileEpochSigmaAccelerometer & \hfuzz=500pt filename & \hfuzz=500pt from vce and outlier detection\\
\hfuzz=500pt\includegraphics[width=1em]{element.pdf}~outputfileAngularAcc & \hfuzz=500pt filename & \hfuzz=500pt angular acceleration observations (bias removed)\\
\hfuzz=500pt\includegraphics[width=1em]{element.pdf}~outputfileSolution & \hfuzz=500pt filename & \hfuzz=500pt estimated parameter (one column for each arc)\\
\hfuzz=500pt\includegraphics[width=1em]{element-mustset.pdf}~inputfileStarCameraReference & \hfuzz=500pt filename & \hfuzz=500pt quaternions as taylor point\\
\hfuzz=500pt\includegraphics[width=1em]{element-mustset.pdf}~inputfileStarCamera & \hfuzz=500pt filename & \hfuzz=500pt star camera observations\\
\hfuzz=500pt\includegraphics[width=1em]{element.pdf}~inputfileStarCameraCovariance & \hfuzz=500pt filename & \hfuzz=500pt star camera observations\\
\hfuzz=500pt\includegraphics[width=1em]{element-mustset.pdf}~inputfileAngularAcc & \hfuzz=500pt filename & \hfuzz=500pt angular acceleration observations\\
\hfuzz=500pt\includegraphics[width=1em]{element.pdf}~correctAccNonQuadratic & \hfuzz=500pt boolean & \hfuzz=500pt apply correction (non-square proof mass)\\
\hfuzz=500pt\includegraphics[width=1em]{element-unbounded.pdf}~accBias & \hfuzz=500pt \hyperref[parametrizationTemporalType]{parametrizationTemporal} & \hfuzz=500pt accelerometer bias per interval and axis\\
\hfuzz=500pt\includegraphics[width=1em]{element-unbounded.pdf}~accScale & \hfuzz=500pt \hyperref[parametrizationTemporalType]{parametrizationTemporal} & \hfuzz=500pt accelerometer scale per interval and axis\\
\hfuzz=500pt\includegraphics[width=1em]{element.pdf}~sigmaStarcamera & \hfuzz=500pt double & \hfuzz=500pt [rad]\\
\hfuzz=500pt\includegraphics[width=1em]{element.pdf}~sigmaAccelerometerX & \hfuzz=500pt double & \hfuzz=500pt [rad/s\textasciicircum{}2]\\
\hfuzz=500pt\includegraphics[width=1em]{element.pdf}~sigmaAccelerometerY & \hfuzz=500pt double & \hfuzz=500pt [rad/s\textasciicircum{}2]\\
\hfuzz=500pt\includegraphics[width=1em]{element.pdf}~sigmaAccelerometerZ & \hfuzz=500pt double & \hfuzz=500pt [rad/s\textasciicircum{}2]\\
\hfuzz=500pt\includegraphics[width=1em]{element.pdf}~estimateSigmaScaPerAxis & \hfuzz=500pt boolean & \hfuzz=500pt separate variance factor for roll, pitch, yaw, instead of one common factor.\\
\hfuzz=500pt\includegraphics[width=1em]{element.pdf}~estimateSigmaAccPerAxis & \hfuzz=500pt boolean & \hfuzz=500pt separate variance factor for each accelerometer axis, instead of one common factor.\\
\hfuzz=500pt\includegraphics[width=1em]{element.pdf}~huber & \hfuzz=500pt double & \hfuzz=500pt residuals \$>\$ huber*sigma0 are downweighted\\
\hfuzz=500pt\includegraphics[width=1em]{element.pdf}~huberPower & \hfuzz=500pt double & \hfuzz=500pt residuals \$>\$ huber: sigma=(e/huber)\textasciicircum{}power*sigma0\\
\hfuzz=500pt\includegraphics[width=1em]{element.pdf}~interpolationDegree & \hfuzz=500pt uint & \hfuzz=500pt \\
\hfuzz=500pt\includegraphics[width=1em]{element.pdf}~iterationCount & \hfuzz=500pt uint & \hfuzz=500pt non linear equation solved iteratively\\
\hline
\end{tabularx}

This program is \reference{parallelized}{general.parallelization}.
\clearpage
%==================================
\section{Programs: Gravityfield}
\subsection{Gravityfield2AbsoluteGravity}\label{Gravityfield2AbsoluteGravity}
This program computes the absolute value of gravity $\left\lVert{\M g}\right\rVert$
of a \configClass{gravityfield}{gravityfieldType} on a given \configClass{grid}{gridType}.
The result is multiplicated with \config{factor}.
To get the full gravity vector in a terrestrial frame add
the centrifugal part, see \configClass{gravityfield:tides:centrifugal}{tidesType:centrifugal}.

The values will be saved together with points expressed as ellipsoidal coordinates (longitude, latitude, height)
based on a reference ellipsoid with parameters \config{R} and \config{inverseFlattening}.

It is intended to compute gravity anomalies from absolute gravity observations.
To visualize the results use \program{PlotMap}.


\keepXColumns
\begin{tabularx}{\textwidth}{N T A}
\hline
Name & Type & Annotation\\
\hline
\hfuzz=500pt\includegraphics[width=1em]{element-mustset.pdf}~outputfileGriddedData & \hfuzz=500pt filename & \hfuzz=500pt \\
\hfuzz=500pt\includegraphics[width=1em]{element-mustset-unbounded.pdf}~grid & \hfuzz=500pt \hyperref[gridType]{grid} & \hfuzz=500pt \\
\hfuzz=500pt\includegraphics[width=1em]{element-mustset-unbounded.pdf}~gravityfield & \hfuzz=500pt \hyperref[gravityfieldType]{gravityfield} & \hfuzz=500pt \\
\hfuzz=500pt\includegraphics[width=1em]{element.pdf}~factor & \hfuzz=500pt double & \hfuzz=500pt the result is multiplied by this factor, set -1 to subtract the field\\
\hfuzz=500pt\includegraphics[width=1em]{element.pdf}~time & \hfuzz=500pt time & \hfuzz=500pt at this time the gravity field will be evaluated\\
\hfuzz=500pt\includegraphics[width=1em]{element.pdf}~R & \hfuzz=500pt double & \hfuzz=500pt reference radius for ellipsoidal coordinates on output\\
\hfuzz=500pt\includegraphics[width=1em]{element.pdf}~inverseFlattening & \hfuzz=500pt double & \hfuzz=500pt reference flattening for ellipsoidal coordinates on output, 0: spherical coordinates\\
\hline
\end{tabularx}

This program is \reference{parallelized}{general.parallelization}.
\clearpage
%==================================
\subsection{Gravityfield2AreaMeanTimeSeries}\label{Gravityfield2AreaMeanTimeSeries}
This program computes a time series of time variable
\configClass{gravityfield}{gravityfieldType} functionals averaged over a given area,
e.g. equivalent water heights in the amazon basin.  The type of functional
(e.g gravity anomalies or geoid heights) can be choosen with \configClass{kernel}{kernelType}.
The average is performed at each time step by a weigthed average over all \configClass{grid}{gridType}
points where the weight is the associated area at each point. If \config{removeMean} is set
the temporal mean is removed from the time series. To speed up the computation
the gravity field can be converted to spherical harmonics before the computation
with \config{convertToHarmonics}.

Additionally the root mean square of the values in the area at each time step
can is computed if \config{compueRms} is set.

Additionally the accuracy of the value at each time step can be computed if \config{compueSigma} is set.

The \configFile{outputfileTimeSeries}{instrument} is an instrument file with one, two, or three data columns.
First data column contains the computed functionals and the following columns contain the RMS and the accuracies (optionally).

To visualize the results use \program{PlotGraph}.

\fig{!hb}{0.8}{gravityfield2AreaMeanTimeSeries}{fig:gravityfield2AreaMeanTimeSeries}{Amazon basin: Area mean values of ITSG-Grace2016 monthly solutions with error bars from computed sigma.}


\keepXColumns
\begin{tabularx}{\textwidth}{N T A}
\hline
Name & Type & Annotation\\
\hline
\hfuzz=500pt\includegraphics[width=1em]{element-mustset.pdf}~outputfileTimeSeries & \hfuzz=500pt filename & \hfuzz=500pt \\
\hfuzz=500pt\includegraphics[width=1em]{element-mustset-unbounded.pdf}~grid & \hfuzz=500pt \hyperref[gridType]{grid} & \hfuzz=500pt \\
\hfuzz=500pt\includegraphics[width=1em]{element-mustset-unbounded.pdf}~timeSeries & \hfuzz=500pt \hyperref[timeSeriesType]{timeSeries} & \hfuzz=500pt \\
\hfuzz=500pt\includegraphics[width=1em]{element-mustset.pdf}~kernel & \hfuzz=500pt \hyperref[kernelType]{kernel} & \hfuzz=500pt \\
\hfuzz=500pt\includegraphics[width=1em]{element-mustset-unbounded.pdf}~gravityfield & \hfuzz=500pt \hyperref[gravityfieldType]{gravityfield} & \hfuzz=500pt \\
\hfuzz=500pt\includegraphics[width=1em]{element.pdf}~convertToHarmonics & \hfuzz=500pt boolean & \hfuzz=500pt gravityfield is converted to spherical harmonics before evaluation, may accelerate the computation\\
\hfuzz=500pt\includegraphics[width=1em]{element.pdf}~multiplyWithArea & \hfuzz=500pt boolean & \hfuzz=500pt multiply time series with total area (useful for mass estimates)\\
\hfuzz=500pt\includegraphics[width=1em]{element.pdf}~removeMean & \hfuzz=500pt boolean & \hfuzz=500pt remove the temporal mean of the series\\
\hfuzz=500pt\includegraphics[width=1em]{element.pdf}~computeRms & \hfuzz=500pt boolean & \hfuzz=500pt additional rms each time step\\
\hfuzz=500pt\includegraphics[width=1em]{element.pdf}~computeSigma & \hfuzz=500pt boolean & \hfuzz=500pt additional error bars at each time step\\
\hline
\end{tabularx}

This program is \reference{parallelized}{general.parallelization}.
\clearpage
%==================================
\subsection{Gravityfield2Deflections}\label{Gravityfield2Deflections}
This program computes the deflections of the vertical $\xi$ in north direction
and $\eta$ in east direction in radian of the \configClass{gravityfield}{gravityfieldType}
vector relative to the ellipsoidal normal.
The \configClass{gravityfield}{gravityfieldType} must provide the full gravity vector
inclusive the centrifugal part, see \configClass{gravityfield:tides:centrifugal}{tidesType:centrifugal}.

The values will be saved together with points expressed as ellipsoidal coordinates (longitude, latitude, height)
based on a reference ellipsoid with parameters \config{R} and \config{inverseFlattening}.


\keepXColumns
\begin{tabularx}{\textwidth}{N T A}
\hline
Name & Type & Annotation\\
\hline
\hfuzz=500pt\includegraphics[width=1em]{element-mustset.pdf}~outputfileGriddedData & \hfuzz=500pt filename & \hfuzz=500pt xi (north), eta (east) [rad]\\
\hfuzz=500pt\includegraphics[width=1em]{element-mustset-unbounded.pdf}~grid & \hfuzz=500pt \hyperref[gridType]{grid} & \hfuzz=500pt \\
\hfuzz=500pt\includegraphics[width=1em]{element-mustset-unbounded.pdf}~gravityfield & \hfuzz=500pt \hyperref[gravityfieldType]{gravityfield} & \hfuzz=500pt \\
\hfuzz=500pt\includegraphics[width=1em]{element.pdf}~time & \hfuzz=500pt time & \hfuzz=500pt at this time the gravity field will be evaluated\\
\hfuzz=500pt\includegraphics[width=1em]{element.pdf}~R & \hfuzz=500pt double & \hfuzz=500pt reference radius for ellipsoidal coordinates on output\\
\hfuzz=500pt\includegraphics[width=1em]{element.pdf}~inverseFlattening & \hfuzz=500pt double & \hfuzz=500pt reference flattening for ellipsoidal coordinates on output, 0: spherical coordinates\\
\hline
\end{tabularx}

This program is \reference{parallelized}{general.parallelization}.
\clearpage
%==================================
\subsection{Gravityfield2DegreeAmplitudes}\label{Gravityfield2DegreeAmplitudes}
This program computes degree amplitudes from a \configClass{gravityfield}{gravityfieldType}
and saves them to a \file{matrix}{matrix} file with three columns: the degree, the degree amplitude, and the formal errors.

The coefficients can be converted to different functionals with \configClass{kernel}{kernelType}.
The gravity field can be evaluated at different altitudes by specifying \config{evaluationRadius}.
Polar regions can be excluded by setting \config{polarGap}.
If set the expansion is limited in the range between \config{minDegree}
and \config{maxDegree} inclusivly.
The coefficients are related to the reference radius~\config{R}
and the Earth gravitational constant \config{GM}.

See also \program{PotentialCoefficients2DegreeAmplitudes}.


\keepXColumns
\begin{tabularx}{\textwidth}{N T A}
\hline
Name & Type & Annotation\\
\hline
\hfuzz=500pt\includegraphics[width=1em]{element-mustset.pdf}~outputfileMatrix & \hfuzz=500pt filename & \hfuzz=500pt three column matrix with degree, signal amplitude, formal error\\
\hfuzz=500pt\includegraphics[width=1em]{element-mustset-unbounded.pdf}~gravityfield & \hfuzz=500pt \hyperref[gravityfieldType]{gravityfield} & \hfuzz=500pt \\
\hfuzz=500pt\includegraphics[width=1em]{element-mustset.pdf}~kernel & \hfuzz=500pt \hyperref[kernelType]{kernel} & \hfuzz=500pt \\
\hfuzz=500pt\includegraphics[width=1em]{element-mustset.pdf}~type & \hfuzz=500pt choice & \hfuzz=500pt type of variances\\
\hfuzz=500pt\includegraphics[width=1em]{connector.pdf}\includegraphics[width=1em]{element-mustset.pdf}~rms & \hfuzz=500pt  & \hfuzz=500pt degree amplitudes (square root of degree variances)\\
\hfuzz=500pt\includegraphics[width=1em]{connector.pdf}\includegraphics[width=1em]{element-mustset.pdf}~accumulation & \hfuzz=500pt  & \hfuzz=500pt cumulate variances over degrees\\
\hfuzz=500pt\includegraphics[width=1em]{connector.pdf}\includegraphics[width=1em]{element-mustset.pdf}~median & \hfuzz=500pt  & \hfuzz=500pt meadian of absolute values per degree\\
\hfuzz=500pt\includegraphics[width=1em]{element.pdf}~time & \hfuzz=500pt time & \hfuzz=500pt at this time the gravity field will be evaluated\\
\hfuzz=500pt\includegraphics[width=1em]{element.pdf}~evaluationRadius & \hfuzz=500pt double & \hfuzz=500pt evaluate the gravity field at this radius (default: evaluate at surface\\
\hfuzz=500pt\includegraphics[width=1em]{element.pdf}~polarGap & \hfuzz=500pt angle & \hfuzz=500pt exclude polar regions (aperture angle in degrees)\\
\hfuzz=500pt\includegraphics[width=1em]{element.pdf}~minDegree & \hfuzz=500pt uint & \hfuzz=500pt \\
\hfuzz=500pt\includegraphics[width=1em]{element.pdf}~maxDegree & \hfuzz=500pt uint & \hfuzz=500pt \\
\hfuzz=500pt\includegraphics[width=1em]{element.pdf}~GM & \hfuzz=500pt double & \hfuzz=500pt Geocentric gravitational constant\\
\hfuzz=500pt\includegraphics[width=1em]{element.pdf}~R & \hfuzz=500pt double & \hfuzz=500pt reference radius\\
\hline
\end{tabularx}

\clearpage
%==================================
\subsection{Gravityfield2DegreeAmplitudesPlotGrid}\label{Gravityfield2DegreeAmplitudesPlotGrid}
This program computes a \configClass{timeSeries}{timeSeriesType}
of a time variable \configClass{gravityfield}{gravityfieldType} and saves it as degree amplitudes.
The expansion is limited in the range between \config{minDegree} and \config{maxDegree} inclusivly
\begin{equation}
  \sigma_n = \frac{GM}{R}\left(\frac{R}{r}\right)^{n+1}k_n\sqrt{\sum_{m=0}^n c_{nm}^2+s_{nm}^2}.
\end{equation}

The \configFile{outputfileTimeSeries}{matrix} is a matrix with
every row containing the time, degree, degree amplitude, and the formal error.

To visualize the results use \program{PlotGraph}).

See also \program{Gravityfield2DegreeAmplitudes}.

\fig{!hb}{0.5}{gravityfield2DegreeAmplitudesPlotGrid}{fig:gravityfield2DegreeAmplitudesPlotGrid}{Degree amplitudes of monthly ITSG-Grace2016 solutions relative to GOCO05s.}


\keepXColumns
\begin{tabularx}{\textwidth}{N T A}
\hline
Name & Type & Annotation\\
\hline
\hfuzz=500pt\includegraphics[width=1em]{element-mustset.pdf}~outputfileTimeSeries & \hfuzz=500pt filename & \hfuzz=500pt each row: mjd, degree, amplitude, formal error\\
\hfuzz=500pt\includegraphics[width=1em]{element-mustset-unbounded.pdf}~gravityfield & \hfuzz=500pt \hyperref[gravityfieldType]{gravityfield} & \hfuzz=500pt \\
\hfuzz=500pt\includegraphics[width=1em]{element-mustset.pdf}~kernel & \hfuzz=500pt \hyperref[kernelType]{kernel} & \hfuzz=500pt \\
\hfuzz=500pt\includegraphics[width=1em]{element-mustset-unbounded.pdf}~timeSeries & \hfuzz=500pt \hyperref[timeSeriesType]{timeSeries} & \hfuzz=500pt \\
\hfuzz=500pt\includegraphics[width=1em]{element.pdf}~evaluationRadius & \hfuzz=500pt double & \hfuzz=500pt evaluate the gravity field at this radius (default: evaluate at surface\\
\hfuzz=500pt\includegraphics[width=1em]{element.pdf}~polarGap & \hfuzz=500pt angle & \hfuzz=500pt exclude polar regions (aperture angle in degrees)\\
\hfuzz=500pt\includegraphics[width=1em]{element-mustset.pdf}~minDegree & \hfuzz=500pt uint & \hfuzz=500pt minimal degree\\
\hfuzz=500pt\includegraphics[width=1em]{element-mustset.pdf}~maxDegree & \hfuzz=500pt uint & \hfuzz=500pt maximal degree\\
\hfuzz=500pt\includegraphics[width=1em]{element.pdf}~GM & \hfuzz=500pt double & \hfuzz=500pt Geocentric gravitational constant\\
\hfuzz=500pt\includegraphics[width=1em]{element.pdf}~R & \hfuzz=500pt double & \hfuzz=500pt reference radius\\
\hline
\end{tabularx}

\clearpage
%==================================
\subsection{Gravityfield2DisplacementTimeSeries}\label{Gravityfield2DisplacementTimeSeries}
This program computes a time series of displacements of a list of stations (\configClass{grid}{gridType})
due to the effect of time variable loading masses. The displacement~$\M u$ of a station is calculated according to
\begin{equation}\label{eq:displacement}
\M u(\M r) = \frac{1}{\gamma}\sum_{n=0}^\infty \left[\frac{h_n}{1+k_n}V_n(\M r)\,\M e_{up}
+ R\frac{l_n}{1+k_n}\left(
 \frac{\partial V_n(\M r)}{\partial \M e_{north}}\M e_{north}
+\frac{\partial V_n(\M r)}{\partial \M e_{east}} \M e_{east}\right)\right],
\end{equation}
where $\gamma$ is the normal gravity, the load Love and Shida numbers $h_n,l_n$ are given by
\configFile{inputfileDeformationLoadLoveNumber}{matrix} and the load Love numbers $k_n$ are given by
\configFile{inputfilePotentialLoadLoveNumber}{matrix}. The $V_n$ are the spherical harmonics expansion of
the full time variable gravitational potential (potential of the loading mass + deformation potential):
\begin{equation}
V(\M r) = \sum_{n=0}^\infty V_n(\M r).
\end{equation}
Deformations due to Earth tide and due to polar tides are computed using the IERS conventions.
Eq.~\eqref{eq:displacement} is not used in these cases.

The \config{outputfileTimeSeries} is an \file{instrument file}{instrument}, MISCVALUES.
The data columns contain the deformation of each station in $x,y,z$ in a global terrestrial
reference frame or alternatively in a local elliposidal frame (north, east, up)
if \config{localReferenceFrame} is set.


\keepXColumns
\begin{tabularx}{\textwidth}{N T A}
\hline
Name & Type & Annotation\\
\hline
\hfuzz=500pt\includegraphics[width=1em]{element-mustset.pdf}~outputfileTimeSeries & \hfuzz=500pt filename & \hfuzz=500pt x,y,z [m] per station\\
\hfuzz=500pt\includegraphics[width=1em]{element-mustset-unbounded.pdf}~grid & \hfuzz=500pt \hyperref[gridType]{grid} & \hfuzz=500pt station list\\
\hfuzz=500pt\includegraphics[width=1em]{element-mustset-unbounded.pdf}~timeSeries & \hfuzz=500pt \hyperref[timeSeriesType]{timeSeries} & \hfuzz=500pt \\
\hfuzz=500pt\includegraphics[width=1em]{element-unbounded.pdf}~gravityfield & \hfuzz=500pt \hyperref[gravityfieldType]{gravityfield} & \hfuzz=500pt \\
\hfuzz=500pt\includegraphics[width=1em]{element-unbounded.pdf}~tides & \hfuzz=500pt \hyperref[tidesType]{tides} & \hfuzz=500pt \\
\hfuzz=500pt\includegraphics[width=1em]{element-mustset.pdf}~earthRotation & \hfuzz=500pt \hyperref[earthRotationType]{earthRotation} & \hfuzz=500pt \\
\hfuzz=500pt\includegraphics[width=1em]{element.pdf}~ephemerides & \hfuzz=500pt \hyperref[ephemeridesType]{ephemerides} & \hfuzz=500pt \\
\hfuzz=500pt\includegraphics[width=1em]{element-mustset.pdf}~inputfileDeformationLoadLoveNumber & \hfuzz=500pt filename & \hfuzz=500pt \\
\hfuzz=500pt\includegraphics[width=1em]{element.pdf}~inputfilePotentialLoadLoveNumber & \hfuzz=500pt filename & \hfuzz=500pt if full potential is given and not only loading potential\\
\hfuzz=500pt\includegraphics[width=1em]{element.pdf}~removeMean & \hfuzz=500pt boolean & \hfuzz=500pt remove the temporal mean of each coordinate\\
\hfuzz=500pt\includegraphics[width=1em]{element.pdf}~localReferenceFrame & \hfuzz=500pt boolean & \hfuzz=500pt local left handed reference frame (north, east, up)\\
\hline
\end{tabularx}

\clearpage
%==================================
\subsection{Gravityfield2EmpiricalCovariance}\label{Gravityfield2EmpiricalCovariance}
This program estimates an spatial and temporal covariance matrix from
a time series of gravity fields.

Firstly for every time step $t_i$
a spherical harmonics vector $\M x_i$ from the time variable gravity field
is generated. The coefficients of the spherical harmonics expansion are
in the sequence given by \configClass{numbering}{sphericalHarmonicsNumberingType}.
If set the expansion is limited in the range between \config{minDegree}
and \config{maxDegree} inclusivly. The coefficients are related to the
reference radius~\config{R} and the Earth gravitational constant \config{GM}.

In the next step the empirical covariance matrix is estimated according to
\begin{equation}
\M\Sigma(\Delta i)_{full} = \frac{1}{N}\sum_{i=1}^N \M x_i \M x_{i+\Delta i}^T,
\end{equation}
where $\Delta i$ is given by \config{differenceStep}.

From the diagonal elements of $\M\Sigma(\Delta i)$ the isotropic accuracies
are computed
\begin{equation}
\sigma_n^2 = \frac{1}{2n+1}\sum_{m=0}^n \sigma_{cnm}^2+\sigma_{snm}^2,
\end{equation}
and a diagonal matrix is constructed $\Sigma_{iso} = \text{diag}(\sigma_2^2,\ldots,\sigma_N^2)$.
The result is computed:
\begin{equation}
\M\Sigma(\Delta i) = \alpha_{full}\M\Sigma(\Delta i)_{full}+\alpha_{iso}\M\Sigma(\Delta i)_{iso}.
\end{equation}


\keepXColumns
\begin{tabularx}{\textwidth}{N T A}
\hline
Name & Type & Annotation\\
\hline
\hfuzz=500pt\includegraphics[width=1em]{element-mustset.pdf}~outputfileCovarianceMatrix & \hfuzz=500pt filename & \hfuzz=500pt \\
\hfuzz=500pt\includegraphics[width=1em]{element.pdf}~outputfilePotentialCoefficients & \hfuzz=500pt filename & \hfuzz=500pt \\
\hfuzz=500pt\includegraphics[width=1em]{element-mustset-unbounded.pdf}~gravityfield & \hfuzz=500pt \hyperref[gravityfieldType]{gravityfield} & \hfuzz=500pt \\
\hfuzz=500pt\includegraphics[width=1em]{element.pdf}~minDegree & \hfuzz=500pt uint & \hfuzz=500pt \\
\hfuzz=500pt\includegraphics[width=1em]{element-mustset.pdf}~maxDegree & \hfuzz=500pt uint & \hfuzz=500pt \\
\hfuzz=500pt\includegraphics[width=1em]{element.pdf}~GM & \hfuzz=500pt double & \hfuzz=500pt Geocentric gravitational constant\\
\hfuzz=500pt\includegraphics[width=1em]{element.pdf}~R & \hfuzz=500pt double & \hfuzz=500pt reference radius\\
\hfuzz=500pt\includegraphics[width=1em]{element-mustset.pdf}~numbering & \hfuzz=500pt \hyperref[sphericalHarmonicsNumberingType]{sphericalHarmonicsNumbering} & \hfuzz=500pt numbering scheme for solution vector\\
\hfuzz=500pt\includegraphics[width=1em]{element.pdf}~removeMean & \hfuzz=500pt boolean & \hfuzz=500pt \\
\hfuzz=500pt\includegraphics[width=1em]{element-mustset-unbounded.pdf}~timeSeries & \hfuzz=500pt \hyperref[timeSeriesType]{timeSeries} & \hfuzz=500pt sampling of the gravityfield\\
\hfuzz=500pt\includegraphics[width=1em]{element.pdf}~differenceStep & \hfuzz=500pt uint & \hfuzz=500pt choose dt for: x,i(t) - x,j(t+dt)\\
\hfuzz=500pt\includegraphics[width=1em]{element.pdf}~factorFullMatrixPart & \hfuzz=500pt double & \hfuzz=500pt \\
\hfuzz=500pt\includegraphics[width=1em]{element.pdf}~factorIsotropicPart & \hfuzz=500pt double & \hfuzz=500pt \\
\hfuzz=500pt\includegraphics[width=1em]{element-unbounded.pdf}~intervals & \hfuzz=500pt \hyperref[timeSeriesType]{timeSeries} & \hfuzz=500pt \\
\hline
\end{tabularx}

\clearpage
%==================================
\subsection{Gravityfield2Gradients}\label{Gravityfield2Gradients}
This program computes gravity gradients from \configClass{gravityfield}{gravityfieldType}
on a \configClass{grid}{gridType} in a global terrestrial reference frame
or alternatively in a local elliposidal frame (north, east, up) if \config{localReferenceFrame} is set.
In \configFile{outputfileGriddedData}{griddedData} the values $[Vxx, Vyy, Vzz, Vxy, Vxz, Vyz]$
will be saved together with points expressed as ellipsoidal coordinates
(longitude, latitude, height) based on a reference ellipsoid with parameters \config{R} and \config{inverseFlattening}.


\keepXColumns
\begin{tabularx}{\textwidth}{N T A}
\hline
Name & Type & Annotation\\
\hline
\hfuzz=500pt\includegraphics[width=1em]{element-mustset.pdf}~outputfileGriddedData & \hfuzz=500pt filename & \hfuzz=500pt Vxx Vyy Vzz Vxy Vxz Vyz\\
\hfuzz=500pt\includegraphics[width=1em]{element-mustset-unbounded.pdf}~grid & \hfuzz=500pt \hyperref[gridType]{grid} & \hfuzz=500pt \\
\hfuzz=500pt\includegraphics[width=1em]{element-mustset-unbounded.pdf}~gravityfield & \hfuzz=500pt \hyperref[gravityfieldType]{gravityfield} & \hfuzz=500pt \\
\hfuzz=500pt\includegraphics[width=1em]{element.pdf}~localReferenceFrame & \hfuzz=500pt boolean & \hfuzz=500pt local left handed reference frame (north, east, up)\\
\hfuzz=500pt\includegraphics[width=1em]{element.pdf}~time & \hfuzz=500pt time & \hfuzz=500pt at this time the gravity field will be evaluated\\
\hfuzz=500pt\includegraphics[width=1em]{element.pdf}~R & \hfuzz=500pt double & \hfuzz=500pt reference radius for ellipsoidal coordinates on output\\
\hfuzz=500pt\includegraphics[width=1em]{element.pdf}~inverseFlattening & \hfuzz=500pt double & \hfuzz=500pt reference flattening for ellipsoidal coordinates on output, 0: spherical coordinates\\
\hline
\end{tabularx}

This program is \reference{parallelized}{general.parallelization}.
\clearpage
%==================================
\subsection{Gravityfield2GridCovarianceMatrix}\label{Gravityfield2GridCovarianceMatrix}
This program propagates the covariance matrix of a \configClass{gravityfield}{gravityfieldType}
evaluated at \config{time} to a \configClass{grid}{gridType}. The full variance-covariance matrix is computed
and written to a \file{matrix file}{matrix}:
\begin{equation}
\mathbf{\Sigma}_\mathbf{y} = \mathbf{F}\mathbf{\Sigma}_\mathbf{x}\mathbf{F}^T
\end{equation}
The \configClass{kernel}{kernelType} determines the quantity of the grid values, for example,
\configClass{kernel:waterHeight}{kernelType:waterHeight}.

See also \program{GravityfieldCovariancesPropagation2GriddedData}, \program{GravityfieldVariancesPropagation2GriddedData}.


\keepXColumns
\begin{tabularx}{\textwidth}{N T A}
\hline
Name & Type & Annotation\\
\hline
\hfuzz=500pt\includegraphics[width=1em]{element-mustset.pdf}~outputfileMatrix & \hfuzz=500pt filename & \hfuzz=500pt symmetric grid covariance matrix\\
\hfuzz=500pt\includegraphics[width=1em]{element-mustset-unbounded.pdf}~grid & \hfuzz=500pt \hyperref[gridType]{grid} & \hfuzz=500pt \\
\hfuzz=500pt\includegraphics[width=1em]{element-mustset.pdf}~kernel & \hfuzz=500pt \hyperref[kernelType]{kernel} & \hfuzz=500pt \\
\hfuzz=500pt\includegraphics[width=1em]{element-mustset-unbounded.pdf}~gravityfield & \hfuzz=500pt \hyperref[gravityfieldType]{gravityfield} & \hfuzz=500pt \\
\hfuzz=500pt\includegraphics[width=1em]{element.pdf}~time & \hfuzz=500pt time & \hfuzz=500pt at this time the gravity field will be evaluated\\
\hline
\end{tabularx}

\clearpage
%==================================
\subsection{Gravityfield2GriddedData}\label{Gravityfield2GriddedData}
This program computes values of a \configClass{gravityfield}{gravityfieldType} on a given \configClass{grid}{gridType}.
The type of value (e.g gravity anomalies or geoid heights) can be choosen with \configClass{kernel}{kernelType}.
If a time is given the gravity field will be evaluated at this point of time otherwise only the static part will be used.
The values will be saved together with points expressed as ellipsoidal coordinates (longitude, latitude, height)
based on a reference ellipsoid with parameters \config{R} and \config{inverseFlattening}.
To speed up the computation the gravity field can be converted to spherical harmonics before the computation
with \config{convertToHarmonics}.

To visualize the results use \program{PlotMap}.


\keepXColumns
\begin{tabularx}{\textwidth}{N T A}
\hline
Name & Type & Annotation\\
\hline
\hfuzz=500pt\includegraphics[width=1em]{element-mustset.pdf}~outputfileGriddedData & \hfuzz=500pt filename & \hfuzz=500pt \\
\hfuzz=500pt\includegraphics[width=1em]{element-mustset-unbounded.pdf}~grid & \hfuzz=500pt \hyperref[gridType]{grid} & \hfuzz=500pt \\
\hfuzz=500pt\includegraphics[width=1em]{element-mustset.pdf}~kernel & \hfuzz=500pt \hyperref[kernelType]{kernel} & \hfuzz=500pt \\
\hfuzz=500pt\includegraphics[width=1em]{element-mustset-unbounded.pdf}~gravityfield & \hfuzz=500pt \hyperref[gravityfieldType]{gravityfield} & \hfuzz=500pt \\
\hfuzz=500pt\includegraphics[width=1em]{element.pdf}~convertToHarmonics & \hfuzz=500pt boolean & \hfuzz=500pt gravityfield is converted to spherical harmonics before evaluation, may accelerate the computation\\
\hfuzz=500pt\includegraphics[width=1em]{element.pdf}~time & \hfuzz=500pt time & \hfuzz=500pt at this time the gravity field will be evaluated\\
\hfuzz=500pt\includegraphics[width=1em]{element.pdf}~R & \hfuzz=500pt double & \hfuzz=500pt reference radius for ellipsoidal coordinates on output\\
\hfuzz=500pt\includegraphics[width=1em]{element.pdf}~inverseFlattening & \hfuzz=500pt double & \hfuzz=500pt reference flattening for ellipsoidal coordinates on output, 0: spherical coordinates\\
\hline
\end{tabularx}

This program is \reference{parallelized}{general.parallelization}.
\clearpage
%==================================
\subsection{Gravityfield2GriddedDataTimeSeries}\label{Gravityfield2GriddedDataTimeSeries}
This program computes values of a \configClass{gravityfield}{gravityfieldType} on a given \configClass{grid}{gridType}
for each time step of \configClass{timeSeries}{timeSeriesType}.
The type of value (e.g gravity anomalies or geoid heights) can be choosen with \configClass{kernel}{kernelType}.
To speed up the computation the gravity field can be converted to spherical harmonics before the computation
with \config{convertToHarmonics}.
The \configFile{outputfileTimeSeries}{instrument} is an instrument (MISCVALUES) file with a data column
for each grid point per epoch.

This program enables the use of all instrument programs like \program{InstrumentFilter},
\program{InstrumentArcStatistics} or \program{InstrumentDetrend} to analyze and manipulate time series of gridded data.

See also \program{TimeSeries2GriddedData}, \program{Gravityfield2GriddedData}


\keepXColumns
\begin{tabularx}{\textwidth}{N T A}
\hline
Name & Type & Annotation\\
\hline
\hfuzz=500pt\includegraphics[width=1em]{element-mustset.pdf}~outputfileTimeSeries & \hfuzz=500pt filename & \hfuzz=500pt each epoch: data of grid points (MISCVALUES)\\
\hfuzz=500pt\includegraphics[width=1em]{element-mustset-unbounded.pdf}~grid & \hfuzz=500pt \hyperref[gridType]{grid} & \hfuzz=500pt \\
\hfuzz=500pt\includegraphics[width=1em]{element-mustset.pdf}~kernel & \hfuzz=500pt \hyperref[kernelType]{kernel} & \hfuzz=500pt \\
\hfuzz=500pt\includegraphics[width=1em]{element-mustset-unbounded.pdf}~gravityfield & \hfuzz=500pt \hyperref[gravityfieldType]{gravityfield} & \hfuzz=500pt \\
\hfuzz=500pt\includegraphics[width=1em]{element.pdf}~convertToHarmonics & \hfuzz=500pt boolean & \hfuzz=500pt gravityfield is converted to spherical harmonics before evaluation, may accelerate the computation\\
\hfuzz=500pt\includegraphics[width=1em]{element-mustset-unbounded.pdf}~timeSeries & \hfuzz=500pt \hyperref[timeSeriesType]{timeSeries} & \hfuzz=500pt \\
\hline
\end{tabularx}

This program is \reference{parallelized}{general.parallelization}.
\clearpage
%==================================
\subsection{Gravityfield2PotentialCoefficients}\label{Gravityfield2PotentialCoefficients}
This program evaluates a time variable \configClass{gravityfield}{gravityfieldType}
at a given \config{time} and saves it as a \file{spherical harmonics file}{potentialCoefficients}.
If set the expansion is limited in the range between \config{minDegree}
and \config{maxDegree} inclusivly.
The coefficients are related to the reference radius~\config{R}
and the Earth gravitational constant \config{GM}.


\keepXColumns
\begin{tabularx}{\textwidth}{N T A}
\hline
Name & Type & Annotation\\
\hline
\hfuzz=500pt\includegraphics[width=1em]{element-mustset.pdf}~outputfilePotentialCoefficients & \hfuzz=500pt filename & \hfuzz=500pt \\
\hfuzz=500pt\includegraphics[width=1em]{element-mustset-unbounded.pdf}~gravityfield & \hfuzz=500pt \hyperref[gravityfieldType]{gravityfield} & \hfuzz=500pt \\
\hfuzz=500pt\includegraphics[width=1em]{element.pdf}~minDegree & \hfuzz=500pt uint & \hfuzz=500pt \\
\hfuzz=500pt\includegraphics[width=1em]{element.pdf}~maxDegree & \hfuzz=500pt uint & \hfuzz=500pt \\
\hfuzz=500pt\includegraphics[width=1em]{element.pdf}~GM & \hfuzz=500pt double & \hfuzz=500pt Geocentric gravitational constant\\
\hfuzz=500pt\includegraphics[width=1em]{element.pdf}~R & \hfuzz=500pt double & \hfuzz=500pt reference radius\\
\hfuzz=500pt\includegraphics[width=1em]{element.pdf}~time & \hfuzz=500pt time & \hfuzz=500pt at this time the gravity field will be evaluated\\
\hline
\end{tabularx}

\clearpage
%==================================
\subsection{Gravityfield2PotentialCoefficientsTimeSeries}\label{Gravityfield2PotentialCoefficientsTimeSeries}
This program computes a \configClass{timeSeries}{timeSeriesType}
of a time variable \configClass{gravityfield}{gravityfieldType}
and converts to coefficients of a spherical harmonics expansion.
The expansion is limited in the range between \config{minDegree}
and \config{maxDegree} inclusivly.
The coefficients are related to the reference radius~\config{R}
and the Earth gravitational constant \config{GM}.

The \configFile{outputfileTimeSeries}{instrument} contains the potential coefficients
as data columns for each epoch in the sequence given by
\configClass{numbering}{sphericalHarmonicsNumberingType}.


\keepXColumns
\begin{tabularx}{\textwidth}{N T A}
\hline
Name & Type & Annotation\\
\hline
\hfuzz=500pt\includegraphics[width=1em]{element-mustset.pdf}~outputfileTimeSeries & \hfuzz=500pt filename & \hfuzz=500pt instrument file (MISCVALUES)\\
\hfuzz=500pt\includegraphics[width=1em]{element-mustset-unbounded.pdf}~gravityfield & \hfuzz=500pt \hyperref[gravityfieldType]{gravityfield} & \hfuzz=500pt \\
\hfuzz=500pt\includegraphics[width=1em]{element-mustset-unbounded.pdf}~timeSeries & \hfuzz=500pt \hyperref[timeSeriesType]{timeSeries} & \hfuzz=500pt \\
\hfuzz=500pt\includegraphics[width=1em]{element-mustset.pdf}~minDegree & \hfuzz=500pt uint & \hfuzz=500pt \\
\hfuzz=500pt\includegraphics[width=1em]{element-mustset.pdf}~maxDegree & \hfuzz=500pt uint & \hfuzz=500pt \\
\hfuzz=500pt\includegraphics[width=1em]{element.pdf}~GM & \hfuzz=500pt double & \hfuzz=500pt Geocentric gravitational constant\\
\hfuzz=500pt\includegraphics[width=1em]{element.pdf}~R & \hfuzz=500pt double & \hfuzz=500pt reference radius\\
\hfuzz=500pt\includegraphics[width=1em]{element-mustset.pdf}~numbering & \hfuzz=500pt \hyperref[sphericalHarmonicsNumberingType]{sphericalHarmonicsNumbering} & \hfuzz=500pt numbering scheme\\
\hline
\end{tabularx}

\clearpage
%==================================
\subsection{Gravityfield2SphericalHarmonicsVector}\label{Gravityfield2SphericalHarmonicsVector}
This program evaluates a time variable \configClass{gravityfield}{gravityfieldType} at a given \config{time}
and saves a \file{vector}{matrix} with the coefficients of a spherical harmonics expansion in the sequence given by
\configClass{numbering}{sphericalHarmonicsNumberingType}.
If set the expansion is limited in the range between \config{minDegree} and \config{maxDegree} inclusively.
The coefficients are related to the reference radius~\config{R} and the Earth gravitational constant \config{GM}.

This coefficients vector can be used as a approximate solution, see \program{NormalsMultiplyAdd},
or as pseudo oberservations for regularization,
see \configClass{normalEquation:regularization}{normalEquationType:regularization}.

For back transformation use \program{Gravityfield2PotentialCoefficients}
with \configClass{gravityfield:fromParametrization}{gravityfieldType:fromParametrization}.


\keepXColumns
\begin{tabularx}{\textwidth}{N T A}
\hline
Name & Type & Annotation\\
\hline
\hfuzz=500pt\includegraphics[width=1em]{element-mustset.pdf}~outputfileVector & \hfuzz=500pt filename & \hfuzz=500pt \\
\hfuzz=500pt\includegraphics[width=1em]{element-mustset-unbounded.pdf}~gravityfield & \hfuzz=500pt \hyperref[gravityfieldType]{gravityfield} & \hfuzz=500pt \\
\hfuzz=500pt\includegraphics[width=1em]{element.pdf}~startIndex & \hfuzz=500pt uint & \hfuzz=500pt start index to put the coefficients in the solution vector\\
\hfuzz=500pt\includegraphics[width=1em]{element-mustset.pdf}~minDegree & \hfuzz=500pt uint & \hfuzz=500pt \\
\hfuzz=500pt\includegraphics[width=1em]{element-mustset.pdf}~maxDegree & \hfuzz=500pt uint & \hfuzz=500pt \\
\hfuzz=500pt\includegraphics[width=1em]{element.pdf}~GM & \hfuzz=500pt double & \hfuzz=500pt Geocentric gravitational constant\\
\hfuzz=500pt\includegraphics[width=1em]{element.pdf}~R & \hfuzz=500pt double & \hfuzz=500pt reference radius\\
\hfuzz=500pt\includegraphics[width=1em]{element-mustset.pdf}~numbering & \hfuzz=500pt \hyperref[sphericalHarmonicsNumberingType]{sphericalHarmonicsNumbering} & \hfuzz=500pt numbering scheme for solution vector\\
\hfuzz=500pt\includegraphics[width=1em]{element.pdf}~time & \hfuzz=500pt time & \hfuzz=500pt at this time the gravity field will be evaluated\\
\hfuzz=500pt\includegraphics[width=1em]{element.pdf}~useSigma & \hfuzz=500pt boolean & \hfuzz=500pt use formal errors instead of coefficients\\
\hline
\end{tabularx}

\clearpage
%==================================
\subsection{Gravityfield2TimeSplines}\label{Gravityfield2TimeSplines}
This program estimates splines in time domain from a time variable gravity field
and writes \configFile{outputfileTimeSplines}{timeSplinesGravityField}.
The \configClass{gravityfield}{gravityfieldType} is sampled at \configClass{sampling}{timeSeriesType}, converted to potential coefficients
in the range between \config{minDegree} and \config{maxDegree} inclusively.
The time series of spherical harmonics can be temporal filtered with \configClass{temporalFilter}{digitalFilterType}.

In the next step temporal splines with \config{splineDegree} and nodal points given
at \configClass{splineTimeSeries}{timeSeriesType} are adjusted to the time series in a least squares sense.
This is very fast for block means (splineDegree = 0) but for other splines a large systems of equations
must be solved. In the adjustment process the time series of gravity fields can be interpreted as samples
at the given times or as continuous function with linear behaviour between sampled points (\config{linearInterpolation}).

To combine a series of potential coefficients to a spline file with block means (splineDegree = 0)
use the fast \program{PotentialCoefficients2BlockMeanTimeSplines} instead.


\keepXColumns
\begin{tabularx}{\textwidth}{N T A}
\hline
Name & Type & Annotation\\
\hline
\hfuzz=500pt\includegraphics[width=1em]{element-mustset.pdf}~outputfileTimeSplines & \hfuzz=500pt filename & \hfuzz=500pt \\
\hfuzz=500pt\includegraphics[width=1em]{element-mustset-unbounded.pdf}~gravityfield & \hfuzz=500pt \hyperref[gravityfieldType]{gravityfield} & \hfuzz=500pt \\
\hfuzz=500pt\includegraphics[width=1em]{element-unbounded.pdf}~temporalFilter & \hfuzz=500pt \hyperref[digitalFilterType]{digitalFilter} & \hfuzz=500pt filter sampled gravity field in time\\
\hfuzz=500pt\includegraphics[width=1em]{element.pdf}~minDegree & \hfuzz=500pt uint & \hfuzz=500pt \\
\hfuzz=500pt\includegraphics[width=1em]{element.pdf}~maxDegree & \hfuzz=500pt uint & \hfuzz=500pt \\
\hfuzz=500pt\includegraphics[width=1em]{element.pdf}~GM & \hfuzz=500pt double & \hfuzz=500pt Geocentric gravitational constant\\
\hfuzz=500pt\includegraphics[width=1em]{element.pdf}~R & \hfuzz=500pt double & \hfuzz=500pt reference radius\\
\hfuzz=500pt\includegraphics[width=1em]{element-mustset-unbounded.pdf}~sampling & \hfuzz=500pt \hyperref[timeSeriesType]{timeSeries} & \hfuzz=500pt gravity field is sampled at these times\\
\hfuzz=500pt\includegraphics[width=1em]{element.pdf}~removeMean & \hfuzz=500pt boolean & \hfuzz=500pt remove the temporal mean of the series before estimating the splines\\
\hfuzz=500pt\includegraphics[width=1em]{element.pdf}~linearInterpolation & \hfuzz=500pt boolean & \hfuzz=500pt assume linear behavior between sampled points\\
\hfuzz=500pt\includegraphics[width=1em]{element-mustset.pdf}~splineDegree & \hfuzz=500pt uint & \hfuzz=500pt degree of splines\\
\hfuzz=500pt\includegraphics[width=1em]{element-mustset-unbounded.pdf}~splineTimeSeries & \hfuzz=500pt \hyperref[timeSeriesType]{timeSeries} & \hfuzz=500pt nodal points of splines in time domain\\
\hline
\end{tabularx}

\clearpage
%==================================
\subsection{GravityfieldCovariancesPropagation2GriddedData}\label{GravityfieldCovariancesPropagation2GriddedData}
This program computes the covariance between a source point given
by longitude/latitude (\config{L}, \config{B}) and the points of a \configClass{grid}{gridType}
in terms of the functional given by \configClass{kernel}{kernelType} from the variance-covariance
matrix of a \configClass{gravityfield}{gravityfieldType} evaluated at \config{time}.

If \config{computeCorrelation} is set, the program returns the correlation according to
\begin{equation}
r_{ij} = \frac{\sigma_{ij}}{\sigma_i \sigma_j}
\end{equation}
in the range of [-1, 1] instead of the covariance.

See also \program{Gravityfield2GridCovarianceMatrix}, \program{GravityfieldVariancesPropagation2GriddedData}.


\keepXColumns
\begin{tabularx}{\textwidth}{N T A}
\hline
Name & Type & Annotation\\
\hline
\hfuzz=500pt\includegraphics[width=1em]{element-mustset.pdf}~outputfileGriddedData & \hfuzz=500pt filename & \hfuzz=500pt gridded data file containing the covariance betwenn source point and grid points\\
\hfuzz=500pt\includegraphics[width=1em]{element-mustset-unbounded.pdf}~grid & \hfuzz=500pt \hyperref[gridType]{grid} & \hfuzz=500pt \\
\hfuzz=500pt\includegraphics[width=1em]{element-mustset.pdf}~kernel & \hfuzz=500pt \hyperref[kernelType]{kernel} & \hfuzz=500pt functional\\
\hfuzz=500pt\includegraphics[width=1em]{element-mustset-unbounded.pdf}~gravityfield & \hfuzz=500pt \hyperref[gravityfieldType]{gravityfield} & \hfuzz=500pt \\
\hfuzz=500pt\includegraphics[width=1em]{element.pdf}~time & \hfuzz=500pt time & \hfuzz=500pt at this time the gravity field will be evaluated\\
\hfuzz=500pt\includegraphics[width=1em]{element.pdf}~L & \hfuzz=500pt angle & \hfuzz=500pt longitude of variance point\\
\hfuzz=500pt\includegraphics[width=1em]{element.pdf}~B & \hfuzz=500pt angle & \hfuzz=500pt latitude of variance point\\
\hfuzz=500pt\includegraphics[width=1em]{element.pdf}~height & \hfuzz=500pt double & \hfuzz=500pt ellipsoidal height of source point\\
\hfuzz=500pt\includegraphics[width=1em]{element.pdf}~computeCorrelation & \hfuzz=500pt boolean & \hfuzz=500pt compute correlations instead of covariances\\
\hfuzz=500pt\includegraphics[width=1em]{element.pdf}~R & \hfuzz=500pt double & \hfuzz=500pt reference radius for ellipsoidal coordinates on output\\
\hfuzz=500pt\includegraphics[width=1em]{element.pdf}~inverseFlattening & \hfuzz=500pt double & \hfuzz=500pt reference flattening for ellipsoidal coordinates on output, 0: spherical coordinates\\
\hline
\end{tabularx}

This program is \reference{parallelized}{general.parallelization}.
\clearpage
%==================================
\subsection{GravityfieldReplacePotentialCoefficients}\label{GravityfieldReplacePotentialCoefficients}
Replaces single potential coefficients in a gravity field.
Both \configClass{gravityfield}{gravityfieldType}
and \configClass{gravityfieldReplacement}{gravityfieldType} are evaluated
at \config{time} and converted to spherical harmonic coefficients.
Single \config{coefficients} are then replaced in \configClass{gravityfield}{gravityfieldType}
by the values from \configClass{gravityfieldReplacement}{gravityfieldType}
and the result is written to \configFile{outputfilePotentialCoefficients}{potentialCoefficients}
from \config{minDegree} to \config{maxDegree},


\keepXColumns
\begin{tabularx}{\textwidth}{N T A}
\hline
Name & Type & Annotation\\
\hline
\hfuzz=500pt\includegraphics[width=1em]{element-mustset.pdf}~outputfilePotentialCoefficients & \hfuzz=500pt filename & \hfuzz=500pt \\
\hfuzz=500pt\includegraphics[width=1em]{element-mustset-unbounded.pdf}~gravityfield & \hfuzz=500pt \hyperref[gravityfieldType]{gravityfield} & \hfuzz=500pt single coefficients are replaced by the other gravityfield\\
\hfuzz=500pt\includegraphics[width=1em]{element-mustset-unbounded.pdf}~gravityfieldReplacement & \hfuzz=500pt \hyperref[gravityfieldType]{gravityfield} & \hfuzz=500pt contains the coefficients for replacement\\
\hfuzz=500pt\includegraphics[width=1em]{element-mustset-unbounded.pdf}~coefficients & \hfuzz=500pt choice & \hfuzz=500pt \\
\hfuzz=500pt\includegraphics[width=1em]{connector.pdf}\includegraphics[width=1em]{element-mustset.pdf}~cnm & \hfuzz=500pt sequence & \hfuzz=500pt \\
\hfuzz=500pt\quad\includegraphics[width=1em]{connector.pdf}\includegraphics[width=1em]{element-mustset.pdf}~degree & \hfuzz=500pt uint & \hfuzz=500pt \\
\hfuzz=500pt\quad\includegraphics[width=1em]{connector.pdf}\includegraphics[width=1em]{element-mustset.pdf}~order & \hfuzz=500pt uint & \hfuzz=500pt \\
\hfuzz=500pt\includegraphics[width=1em]{connector.pdf}\includegraphics[width=1em]{element-mustset.pdf}~snm & \hfuzz=500pt sequence & \hfuzz=500pt \\
\hfuzz=500pt\quad\includegraphics[width=1em]{connector.pdf}\includegraphics[width=1em]{element-mustset.pdf}~degree & \hfuzz=500pt uint & \hfuzz=500pt \\
\hfuzz=500pt\quad\includegraphics[width=1em]{connector.pdf}\includegraphics[width=1em]{element-mustset.pdf}~order & \hfuzz=500pt uint & \hfuzz=500pt \\
\hfuzz=500pt\includegraphics[width=1em]{element.pdf}~minDegree & \hfuzz=500pt uint & \hfuzz=500pt \\
\hfuzz=500pt\includegraphics[width=1em]{element.pdf}~maxDegree & \hfuzz=500pt uint & \hfuzz=500pt \\
\hfuzz=500pt\includegraphics[width=1em]{element.pdf}~GM & \hfuzz=500pt double & \hfuzz=500pt Geocentric gravitational constant\\
\hfuzz=500pt\includegraphics[width=1em]{element.pdf}~R & \hfuzz=500pt double & \hfuzz=500pt reference radius\\
\hfuzz=500pt\includegraphics[width=1em]{element.pdf}~time & \hfuzz=500pt time & \hfuzz=500pt at this time the gravity field will be evaluated\\
\hline
\end{tabularx}

\clearpage
%==================================
\subsection{GravityfieldVariancesPropagation2GriddedData}\label{GravityfieldVariancesPropagation2GriddedData}
This program propagates variance-covariance matrix of a \configClass{gravityfield}{gravityfieldType}
evaluated at \config{time} to the points of a \configClass{grid}{gridType} in terms of the functional
given by \configClass{kernel}{kernelType}.
The resulting \file{outputfileGriddedData}{griddedData} contains the standard deviations of the grid
points.

See also \program{Gravityfield2GridCovarianceMatrix}, \program{GravityfieldCovariancesPropagation2GriddedData}.


\keepXColumns
\begin{tabularx}{\textwidth}{N T A}
\hline
Name & Type & Annotation\\
\hline
\hfuzz=500pt\includegraphics[width=1em]{element-mustset.pdf}~outputfileGriddedData & \hfuzz=500pt filename & \hfuzz=500pt standard deviation at each grid point\\
\hfuzz=500pt\includegraphics[width=1em]{element-mustset-unbounded.pdf}~grid & \hfuzz=500pt \hyperref[gridType]{grid} & \hfuzz=500pt \\
\hfuzz=500pt\includegraphics[width=1em]{element-mustset.pdf}~kernel & \hfuzz=500pt \hyperref[kernelType]{kernel} & \hfuzz=500pt functional\\
\hfuzz=500pt\includegraphics[width=1em]{element-mustset-unbounded.pdf}~gravityfield & \hfuzz=500pt \hyperref[gravityfieldType]{gravityfield} & \hfuzz=500pt \\
\hfuzz=500pt\includegraphics[width=1em]{element.pdf}~time & \hfuzz=500pt time & \hfuzz=500pt at this time the gravity field will be evaluated\\
\hfuzz=500pt\includegraphics[width=1em]{element.pdf}~R & \hfuzz=500pt double & \hfuzz=500pt reference radius for ellipsoidal coordinates on output\\
\hfuzz=500pt\includegraphics[width=1em]{element.pdf}~inverseFlattening & \hfuzz=500pt double & \hfuzz=500pt reference flattening for ellipsoidal coordinates on output, 0: spherical coordinates\\
\hline
\end{tabularx}

This program is \reference{parallelized}{general.parallelization}.
\clearpage
%==================================
\section{Programs: Grid}
\subsection{GriddedData2AreaMeanTimeSeries}\label{GriddedData2AreaMeanTimeSeries}
This program computes a time series of area mean values
in a basin represented by \configClass{border}{borderType} from a sequence of grid files.
If a file is not found, the epoch is skipped. Per default
the weighted average of all points in the given border is computed where the points are weighted by their area element.

If \config{computeMean} is set, the time average of each grid points is subtracted before the computation.
If \config{multiplyWithArea} is set, the weighted average is multiplied with the total basin area.
This is useful for computing the total mass in the basin.

The \configFile{outputfileTimeSeries}{instrument} is an instrument file, where the first columns are the
mean value each data column in the grid files, followed by the the weighted RMS
for each data column in the grid files if \config{computeRms} is set.
If the number of data columns differs between the grid files, the remaining columns are padded with zeros.

See also \program{Gravityfield2AreaMeanTimeSeries}.


\keepXColumns
\begin{tabularx}{\textwidth}{N T A}
\hline
Name & Type & Annotation\\
\hline
\hfuzz=500pt\includegraphics[width=1em]{element-mustset.pdf}~outputfileTimeSeries & \hfuzz=500pt filename & \hfuzz=500pt \\
\hfuzz=500pt\includegraphics[width=1em]{element-mustset-unbounded.pdf}~inputfileGriddedData & \hfuzz=500pt filename & \hfuzz=500pt \\
\hfuzz=500pt\includegraphics[width=1em]{element-mustset-unbounded.pdf}~border & \hfuzz=500pt \hyperref[borderType]{border} & \hfuzz=500pt \\
\hfuzz=500pt\includegraphics[width=1em]{element-mustset-unbounded.pdf}~timeSeries & \hfuzz=500pt \hyperref[timeSeriesType]{timeSeries} & \hfuzz=500pt \\
\hfuzz=500pt\includegraphics[width=1em]{element.pdf}~multiplyWithArea & \hfuzz=500pt boolean & \hfuzz=500pt multiply time series with total area (useful for mass estimates)\\
\hfuzz=500pt\includegraphics[width=1em]{element.pdf}~removeMean & \hfuzz=500pt boolean & \hfuzz=500pt remove the temporal mean of the series\\
\hfuzz=500pt\includegraphics[width=1em]{element.pdf}~computeRms & \hfuzz=500pt boolean & \hfuzz=500pt additional rms each time step\\
\hline
\end{tabularx}

\clearpage
%==================================
\subsection{GriddedData2GriddedDataStatistics}\label{GriddedData2GriddedDataStatistics}
This program assigns values \configFile{inputfileGriddedData}{griddedData} to the nearest points
of a new \configClass{grid}{gridType}. If some of the new points are not filled in with data
\config{emptyValue} is used instead. If multiple points of the input fall on the same node
the result can be selected with \config{statistics} (e.g. mean, root mean square, min, max, \ldots).
It also is possible to simply count the number of data points that were assigned to each point.

Be aware in case borders are given within \configClass{grid}{gridType}, the \configFile{outputfileGriddedData}{griddedData} will have points excluded before the assignement of old points to the new points.
The data from \configFile{inputfileGriddedData}{griddedData} will not be limited by the given borders! See \reference{GriddedDataConcatenate}{GriddedDataConcatenate} to limit the
\configFile{inputfileGriddedData}{griddedData} to given borders.

\fig{!hb}{0.8}{griddedData2GriddedDataStatistics}{fig:griddedData2GriddedDataStatistics}{Assignement of irregular distributed data to grid.}


\keepXColumns
\begin{tabularx}{\textwidth}{N T A}
\hline
Name & Type & Annotation\\
\hline
\hfuzz=500pt\includegraphics[width=1em]{element-mustset.pdf}~outputfileGriddedData & \hfuzz=500pt filename & \hfuzz=500pt \\
\hfuzz=500pt\includegraphics[width=1em]{element-mustset.pdf}~inputfileGriddedData & \hfuzz=500pt filename & \hfuzz=500pt \\
\hfuzz=500pt\includegraphics[width=1em]{element-mustset-unbounded.pdf}~grid & \hfuzz=500pt \hyperref[gridType]{grid} & \hfuzz=500pt \\
\hfuzz=500pt\includegraphics[width=1em]{element-mustset.pdf}~statistic & \hfuzz=500pt choice & \hfuzz=500pt statistic used if multiple values fall on the same cell\\
\hfuzz=500pt\includegraphics[width=1em]{connector.pdf}\includegraphics[width=1em]{element-mustset.pdf}~mean & \hfuzz=500pt  & \hfuzz=500pt mean\\
\hfuzz=500pt\includegraphics[width=1em]{connector.pdf}\includegraphics[width=1em]{element-mustset.pdf}~wmean & \hfuzz=500pt  & \hfuzz=500pt area weighted mean\\
\hfuzz=500pt\includegraphics[width=1em]{connector.pdf}\includegraphics[width=1em]{element-mustset.pdf}~rms & \hfuzz=500pt  & \hfuzz=500pt root mean square\\
\hfuzz=500pt\includegraphics[width=1em]{connector.pdf}\includegraphics[width=1em]{element-mustset.pdf}~wrms & \hfuzz=500pt  & \hfuzz=500pt area weighted root mean square\\
\hfuzz=500pt\includegraphics[width=1em]{connector.pdf}\includegraphics[width=1em]{element-mustset.pdf}~std & \hfuzz=500pt  & \hfuzz=500pt standard deviation\\
\hfuzz=500pt\includegraphics[width=1em]{connector.pdf}\includegraphics[width=1em]{element-mustset.pdf}~wstd & \hfuzz=500pt  & \hfuzz=500pt area weighted standard deviation\\
\hfuzz=500pt\includegraphics[width=1em]{connector.pdf}\includegraphics[width=1em]{element-mustset.pdf}~sum & \hfuzz=500pt  & \hfuzz=500pt sum\\
\hfuzz=500pt\includegraphics[width=1em]{connector.pdf}\includegraphics[width=1em]{element-mustset.pdf}~min & \hfuzz=500pt  & \hfuzz=500pt minimum value\\
\hfuzz=500pt\includegraphics[width=1em]{connector.pdf}\includegraphics[width=1em]{element-mustset.pdf}~max & \hfuzz=500pt  & \hfuzz=500pt maximum value\\
\hfuzz=500pt\includegraphics[width=1em]{connector.pdf}\includegraphics[width=1em]{element-mustset.pdf}~count & \hfuzz=500pt  & \hfuzz=500pt number of values\\
\hfuzz=500pt\includegraphics[width=1em]{connector.pdf}\includegraphics[width=1em]{element-mustset.pdf}~first & \hfuzz=500pt  & \hfuzz=500pt first value\\
\hfuzz=500pt\includegraphics[width=1em]{connector.pdf}\includegraphics[width=1em]{element-mustset.pdf}~last & \hfuzz=500pt  & \hfuzz=500pt last value\\
\hfuzz=500pt\includegraphics[width=1em]{element.pdf}~emptyValue & \hfuzz=500pt double & \hfuzz=500pt value for nodes without data\\
\hfuzz=500pt\includegraphics[width=1em]{element.pdf}~R & \hfuzz=500pt double & \hfuzz=500pt reference radius for ellipsoidal coordinates\\
\hfuzz=500pt\includegraphics[width=1em]{element.pdf}~inverseFlattening & \hfuzz=500pt double & \hfuzz=500pt reference flattening for ellipsoidal coordinates\\
\hline
\end{tabularx}

\clearpage
%==================================
\subsection{GriddedData2GriddedDataTimeSeries}\label{GriddedData2GriddedDataTimeSeries}
Write a series of \configFile{inputfileGriddedData}{griddedData}
with the corresponding \configClass{timeSeries}{timeSeriesType}
as a single \file{gridded data time series file}{griddedDataTimeSeries}.
The \config{splineDegree} defines the possible temporal interpolation of data in the output file.
For a file with spline degree 0 (temporal block means) the time intervals
in which the grids are valid are defined between adjacent points in time.
Therefore one more point in time is needed than the number of input grid files for degree 0.

See also \program{GriddedDataTimeSeries2GriddedData}.


\keepXColumns
\begin{tabularx}{\textwidth}{N T A}
\hline
Name & Type & Annotation\\
\hline
\hfuzz=500pt\includegraphics[width=1em]{element-mustset.pdf}~outputfileGriddedDataTimeSeries & \hfuzz=500pt filename & \hfuzz=500pt \\
\hfuzz=500pt\includegraphics[width=1em]{element-mustset-unbounded.pdf}~inputfileGriddedData & \hfuzz=500pt filename & \hfuzz=500pt file count must agree with number of times+splineDegre-1\\
\hfuzz=500pt\includegraphics[width=1em]{element-mustset-unbounded.pdf}~timeSeries & \hfuzz=500pt \hyperref[timeSeriesType]{timeSeries} & \hfuzz=500pt \\
\hfuzz=500pt\includegraphics[width=1em]{element.pdf}~splineDegree & \hfuzz=500pt uint & \hfuzz=500pt degree of splines\\
\hline
\end{tabularx}

\clearpage
%==================================
\subsection{GriddedData2Matrix}\label{GriddedData2Matrix}
This program converts \configFile{inputfileGriddedData}{griddedData}
to \configFile{outputfileMatrix}{matrix} with data columns.
The grid is expressed as ellipsoidal coordinates (longitude, latitude, height)
based on a reference ellipsoid with parameters \config{R} and \config{inverseFlattening}.
The content of the output matrix can be controlled by \config{outColumn} expressions
applied to every grid point. The common data variables for grids are available,
see \reference{dataVariables}{general.parser:dataVariables}.


\keepXColumns
\begin{tabularx}{\textwidth}{N T A}
\hline
Name & Type & Annotation\\
\hline
\hfuzz=500pt\includegraphics[width=1em]{element-mustset.pdf}~outputfileMatrix & \hfuzz=500pt filename & \hfuzz=500pt point list as matrix with longitude and latitude values in columns and possible additional columns\\
\hfuzz=500pt\includegraphics[width=1em]{element-mustset.pdf}~inputfileGriddedData & \hfuzz=500pt filename & \hfuzz=500pt \\
\hfuzz=500pt\includegraphics[width=1em]{element.pdf}~R & \hfuzz=500pt double & \hfuzz=500pt reference radius for ellipsoidal coordinates on output\\
\hfuzz=500pt\includegraphics[width=1em]{element.pdf}~inverseFlattening & \hfuzz=500pt double & \hfuzz=500pt reference flattening for ellipsoidal coordinates on output, 0: spherical coordinates\\
\hfuzz=500pt\includegraphics[width=1em]{element.pdf}~outColumn & \hfuzz=500pt expression & \hfuzz=500pt expression (variables: longitude, latitude, height, area, data0, data1, ...)\\
\hfuzz=500pt\includegraphics[width=1em]{element.pdf}~outColumn & \hfuzz=500pt expression & \hfuzz=500pt expression (variables: longitude, latitude, height, area, data0, data1, ...)\\
\hfuzz=500pt\includegraphics[width=1em]{element.pdf}~outColumn & \hfuzz=500pt expression & \hfuzz=500pt expression (variables: longitude, latitude, height, area, data0, data1, ...)\\
\hfuzz=500pt\includegraphics[width=1em]{element-unbounded.pdf}~outColumn & \hfuzz=500pt expression & \hfuzz=500pt expression (variables: longitude, latitude, height, area, data0, data1, ...)\\
\hline
\end{tabularx}

\clearpage
%==================================
\subsection{GriddedData2PotentialCoefficients}\label{GriddedData2PotentialCoefficients}
This program estimate potential coefficients from \configFile{inputfileGriddedData}{griddedData}
gravity field functionals. It used a simple quadrature formular
\begin{equation}
  c_{nm} = \frac{1}{4\pi}\frac{R}{GM} \sum_i f_i \left(\frac{r_i}{R}\right)^{n+1} k_n C_{nm}(\lambda_i,\vartheta_i)\,\Delta\Phi_i
\end{equation}
or a \config{leastSquares} adjustment with block diagonal normal matrix (order by order).
For the latter one the data must be regular distributed.

The \config{value}s $f_i$ and the \config{weight}s $\Delta\Phi_i$ are expressions
using the common data variables for grids, see \reference{dataVariables}{general.parser:dataVariables}.
The type of the gridded data (e.g gravity anomalies or geoid heights)
must be set with \configClass{kernel}{kernelType} $k_n$.

The expansion is limited in the range between \config{minDegree}
and \config{maxDegree} inclusively. The coefficients are related
to the reference radius~\config{R} and the Earth gravitational constant \config{GM}.

For irregular distributed data and using the full variance covariance matrix use
\program{NormalsSolverVCE} together with \configClass{oberservation:terrestrial}{observationType:terrestrial}
and \configClass{parametrizationGravity:sphericalHarmonics}{parametrizationGravityType:sphericalHarmonics}.


\keepXColumns
\begin{tabularx}{\textwidth}{N T A}
\hline
Name & Type & Annotation\\
\hline
\hfuzz=500pt\includegraphics[width=1em]{element-mustset.pdf}~outputfilePotentialCoefficients & \hfuzz=500pt filename & \hfuzz=500pt \\
\hfuzz=500pt\includegraphics[width=1em]{element-mustset.pdf}~inputfileGriddedData & \hfuzz=500pt filename & \hfuzz=500pt \\
\hfuzz=500pt\includegraphics[width=1em]{element-mustset.pdf}~value & \hfuzz=500pt expression & \hfuzz=500pt expression to compute values (input columns are named data0, data1, ...)\\
\hfuzz=500pt\includegraphics[width=1em]{element-mustset.pdf}~weight & \hfuzz=500pt expression & \hfuzz=500pt expression to compute values (input columns are named data0, data1, ...)\\
\hfuzz=500pt\includegraphics[width=1em]{element-mustset.pdf}~kernel & \hfuzz=500pt \hyperref[kernelType]{kernel} & \hfuzz=500pt data type of input values\\
\hfuzz=500pt\includegraphics[width=1em]{element.pdf}~minDegree & \hfuzz=500pt uint & \hfuzz=500pt \\
\hfuzz=500pt\includegraphics[width=1em]{element-mustset.pdf}~maxDegree & \hfuzz=500pt uint & \hfuzz=500pt \\
\hfuzz=500pt\includegraphics[width=1em]{element.pdf}~GM & \hfuzz=500pt double & \hfuzz=500pt Geocentric gravitational constant\\
\hfuzz=500pt\includegraphics[width=1em]{element.pdf}~R & \hfuzz=500pt double & \hfuzz=500pt reference radius\\
\hfuzz=500pt\includegraphics[width=1em]{element.pdf}~leastSquares & \hfuzz=500pt boolean & \hfuzz=500pt false: quadrature formular, true: least squares adjustment order by order\\
\hline
\end{tabularx}

This program is \reference{parallelized}{general.parallelization}.
\clearpage
%==================================
\subsection{GriddedData2SphericalDistance}\label{GriddedData2SphericalDistance}
Compute the spherical distance on the unit sphere in radians between all point pairs of two grids.
The spherical distance is computed by
\begin{equation}
  \psi_{12} = \arccos(\M n_1 \cdot \M n_2),
\end{equation}
where $\M n_i$ is the (normalized) position. This implies that all points are projected onto the unit sphere.


\keepXColumns
\begin{tabularx}{\textwidth}{N T A}
\hline
Name & Type & Annotation\\
\hline
\hfuzz=500pt\includegraphics[width=1em]{element-mustset.pdf}~outputfileMatrix & \hfuzz=500pt filename & \hfuzz=500pt matrix containing the spherical distance between all point pairs [rad]\\
\hfuzz=500pt\includegraphics[width=1em]{element-mustset-unbounded.pdf}~grid1 & \hfuzz=500pt \hyperref[gridType]{grid} & \hfuzz=500pt \\
\hfuzz=500pt\includegraphics[width=1em]{element-mustset-unbounded.pdf}~grid2 & \hfuzz=500pt \hyperref[gridType]{grid} & \hfuzz=500pt \\
\hline
\end{tabularx}

\clearpage
%==================================
\subsection{GriddedData2TimeSeries}\label{GriddedData2TimeSeries}
Write a series of \configFile{inputfileGriddedData}{griddedData} with the corresponding
\configClass{timeSeries}{timeSeriesType} as a single time series file
(\file{instrument}{instrument}, MISCVALUES).

If \config{groupDataByPoints} is true the \config{outputfileTimeSeries} starts
for each epoch with all data (\verb|data0|, \verb|data1|\ldots) for the first point,
followed by all data of the second point and so on.
If \config{groupDataByPoints} is false, the file starts with \verb|data0|
for all points, followed by all \verb|data1| and so on.

This enables the use of all instrument programs like \program{InstrumentFilter} or
\program{InstrumentDetrend} to analyze and manipulate time series of gridded data.

See also \program{TimeSeries2GriddedData}.


\keepXColumns
\begin{tabularx}{\textwidth}{N T A}
\hline
Name & Type & Annotation\\
\hline
\hfuzz=500pt\includegraphics[width=1em]{element-mustset.pdf}~outputfileTimeSeries & \hfuzz=500pt filename & \hfuzz=500pt each epoch: multiple data for points (MISCVALUES)\\
\hfuzz=500pt\includegraphics[width=1em]{element-mustset-unbounded.pdf}~inputfileGriddedData & \hfuzz=500pt filename & \hfuzz=500pt file count must agree with number of times\\
\hfuzz=500pt\includegraphics[width=1em]{element-mustset-unbounded.pdf}~timeSeries & \hfuzz=500pt \hyperref[timeSeriesType]{timeSeries} & \hfuzz=500pt \\
\hfuzz=500pt\includegraphics[width=1em]{element.pdf}~groupDataByPoints & \hfuzz=500pt boolean & \hfuzz=500pt multiple data are given point by point, otherwise: data0 for all points, followed by all data1\\
\hline
\end{tabularx}

\clearpage
%==================================
\subsection{GriddedDataCalculate}\label{GriddedDataCalculate}
This program manipulates \file{grid files}{griddedData} with data in columns similar to
\program{FunctionsCalculate}, see there for more details.
If several \config{inputfile}s are given the data columns are copied side by side.
All \config{inputfile}s must contain the same grid points.
The columns are enumerated by \verb|data0|,~\verb|data1|,~\ldots.

The content of \configFile{outputfileGriddedData}{griddedData} is controlled by \config{outColumn}.
The algorithm to compute the output is as follows:
The expressions in \config{outColumn} are evaluated once for each grid point of the input.
The variables \verb|data0|,~\verb|data1|,~\ldots are replaced by the according values
from the input columns before.
Additional variables are available, e.g. \verb|index|, \verb|data0rms|,
see~\reference{dataVariables}{general.parser:dataVariables}.

For a simplified handling \config{constant}s can be defined by \verb|name=value|.
It is also possible to estimate \config{parameter}s in a least squares adjustment.
The \config{leastSquares} serves as template for observation equations for every point.
The expression \config{leastSquares} is evaluated for each grid point.
The variables \verb|data0|,~\verb|data1|,~\ldots are replaced by the according values from the input columns before.
In the next step the parameters are estimated in order to minimize the expressions in \config{leastSquares}
in the sense of least squares.

Afterwards grid points are removed if one of the \config{removalCriteria} expressions
for this grid point evaluates true (not zero).

An extra \configFile{statistics:outputfile}{matrix} can be generated with one row of data.
For the computation of the \config{outColumn} values
all~\reference{dataVariables}{general.parser:dataVariables} are available
(e.g. \verb|data3mean|, \verb|data4std|) inclusively the \config{constant}s and
estimated \config{parameter}s but without the \verb|data0|,~\verb|data1|,~\ldots itself.
The variables and the numbering of the columns refers to the \configFile{outputfileGriddedData}{griddedData}.

See also \program{FunctionsCalculate}, \program{InstrumentArcCalculate}, \program{MatrixCalculate}.


\keepXColumns
\begin{tabularx}{\textwidth}{N T A}
\hline
Name & Type & Annotation\\
\hline
\hfuzz=500pt\includegraphics[width=1em]{element.pdf}~outputfileGriddedData & \hfuzz=500pt filename & \hfuzz=500pt \\
\hfuzz=500pt\includegraphics[width=1em]{element-mustset-unbounded.pdf}~inputfileGriddedData & \hfuzz=500pt filename & \hfuzz=500pt \\
\hfuzz=500pt\includegraphics[width=1em]{element-unbounded.pdf}~constant & \hfuzz=500pt expression & \hfuzz=500pt define a constant by name=value\\
\hfuzz=500pt\includegraphics[width=1em]{element-unbounded.pdf}~parameter & \hfuzz=500pt expression & \hfuzz=500pt define a parameter by name[=value]\\
\hfuzz=500pt\includegraphics[width=1em]{element-unbounded.pdf}~leastSquares & \hfuzz=500pt expression & \hfuzz=500pt try to minimize the expression by adjustment of the parameters\\
\hfuzz=500pt\includegraphics[width=1em]{element-unbounded.pdf}~removalCriteria & \hfuzz=500pt expression & \hfuzz=500pt points are removed if one criterion evaluates true. data0 is the first data field.\\
\hfuzz=500pt\includegraphics[width=1em]{element-mustset.pdf}~longitude & \hfuzz=500pt expression & \hfuzz=500pt expression\\
\hfuzz=500pt\includegraphics[width=1em]{element-mustset.pdf}~latitude & \hfuzz=500pt expression & \hfuzz=500pt expression\\
\hfuzz=500pt\includegraphics[width=1em]{element-mustset.pdf}~height & \hfuzz=500pt expression & \hfuzz=500pt expression\\
\hfuzz=500pt\includegraphics[width=1em]{element.pdf}~area & \hfuzz=500pt expression & \hfuzz=500pt expression: e.g. deltaL * 2.0 * sin(deltaB/2.0) * cos(latitude/rho)\\
\hfuzz=500pt\includegraphics[width=1em]{element-unbounded.pdf}~value & \hfuzz=500pt expression & \hfuzz=500pt expression to compute values (input columns are named data0, data1, ...)\\
\hfuzz=500pt\includegraphics[width=1em]{element.pdf}~computeArea & \hfuzz=500pt boolean & \hfuzz=500pt automatically area computation of rectangular grids (overwrite area)\\
\hfuzz=500pt\includegraphics[width=1em]{element.pdf}~R & \hfuzz=500pt double & \hfuzz=500pt reference radius for ellipsoidal coordinates\\
\hfuzz=500pt\includegraphics[width=1em]{element.pdf}~inverseFlattening & \hfuzz=500pt double & \hfuzz=500pt reference flattening for ellipsoidal coordinates\\
\hfuzz=500pt\includegraphics[width=1em]{element.pdf}~statistics & \hfuzz=500pt sequence & \hfuzz=500pt \\
\hfuzz=500pt\includegraphics[width=1em]{connector.pdf}\includegraphics[width=1em]{element-mustset.pdf}~outputfile & \hfuzz=500pt filename & \hfuzz=500pt matrix with one row, columns are user defined\\
\hfuzz=500pt\includegraphics[width=1em]{connector.pdf}\includegraphics[width=1em]{element-mustset-unbounded.pdf}~outColumn & \hfuzz=500pt expression & \hfuzz=500pt expression to compute statistics columns, data* are the outputColumns\\
\hline
\end{tabularx}

\clearpage
%==================================
\subsection{GriddedDataConcatenate}\label{GriddedDataConcatenate}
This program concatenate grid from several \configFile{inputfileGriddedData}{griddedData}
and write it to a new \configFile{outputfileGriddedData}{griddedData}.
Input files must have the same number of data columns.
If \config{sort} is enabled, the points are sorted by latitudes starting from north/west to south east.
Identical points (within a \config{margin}) can be removed with \config{removeDuplicates}.


\keepXColumns
\begin{tabularx}{\textwidth}{N T A}
\hline
Name & Type & Annotation\\
\hline
\hfuzz=500pt\includegraphics[width=1em]{element-mustset.pdf}~outputfileGriddedData & \hfuzz=500pt filename & \hfuzz=500pt \\
\hfuzz=500pt\includegraphics[width=1em]{element-mustset-unbounded.pdf}~inputfileGriddedData & \hfuzz=500pt filename & \hfuzz=500pt \\
\hfuzz=500pt\includegraphics[width=1em]{element-unbounded.pdf}~border & \hfuzz=500pt \hyperref[borderType]{border} & \hfuzz=500pt \\
\hfuzz=500pt\includegraphics[width=1em]{element.pdf}~sortPoints & \hfuzz=500pt boolean & \hfuzz=500pt sort from north/west to south east\\
\hfuzz=500pt\includegraphics[width=1em]{element.pdf}~removeDuplicates & \hfuzz=500pt choice & \hfuzz=500pt remove duplicate points\\
\hfuzz=500pt\includegraphics[width=1em]{connector.pdf}\includegraphics[width=1em]{element-mustset.pdf}~keepFirst & \hfuzz=500pt sequence & \hfuzz=500pt keep first point, remove all other identicals\\
\hfuzz=500pt\quad\includegraphics[width=1em]{connector.pdf}\includegraphics[width=1em]{element.pdf}~margin & \hfuzz=500pt double & \hfuzz=500pt margin distance for identical points [m]\\
\hfuzz=500pt\includegraphics[width=1em]{connector.pdf}\includegraphics[width=1em]{element-mustset.pdf}~keepLast & \hfuzz=500pt sequence & \hfuzz=500pt keep last point, remove all other identicals\\
\hfuzz=500pt\quad\includegraphics[width=1em]{connector.pdf}\includegraphics[width=1em]{element.pdf}~margin & \hfuzz=500pt double & \hfuzz=500pt margin distance for identical points [m]\\
\hfuzz=500pt\includegraphics[width=1em]{element.pdf}~R & \hfuzz=500pt double & \hfuzz=500pt reference radius for ellipsoidal coordinates\\
\hfuzz=500pt\includegraphics[width=1em]{element.pdf}~inverseFlattening & \hfuzz=500pt double & \hfuzz=500pt reference flattening for ellipsoidal coordinates\\
\hline
\end{tabularx}

\clearpage
%==================================
\subsection{GriddedDataCreate}\label{GriddedDataCreate}
This program creates a \configClass{grid}{gridType} and writes it to \configFile{outputfileGrid}{griddedData}.
The grid is expressed as ellipsoidal coordinates (longitude, latitude, height)
based on a reference ellipsoid with parameters \config{R} and \config{inverseFlattening}.
Extra \config{value} columns can be appended using expressions
with the common \reference{data variables}{general.parser:dataVariables} for gridded data.


\keepXColumns
\begin{tabularx}{\textwidth}{N T A}
\hline
Name & Type & Annotation\\
\hline
\hfuzz=500pt\includegraphics[width=1em]{element-mustset.pdf}~outputfileGrid & \hfuzz=500pt filename & \hfuzz=500pt \\
\hfuzz=500pt\includegraphics[width=1em]{element-mustset-unbounded.pdf}~grid & \hfuzz=500pt \hyperref[gridType]{grid} & \hfuzz=500pt \\
\hfuzz=500pt\includegraphics[width=1em]{element.pdf}~R & \hfuzz=500pt double & \hfuzz=500pt reference radius for ellipsoidal coordinates on output\\
\hfuzz=500pt\includegraphics[width=1em]{element.pdf}~inverseFlattening & \hfuzz=500pt double & \hfuzz=500pt reference flattening for ellipsoidal coordinates on output, 0: spherical coordinates\\
\hfuzz=500pt\includegraphics[width=1em]{element-unbounded.pdf}~value & \hfuzz=500pt expression & \hfuzz=500pt expression (variables as 'longitude', 'height', 'area' are taken from the gridded data)\\
\hline
\end{tabularx}

\clearpage
%==================================
\subsection{GriddedDataInterpolate}\label{GriddedDataInterpolate}
Interpolate values of a regular rectangular \configFile{inputfileGriddedData}{griddedData}
to new points given by \configClass{grid}{gridType} and write as \configFile{outputfileGriddedData}{griddedData}.
Only longitude and latitude of points are considered; the height is ignored for interpolation.

(Only nearest neighbor method is implemented at the moment.)

\fig{!hb}{0.8}{griddedDataInterpolate}{fig:griddedDataInterpolate}{Interpolation of point data from rectangular gridded data.}


\keepXColumns
\begin{tabularx}{\textwidth}{N T A}
\hline
Name & Type & Annotation\\
\hline
\hfuzz=500pt\includegraphics[width=1em]{element-mustset.pdf}~outputfileGriddedData & \hfuzz=500pt filename & \hfuzz=500pt \\
\hfuzz=500pt\includegraphics[width=1em]{element-mustset.pdf}~inputfileGriddedData & \hfuzz=500pt filename & \hfuzz=500pt must be rectangular\\
\hfuzz=500pt\includegraphics[width=1em]{element-mustset-unbounded.pdf}~grid & \hfuzz=500pt \hyperref[gridType]{grid} & \hfuzz=500pt \\
\hfuzz=500pt\includegraphics[width=1em]{element-mustset.pdf}~method & \hfuzz=500pt choice & \hfuzz=500pt \\
\hfuzz=500pt\includegraphics[width=1em]{connector.pdf}\includegraphics[width=1em]{element-mustset.pdf}~nearestNeighbor & \hfuzz=500pt  & \hfuzz=500pt \\
\hline
\end{tabularx}

\clearpage
%==================================
\subsection{GriddedDataReduceSampling}\label{GriddedDataReduceSampling}
Generate coarse grid by computing mean values.
The number of points is decimated by averaging integer multiplies of grid points
(\config{multiplierLongitude}, \config{multiplierLatitude}).
The fine grid can be written, where the coarse grid values are additionally appended.


\keepXColumns
\begin{tabularx}{\textwidth}{N T A}
\hline
Name & Type & Annotation\\
\hline
\hfuzz=500pt\includegraphics[width=1em]{element.pdf}~outputfileCoarseGridRectangular & \hfuzz=500pt filename & \hfuzz=500pt coarse grid\\
\hfuzz=500pt\includegraphics[width=1em]{element.pdf}~outputfileFineGridRectangular & \hfuzz=500pt filename & \hfuzz=500pt fine grid with additional coarse grid values\\
\hfuzz=500pt\includegraphics[width=1em]{element-mustset.pdf}~inputfileFineGridRectangular & \hfuzz=500pt filename & \hfuzz=500pt Digital Terrain Model\\
\hfuzz=500pt\includegraphics[width=1em]{element-mustset.pdf}~multiplierLongitude & \hfuzz=500pt uint & \hfuzz=500pt Generalizing factor\\
\hfuzz=500pt\includegraphics[width=1em]{element-mustset.pdf}~multiplierLatitude & \hfuzz=500pt uint & \hfuzz=500pt Generalizing factor\\
\hline
\end{tabularx}

\clearpage
%==================================
\subsection{GriddedDataTimeSeries2GriddedData}\label{GriddedDataTimeSeries2GriddedData}
Read a \configFile{inputfileGriddedDataTimeSeries}{griddedDataTimeSeries}
and write for each epoch a \file{gridded data file}{griddedData} where
the \config{variableLoopTime} and \config{variableLoopIndex} are expanded for
each point of the given \configClass{timeSeries}{timeSeriesType}
to create the file name for this epoch (see \reference{text parser}{general.parser:text}).

If \configClass{timeSeries}{timeSeriesType} is not set
the temporal nodal points from the inputfile are used.

See also \program{GriddedData2GriddedDataTimeSeries}.


\keepXColumns
\begin{tabularx}{\textwidth}{N T A}
\hline
Name & Type & Annotation\\
\hline
\hfuzz=500pt\includegraphics[width=1em]{element-mustset.pdf}~outputfilesGriddedData & \hfuzz=500pt filename & \hfuzz=500pt for each epoch\\
\hfuzz=500pt\includegraphics[width=1em]{element.pdf}~variableLoopTime & \hfuzz=500pt string & \hfuzz=500pt variable with time of each epoch\\
\hfuzz=500pt\includegraphics[width=1em]{element.pdf}~variableLoopIndex & \hfuzz=500pt string & \hfuzz=500pt variable with index of current epoch (starts with zero)\\
\hfuzz=500pt\includegraphics[width=1em]{element.pdf}~variableLoopCount & \hfuzz=500pt string & \hfuzz=500pt variable with total number of epochs\\
\hfuzz=500pt\includegraphics[width=1em]{element-mustset.pdf}~inputfileGriddedDataTimeSeries & \hfuzz=500pt filename & \hfuzz=500pt \\
\hfuzz=500pt\includegraphics[width=1em]{element-unbounded.pdf}~timeSeries & \hfuzz=500pt \hyperref[timeSeriesType]{timeSeries} & \hfuzz=500pt otherwise times from inputfile are used\\
\hline
\end{tabularx}

\clearpage
%==================================
\subsection{GriddedTopography2AtmospherePotentialCoefficients}\label{GriddedTopography2AtmospherePotentialCoefficients}
Estimate interior and exterior potential coefficients for atmosphere above digital terrain models.
Coefficients for interior $(1/r)^{n+1}$ and exterior ($r^n$) are computed.
The density of the atmosphere is assumed to be (Sjöberg, 1998)
\begin{equation}
\rho_0\left(\frac{R}{R+h}\right)^\nu,
\end{equation}
where $R$ is the radial distance of the ellipsoid at each point, $h$ the radial height above the ellipsoid,
$\rho_0$ is \config{densitySeaLevel} and \config{nu} $\nu$ is a constant factor. The density is integrated
from \config{radialLowerBound} and \config{upperAtmosphericBoundary} above the ellipsoid.
The \config{radialLowerBound} is typically the topography and can be computed as expression at every point
from \configFile{inputfileGriddedData}{griddedData}.


\keepXColumns
\begin{tabularx}{\textwidth}{N T A}
\hline
Name & Type & Annotation\\
\hline
\hfuzz=500pt\includegraphics[width=1em]{element-mustset.pdf}~outputfilePotentialCoefficientsExterior & \hfuzz=500pt filename & \hfuzz=500pt \\
\hfuzz=500pt\includegraphics[width=1em]{element-mustset.pdf}~outputfilePotentialCoefficientsInterior & \hfuzz=500pt filename & \hfuzz=500pt \\
\hfuzz=500pt\includegraphics[width=1em]{element-mustset.pdf}~inputfileGriddedData & \hfuzz=500pt filename & \hfuzz=500pt Digital Terrain Model\\
\hfuzz=500pt\includegraphics[width=1em]{element.pdf}~densitySeaLevel & \hfuzz=500pt double & \hfuzz=500pt [kg/m**3]\\
\hfuzz=500pt\includegraphics[width=1em]{element.pdf}~ny & \hfuzz=500pt double & \hfuzz=500pt Constant for Atmosphere\\
\hfuzz=500pt\includegraphics[width=1em]{element.pdf}~radialLowerBound & \hfuzz=500pt expression & \hfuzz=500pt expression (variables 'L', 'B', 'height', 'data', and 'area' are taken from the gridded data\\
\hfuzz=500pt\includegraphics[width=1em]{element.pdf}~upperAtmosphericBoundary & \hfuzz=500pt double & \hfuzz=500pt constant upper bound [m]\\
\hfuzz=500pt\includegraphics[width=1em]{element.pdf}~factor & \hfuzz=500pt double & \hfuzz=500pt the result is multiplied by this factor, set -1 to subtract the field\\
\hfuzz=500pt\includegraphics[width=1em]{element.pdf}~minDegree & \hfuzz=500pt uint & \hfuzz=500pt \\
\hfuzz=500pt\includegraphics[width=1em]{element-mustset.pdf}~maxDegree & \hfuzz=500pt uint & \hfuzz=500pt \\
\hfuzz=500pt\includegraphics[width=1em]{element.pdf}~GM & \hfuzz=500pt double & \hfuzz=500pt Geocentric gravitational constant\\
\hfuzz=500pt\includegraphics[width=1em]{element.pdf}~R & \hfuzz=500pt double & \hfuzz=500pt reference radius\\
\hline
\end{tabularx}

This program is \reference{parallelized}{general.parallelization}.
\clearpage
%==================================
\subsection{GriddedTopography2PotentialCoefficients}\label{GriddedTopography2PotentialCoefficients}
Estimate potential coefficients from digital terrain models.
Coefficients for interior $(1/r)^{n+1}$ and exterior ($r^n$) are computed.


\keepXColumns
\begin{tabularx}{\textwidth}{N T A}
\hline
Name & Type & Annotation\\
\hline
\hfuzz=500pt\includegraphics[width=1em]{element.pdf}~outputfilePotentialCoefficients & \hfuzz=500pt filename & \hfuzz=500pt \\
\hfuzz=500pt\includegraphics[width=1em]{element.pdf}~outputfilePotentialCoefficientsInterior & \hfuzz=500pt filename & \hfuzz=500pt \\
\hfuzz=500pt\includegraphics[width=1em]{element-mustset.pdf}~inputfileGriddedData & \hfuzz=500pt filename & \hfuzz=500pt Digital Terrain Model\\
\hfuzz=500pt\includegraphics[width=1em]{element.pdf}~density & \hfuzz=500pt expression & \hfuzz=500pt expression [kg/m**3]\\
\hfuzz=500pt\includegraphics[width=1em]{element.pdf}~radialUpperBound & \hfuzz=500pt expression & \hfuzz=500pt expression (variables 'L', 'B', 'height', 'data', and 'area' are taken from the gridded data\\
\hfuzz=500pt\includegraphics[width=1em]{element.pdf}~radialLowerBound & \hfuzz=500pt expression & \hfuzz=500pt expression (variables 'L', 'B', 'height', 'data', and 'area' are taken from the gridded data\\
\hfuzz=500pt\includegraphics[width=1em]{element.pdf}~factor & \hfuzz=500pt double & \hfuzz=500pt the result is multiplied by this factor\\
\hfuzz=500pt\includegraphics[width=1em]{element.pdf}~minDegree & \hfuzz=500pt uint & \hfuzz=500pt \\
\hfuzz=500pt\includegraphics[width=1em]{element-mustset.pdf}~maxDegree & \hfuzz=500pt uint & \hfuzz=500pt \\
\hfuzz=500pt\includegraphics[width=1em]{element.pdf}~GM & \hfuzz=500pt double & \hfuzz=500pt Geocentric gravitational constant\\
\hfuzz=500pt\includegraphics[width=1em]{element.pdf}~R & \hfuzz=500pt double & \hfuzz=500pt reference radius\\
\hline
\end{tabularx}

This program is \reference{parallelized}{general.parallelization}.
\clearpage
%==================================
\subsection{GriddedTopographyEllipsoidal2Radial}\label{GriddedTopographyEllipsoidal2Radial}
Interpolate digital terrain models from ellipoidal heights to radial heights.


\keepXColumns
\begin{tabularx}{\textwidth}{N T A}
\hline
Name & Type & Annotation\\
\hline
\hfuzz=500pt\includegraphics[width=1em]{element-mustset.pdf}~outputfileGriddedData & \hfuzz=500pt filename & \hfuzz=500pt \\
\hfuzz=500pt\includegraphics[width=1em]{element-mustset.pdf}~inputfileGriddedData & \hfuzz=500pt filename & \hfuzz=500pt Digital Terrain Model\\
\hline
\end{tabularx}

\clearpage
%==================================
\subsection{Matrix2GriddedData}\label{Matrix2GriddedData}
This program reads a \file{matrix file}{matrix} with data in columns
and convert into \file{gridded data}{griddedData}.
The input columns are enumerated by \verb|data0|,~\verb|data1|,~\ldots,
see~\reference{dataVariables}{general.parser:dataVariables}.


\keepXColumns
\begin{tabularx}{\textwidth}{N T A}
\hline
Name & Type & Annotation\\
\hline
\hfuzz=500pt\includegraphics[width=1em]{element-mustset.pdf}~outputfileGriddedData & \hfuzz=500pt filename & \hfuzz=500pt \\
\hfuzz=500pt\includegraphics[width=1em]{element-mustset.pdf}~inputfileMatrix & \hfuzz=500pt filename & \hfuzz=500pt \\
\hfuzz=500pt\includegraphics[width=1em]{element-mustset.pdf}~points & \hfuzz=500pt choice & \hfuzz=500pt \\
\hfuzz=500pt\includegraphics[width=1em]{connector.pdf}\includegraphics[width=1em]{element-mustset.pdf}~ellipsoidal & \hfuzz=500pt sequence & \hfuzz=500pt \\
\hfuzz=500pt\quad\includegraphics[width=1em]{connector.pdf}\includegraphics[width=1em]{element-mustset.pdf}~longitude & \hfuzz=500pt expression & \hfuzz=500pt expression\\
\hfuzz=500pt\quad\includegraphics[width=1em]{connector.pdf}\includegraphics[width=1em]{element-mustset.pdf}~latitude & \hfuzz=500pt expression & \hfuzz=500pt expression\\
\hfuzz=500pt\quad\includegraphics[width=1em]{connector.pdf}\includegraphics[width=1em]{element-mustset.pdf}~height & \hfuzz=500pt expression & \hfuzz=500pt expression\\
\hfuzz=500pt\includegraphics[width=1em]{connector.pdf}\includegraphics[width=1em]{element-mustset.pdf}~cartesian & \hfuzz=500pt sequence & \hfuzz=500pt \\
\hfuzz=500pt\quad\includegraphics[width=1em]{connector.pdf}\includegraphics[width=1em]{element-mustset.pdf}~x & \hfuzz=500pt expression & \hfuzz=500pt expression\\
\hfuzz=500pt\quad\includegraphics[width=1em]{connector.pdf}\includegraphics[width=1em]{element-mustset.pdf}~y & \hfuzz=500pt expression & \hfuzz=500pt expression\\
\hfuzz=500pt\quad\includegraphics[width=1em]{connector.pdf}\includegraphics[width=1em]{element-mustset.pdf}~z & \hfuzz=500pt expression & \hfuzz=500pt expression\\
\hfuzz=500pt\includegraphics[width=1em]{element.pdf}~area & \hfuzz=500pt expression & \hfuzz=500pt expression (e.g. deltaL*2*sin(deltaB/2)*cos(data1/RHO))\\
\hfuzz=500pt\includegraphics[width=1em]{element-unbounded.pdf}~value & \hfuzz=500pt expression & \hfuzz=500pt expression\\
\hfuzz=500pt\includegraphics[width=1em]{element.pdf}~sortPoints & \hfuzz=500pt boolean & \hfuzz=500pt sort from north/west to south east\\
\hfuzz=500pt\includegraphics[width=1em]{element.pdf}~computeArea & \hfuzz=500pt boolean & \hfuzz=500pt the area can be computed automatically for rectangular grids\\
\hfuzz=500pt\includegraphics[width=1em]{element.pdf}~R & \hfuzz=500pt double & \hfuzz=500pt reference radius for ellipsoidal coordinates\\
\hfuzz=500pt\includegraphics[width=1em]{element.pdf}~inverseFlattening & \hfuzz=500pt double & \hfuzz=500pt reference flattening for ellipsoidal coordinates\\
\hline
\end{tabularx}

\clearpage
%==================================
\subsection{MatrixRectangular2GriddedData}\label{MatrixRectangular2GriddedData}
Read gridded data (matrix).


\keepXColumns
\begin{tabularx}{\textwidth}{N T A}
\hline
Name & Type & Annotation\\
\hline
\hfuzz=500pt\includegraphics[width=1em]{element-mustset.pdf}~outputfileGriddedData & \hfuzz=500pt filename & \hfuzz=500pt \\
\hfuzz=500pt\includegraphics[width=1em]{element-mustset.pdf}~inputfileMatrix & \hfuzz=500pt filename & \hfuzz=500pt \\
\hfuzz=500pt\includegraphics[width=1em]{element.pdf}~rowMajor & \hfuzz=500pt boolean & \hfuzz=500pt true: data is ordered row by row, false: columnwise\\
\hfuzz=500pt\includegraphics[width=1em]{element-mustset.pdf}~startLongitude & \hfuzz=500pt angle & \hfuzz=500pt longitude of upper left corner of the grid\\
\hfuzz=500pt\includegraphics[width=1em]{element-mustset.pdf}~startLatitude & \hfuzz=500pt angle & \hfuzz=500pt latitude of upper left corner of the grid\\
\hfuzz=500pt\includegraphics[width=1em]{element-mustset.pdf}~deltaLongitude & \hfuzz=500pt angle & \hfuzz=500pt sampling, negative for east to west data\\
\hfuzz=500pt\includegraphics[width=1em]{element-mustset.pdf}~deltaLatitude & \hfuzz=500pt angle & \hfuzz=500pt sampling, negative for south to north data\\
\hfuzz=500pt\includegraphics[width=1em]{element.pdf}~R & \hfuzz=500pt double & \hfuzz=500pt reference radius for ellipsoidal coordinates\\
\hfuzz=500pt\includegraphics[width=1em]{element.pdf}~inverseFlattening & \hfuzz=500pt double & \hfuzz=500pt reference flattening for ellipsoidal coordinates\\
\hline
\end{tabularx}

\clearpage
%==================================
\subsection{TimeSeries2GriddedData}\label{TimeSeries2GriddedData}
Interpret the data columns of \configFile{inputfileTimeSeries}{instrument}
as data points of a corresponding \configClass{grid}{gridType}.

For each epoch a \file{gridded data file}{griddedData} is written where
the \config{variableLoopTime} and \config{variableLoopIndex} are expanded for
each point of the given time series to create the file name for this epoch
(see \reference{text parser}{general.parser:text}).

The number of input data columns must be a multiple of the number $n$ of grid points.
If \config{isGroupedDataByPoint} is true the \configFile{inputfileTimeSeries}{instrument} starts
with all data (\verb|data0|, \verb|data1|\ldots) for the first point, followed by all data of the second point and so on.
If \config{isGroupedDataByPoint} is false, the file starts with \verb|data0| for all points, followed by all \verb|data1| and so on.

See also \program{GriddedData2TimeSeries}.


\keepXColumns
\begin{tabularx}{\textwidth}{N T A}
\hline
Name & Type & Annotation\\
\hline
\hfuzz=500pt\includegraphics[width=1em]{element-mustset.pdf}~outputfilesGriddedData & \hfuzz=500pt filename & \hfuzz=500pt for each epoch\\
\hfuzz=500pt\includegraphics[width=1em]{element.pdf}~variableLoopTime & \hfuzz=500pt string & \hfuzz=500pt variable with time of each epoch\\
\hfuzz=500pt\includegraphics[width=1em]{element.pdf}~variableLoopIndex & \hfuzz=500pt string & \hfuzz=500pt variable with index of current epoch (starts with zero)\\
\hfuzz=500pt\includegraphics[width=1em]{element.pdf}~variableLoopCount & \hfuzz=500pt string & \hfuzz=500pt variable with total number of epochs\\
\hfuzz=500pt\includegraphics[width=1em]{element-mustset.pdf}~inputfileTimeSeries & \hfuzz=500pt filename & \hfuzz=500pt each epoch: multiple data for points (MISCVALUES)\\
\hfuzz=500pt\includegraphics[width=1em]{element-mustset-unbounded.pdf}~grid & \hfuzz=500pt \hyperref[gridType]{grid} & \hfuzz=500pt corresponding grid points\\
\hfuzz=500pt\includegraphics[width=1em]{element.pdf}~isDataGroupedByPoint & \hfuzz=500pt boolean & \hfuzz=500pt multiple data are given point by point, otherwise: first data0 for all points, followed by all data1\\
\hfuzz=500pt\includegraphics[width=1em]{element.pdf}~R & \hfuzz=500pt double & \hfuzz=500pt reference radius for ellipsoidal coordinates on output\\
\hfuzz=500pt\includegraphics[width=1em]{element.pdf}~inverseFlattening & \hfuzz=500pt double & \hfuzz=500pt reference flattening for ellipsoidal coordinates on output, 0: spherical coordinates\\
\hline
\end{tabularx}

\clearpage
%==================================
\section{Programs: Instrument}
\subsection{Instrument2AllanVariance}\label{Instrument2AllanVariance}
This program computes the overlapping Allan variance from an
\configFile{inputfileInstrument}{instrument}.
The estimate is averaged over all arcs (arcs are assumed to contain no data gaps).

The overlapping Allan variance is defined as
\begin{equation}
  \sigma^2(m\tau_0) = \frac{1}{2(m\tau_0)^2(N-2m)} \sum_{n=1}^{N-2m}(x_{n+2m}-2x_{n+m}+x_n)^2,
\end{equation}
where $m\tau_0$ is the averaging interval defined by the median sampling $\tau_0$.


\keepXColumns
\begin{tabularx}{\textwidth}{N T A}
\hline
Name & Type & Annotation\\
\hline
\hfuzz=500pt\includegraphics[width=1em]{element-mustset.pdf}~outputfileAllanVariance & \hfuzz=500pt filename & \hfuzz=500pt column 0: averaging interval [seconds], column 1-(n-1): Allan variance for each data column\\
\hfuzz=500pt\includegraphics[width=1em]{element-mustset.pdf}~inputfileInstrument & \hfuzz=500pt filename & \hfuzz=500pt \\
\hline
\end{tabularx}

This program is \reference{parallelized}{general.parallelization}.
\clearpage
%==================================
\subsection{Instrument2CovarianceFunctionVCE}\label{Instrument2CovarianceFunctionVCE}
This estimates a covariance function of \configFile{inputfileInstrument}{instrument}
for all selected columns with \config{startDataFields} and \config{countDataFields}.
The estimation is performed robustly via variance component estimation.
Bad arcs are downweigthed and the accuracies can be written with \configFile{outputfileSigmasPerArc}{matrix}.
The length of the covariance functions are determined by the longest arc.
Additionaly the data can be detrended with \configClass{parameter}{parametrizationTemporalType}
and \configClass{parameterPerArc}{parametrizationTemporalType}.


\keepXColumns
\begin{tabularx}{\textwidth}{N T A}
\hline
Name & Type & Annotation\\
\hline
\hfuzz=500pt\includegraphics[width=1em]{element-mustset.pdf}~outputfileCovarianceFunction & \hfuzz=500pt filename & \hfuzz=500pt covariance functions\\
\hfuzz=500pt\includegraphics[width=1em]{element.pdf}~outputfileSigmasPerArc & \hfuzz=500pt filename & \hfuzz=500pt accuracies of each arc\\
\hfuzz=500pt\includegraphics[width=1em]{element.pdf}~outputfileResiduals & \hfuzz=500pt filename & \hfuzz=500pt \\
\hfuzz=500pt\includegraphics[width=1em]{element.pdf}~outputfileSolution & \hfuzz=500pt filename & \hfuzz=500pt estimated parameter vector (global part only)\\
\hfuzz=500pt\includegraphics[width=1em]{element-mustset.pdf}~inputfileInstrument & \hfuzz=500pt filename & \hfuzz=500pt \\
\hfuzz=500pt\includegraphics[width=1em]{element.pdf}~startDataFields & \hfuzz=500pt uint & \hfuzz=500pt start\\
\hfuzz=500pt\includegraphics[width=1em]{element.pdf}~countDataFields & \hfuzz=500pt uint & \hfuzz=500pt number of data fields (default: all after start)\\
\hfuzz=500pt\includegraphics[width=1em]{element-unbounded.pdf}~parameter & \hfuzz=500pt \hyperref[parametrizationTemporalType]{parametrizationTemporal} & \hfuzz=500pt data is reduced by temporal representation\\
\hfuzz=500pt\includegraphics[width=1em]{element-unbounded.pdf}~parameterPerArc & \hfuzz=500pt \hyperref[parametrizationTemporalType]{parametrizationTemporal} & \hfuzz=500pt data is reduced by temporal representation\\
\hfuzz=500pt\includegraphics[width=1em]{element.pdf}~iterationCount & \hfuzz=500pt uint & \hfuzz=500pt number of iterations for the estimation\\
\hline
\end{tabularx}

This program is \reference{parallelized}{general.parallelization}.
\clearpage
%==================================
\subsection{Instrument2CrossCorrelationFunction}\label{Instrument2CrossCorrelationFunction}
This program computes the cross correlation between all corresponding data columns
in two \file{instrument files}{instrument}. The instrument files must be synchronized (\program{InstrumentSynchronize}).
The \configFile{outputfileCorrelation}{matrix} is a matrix with the first column containing the time lag followed by
cross-correlation function for each data column. The maximum lag is defined by the maximum arc length.

The correlation is based on the unbiased estimate of the cross-covariance between data columns $x$ and $y$,
\begin{equation}
  \sigma_{xy}(h) = \frac{1}{N}\sum_{k=1} x_{k+h} y_k,
\end{equation}
which is averaged over all arcs. From this estimate, the correlation for each lag is then computed via
\begin{equation}
  r_{xy}(h) = \frac{\sigma_{xy}(h)}{\sigma_x(0)\sigma_y(0)},
\end{equation}
which is the ratio between the biased estimates of the cross-covariance at lag $h$ and the auto-covariance of the individual data columns.

For instrument with data gaps, lag bins without any data are set to NAN.


\keepXColumns
\begin{tabularx}{\textwidth}{N T A}
\hline
Name & Type & Annotation\\
\hline
\hfuzz=500pt\includegraphics[width=1em]{element-mustset.pdf}~outputfileCorrelation & \hfuzz=500pt filename & \hfuzz=500pt column 1: time lag, column 2..n cross-correlation\\
\hfuzz=500pt\includegraphics[width=1em]{element-mustset.pdf}~inputfileInstrument & \hfuzz=500pt filename & \hfuzz=500pt \\
\hfuzz=500pt\includegraphics[width=1em]{element-mustset.pdf}~inputfileInstrumentReference & \hfuzz=500pt filename & \hfuzz=500pt \\
\hline
\end{tabularx}

This program is \reference{parallelized}{general.parallelization}.
\clearpage
%==================================
\subsection{Instrument2Histogram}\label{Instrument2Histogram}
This program computes the arc-wise histogram from an \file{instrument file}{instrument}.
The output is a \file{matrix}{matrix} with the first column containing the lower bound of each bin.
The other columns contain the histograms for each arc.

\fig{!hb}{0.8}{instrument2Histogram}{fig:instrument2Histogram}{GRACE range-rate residuals of one month (one arc) divided into 50 bins.}


\keepXColumns
\begin{tabularx}{\textwidth}{N T A}
\hline
Name & Type & Annotation\\
\hline
\hfuzz=500pt\includegraphics[width=1em]{element-mustset.pdf}~outputfileMatrix & \hfuzz=500pt filename & \hfuzz=500pt column 1: lower bin bound; columns 2 to N: histogram of each arc\\
\hfuzz=500pt\includegraphics[width=1em]{element-mustset.pdf}~inputfileInstrument & \hfuzz=500pt filename & \hfuzz=500pt \\
\hfuzz=500pt\includegraphics[width=1em]{element.pdf}~selectDataField & \hfuzz=500pt uint & \hfuzz=500pt select channel for histogram computation\\
\hfuzz=500pt\includegraphics[width=1em]{element.pdf}~binCount & \hfuzz=500pt uint & \hfuzz=500pt (default: Freedman-Diaconis' choice, maximum of all channels)\\
\hfuzz=500pt\includegraphics[width=1em]{element.pdf}~lowerBound & \hfuzz=500pt expression & \hfuzz=500pt lower bound for bins (default: global minimum, data values outside are ignored)\\
\hfuzz=500pt\includegraphics[width=1em]{element.pdf}~upperBound & \hfuzz=500pt expression & \hfuzz=500pt upper bound for bins (default: global maximum, data values outside are ignored)\\
\hfuzz=500pt\includegraphics[width=1em]{element.pdf}~relative & \hfuzz=500pt boolean & \hfuzz=500pt output relative frequencies\\
\hfuzz=500pt\includegraphics[width=1em]{element.pdf}~cumulative & \hfuzz=500pt boolean & \hfuzz=500pt accumulate frequencies\\
\hline
\end{tabularx}

This program is \reference{parallelized}{general.parallelization}.
\clearpage
%==================================
\subsection{Instrument2PowerSpectralDensity}\label{Instrument2PowerSpectralDensity}
This program computes the power spectral density (PSD) for all data fields in an \file{instrument file}{instrument}.
The PSD is computed using Lomb's method. For each arc and each frequency $f$, a sinusoid is fit to the data
\begin{equation}
  l_i = a \cos(2\pi f t_i) + b \sin(2\pi f t_i) + e_i
\end{equation}

The PSD for this frequency is then computed by forming the square sum of adjusted observations:
\begin{equation}
  P(f) = \sum_i \hat{l}^2_i.
\end{equation}

The resulting PSD is the average over all arcs. For regularly sampled time series,
this method yields the same results as FFT based PSD estimates.

A regular frequency grid based on the longest arc and the median sampling is computed.
The maximum number of epochs per arc is determined by
\begin{equation}
  N = \frac{t_{\text{end}} - t_{\text{start}}}{\Delta t_{\text{median}} } + 1,
\end{equation}
the Nyquist frequency is given by
\begin{equation}
  f_{\text{nyq}} = \frac{1}{2\Delta t_{\text{median}}}.
\end{equation}

If it is suspected that \configFile{inputfileInstrument}{instrument} contains secular variations,
the input should be detrended using \program{InstrumentDetrend}.

See also \program{Instrument2CovarianceFunctionVCE},
\program{CovarianceFunction2PowerSpectralDensity}, \program{PowerSpectralDensity2CovarianceFunction}.


\keepXColumns
\begin{tabularx}{\textwidth}{N T A}
\hline
Name & Type & Annotation\\
\hline
\hfuzz=500pt\includegraphics[width=1em]{element-mustset.pdf}~outputfilePSD & \hfuzz=500pt filename & \hfuzz=500pt estimated PSD: column 0: frequency vector, column 1-(n-1): PSD estimate for each channel\\
\hfuzz=500pt\includegraphics[width=1em]{element-mustset.pdf}~inputfileInstrument & \hfuzz=500pt filename & \hfuzz=500pt \\
\hline
\end{tabularx}

This program is \reference{parallelized}{general.parallelization}.
\clearpage
%==================================
\subsection{Instrument2RmsPlotGrid}\label{Instrument2RmsPlotGrid}
This program computes an RMS plot grid from one or more \configFile{inputfileInstrument}{instrument}
containing 3D data (e.g. orbits or station positions), which can then be plotted as gridded data in \program{PlotGraph}.
The RMS is computed from the difference between \configFile{inputfileInstrument}{instrument} and
\configFile{inputfileInstrumentReference}{instrument}.
All instrument files must be synchronized (see \program{InstrumentSynchronize}).

Each separate \configFile{inputfileInstrument}{instrument} represents an entry (e.g. a satellite or station)
in the resulting grid. Therefore, providing, for example, 32 orbit files of GPS satellites
results in a grid with columns: mjd, id (0-31), rms.

The first three data columns of the instrument data are considered for computation of the RMS values.
The \config{factor} can be set to, for example, sqrt(3) to get 3D instead of 1D RMS values.

If \configClass{timeIntervals}{timeSeriesType} are provided, each \configFile{inputfileInstrument}{instrument}
and \configFile{inputfileInstrumentReference}{instrument} serves as a template with variable \verb|loopTime|.
This allows concatenation of instrument files, for example to create a month-long RMS plot grid from daily GPS
orbit files (see below).

Helmert parameters between the two frames can be estimated each epoch optionally if
\config{estimateShift}, \config{estimateScale}, or \config{estimateRotation} are set.
It uses a \reference{robust least squares adjustment}{fundamentals.robustLeastSquares}.

\fig{!hb}{0.8}{instrument2RmsPlotGrid}{fig:instrument2RmsPlotGrid}{Comparison of estimated GPS orbits with IGS final solution.}


\keepXColumns
\begin{tabularx}{\textwidth}{N T A}
\hline
Name & Type & Annotation\\
\hline
\hfuzz=500pt\includegraphics[width=1em]{element.pdf}~outputfileRmsPlotGrid & \hfuzz=500pt filename & \hfuzz=500pt columns: mjd, id, rms\\
\hfuzz=500pt\includegraphics[width=1em]{element.pdf}~outputfileHelmertTimeSeries & \hfuzz=500pt filename & \hfuzz=500pt columns: mjd, tx, ty, tz, scale, rx, ry, rz\\
\hfuzz=500pt\includegraphics[width=1em]{element-mustset-unbounded.pdf}~inputfileInstrument & \hfuzz=500pt filename & \hfuzz=500pt one file per satellite/station\\
\hfuzz=500pt\includegraphics[width=1em]{element-mustset-unbounded.pdf}~inputfileInstrumentReference & \hfuzz=500pt filename & \hfuzz=500pt one file per satellite/station, same order as above\\
\hfuzz=500pt\includegraphics[width=1em]{element-unbounded.pdf}~timeIntervals & \hfuzz=500pt \hyperref[timeSeriesType]{timeSeries} & \hfuzz=500pt for \{loopTime\} variable in inputfile\\
\hfuzz=500pt\includegraphics[width=1em]{element.pdf}~factor & \hfuzz=500pt double & \hfuzz=500pt e.g. sqrt(3) for 3D RMS\\
\hfuzz=500pt\includegraphics[width=1em]{element.pdf}~estimateShift & \hfuzz=500pt boolean & \hfuzz=500pt coordinate center every epoch\\
\hfuzz=500pt\includegraphics[width=1em]{element.pdf}~estimateScale & \hfuzz=500pt boolean & \hfuzz=500pt scale factor of position every epoch\\
\hfuzz=500pt\includegraphics[width=1em]{element.pdf}~estimateRotation & \hfuzz=500pt boolean & \hfuzz=500pt rotation every epoch\\
\hfuzz=500pt\includegraphics[width=1em]{element.pdf}~huber & \hfuzz=500pt double & \hfuzz=500pt for robust least squares\\
\hfuzz=500pt\includegraphics[width=1em]{element.pdf}~huberPower & \hfuzz=500pt double & \hfuzz=500pt for robust least squares\\
\hfuzz=500pt\includegraphics[width=1em]{element.pdf}~huberMaxIteration & \hfuzz=500pt uint & \hfuzz=500pt (maximum) number of iterations for robust estimation\\
\hline
\end{tabularx}

\clearpage
%==================================
\subsection{Instrument2Scaleogram}\label{Instrument2Scaleogram}
This program computes the wavelet transform of a time series up to a \config{maxLevel}.
The scalogram is written to a matrix which can be plotted by using a gridded layer in \program{PlotGraph}.
Individual detail levels can be written to matrix files by setting \configFile{outputfileLevels}{matrix}.
The data column to be decomposed must be set by \config{selectDataField}.

The wavelet transform is implemented as a filter bank, so care should be taken when the input contains data gaps.
Low/highpass wavelet filters are applied in forward and backward direction, input is padded symmetric.
See \configClass{digitalFilter}{digitalFilterType} for details.

\fig{!hb}{0.8}{instrument2Scaleogram}{fig:Instrument2Scaleogram}{GRACE range-rate residuals of one month.}


\keepXColumns
\begin{tabularx}{\textwidth}{N T A}
\hline
Name & Type & Annotation\\
\hline
\hfuzz=500pt\includegraphics[width=1em]{element-mustset.pdf}~outputfileScaleogram & \hfuzz=500pt filename & \hfuzz=500pt matrix columns: mjd, level, value\\
\hfuzz=500pt\includegraphics[width=1em]{element.pdf}~outputfileLevels & \hfuzz=500pt filename & \hfuzz=500pt use loopLevel as variable\\
\hfuzz=500pt\includegraphics[width=1em]{element-mustset.pdf}~inputfileInstrument & \hfuzz=500pt filename & \hfuzz=500pt \\
\hfuzz=500pt\includegraphics[width=1em]{element-mustset.pdf}~inputfileWavelet & \hfuzz=500pt filename & \hfuzz=500pt wavelet coefficients\\
\hfuzz=500pt\includegraphics[width=1em]{element.pdf}~selectDataField & \hfuzz=500pt uint & \hfuzz=500pt data column to transform\\
\hfuzz=500pt\includegraphics[width=1em]{element.pdf}~maxLevel & \hfuzz=500pt uint & \hfuzz=500pt maximum level of decomposition (default: full)\\
\hline
\end{tabularx}

\clearpage
%==================================
\subsection{Instrument2SpectralCoherence}\label{Instrument2SpectralCoherence}
This program computes the spectral coherence between two \file{instrument files}{instrument}.

The (magnitude-squared) coherence is defined as
\begin{equation}
  C_{xy}(f) = \frac{|P_{xy}(f)|^2}{P_{xx}(f)P_{yy}(f)}
\end{equation}
and is a measure in the range [0, 1] for the similarity of the signals $x$ and $y$ in frequency domain.
$P_{xy}$ is the cross-spectral density between $x$ and $y$ and $P_{xx}$, $P_{yy}$ are auto-spectral densities.
Auto- and cross-spectral densities are computed using Lomb's method (see \program{Instrument2PowerSpectralDensity} for details).

The resulting PSD is the average over all arcs. For regularly sampled time series,
this method yields the same results as FFT based PSD estimates.

A regular frequency grid based on the longest arc and the median sampling is computed.
The maximum number of epochs per arc is determined by
\begin{equation}
  N = \frac{t_{\text{end}} - t_{\text{start}}}{\Delta t_{\text{median}} } + 1,
\end{equation}
the Nyquist frequency is given by
\begin{equation}
  f_{\text{nyq}} = \frac{1}{2\Delta t_{\text{median}}}.
\end{equation}

If it is suspected that \configFile{inputfileInstrument}{instrument} contains secular variations,
the input should be detrended using \program{InstrumentDetrend}.

The \configFile{outputfileCoherence}{matrix} contains a matrix with the frequency vector as first column,
the coherence for each instrument channel is saved in the following columns.


\keepXColumns
\begin{tabularx}{\textwidth}{N T A}
\hline
Name & Type & Annotation\\
\hline
\hfuzz=500pt\includegraphics[width=1em]{element-mustset.pdf}~outputfileCoherence & \hfuzz=500pt filename & \hfuzz=500pt column 1: frequency, column 2-n coherence\\
\hfuzz=500pt\includegraphics[width=1em]{element-mustset.pdf}~inputfileInstrument & \hfuzz=500pt filename & \hfuzz=500pt \\
\hfuzz=500pt\includegraphics[width=1em]{element-mustset.pdf}~inputfileInstrumentReference & \hfuzz=500pt filename & \hfuzz=500pt \\
\hline
\end{tabularx}

This program is \reference{parallelized}{general.parallelization}.
\clearpage
%==================================
\subsection{Instrument2Spectrogram}\label{Instrument2Spectrogram}
This program applies the Short Time Fourier Transform (STFT) to selected data columns
of \configFile{inputfileInstrument}{instrument} and computes the spectrogram.
The STFT is computed at centered \configClass{timeSeries}{timeSeriesType} with
an (possible overlapping) rectangular window with \config{windowLength} seconds.
Data gaps are zero padded within the window.

The \configFile{outputfileSpectrogram}{matrix} is a matrix with each row the time (MJD),
the frequency $[Hz]$, and the amplitudes $[unit/\sqrt{Hz}]$ for the selected data columns.
It can be plotted with \program{PlotGraph}.

\fig{!hb}{0.8}{instrument2Spectrogram}{fig:instrument2Spectrogram}{GRACE range-rate residuals of one month (window of 6 hours).}


\keepXColumns
\begin{tabularx}{\textwidth}{N T A}
\hline
Name & Type & Annotation\\
\hline
\hfuzz=500pt\includegraphics[width=1em]{element-mustset.pdf}~outputfileSpectrogram & \hfuzz=500pt filename & \hfuzz=500pt mjd, freq, ampl0, ampl1, ...\\
\hfuzz=500pt\includegraphics[width=1em]{element-mustset.pdf}~inputfileInstrument & \hfuzz=500pt filename & \hfuzz=500pt \\
\hfuzz=500pt\includegraphics[width=1em]{element-mustset-unbounded.pdf}~timeSeries & \hfuzz=500pt \hyperref[timeSeriesType]{timeSeries} & \hfuzz=500pt center of SFFT window\\
\hfuzz=500pt\includegraphics[width=1em]{element-mustset.pdf}~windowLength & \hfuzz=500pt double & \hfuzz=500pt [seconds]\\
\hfuzz=500pt\includegraphics[width=1em]{element.pdf}~startDataFields & \hfuzz=500pt uint & \hfuzz=500pt start\\
\hfuzz=500pt\includegraphics[width=1em]{element.pdf}~countDataFields & \hfuzz=500pt uint & \hfuzz=500pt number of data fields (default: all)\\
\hline
\end{tabularx}

This program is \reference{parallelized}{general.parallelization}.
\clearpage
%==================================
\subsection{InstrumentAccelerometer2ThermosphericDensity}\label{InstrumentAccelerometer2ThermosphericDensity}
This program estimates neutral mass densities along the satellite trajectory based on \file{accelerometer data}{instrument}.
In order to determine the neutral mass density the accelerometer input should only reflect the accelerations due to drag
(e.g. \configClass{miscAccelerations:atmosphericDrag}{miscAccelerationsType:atmosphericDrag}).
Thus, influences from solar and Earth radiation pressure must be reduced beforehand.


\keepXColumns
\begin{tabularx}{\textwidth}{N T A}
\hline
Name & Type & Annotation\\
\hline
\hfuzz=500pt\includegraphics[width=1em]{element-mustset.pdf}~outputfileDensity & \hfuzz=500pt filename & \hfuzz=500pt MISCVALUE (kg/m\textasciicircum{}3)\\
\hfuzz=500pt\includegraphics[width=1em]{element.pdf}~satelliteModel & \hfuzz=500pt filename & \hfuzz=500pt satellite macro model\\
\hfuzz=500pt\includegraphics[width=1em]{element-mustset.pdf}~inputfileOrbit & \hfuzz=500pt filename & \hfuzz=500pt \\
\hfuzz=500pt\includegraphics[width=1em]{element-mustset.pdf}~inputfileStarCamera & \hfuzz=500pt filename & \hfuzz=500pt \\
\hfuzz=500pt\includegraphics[width=1em]{element-mustset.pdf}~inputfileAccelerometer & \hfuzz=500pt filename & \hfuzz=500pt add non-gravitational forces in satellite reference frame\\
\hfuzz=500pt\includegraphics[width=1em]{element-mustset.pdf}~thermosphere & \hfuzz=500pt \hyperref[thermosphereType]{thermosphere} & \hfuzz=500pt used to compute temperature and wind\\
\hfuzz=500pt\includegraphics[width=1em]{element.pdf}~considerTemperature & \hfuzz=500pt boolean & \hfuzz=500pt compute drag and lift, otherwise simple drag coefficient is used\\
\hfuzz=500pt\includegraphics[width=1em]{element.pdf}~considerWind & \hfuzz=500pt boolean & \hfuzz=500pt \\
\hfuzz=500pt\includegraphics[width=1em]{element-mustset.pdf}~earthRotation & \hfuzz=500pt \hyperref[earthRotationType]{earthRotation} & \hfuzz=500pt \\
\hfuzz=500pt\includegraphics[width=1em]{element.pdf}~ephemerides & \hfuzz=500pt \hyperref[ephemeridesType]{ephemerides} & \hfuzz=500pt \\
\hline
\end{tabularx}

This program is \reference{parallelized}{general.parallelization}.
\clearpage
%==================================
\subsection{InstrumentAccelerometerApplyEstimatedParameters}\label{InstrumentAccelerometerApplyEstimatedParameters}
This program evaluates estimated satellite parameters and writes the result to an accelerometer file.


\keepXColumns
\begin{tabularx}{\textwidth}{N T A}
\hline
Name & Type & Annotation\\
\hline
\hfuzz=500pt\includegraphics[width=1em]{element-mustset.pdf}~outputfileAccelerometer & \hfuzz=500pt filename & \hfuzz=500pt \\
\hfuzz=500pt\includegraphics[width=1em]{element.pdf}~inputfileSatelliteModel & \hfuzz=500pt filename & \hfuzz=500pt satellite macro model\\
\hfuzz=500pt\includegraphics[width=1em]{element.pdf}~inputfileOrbit & \hfuzz=500pt filename & \hfuzz=500pt \\
\hfuzz=500pt\includegraphics[width=1em]{element.pdf}~inputfileStarCamera & \hfuzz=500pt filename & \hfuzz=500pt \\
\hfuzz=500pt\includegraphics[width=1em]{element.pdf}~inputfileAccelerometer & \hfuzz=500pt filename & \hfuzz=500pt add non-gravitational forces in satellite reference frame\\
\hfuzz=500pt\includegraphics[width=1em]{element.pdf}~earthRotation & \hfuzz=500pt \hyperref[earthRotationType]{earthRotation} & \hfuzz=500pt \\
\hfuzz=500pt\includegraphics[width=1em]{element.pdf}~ephemerides & \hfuzz=500pt \hyperref[ephemeridesType]{ephemerides} & \hfuzz=500pt may be needed by parametrizationAcceleration\\
\hfuzz=500pt\includegraphics[width=1em]{element-mustset-unbounded.pdf}~parametrizationAcceleration & \hfuzz=500pt \hyperref[parametrizationAccelerationType]{parametrizationAcceleration} & \hfuzz=500pt orbit force parameters\\
\hfuzz=500pt\includegraphics[width=1em]{element-mustset.pdf}~inputfileParameter & \hfuzz=500pt filename & \hfuzz=500pt estimated orbit force parameters\\
\hfuzz=500pt\includegraphics[width=1em]{element.pdf}~indexStart & \hfuzz=500pt int & \hfuzz=500pt position in the solution vector\\
\hfuzz=500pt\includegraphics[width=1em]{element.pdf}~rightSide & \hfuzz=500pt int & \hfuzz=500pt if solution contains several right hand sides, select one\\
\hfuzz=500pt\includegraphics[width=1em]{element.pdf}~factor & \hfuzz=500pt double & \hfuzz=500pt the result is multiplied by this factor\\
\hline
\end{tabularx}

This program is \reference{parallelized}{general.parallelization}.
\clearpage
%==================================
\subsection{InstrumentAccelerometerEstimateBiasScale}\label{InstrumentAccelerometerEstimateBiasScale}
This program calibrates \configFile{inputfileAccelerometer}{instrument} with respect to
simulated accelerometer data, see \program{SimulateAccelerometer}.
The parameters \configFile{outputfileSolution}{matrix}
of \configClass{parametrizationAcceleration}{parametrizationAccelerationType}
are estimated and the effect is reduced to calibrate the \file{accelerometer data}{instrument}.

If \configFile{inputfileThruster}{instrument} is given, the corresponding epochs
(within \config{marginThruster}) are not used for the parameter estimation,
but the accelerometer epochs are still calibrated afterwards.
An arbitrary instrument file is allowed here.

The \configFile{inputfileOrbit}{instrument}, \configFile{inputfileStarCamera}{instrument},
\configClass{earthRotation}{earthRotationType}, \configClass{ephemerides}{ephemeridesType},
and \configFile{satelliteModel}{satelliteModel} are only needed for some special parametrizations.


\keepXColumns
\begin{tabularx}{\textwidth}{N T A}
\hline
Name & Type & Annotation\\
\hline
\hfuzz=500pt\includegraphics[width=1em]{element-mustset.pdf}~outputfileAccelerometer & \hfuzz=500pt filename & \hfuzz=500pt \\
\hfuzz=500pt\includegraphics[width=1em]{element.pdf}~outputfileSolution & \hfuzz=500pt filename & \hfuzz=500pt \\
\hfuzz=500pt\includegraphics[width=1em]{element-mustset.pdf}~inputfileAccelerometer & \hfuzz=500pt filename & \hfuzz=500pt \\
\hfuzz=500pt\includegraphics[width=1em]{element-mustset.pdf}~inputfileAccelerometerSim & \hfuzz=500pt filename & \hfuzz=500pt \\
\hfuzz=500pt\includegraphics[width=1em]{element.pdf}~inputfileThruster & \hfuzz=500pt filename & \hfuzz=500pt remove thruster events\\
\hfuzz=500pt\includegraphics[width=1em]{element.pdf}~marginThruster & \hfuzz=500pt double & \hfuzz=500pt margin size (on both sides) [seconds]\\
\hfuzz=500pt\includegraphics[width=1em]{element.pdf}~inputfileOrbit & \hfuzz=500pt filename & \hfuzz=500pt \\
\hfuzz=500pt\includegraphics[width=1em]{element.pdf}~inputfileStarCamera & \hfuzz=500pt filename & \hfuzz=500pt \\
\hfuzz=500pt\includegraphics[width=1em]{element.pdf}~earthRotation & \hfuzz=500pt \hyperref[earthRotationType]{earthRotation} & \hfuzz=500pt \\
\hfuzz=500pt\includegraphics[width=1em]{element.pdf}~ephemerides & \hfuzz=500pt \hyperref[ephemeridesType]{ephemerides} & \hfuzz=500pt may be needed by parametrizationAcceleration\\
\hfuzz=500pt\includegraphics[width=1em]{element.pdf}~inputfileSatelliteModel & \hfuzz=500pt filename & \hfuzz=500pt satellite macro model\\
\hfuzz=500pt\includegraphics[width=1em]{element-mustset-unbounded.pdf}~parametrizationAcceleration & \hfuzz=500pt \hyperref[parametrizationAccelerationType]{parametrizationAcceleration} & \hfuzz=500pt \\
\hline
\end{tabularx}

This program is \reference{parallelized}{general.parallelization}.
\clearpage
%==================================
\subsection{InstrumentAccelerometerEstimateParameters}\label{InstrumentAccelerometerEstimateParameters}
This program estimates calibration parameters for acceleration data given given an optional reference acceleration.
Specifically, the program solves the equation
\begin{equation}
  \mathbf{a} - \mathbf{a}_\text{ref} = \mathbf{f}(\mathbf{x}) + \mathbf{e}
\end{equation}
for the unknown parameters $\mathbf{x}$, where $\mathbf{a}$ is given in \configFile{inputfileAccelerometer}{instrument} and
$\mathbf{a}_\text{ref}$ is given in  \configFile{inputfileAccelerometerReference}{instrument}.
The parametrization of $\mathbf{x}$ can be set via \configClass{parametrizationAcceleration}{parametrizationAccelerationType}.
Optionally, the empirical covariance functions for the accelerations $\mathbf{a}$ can be estimated by enabling \config{estimateCovarianceFunctions}.

The estimated parameters are written to the file \configFile{outputfileSolution}{matrix} and can be used by
\program{InstrumentAccelerometerApplyEstimatedParameters} to calibrate accelerometer measurements.


\keepXColumns
\begin{tabularx}{\textwidth}{N T A}
\hline
Name & Type & Annotation\\
\hline
\hfuzz=500pt\includegraphics[width=1em]{element-mustset.pdf}~outputfileSolution & \hfuzz=500pt filename & \hfuzz=500pt values for estimated parameters\\
\hfuzz=500pt\includegraphics[width=1em]{element.pdf}~outputfileParameterNames & \hfuzz=500pt filename & \hfuzz=500pt names of the estimated parameters\\
\hfuzz=500pt\includegraphics[width=1em]{element.pdf}~estimateArcSigmas & \hfuzz=500pt sequence & \hfuzz=500pt \\
\hfuzz=500pt\includegraphics[width=1em]{connector.pdf}\includegraphics[width=1em]{element.pdf}~outputfileArcSigmas & \hfuzz=500pt filename & \hfuzz=500pt accuracies of each arc\\
\hfuzz=500pt\includegraphics[width=1em]{element.pdf}~estimateEpochSigmas & \hfuzz=500pt sequence & \hfuzz=500pt \\
\hfuzz=500pt\includegraphics[width=1em]{connector.pdf}\includegraphics[width=1em]{element.pdf}~outputfileEpochSigmas & \hfuzz=500pt filename & \hfuzz=500pt estimated epoch-wise sigmas\\
\hfuzz=500pt\includegraphics[width=1em]{element.pdf}~estimateCovarianceFunctions & \hfuzz=500pt sequence & \hfuzz=500pt \\
\hfuzz=500pt\includegraphics[width=1em]{connector.pdf}\includegraphics[width=1em]{element.pdf}~outputfileCovarianceFunction & \hfuzz=500pt filename & \hfuzz=500pt covariance functions for x, y, z direction\\
\hfuzz=500pt\includegraphics[width=1em]{element-mustset.pdf}~inputfileAccelerometer & \hfuzz=500pt filename & \hfuzz=500pt \\
\hfuzz=500pt\includegraphics[width=1em]{element.pdf}~inputfileAccelerometerReference & \hfuzz=500pt filename & \hfuzz=500pt if not given, reference acceleration is assumed zero\\
\hfuzz=500pt\includegraphics[width=1em]{element.pdf}~inputfileOrbit & \hfuzz=500pt filename & \hfuzz=500pt may be needed by parametrizationAcceleration\\
\hfuzz=500pt\includegraphics[width=1em]{element.pdf}~inputfileStarCamera & \hfuzz=500pt filename & \hfuzz=500pt may be needed by parametrizationAcceleration\\
\hfuzz=500pt\includegraphics[width=1em]{element.pdf}~inputfileSatelliteModel & \hfuzz=500pt filename & \hfuzz=500pt satellite macro model (may be needed by parametrizationAcceleration)\\
\hfuzz=500pt\includegraphics[width=1em]{element.pdf}~earthRotation & \hfuzz=500pt \hyperref[earthRotationType]{earthRotation} & \hfuzz=500pt may be needed by parametrizationAcceleration\\
\hfuzz=500pt\includegraphics[width=1em]{element.pdf}~ephemerides & \hfuzz=500pt \hyperref[ephemeridesType]{ephemerides} & \hfuzz=500pt may be needed by parametrizationAcceleration\\
\hfuzz=500pt\includegraphics[width=1em]{element-mustset-unbounded.pdf}~parametrizationAcceleration & \hfuzz=500pt \hyperref[parametrizationAccelerationType]{parametrizationAcceleration} & \hfuzz=500pt \\
\hfuzz=500pt\includegraphics[width=1em]{element.pdf}~sigmaX & \hfuzz=500pt double & \hfuzz=500pt apriori accuracy in x-axis\\
\hfuzz=500pt\includegraphics[width=1em]{element.pdf}~sigmaY & \hfuzz=500pt double & \hfuzz=500pt apriori accuracy in y-axis\\
\hfuzz=500pt\includegraphics[width=1em]{element.pdf}~sigmaZ & \hfuzz=500pt double & \hfuzz=500pt apriori accuracy in z-axis\\
\hfuzz=500pt\includegraphics[width=1em]{element.pdf}~iterationCount & \hfuzz=500pt uint & \hfuzz=500pt iteration count for determining the covariance function\\
\hline
\end{tabularx}

This program is \reference{parallelized}{general.parallelization}.
\clearpage
%==================================
\subsection{InstrumentApplyTimeOffset}\label{InstrumentApplyTimeOffset}
This program applies a \configFile{inputfileTimeOffset}{instrument} (MISCVALUE)
to an \configFile{inputfileInstrument}{instrument}.
The time offsets in seconds are multiplicated with a \config{factor}.
The instrument files must be synchronized (see \program{InstrumentSynchronize}).


\keepXColumns
\begin{tabularx}{\textwidth}{N T A}
\hline
Name & Type & Annotation\\
\hline
\hfuzz=500pt\includegraphics[width=1em]{element-mustset.pdf}~outputfileInstrument & \hfuzz=500pt filename & \hfuzz=500pt \\
\hfuzz=500pt\includegraphics[width=1em]{element-mustset.pdf}~inputfileInstrument & \hfuzz=500pt filename & \hfuzz=500pt \\
\hfuzz=500pt\includegraphics[width=1em]{element-mustset.pdf}~inputfileTimeOffset & \hfuzz=500pt filename & \hfuzz=500pt MISCVALUE with time offset in seconds\\
\hfuzz=500pt\includegraphics[width=1em]{element.pdf}~factor & \hfuzz=500pt double & \hfuzz=500pt applied to time offset\\
\hline
\end{tabularx}

\clearpage
%==================================
\subsection{InstrumentArcCalculate}\label{InstrumentArcCalculate}
This program manipulates the data columns every arc of an \file{instrument file}{instrument} similar to
\program{FunctionsCalculate}, see there for more details.
If several \configFile{inputfileInstrument}{instrument}s are given the data columns are copied side by side.
For this the instrument files must be synchronized (see \program{InstrumentSynchronize}). For the data
columns the standard data variables are available, see~\reference{dataVariables}{general.parser:dataVariables}.
For the time column (MJD) a variable \verb|epoch| (together with \verb|epochmean|, \verb|epochmin|, \ldots)
is defined additionally.

The content of \configFile{outputfileInstrument}{instrument} is controlled by \config{outColumn}.
The number of \config{outColumn} must agree with the selected \configClass{outType}{instrumentTypeType}.
The algorithm to compute the output is as follows:
The expressions in \config{outColumn} are evaluated once for each epoch of the input.
The variables \verb|data0|,~\verb|data1|,~\ldots are replaced by the according values from the input columns before.
If no \config{outColumn} are specified all input columns are used instead directly.
The \configClass{instrument type}{instrumentTypeType} can be specified with \config{outType} and must be agree with the number of columns.

An extra \config{statistics} file can be generated with one mid epoch per arc. For the computation of the \config{outColumn} values
all~\reference{dataVariables}{general.parser:dataVariables} are available (e.g. \verb|epochmin|, \verb|data0mean|, \verb|data1std|, \ldots)
inclusively the \config{constant}s and estimated \config{parameter}s but without the \verb|data0|,~\verb|data1|,~\ldots itself.
The variables and the numbering of the columns refers to the \configFile{outputfileInstrument}{instrument}.

See also \program{FunctionsCalculate}, \program{MatrixCalculate}.


\keepXColumns
\begin{tabularx}{\textwidth}{N T A}
\hline
Name & Type & Annotation\\
\hline
\hfuzz=500pt\includegraphics[width=1em]{element.pdf}~outputfileInstrument & \hfuzz=500pt filename & \hfuzz=500pt \\
\hfuzz=500pt\includegraphics[width=1em]{element-mustset-unbounded.pdf}~inputfileInstrument & \hfuzz=500pt filename & \hfuzz=500pt data columns are appended to the right\\
\hfuzz=500pt\includegraphics[width=1em]{element-unbounded.pdf}~constant & \hfuzz=500pt expression & \hfuzz=500pt define a constant by name=value\\
\hfuzz=500pt\includegraphics[width=1em]{element-unbounded.pdf}~parameter & \hfuzz=500pt expression & \hfuzz=500pt define a parameter by name[=value]\\
\hfuzz=500pt\includegraphics[width=1em]{element-unbounded.pdf}~leastSquares & \hfuzz=500pt expression & \hfuzz=500pt try to minimize the expression by adjustment of the parameters\\
\hfuzz=500pt\includegraphics[width=1em]{element-unbounded.pdf}~removalCriteria & \hfuzz=500pt expression & \hfuzz=500pt row is removed if one criterion evaluates true.\\
\hfuzz=500pt\includegraphics[width=1em]{element.pdf}~outType & \hfuzz=500pt \hyperref[instrumentTypeType]{instrumentType} & \hfuzz=500pt \\
\hfuzz=500pt\includegraphics[width=1em]{element-unbounded.pdf}~outColumn & \hfuzz=500pt expression & \hfuzz=500pt expression of output columns, extra 'epoch' variable\\
\hfuzz=500pt\includegraphics[width=1em]{element.pdf}~statistics & \hfuzz=500pt sequence & \hfuzz=500pt \\
\hfuzz=500pt\includegraphics[width=1em]{connector.pdf}\includegraphics[width=1em]{element-mustset.pdf}~outputfileInstrument & \hfuzz=500pt filename & \hfuzz=500pt instrument file with mid epoch per arc, data columns are user defined\\
\hfuzz=500pt\includegraphics[width=1em]{connector.pdf}\includegraphics[width=1em]{element-mustset-unbounded.pdf}~outColumn & \hfuzz=500pt expression & \hfuzz=500pt expression to compute statistics columns, data* are from outColumn\\
\hline
\end{tabularx}

This program is \reference{parallelized}{general.parallelization}.
\clearpage
%==================================
\subsection{InstrumentArcCrossStatistics}\label{InstrumentArcCrossStatistics}
Computes statistics of selected data columns between two \file{instrument files}{instrument} arc wise.
The \configFile{outputfileStatisticsTimeSeries}{instrument} contains for every arc one (mid) epoch
with statistics column(s). Possible statistics are
\begin{itemize}
  \item Correlation
  \begin{equation}
    \rho = \frac{\sum_i x_i y_i}{\sqrt{(\sum_i x_i^2) (\sum_i y_i^2})},
  \end{equation}
  \item Error RMS
  \begin{equation}
    rms = \sqrt{\frac{1}{N}\sum_i (x_i-y_i)^2},
  \end{equation}
  \item Nash-Sutcliffe coefficient (NSC)
  \begin{equation}
    nsc = 1- \frac{\sum_i (x_i-y_i)^2}{\sum_i (y_i-\bar{y})^2}.
  \end{equation}
\end{itemize}
With \config{removeArcMean} the mean of each data column of each arc is reduced before.

With \config{perColumn} separate statistics for each selected data column are computed,
otherwise an overall value is computed.

See also \program{InstrumentArcStatistics}, \program{InstrumentStatisticsTimeSeries}.


\keepXColumns
\begin{tabularx}{\textwidth}{N T A}
\hline
Name & Type & Annotation\\
\hline
\hfuzz=500pt\includegraphics[width=1em]{element-mustset.pdf}~outputfileStatisticsTimeSeries & \hfuzz=500pt filename & \hfuzz=500pt statistics column(s) per arc, MISCVALUES\\
\hfuzz=500pt\includegraphics[width=1em]{element-mustset.pdf}~inputfileInstrument & \hfuzz=500pt filename & \hfuzz=500pt \\
\hfuzz=500pt\includegraphics[width=1em]{element-mustset.pdf}~inputfileInstrumentReference & \hfuzz=500pt filename & \hfuzz=500pt \\
\hfuzz=500pt\includegraphics[width=1em]{element-mustset.pdf}~statistics & \hfuzz=500pt choice & \hfuzz=500pt \\
\hfuzz=500pt\includegraphics[width=1em]{connector.pdf}\includegraphics[width=1em]{element-mustset.pdf}~correlation & \hfuzz=500pt  & \hfuzz=500pt \\
\hfuzz=500pt\includegraphics[width=1em]{connector.pdf}\includegraphics[width=1em]{element-mustset.pdf}~errorRMS & \hfuzz=500pt  & \hfuzz=500pt rms of differences\\
\hfuzz=500pt\includegraphics[width=1em]{connector.pdf}\includegraphics[width=1em]{element-mustset.pdf}~nashSutcliffe & \hfuzz=500pt  & \hfuzz=500pt with respect to reference field\\
\hfuzz=500pt\includegraphics[width=1em]{element.pdf}~removeArcMean & \hfuzz=500pt boolean & \hfuzz=500pt \\
\hfuzz=500pt\includegraphics[width=1em]{element.pdf}~startDataFields & \hfuzz=500pt uint & \hfuzz=500pt start\\
\hfuzz=500pt\includegraphics[width=1em]{element.pdf}~countDataFields & \hfuzz=500pt uint & \hfuzz=500pt number of data fields (default: all)\\
\hfuzz=500pt\includegraphics[width=1em]{element.pdf}~perColumn & \hfuzz=500pt boolean & \hfuzz=500pt compute statistic per column\\
\hline
\end{tabularx}

This program is \reference{parallelized}{general.parallelization}.
\clearpage
%==================================
\subsection{InstrumentArcStatistics}\label{InstrumentArcStatistics}
Computes statistics of selected data columns of \configFile{inputfileInstrument}{instrument} arc wise.
The \configFile{outputfileStatisticsTimeSeries}{instrument} contains for every arc one (mid) epoch
with statistics column(s). Possible statistics are root mean square, standard deviation,
mean, median, min, and max.

With \config{perColumn} separate statistics for each selected data column are computed,
otherwise an overall value is computed.

See also \program{InstrumentArcCrossStatistics}, \program{InstrumentStatisticsTimeSeries}.


\keepXColumns
\begin{tabularx}{\textwidth}{N T A}
\hline
Name & Type & Annotation\\
\hline
\hfuzz=500pt\includegraphics[width=1em]{element-mustset.pdf}~outputfileStatisticsTimeSeries & \hfuzz=500pt filename & \hfuzz=500pt columns: mjd, statistics column(s) per instrument file\\
\hfuzz=500pt\includegraphics[width=1em]{element-mustset.pdf}~inputfileInstrument & \hfuzz=500pt filename & \hfuzz=500pt \\
\hfuzz=500pt\includegraphics[width=1em]{element-mustset.pdf}~statistics & \hfuzz=500pt choice & \hfuzz=500pt \\
\hfuzz=500pt\includegraphics[width=1em]{connector.pdf}\includegraphics[width=1em]{element-mustset.pdf}~rootMeanSquare & \hfuzz=500pt  & \hfuzz=500pt \\
\hfuzz=500pt\includegraphics[width=1em]{connector.pdf}\includegraphics[width=1em]{element-mustset.pdf}~standardDeviation & \hfuzz=500pt  & \hfuzz=500pt \\
\hfuzz=500pt\includegraphics[width=1em]{connector.pdf}\includegraphics[width=1em]{element-mustset.pdf}~mean & \hfuzz=500pt  & \hfuzz=500pt \\
\hfuzz=500pt\includegraphics[width=1em]{connector.pdf}\includegraphics[width=1em]{element-mustset.pdf}~median & \hfuzz=500pt  & \hfuzz=500pt \\
\hfuzz=500pt\includegraphics[width=1em]{connector.pdf}\includegraphics[width=1em]{element-mustset.pdf}~min & \hfuzz=500pt  & \hfuzz=500pt \\
\hfuzz=500pt\includegraphics[width=1em]{connector.pdf}\includegraphics[width=1em]{element-mustset.pdf}~max & \hfuzz=500pt  & \hfuzz=500pt \\
\hfuzz=500pt\includegraphics[width=1em]{connector.pdf}\includegraphics[width=1em]{element-mustset.pdf}~epochCount & \hfuzz=500pt  & \hfuzz=500pt \\
\hfuzz=500pt\includegraphics[width=1em]{element.pdf}~startDataFields & \hfuzz=500pt uint & \hfuzz=500pt start\\
\hfuzz=500pt\includegraphics[width=1em]{element.pdf}~countDataFields & \hfuzz=500pt uint & \hfuzz=500pt number of data fields (default: all)\\
\hfuzz=500pt\includegraphics[width=1em]{element.pdf}~perColumn & \hfuzz=500pt boolean & \hfuzz=500pt compute statistic per column\\
\hfuzz=500pt\includegraphics[width=1em]{element.pdf}~ignoreNan & \hfuzz=500pt boolean & \hfuzz=500pt ignore NaN values in input\\
\hline
\end{tabularx}

This program is \reference{parallelized}{general.parallelization}.
\clearpage
%==================================
\subsection{InstrumentConcatenate}\label{InstrumentConcatenate}
This program concatenate the arcs from several \file{instrument files}{instrument}
and write it to a new \file{file}{instrument}. Input files must be of the same type.
The arcs are merged to one arc even though there is a gap inbetween.
To split the data into arcs use \program{InstrumentSynchronize}.
Three options are available: \config{sort}, \config{removeDuplicates} and \config{checkForNaNs}.
If \config{sort} is enabled, the program reads all files, no matter if they are sorted correctly in time, and
then sorts the epochs. If \config{removeDuplicates} is enabled, the program checks the whole data set
for epochs that are contained twice. And if \config{checkForNaNs} is enabled the data set is checked for
invalid epochs containing NaNs.


\keepXColumns
\begin{tabularx}{\textwidth}{N T A}
\hline
Name & Type & Annotation\\
\hline
\hfuzz=500pt\includegraphics[width=1em]{element-mustset.pdf}~outputfile & \hfuzz=500pt filename & \hfuzz=500pt \\
\hfuzz=500pt\includegraphics[width=1em]{element-mustset-unbounded.pdf}~inputfile & \hfuzz=500pt filename & \hfuzz=500pt \\
\hfuzz=500pt\includegraphics[width=1em]{element.pdf}~sort & \hfuzz=500pt boolean & \hfuzz=500pt sort epochs with increasing time\\
\hfuzz=500pt\includegraphics[width=1em]{element.pdf}~removeDuplicates & \hfuzz=500pt choice & \hfuzz=500pt remove duplicate epochs\\
\hfuzz=500pt\includegraphics[width=1em]{connector.pdf}\includegraphics[width=1em]{element-mustset.pdf}~keepFirst & \hfuzz=500pt sequence & \hfuzz=500pt keep first epoch with the same time stamp, remove all others\\
\hfuzz=500pt\quad\includegraphics[width=1em]{connector.pdf}\includegraphics[width=1em]{element.pdf}~margin & \hfuzz=500pt double & \hfuzz=500pt margin for identical times [seconds]\\
\hfuzz=500pt\includegraphics[width=1em]{connector.pdf}\includegraphics[width=1em]{element-mustset.pdf}~keepLast & \hfuzz=500pt sequence & \hfuzz=500pt keep last epoch with the same time stamp, remove all others\\
\hfuzz=500pt\quad\includegraphics[width=1em]{connector.pdf}\includegraphics[width=1em]{element.pdf}~margin & \hfuzz=500pt double & \hfuzz=500pt margin for identical times [seconds]\\
\hfuzz=500pt\includegraphics[width=1em]{element.pdf}~checkForNaNs & \hfuzz=500pt boolean & \hfuzz=500pt remove epochs with NaN values in one of the data fields\\
\hline
\end{tabularx}

\clearpage
%==================================
\subsection{InstrumentCovarianceCheck}\label{InstrumentCovarianceCheck}
This program checks \configFile{inputfileCovariance3d}{instrument}
3x3 covariance matrices if they are invertible or not and removes the invalid epochs.


\keepXColumns
\begin{tabularx}{\textwidth}{N T A}
\hline
Name & Type & Annotation\\
\hline
\hfuzz=500pt\includegraphics[width=1em]{element-mustset.pdf}~outputfileCovariance3d & \hfuzz=500pt filename & \hfuzz=500pt \\
\hfuzz=500pt\includegraphics[width=1em]{element-mustset.pdf}~inputfileCovariance3d & \hfuzz=500pt filename & \hfuzz=500pt \\
\hline
\end{tabularx}

This program is \reference{parallelized}{general.parallelization}.
\clearpage
%==================================
\subsection{InstrumentDetrend}\label{InstrumentDetrend}
Reduces \configClass{parametrizationTemporal}{parametrizationTemporalType} (e.g. const, trend, polynomial)
per arc from selected data columns of \configFile{inputfileInstrument}{instrument}
using a robust \reference{robust least squares adjustment}{fundamentals.robustLeastSquares}.

The \configFile{outputfileTimeSeriesArcParameters}{instrument} contains for every arc one (mid) epoch
with the estimated parameters. The order is: first all data (\verb|data0|, \verb|data1|, \ldots)
of first temporal parameter, followed by all data of the second temporal parameter and so on.


\keepXColumns
\begin{tabularx}{\textwidth}{N T A}
\hline
Name & Type & Annotation\\
\hline
\hfuzz=500pt\includegraphics[width=1em]{element-mustset.pdf}~outputfileInstrument & \hfuzz=500pt filename & \hfuzz=500pt detrended instrument time series\\
\hfuzz=500pt\includegraphics[width=1em]{element.pdf}~outputfileTimeSeriesArcParameters & \hfuzz=500pt filename & \hfuzz=500pt time series of estimated parameters per arc\\
\hfuzz=500pt\includegraphics[width=1em]{element-mustset.pdf}~inputfileInstrument & \hfuzz=500pt filename & \hfuzz=500pt \\
\hfuzz=500pt\includegraphics[width=1em]{element-mustset-unbounded.pdf}~parametrizationTemporal & \hfuzz=500pt \hyperref[parametrizationTemporalType]{parametrizationTemporal} & \hfuzz=500pt per arc, data is reduced by temporal representation\\
\hfuzz=500pt\includegraphics[width=1em]{element.pdf}~startDataFields & \hfuzz=500pt uint & \hfuzz=500pt start\\
\hfuzz=500pt\includegraphics[width=1em]{element.pdf}~countDataFields & \hfuzz=500pt uint & \hfuzz=500pt number of data fields (default: all after start)\\
\hfuzz=500pt\includegraphics[width=1em]{element.pdf}~huber & \hfuzz=500pt double & \hfuzz=500pt for robust least squares\\
\hfuzz=500pt\includegraphics[width=1em]{element.pdf}~huberPower & \hfuzz=500pt double & \hfuzz=500pt for robust least squares\\
\hfuzz=500pt\includegraphics[width=1em]{element.pdf}~huberMaxIteration & \hfuzz=500pt uint & \hfuzz=500pt (maximum) number of iterations for robust estimation\\
\hline
\end{tabularx}

This program is \reference{parallelized}{general.parallelization}.
\clearpage
%==================================
\subsection{InstrumentEarthRotation}\label{InstrumentEarthRotation}
Precompute Earth rotation matrix from celestial to terrestrial frame
and save as \file{StarCamera file}{instrument}.


\keepXColumns
\begin{tabularx}{\textwidth}{N T A}
\hline
Name & Type & Annotation\\
\hline
\hfuzz=500pt\includegraphics[width=1em]{element-mustset.pdf}~outputfileStarCamera & \hfuzz=500pt filename & \hfuzz=500pt rotation from CRF to TRF\\
\hfuzz=500pt\includegraphics[width=1em]{element-mustset.pdf}~earthRotation & \hfuzz=500pt \hyperref[earthRotationType]{earthRotation} & \hfuzz=500pt \\
\hfuzz=500pt\includegraphics[width=1em]{element-mustset-unbounded.pdf}~timeSeries & \hfuzz=500pt \hyperref[timeSeriesType]{timeSeries} & \hfuzz=500pt \\
\hline
\end{tabularx}

\clearpage
%==================================
\subsection{InstrumentEstimateEmpiricalCovariance}\label{InstrumentEstimateEmpiricalCovariance}
This program estimates the empirical auto- and cross-covariance of selected data columns per arc
of \configFile{inputfileInstrument}{instrument}.
The maximum computed lag is determined by the number of \configFile{outputfileCovarianceMatrix}{matrix} specified
(for a single output file only the auto-covariance is determined, for two output files auto- and cross-covariance is computed and so on).

Stationarity is assumed for the input time series, which means the temporal covariance matrix has Toeplitz structure.
\begin{equation}
\begin{bmatrix}
\Sigma & \Sigma_{\Delta_1} & \Sigma_{\Delta_2} & \Sigma_{\Delta_3} & \Sigma_{\Delta_4} \\
       & \Sigma            & \Sigma_{\Delta_1} & \Sigma_{\Delta_2} & \Sigma_{\Delta_3} \\
       &                   & \Sigma            & \Sigma_{\Delta_1} & \Sigma_{\Delta_2} \\
       &                   &                   & \Sigma            & \Sigma_{\Delta_1} \\
       &                   &                   &                   & \Sigma            \\
\end{bmatrix}
\end{equation}

The matrix for lag $h$ describes the covariance between $x_{t-h}$ and $x_{t}$, i.e. $\Sigma(t-h, t)$.

To get a reliable estimate, \program{InstrumentDetrend} should be called first.


\keepXColumns
\begin{tabularx}{\textwidth}{N T A}
\hline
Name & Type & Annotation\\
\hline
\hfuzz=500pt\includegraphics[width=1em]{element-mustset-unbounded.pdf}~outputfileCovarianceMatrix & \hfuzz=500pt filename & \hfuzz=500pt \\
\hfuzz=500pt\includegraphics[width=1em]{element-mustset.pdf}~inputfileInstrument & \hfuzz=500pt filename & \hfuzz=500pt \\
\hfuzz=500pt\includegraphics[width=1em]{element.pdf}~startDataFields & \hfuzz=500pt uint & \hfuzz=500pt start\\
\hfuzz=500pt\includegraphics[width=1em]{element.pdf}~countDataFields & \hfuzz=500pt uint & \hfuzz=500pt number of data fields (default: all after start)\\
\hline
\end{tabularx}

This program is \reference{parallelized}{general.parallelization}.
\clearpage
%==================================
\subsection{InstrumentEstimateHelmertTransformation}\label{InstrumentEstimateHelmertTransformation}
This program estimates a 3D Helmert transformation between two networks
(frame realizations, e.g. GNSS satellite or station network).
Each separate \config{data} represents a satellite/station/\ldots (e.g. 32 GPS satellites).
The instrument data (x,y,z position) considered can be set with \config{startData}.
The Helmert parameters are set up according to \configClass{parametrizationTemporal}{parametrizationTemporalType}
for each \configClass{timeIntervals}{timeSeriesType} and are estimated using a
\reference{robust least squares adjustment}{fundamentals.robustLeastSquares}.


\keepXColumns
\begin{tabularx}{\textwidth}{N T A}
\hline
Name & Type & Annotation\\
\hline
\hfuzz=500pt\includegraphics[width=1em]{element.pdf}~outputfileHelmertTimeSeries & \hfuzz=500pt filename & \hfuzz=500pt columns: mjd, Tx,Ty,Tz,s,Rx,Ry,Rz according to temporal parametrization\\
\hfuzz=500pt\includegraphics[width=1em]{element-mustset-unbounded.pdf}~data & \hfuzz=500pt sequence & \hfuzz=500pt e.g. satellite, station\\
\hfuzz=500pt\includegraphics[width=1em]{connector.pdf}\includegraphics[width=1em]{element.pdf}~outputfileInstrument & \hfuzz=500pt filename & \hfuzz=500pt transformed positions as instrument type Vector3d\\
\hfuzz=500pt\includegraphics[width=1em]{connector.pdf}\includegraphics[width=1em]{element.pdf}~outputfileInstrumentDiff & \hfuzz=500pt filename & \hfuzz=500pt position difference as instrument type Vector3d\\
\hfuzz=500pt\includegraphics[width=1em]{connector.pdf}\includegraphics[width=1em]{element-mustset.pdf}~inputfileInstrument & \hfuzz=500pt filename & \hfuzz=500pt \\
\hfuzz=500pt\includegraphics[width=1em]{connector.pdf}\includegraphics[width=1em]{element-mustset.pdf}~inputfileInstrumentReference & \hfuzz=500pt filename & \hfuzz=500pt \\
\hfuzz=500pt\includegraphics[width=1em]{connector.pdf}\includegraphics[width=1em]{element.pdf}~startDataFields & \hfuzz=500pt uint & \hfuzz=500pt start index of position (x,y,z) columns\\
\hfuzz=500pt\includegraphics[width=1em]{element-mustset-unbounded.pdf}~timeIntervals & \hfuzz=500pt \hyperref[timeSeriesType]{timeSeries} & \hfuzz=500pt parameters are estimated per interval\\
\hfuzz=500pt\includegraphics[width=1em]{element-mustset-unbounded.pdf}~parametrizationTemporal & \hfuzz=500pt \hyperref[parametrizationTemporalType]{parametrizationTemporal} & \hfuzz=500pt temporal parametrization\\
\hfuzz=500pt\includegraphics[width=1em]{element.pdf}~estimateShift & \hfuzz=500pt boolean & \hfuzz=500pt coordinate center\\
\hfuzz=500pt\includegraphics[width=1em]{element.pdf}~estimateScale & \hfuzz=500pt boolean & \hfuzz=500pt scale factor of position\\
\hfuzz=500pt\includegraphics[width=1em]{element.pdf}~estimateRotation & \hfuzz=500pt boolean & \hfuzz=500pt rotation\\
\hfuzz=500pt\includegraphics[width=1em]{element.pdf}~huber & \hfuzz=500pt double & \hfuzz=500pt for robust least squares\\
\hfuzz=500pt\includegraphics[width=1em]{element.pdf}~huberPower & \hfuzz=500pt double & \hfuzz=500pt for robust least squares\\
\hfuzz=500pt\includegraphics[width=1em]{element.pdf}~huberMaxIteration & \hfuzz=500pt uint & \hfuzz=500pt (maximum) number of iterations for robust estimation\\
\hline
\end{tabularx}

\clearpage
%==================================
\subsection{InstrumentFilter}\label{InstrumentFilter}
This program filter selected data columns of \configFile{inputfileInstrument}{instrument}
with \configClass{digitalFilter}{digitalFilterType} arc wise.


\keepXColumns
\begin{tabularx}{\textwidth}{N T A}
\hline
Name & Type & Annotation\\
\hline
\hfuzz=500pt\includegraphics[width=1em]{element-mustset.pdf}~outputfileInstrument & \hfuzz=500pt filename & \hfuzz=500pt \\
\hfuzz=500pt\includegraphics[width=1em]{element-mustset.pdf}~inputfileInstrument & \hfuzz=500pt filename & \hfuzz=500pt \\
\hfuzz=500pt\includegraphics[width=1em]{element-mustset-unbounded.pdf}~digitalFilter & \hfuzz=500pt \hyperref[digitalFilterType]{digitalFilter} & \hfuzz=500pt \\
\hfuzz=500pt\includegraphics[width=1em]{element.pdf}~startDataFields & \hfuzz=500pt uint & \hfuzz=500pt start\\
\hfuzz=500pt\includegraphics[width=1em]{element.pdf}~countDataFields & \hfuzz=500pt uint & \hfuzz=500pt number of data fields (default: all after start)\\
\hline
\end{tabularx}

This program is \reference{parallelized}{general.parallelization}.
\clearpage
%==================================
\subsection{InstrumentInsertNAN}\label{InstrumentInsertNAN}
This program inserts NAN epochs into \configFile{inputfileInstrument}{instrument} files,
either at specific \configClass{times}{timeSeriesType} or where gaps in the instrument are detected.


\keepXColumns
\begin{tabularx}{\textwidth}{N T A}
\hline
Name & Type & Annotation\\
\hline
\hfuzz=500pt\includegraphics[width=1em]{element-mustset.pdf}~outputfileInstrument & \hfuzz=500pt filename & \hfuzz=500pt \\
\hfuzz=500pt\includegraphics[width=1em]{element-mustset.pdf}~inputfileInstrument & \hfuzz=500pt filename & \hfuzz=500pt \\
\hfuzz=500pt\includegraphics[width=1em]{element-unbounded.pdf}~times & \hfuzz=500pt \hyperref[timeSeriesType]{timeSeries} & \hfuzz=500pt Insert NAN at specific times.\\
\hfuzz=500pt\includegraphics[width=1em]{element.pdf}~atGaps & \hfuzz=500pt boolean & \hfuzz=500pt Insert NAN where epochs are more than 1.5 times the median sampling apart.\\
\hfuzz=500pt\includegraphics[width=1em]{element.pdf}~atArcEnds & \hfuzz=500pt boolean & \hfuzz=500pt Insert one epoch with data NAN at arc ends\\
\hline
\end{tabularx}

This program is \reference{parallelized}{general.parallelization}.
\clearpage
%==================================
\subsection{InstrumentMultiplyAdd}\label{InstrumentMultiplyAdd}
This program multiply \file{instrument data}{instrument} with a factor and add them together.
Afterwards the mean of each arc and data column can be removed with \config{removeArcMean}.
The instrument files must be synchronized (\program{InstrumentSynchronize}).

See also \program{InstrumentArcCalculate}.


\keepXColumns
\begin{tabularx}{\textwidth}{N T A}
\hline
Name & Type & Annotation\\
\hline
\hfuzz=500pt\includegraphics[width=1em]{element-mustset.pdf}~outputfileInstrument & \hfuzz=500pt filename & \hfuzz=500pt \\
\hfuzz=500pt\includegraphics[width=1em]{element-mustset-unbounded.pdf}~instrument & \hfuzz=500pt sequence & \hfuzz=500pt \\
\hfuzz=500pt\includegraphics[width=1em]{connector.pdf}\includegraphics[width=1em]{element-mustset.pdf}~inputfileInstrument & \hfuzz=500pt filename & \hfuzz=500pt \\
\hfuzz=500pt\includegraphics[width=1em]{connector.pdf}\includegraphics[width=1em]{element.pdf}~factor & \hfuzz=500pt double & \hfuzz=500pt \\
\hfuzz=500pt\includegraphics[width=1em]{element.pdf}~removeArcMean & \hfuzz=500pt boolean & \hfuzz=500pt remove mean value of each arc\\
\hline
\end{tabularx}

This program is \reference{parallelized}{general.parallelization}.
\clearpage
%==================================
\subsection{InstrumentReduceSampling}\label{InstrumentReduceSampling}
This program reduce the sampling of a instrument file. Only epochs with a time stamp
with a division by \config{sampling} without remainder are kept (inside \config{margin}).


\keepXColumns
\begin{tabularx}{\textwidth}{N T A}
\hline
Name & Type & Annotation\\
\hline
\hfuzz=500pt\includegraphics[width=1em]{element-mustset.pdf}~outputfileInstrument & \hfuzz=500pt filename & \hfuzz=500pt \\
\hfuzz=500pt\includegraphics[width=1em]{element-mustset.pdf}~inputfileInstrument & \hfuzz=500pt filename & \hfuzz=500pt \\
\hfuzz=500pt\includegraphics[width=1em]{element-mustset.pdf}~sampling & \hfuzz=500pt double & \hfuzz=500pt new sampling in seconds\\
\hfuzz=500pt\includegraphics[width=1em]{element.pdf}~margin & \hfuzz=500pt double & \hfuzz=500pt margin around the new sampling in seconds\\
\hfuzz=500pt\includegraphics[width=1em]{element.pdf}~relative2FirstEpoch & \hfuzz=500pt boolean & \hfuzz=500pt compute sampling relative to time of first epoch\\
\hline
\end{tabularx}

\clearpage
%==================================
\subsection{InstrumentRemoveEpochsByCriteria}\label{InstrumentRemoveEpochsByCriteria}
This program removes epochs from \configFile{inputfileInstrument}{instrument}
by evaluating a set of \config{removalCriteria} expressions. For the data
columns the standard data variables are available,
see~\reference{dataVariables}{general.parser:dataVariables}.

The instrument data can be reduced by data from \configFile{inputfileInstrumentReference}{instrument}
prior to evaluation of the expressions.

To reduce the data by its median, use an expression like \verb|data1-data1mean|.
To remove epochs that deviate by more than 3 sigma use \verb|abs(data1)>3*data1std|
or \verb|abs(data0-data0median)>3*1.4826*data0mad|.

All arcs in the input instrument file are concatenated, meaning expressions
like \verb|data1mean| refer to the complete dataset. The removed epochs can be saved
in a separate \configFile{outputfileInstrumentRemovedEpochs}{instrument}.


\keepXColumns
\begin{tabularx}{\textwidth}{N T A}
\hline
Name & Type & Annotation\\
\hline
\hfuzz=500pt\includegraphics[width=1em]{element.pdf}~outputfileInstrument & \hfuzz=500pt filename & \hfuzz=500pt all data is stored in one arc\\
\hfuzz=500pt\includegraphics[width=1em]{element.pdf}~outputfileInstrumentRemovedEpochs & \hfuzz=500pt filename & \hfuzz=500pt all data is stored in one arc\\
\hfuzz=500pt\includegraphics[width=1em]{element-mustset.pdf}~inputfileInstrument & \hfuzz=500pt filename & \hfuzz=500pt arcs are concatenated for processing\\
\hfuzz=500pt\includegraphics[width=1em]{element.pdf}~inputfileInstrumentReference & \hfuzz=500pt filename & \hfuzz=500pt if given, the reference data is reduced prior to the expressions being evaluated\\
\hfuzz=500pt\includegraphics[width=1em]{element-mustset-unbounded.pdf}~removalCriteria & \hfuzz=500pt expression & \hfuzz=500pt epochs are removed if one criterion evaluates true. data0 is the first data field.\\
\hfuzz=500pt\includegraphics[width=1em]{element.pdf}~margin & \hfuzz=500pt double & \hfuzz=500pt remove data around identified epochs (on both sides) [seconds]\\
\hline
\end{tabularx}

\clearpage
%==================================
\subsection{InstrumentRemoveEpochsByTimes}\label{InstrumentRemoveEpochsByTimes}
This program compares an \file{instrument file}{instrument} with a
\configClass{time series}{timeSeriesType}.
Epochs contained within the time series (including a defined margin)
are removed from the instrument file. The margin is added on
both sides of the epochs. The arcs of the instrument file are
concatenated to one arc. The removed epochs can be saved
in a separate instrument file.


\keepXColumns
\begin{tabularx}{\textwidth}{N T A}
\hline
Name & Type & Annotation\\
\hline
\hfuzz=500pt\includegraphics[width=1em]{element.pdf}~outputfileInstrument & \hfuzz=500pt filename & \hfuzz=500pt all epochs are concatenated in one arc\\
\hfuzz=500pt\includegraphics[width=1em]{element.pdf}~outputfileInstrumentRemovedEpochs & \hfuzz=500pt filename & \hfuzz=500pt all epochs are concatenated in one arc\\
\hfuzz=500pt\includegraphics[width=1em]{element-mustset.pdf}~inputfileInstrument & \hfuzz=500pt filename & \hfuzz=500pt \\
\hfuzz=500pt\includegraphics[width=1em]{element-mustset-unbounded.pdf}~timePoints & \hfuzz=500pt \hyperref[timeSeriesType]{timeSeries} & \hfuzz=500pt \\
\hfuzz=500pt\includegraphics[width=1em]{element.pdf}~margin & \hfuzz=500pt double & \hfuzz=500pt margin size (on both sides) [seconds]\\
\hline
\end{tabularx}

\clearpage
%==================================
\subsection{InstrumentRemoveEpochsThruster}\label{InstrumentRemoveEpochsThruster}
This program remove epochs from an \file{instrument file}{instrument}.
The epochs are defined by a \file{thruster file}{instrument}
plus a defined margin before and after the thruster firings.
The arcs of the instrument file are concatenated to one arc.
The removed epochs can be saved in a separate instrument file.


\keepXColumns
\begin{tabularx}{\textwidth}{N T A}
\hline
Name & Type & Annotation\\
\hline
\hfuzz=500pt\includegraphics[width=1em]{element.pdf}~outputfileInstrument & \hfuzz=500pt filename & \hfuzz=500pt all epochs are concatenated in one arc\\
\hfuzz=500pt\includegraphics[width=1em]{element.pdf}~outputfileInstrumentRemovedEpochs & \hfuzz=500pt filename & \hfuzz=500pt all epochs are concatenated in one arc\\
\hfuzz=500pt\includegraphics[width=1em]{element-mustset.pdf}~inputfileInstrument & \hfuzz=500pt filename & \hfuzz=500pt \\
\hfuzz=500pt\includegraphics[width=1em]{element-mustset.pdf}~inputfileThruster & \hfuzz=500pt filename & \hfuzz=500pt THRUSTER\\
\hfuzz=500pt\includegraphics[width=1em]{element.pdf}~marginBefore & \hfuzz=500pt double & \hfuzz=500pt margin before start of firing [seconds]\\
\hfuzz=500pt\includegraphics[width=1em]{element.pdf}~marginAfter & \hfuzz=500pt double & \hfuzz=500pt margin after end of firing [seconds]\\
\hline
\end{tabularx}

\clearpage
%==================================
\subsection{InstrumentResample}\label{InstrumentResample}
This program resamples \file{instrument data}{instrument} to a given
\configClass{timeSeries}{timeSeriesType} using a resampling
\configClass{method}{interpolatorTimeSeriesType}.

This program can also be used to reduce the sampling of an instrument file,
but a better way to reduce the sampling of noisy data with regular sampling
is to use a low pass filter first with \program{InstrumentFilter} and then thin
out the data with \program{InstrumentReduceSampling}.


\keepXColumns
\begin{tabularx}{\textwidth}{N T A}
\hline
Name & Type & Annotation\\
\hline
\hfuzz=500pt\includegraphics[width=1em]{element-mustset.pdf}~outputfileInstrument & \hfuzz=500pt filename & \hfuzz=500pt \\
\hfuzz=500pt\includegraphics[width=1em]{element-mustset.pdf}~inputfileInstrument & \hfuzz=500pt filename & \hfuzz=500pt \\
\hfuzz=500pt\includegraphics[width=1em]{element-mustset.pdf}~method & \hfuzz=500pt \hyperref[interpolatorTimeSeriesType]{interpolatorTimeSeries} & \hfuzz=500pt resampling method\\
\hfuzz=500pt\includegraphics[width=1em]{element-mustset-unbounded.pdf}~timeSeries & \hfuzz=500pt \hyperref[timeSeriesType]{timeSeries} & \hfuzz=500pt resampled points in time\\
\hline
\end{tabularx}

\clearpage
%==================================
\subsection{InstrumentRotate}\label{InstrumentRotate}
This program rotates \file{instrument data}{instrument} into a new reference frame
(using \configFile{inputfileStarCamera}{instrument}).
The rotation is usually done from satellite frame into inertial frame.
To apply Earth rotation use \program{InstrumentEarthRotation}.


\keepXColumns
\begin{tabularx}{\textwidth}{N T A}
\hline
Name & Type & Annotation\\
\hline
\hfuzz=500pt\includegraphics[width=1em]{element-mustset.pdf}~outputfileInstrument & \hfuzz=500pt filename & \hfuzz=500pt \\
\hfuzz=500pt\includegraphics[width=1em]{element-mustset.pdf}~inputfileInstrument & \hfuzz=500pt filename & \hfuzz=500pt \\
\hfuzz=500pt\includegraphics[width=1em]{element-mustset.pdf}~inputfileStarCamera & \hfuzz=500pt filename & \hfuzz=500pt \\
\hfuzz=500pt\includegraphics[width=1em]{element.pdf}~inverseRotate & \hfuzz=500pt boolean & \hfuzz=500pt \\
\hline
\end{tabularx}

This program is \reference{parallelized}{general.parallelization}.
\clearpage
%==================================
\subsection{InstrumentSetType}\label{InstrumentSetType}
Convert \file{instrument data}{instrument} into instrument data with new \configClass{type}{instrumentTypeType}.
The selected number of data columns must agree with the \configClass{type}{instrumentTypeType}.


\keepXColumns
\begin{tabularx}{\textwidth}{N T A}
\hline
Name & Type & Annotation\\
\hline
\hfuzz=500pt\includegraphics[width=1em]{element-mustset.pdf}~outputfileInstrument & \hfuzz=500pt filename & \hfuzz=500pt \\
\hfuzz=500pt\includegraphics[width=1em]{element-mustset.pdf}~inputfileInstrument & \hfuzz=500pt filename & \hfuzz=500pt \\
\hfuzz=500pt\includegraphics[width=1em]{element.pdf}~type & \hfuzz=500pt \hyperref[instrumentTypeType]{instrumentType} & \hfuzz=500pt \\
\hfuzz=500pt\includegraphics[width=1em]{element.pdf}~startDataFields & \hfuzz=500pt uint & \hfuzz=500pt start\\
\hfuzz=500pt\includegraphics[width=1em]{element.pdf}~countDataFields & \hfuzz=500pt uint & \hfuzz=500pt number of data fields (default: all after start)\\
\hline
\end{tabularx}

\clearpage
%==================================
\subsection{InstrumentStarCamera2AccAngularRate}\label{InstrumentStarCamera2AccAngularRate}
This program derivate from a time series of quaternions
a series of angular rates and angular accelerations.
The derivatives are computed by a polynomial interpolation
with \config{interpolationDegree} of the quaternions.


\keepXColumns
\begin{tabularx}{\textwidth}{N T A}
\hline
Name & Type & Annotation\\
\hline
\hfuzz=500pt\includegraphics[width=1em]{element.pdf}~outputfileAngularRate & \hfuzz=500pt filename & \hfuzz=500pt [rad/s], VECTOR3D\\
\hfuzz=500pt\includegraphics[width=1em]{element.pdf}~outputfileAngularAcc & \hfuzz=500pt filename & \hfuzz=500pt [rad/s**2], VECTOR3D\\
\hfuzz=500pt\includegraphics[width=1em]{element-mustset.pdf}~inputfileStarCamera & \hfuzz=500pt filename & \hfuzz=500pt \\
\hfuzz=500pt\includegraphics[width=1em]{element.pdf}~interpolationDegree & \hfuzz=500pt uint & \hfuzz=500pt derivation by polynomial interpolation of degree n\\
\hline
\end{tabularx}

\clearpage
%==================================
\subsection{InstrumentStarCamera2RollPitchYaw}\label{InstrumentStarCamera2RollPitchYaw}
Compute roll, pitch, yaw angles from \configFile{inputfileStarCamera}{instrument} data.
Optional the angles are computed relative
to a \configFile{inputfileStarCameraReference}{instrument}.

See also \program{SimulateStarCamera}.


\keepXColumns
\begin{tabularx}{\textwidth}{N T A}
\hline
Name & Type & Annotation\\
\hline
\hfuzz=500pt\includegraphics[width=1em]{element-mustset.pdf}~outputfileInstrument & \hfuzz=500pt filename & \hfuzz=500pt roll, pitch, yaw [rad], VECTOR3D\\
\hfuzz=500pt\includegraphics[width=1em]{element-mustset.pdf}~inputfileStarCamera & \hfuzz=500pt filename & \hfuzz=500pt \\
\hfuzz=500pt\includegraphics[width=1em]{element.pdf}~inputfileStarCameraReference & \hfuzz=500pt filename & \hfuzz=500pt nominal orientation\\
\hline
\end{tabularx}

This program is \reference{parallelized}{general.parallelization}.
\clearpage
%==================================
\subsection{InstrumentStarCamera2RotaryMatrix}\label{InstrumentStarCamera2RotaryMatrix}
Write \configFile{inputfileStarCamera}{instrument} rotations
as \configFile{outputfileInstrument}{instrument} rotary matrices
(for each epoch $xx, xy, xz, yx, yy, yz, zx, zy, zz$).


\keepXColumns
\begin{tabularx}{\textwidth}{N T A}
\hline
Name & Type & Annotation\\
\hline
\hfuzz=500pt\includegraphics[width=1em]{element-mustset.pdf}~outputfileInstrument & \hfuzz=500pt filename & \hfuzz=500pt xx, xy, xz, yx, yy, yz, zx, zy, zz (MISCVALUES)\\
\hfuzz=500pt\includegraphics[width=1em]{element-mustset.pdf}~inputfileStarCamera & \hfuzz=500pt filename & \hfuzz=500pt \\
\hline
\end{tabularx}

This program is \reference{parallelized}{general.parallelization}.
\clearpage
%==================================
\subsection{InstrumentStarCameraMultiply}\label{InstrumentStarCameraMultiply}
This program applies several rotations given by \configFile{inputfileStarCamera}{instrument}.
The resulting rotation is written as \configFile{outputfileStarCamera}{instrument}.
All instrument files must be synchronized (\program{InstrumentSynchronize}).


\keepXColumns
\begin{tabularx}{\textwidth}{N T A}
\hline
Name & Type & Annotation\\
\hline
\hfuzz=500pt\includegraphics[width=1em]{element-mustset.pdf}~outputfileStarCamera & \hfuzz=500pt filename & \hfuzz=500pt \\
\hfuzz=500pt\includegraphics[width=1em]{element-mustset-unbounded.pdf}~instrument & \hfuzz=500pt sequence & \hfuzz=500pt \\
\hfuzz=500pt\includegraphics[width=1em]{connector.pdf}\includegraphics[width=1em]{element-mustset.pdf}~inputfileStarCamera & \hfuzz=500pt filename & \hfuzz=500pt \\
\hfuzz=500pt\includegraphics[width=1em]{connector.pdf}\includegraphics[width=1em]{element.pdf}~inverse & \hfuzz=500pt boolean & \hfuzz=500pt \\
\hline
\end{tabularx}

\clearpage
%==================================
\subsection{InstrumentStatisticsTimeSeries}\label{InstrumentStatisticsTimeSeries}
This program computes a time series of statistics for one or more instrument files.
Possible statistics are root mean square, standard deviation, mean, median, min, and max.
The columns of the output time series are defined either as one per \configFile{inputfileInstrument}{instrument}
or, if \config{perColumn} is true, statistics are computed per column for each file.
Providing e.g. 32 orbit files of GPS satellites results in a time series matrix
with columns: mjd, statisticsG01, statisticsG02, ..., statisticsG32.
If \config{intervals} are provided, the input data is split into these intervals
and one statistic is computed per interval. Otherwise, overall statistics are computed.
The instrument data considered for computation of the component-wise statistics
can be set with \config{startDataFields} and \config{countDataFields}.
The \config{factor} can be set to e.g. sqrt(3) to get 3D instead of 1D RMS values.

See also \program{InstrumentArcStatistics}, \program{InstrumentArcCrossStatistics}.


\keepXColumns
\begin{tabularx}{\textwidth}{N T A}
\hline
Name & Type & Annotation\\
\hline
\hfuzz=500pt\includegraphics[width=1em]{element-mustset.pdf}~outputfileStatisticsTimeSeries & \hfuzz=500pt filename & \hfuzz=500pt columns: mjd, statistics column(s) per instrument file\\
\hfuzz=500pt\includegraphics[width=1em]{element-mustset-unbounded.pdf}~inputfileInstrument & \hfuzz=500pt filename & \hfuzz=500pt \\
\hfuzz=500pt\includegraphics[width=1em]{element-mustset.pdf}~statistics & \hfuzz=500pt choice & \hfuzz=500pt \\
\hfuzz=500pt\includegraphics[width=1em]{connector.pdf}\includegraphics[width=1em]{element-mustset.pdf}~rootMeanSquare & \hfuzz=500pt  & \hfuzz=500pt \\
\hfuzz=500pt\includegraphics[width=1em]{connector.pdf}\includegraphics[width=1em]{element-mustset.pdf}~standardDeviation & \hfuzz=500pt  & \hfuzz=500pt \\
\hfuzz=500pt\includegraphics[width=1em]{connector.pdf}\includegraphics[width=1em]{element-mustset.pdf}~mean & \hfuzz=500pt  & \hfuzz=500pt \\
\hfuzz=500pt\includegraphics[width=1em]{connector.pdf}\includegraphics[width=1em]{element-mustset.pdf}~median & \hfuzz=500pt  & \hfuzz=500pt \\
\hfuzz=500pt\includegraphics[width=1em]{connector.pdf}\includegraphics[width=1em]{element-mustset.pdf}~sum & \hfuzz=500pt  & \hfuzz=500pt \\
\hfuzz=500pt\includegraphics[width=1em]{connector.pdf}\includegraphics[width=1em]{element-mustset.pdf}~min & \hfuzz=500pt  & \hfuzz=500pt \\
\hfuzz=500pt\includegraphics[width=1em]{connector.pdf}\includegraphics[width=1em]{element-mustset.pdf}~max & \hfuzz=500pt  & \hfuzz=500pt \\
\hfuzz=500pt\includegraphics[width=1em]{connector.pdf}\includegraphics[width=1em]{element-mustset.pdf}~epochCount & \hfuzz=500pt  & \hfuzz=500pt \\
\hfuzz=500pt\includegraphics[width=1em]{element.pdf}~startDataFields & \hfuzz=500pt uint & \hfuzz=500pt start\\
\hfuzz=500pt\includegraphics[width=1em]{element.pdf}~countDataFields & \hfuzz=500pt uint & \hfuzz=500pt number of data fields (default: all)\\
\hfuzz=500pt\includegraphics[width=1em]{element.pdf}~perColumn & \hfuzz=500pt boolean & \hfuzz=500pt compute statistic per column\\
\hfuzz=500pt\includegraphics[width=1em]{element.pdf}~ignoreNan & \hfuzz=500pt boolean & \hfuzz=500pt ignore NaN values in statistic computation\\
\hfuzz=500pt\includegraphics[width=1em]{element-unbounded.pdf}~intervals & \hfuzz=500pt \hyperref[timeSeriesType]{timeSeries} & \hfuzz=500pt intervals for statistics computation (one statistic per interval)\\
\hfuzz=500pt\includegraphics[width=1em]{element.pdf}~factor & \hfuzz=500pt double & \hfuzz=500pt e.g. sqrt(3) for 3D RMS\\
\hline
\end{tabularx}

\clearpage
%==================================
\subsection{InstrumentSynchronize}\label{InstrumentSynchronize}
This program reads several \file{instrument files}{instrument} and synchronize the data.
Every epoch with some missing data will be deleted so the remaining epochs
have data from every instrument.

In a second step the epochs are divided into arcs with maximal epochs
(or \config{maxArcLen}) without having a gap inside an arc.
A Gap is defined by a time step with at least \config{minGap} seconds
between consecutive epochs or if not set the 1.5 of the median sampling.
Arc with an epoch count less than \config{minArcLen} will be rejected.

A specific region can be selected with \configClass{border}{borderType}.
In this case one of the instrument data must an orbit.

If \configClass{timeIntervals}{timeSeriesType} is given the data are also divided into time bins.
The assignment of arcs to the bins can be saved in \configFile{outputfileArcList}{arcList}.
This file can be used for the variational equation approach or \program{KalmanBuildNormals}.

Instrument files from \config{irregularData} are not synchronized but
divided into the same number of arcs within the same time intervals.
Data outside the defined arcs will be deleted.


\keepXColumns
\begin{tabularx}{\textwidth}{N T A}
\hline
Name & Type & Annotation\\
\hline
\hfuzz=500pt\includegraphics[width=1em]{element-mustset-unbounded.pdf}~data & \hfuzz=500pt sequence & \hfuzz=500pt \\
\hfuzz=500pt\includegraphics[width=1em]{connector.pdf}\includegraphics[width=1em]{element.pdf}~outputfileInstrument & \hfuzz=500pt filename & \hfuzz=500pt \\
\hfuzz=500pt\includegraphics[width=1em]{connector.pdf}\includegraphics[width=1em]{element-mustset.pdf}~inputfileInstrument & \hfuzz=500pt filename & \hfuzz=500pt \\
\hfuzz=500pt\includegraphics[width=1em]{element.pdf}~margin & \hfuzz=500pt double & \hfuzz=500pt margin for identical times [seconds]\\
\hfuzz=500pt\includegraphics[width=1em]{element.pdf}~minGap & \hfuzz=500pt double & \hfuzz=500pt minimal time to define a gap and to begin a new arc, 0: no dividing [seconds], if not set 1.5*median sampling is used\\
\hfuzz=500pt\includegraphics[width=1em]{element.pdf}~minArcLength & \hfuzz=500pt uint & \hfuzz=500pt minimal number of epochs of an arc\\
\hfuzz=500pt\includegraphics[width=1em]{element.pdf}~maxArcLength & \hfuzz=500pt uint & \hfuzz=500pt maximal number of epochs of an arc\\
\hfuzz=500pt\includegraphics[width=1em]{element.pdf}~arcType & \hfuzz=500pt choice & \hfuzz=500pt all arcs or only ascending or descending arcs are selected\\
\hfuzz=500pt\includegraphics[width=1em]{connector.pdf}\includegraphics[width=1em]{element-mustset.pdf}~ascending & \hfuzz=500pt  & \hfuzz=500pt \\
\hfuzz=500pt\includegraphics[width=1em]{connector.pdf}\includegraphics[width=1em]{element-mustset.pdf}~descending & \hfuzz=500pt  & \hfuzz=500pt \\
\hfuzz=500pt\includegraphics[width=1em]{element-unbounded.pdf}~border & \hfuzz=500pt \hyperref[borderType]{border} & \hfuzz=500pt only data in a specific region is selected\\
\hfuzz=500pt\includegraphics[width=1em]{element-unbounded.pdf}~timeIntervals & \hfuzz=500pt \hyperref[timeSeriesType]{timeSeries} & \hfuzz=500pt divide data into time bins\\
\hfuzz=500pt\includegraphics[width=1em]{element.pdf}~outputfileArcList & \hfuzz=500pt filename & \hfuzz=500pt arc and time bin mapping\\
\hfuzz=500pt\includegraphics[width=1em]{element-unbounded.pdf}~irregularData & \hfuzz=500pt sequence & \hfuzz=500pt instrument files with irregular sampling\\
\hfuzz=500pt\includegraphics[width=1em]{connector.pdf}\includegraphics[width=1em]{element.pdf}~outputfileInstrument & \hfuzz=500pt filename & \hfuzz=500pt \\
\hfuzz=500pt\includegraphics[width=1em]{connector.pdf}\includegraphics[width=1em]{element-mustset.pdf}~inputfileInstrument & \hfuzz=500pt filename & \hfuzz=500pt \\
\hfuzz=500pt\includegraphics[width=1em]{connector.pdf}\includegraphics[width=1em]{element.pdf}~minArcLength & \hfuzz=500pt uint & \hfuzz=500pt minimal number of epochs in an arc\\
\hline
\end{tabularx}

\clearpage
%==================================
\subsection{InstrumentWaveletDecomposition}\label{InstrumentWaveletDecomposition}
This program performs a multilevel one-dimensional wavelet analysis on one \config{selectDataField}
data column of \configFile{inputfileInstrument}{instrument}.
The \configFile{outputfileInstrument}{instrument} contains the decomposed levels in time domain ${a_J,d_J,...,d_1}$


\keepXColumns
\begin{tabularx}{\textwidth}{N T A}
\hline
Name & Type & Annotation\\
\hline
\hfuzz=500pt\includegraphics[width=1em]{element-mustset.pdf}~outputfileInstrument & \hfuzz=500pt filename & \hfuzz=500pt MISCVALUES, decomposed levels in time domain a\_J,d\_J,...,d\_1\\
\hfuzz=500pt\includegraphics[width=1em]{element-mustset.pdf}~inputfileInstrument & \hfuzz=500pt filename & \hfuzz=500pt \\
\hfuzz=500pt\includegraphics[width=1em]{element.pdf}~selectDataField & \hfuzz=500pt uint & \hfuzz=500pt select a data column for decomposition\\
\hfuzz=500pt\includegraphics[width=1em]{element-mustset.pdf}~inputfileWavelet & \hfuzz=500pt filename & \hfuzz=500pt wavelet coefficients\\
\hfuzz=500pt\includegraphics[width=1em]{element-mustset.pdf}~level & \hfuzz=500pt uint & \hfuzz=500pt level of decomposition\\
\hline
\end{tabularx}

\clearpage
%==================================
\subsection{LocalLevelFrame2StarCamera}\label{LocalLevelFrame2StarCamera}
Compute rotation (\file{StarCamera file}{instrument}) from local level frame (ellipsoidal north, east, down)
to TRF for positions given in \configFile{inputfileInstrument}{instrument} (first 3 data columns).


\keepXColumns
\begin{tabularx}{\textwidth}{N T A}
\hline
Name & Type & Annotation\\
\hline
\hfuzz=500pt\includegraphics[width=1em]{element-mustset.pdf}~outputfileStarCamera & \hfuzz=500pt filename & \hfuzz=500pt rotation matrix from local level frame (ellipsoidal north, east, down) to TRF\\
\hfuzz=500pt\includegraphics[width=1em]{element-mustset.pdf}~inputfileInstrument & \hfuzz=500pt filename & \hfuzz=500pt origin of local level frame\\
\hfuzz=500pt\includegraphics[width=1em]{element.pdf}~constantOriginPerArc & \hfuzz=500pt boolean & \hfuzz=500pt use constant origin for all epochs of an arc (median position)\\
\hfuzz=500pt\includegraphics[width=1em]{element.pdf}~R & \hfuzz=500pt double & \hfuzz=500pt reference radius for ellipsoidal coordinates\\
\hfuzz=500pt\includegraphics[width=1em]{element.pdf}~inverseFlattening & \hfuzz=500pt double & \hfuzz=500pt reference flattening for ellipsoidal coordinates, 0: spherical coordinates\\
\hline
\end{tabularx}

This program is \reference{parallelized}{general.parallelization}.
\clearpage
%==================================
\section{Programs: KalmanFilter}
\subsection{KalmanBuildNormals}\label{KalmanBuildNormals}
This program sets up normal equations based on \configClass{observation}{observationType}
for short-term gravity field variations.
It computes the normal equations based on the intervals $i \in \{1, ..., N\}$ given in the \configFile{arcList}{arcList}.
It sets up the least squares adjustment
\begin{equation}
    \begin{bmatrix}
    \mathbf{l}_1 \\
    \mathbf{l}_2 \\
    \vdots \\
    \mathbf{l}_N \\
  \end{bmatrix}
  =
  \begin{bmatrix}
    \mathbf{A}_1  &  & & \\
    & \mathbf{A}_2  & &\\
    &  & \ddots & \\
    & & & \mathbf{A}_N \\
  \end{bmatrix}
  \begin{bmatrix}
    \mathbf{x}^{(1)} \\
    \mathbf{x}^{(2)} \\
    \vdots \\
    \mathbf{x}^{(N)} \\
  \end{bmatrix}
  +
  \begin{bmatrix}
    \mathbf{e}_1 \\
    \mathbf{e}_2 \\
    \vdots \\
    \mathbf{e}_N \\
  \end{bmatrix},
\end{equation}
and subsequently computes the normal equations $\mathbf{N}_i, \mathbf{n}_i$ for each interval.
If \config{eliminateNonGravityParameters} is true, all non-gravity parameters are eliminated before the normals
are written to \configFile{outputfileNormalEquation}{normalEquation}.
For each time interval in \config{arcList} a single \file{normal equation file}{normalEquation} is written.

This program computes the input normals for \program{KalmanFilter} and \program{KalmanSmootherLeastSquares}.


\keepXColumns
\begin{tabularx}{\textwidth}{N T A}
\hline
Name & Type & Annotation\\
\hline
\hfuzz=500pt\includegraphics[width=1em]{element-mustset.pdf}~outputfileNormalEquation & \hfuzz=500pt filename & \hfuzz=500pt outputfile for normal equations\\
\hfuzz=500pt\includegraphics[width=1em]{element-mustset.pdf}~observation & \hfuzz=500pt \hyperref[observationType]{observation} & \hfuzz=500pt \\
\hfuzz=500pt\includegraphics[width=1em]{element-mustset.pdf}~inputfileArcList & \hfuzz=500pt filename & \hfuzz=500pt list to correspond points of time to arc numbers\\
\hfuzz=500pt\includegraphics[width=1em]{element.pdf}~eliminateNonGravityParameters & \hfuzz=500pt boolean & \hfuzz=500pt eliminate additional parameters from normals, 0: all parameter are saved\\
\hline
\end{tabularx}

This program is \reference{parallelized}{general.parallelization}.
\clearpage
%==================================
\subsection{KalmanFilter}\label{KalmanFilter}
The program computes time variable gravity fields using the Kalman filter approach of

Kurtenbach, E., Eicker, A., Mayer-Gürr, T., Holschneider, M., Hayn, M., Fuhrmann, M., and Kusche, J. (2012).
Improved daily GRACE gravity field solutions using a Kalman smoother. Journal of Geodynamics, 59–60, 39–48.
\url{https://doi.org/10.1016/j.jog.2012.02.006}.

The updated state $\mathbf{x}_t^+$ is determined by solving the least squares adjustment
\begin{equation}
\mathbf{l}_t = \mathbf{A}_t \mathbf{x}_t + \mathbf{e}_t \hspace{25pt} \mathbf{e}_t \sim \mathcal{N}(0, \mathbf{R}_t)\\
\mathbf{B} \mathbf{x}^+_{t-1} = \mathbf{I} \mathbf{x}_t + \mathbf{v}_t\hspace{25pt} \mathbf{v} \sim \mathcal{N}(0,\mathbf{Q} + \mathbf{B} \mathbf{P}^+_{t-1}\mathbf{B}^T).
\end{equation}
In normal equation form this can be written as
\begin{equation}
\hat{\mathbf{x}}_t = \mathbf{x}^+_t = (\mathbf{N}_t + \mathbf{P}^{-^{-1}}_t)^{-1}(\mathbf{n}_t + \mathbf{P}^{-^{-1}}_t \mathbf{x}^-_t),
\end{equation}
where $\mathbf{x}_t^- = \mathbf{B} \mathbf{x}^+_{t-1}$ and $\mathbf{P}_t^{-} = \mathbf{Q} + \mathbf{B} \mathbf{P}^+_{t-1}\mathbf{B}^T$
are the predicted state and its covariance matrix.

The process dynamic $\mathbf{B}, \mathbf{Q}$ is represented as an \reference{autoregressive model}{fundamentals.autoregressiveModel},
and passed to the program through \configFile{inputfileAutoregressiveModel}{matrix}.
The sequence of normal equations $\mathbf{N}_t, \mathbf{n}_t$ are given as list of \configFile{inputfileNormalEquations}{normalEquation},
which can be generated using \configClass{loops}{loopType}.
In the same way, the \file{matrix files}{matrix} for \config{outputfileUpdatedState} and \config{inputfileUpdatedStateCovariance}
can also be specified using \configClass{loops}{loopType}.

If no \configFile{inputfileInitialState}{matrix} is set, a zero vector with appropriate dimensions is used.
The \configFile{inputfileInitialStateCovarianceMatrix}{matrix} however must be given.

See also \program{KalmanBuildNormals}, \program{KalmanSmoother}.


\keepXColumns
\begin{tabularx}{\textwidth}{N T A}
\hline
Name & Type & Annotation\\
\hline
\hfuzz=500pt\includegraphics[width=1em]{element-mustset-unbounded.pdf}~outputfileUpdatedState & \hfuzz=500pt filename & \hfuzz=500pt estimated state x+ (nx1-matrix)\\
\hfuzz=500pt\includegraphics[width=1em]{element-unbounded.pdf}~outputfileUpdatedStateCovarianceMatrix & \hfuzz=500pt filename & \hfuzz=500pt estimated state' s covariance matrix Cov(x+)\\
\hfuzz=500pt\includegraphics[width=1em]{element-mustset-unbounded.pdf}~inputfileNormalEquations & \hfuzz=500pt filename & \hfuzz=500pt normal equations input file\\
\hfuzz=500pt\includegraphics[width=1em]{element.pdf}~inputfileInitialState & \hfuzz=500pt filename & \hfuzz=500pt initial state x0\\
\hfuzz=500pt\includegraphics[width=1em]{element-mustset.pdf}~inputfileInitialStateCovarianceMatrix & \hfuzz=500pt filename & \hfuzz=500pt initial state's covariance matrix Cov(x0)\\
\hfuzz=500pt\includegraphics[width=1em]{element-mustset.pdf}~inputfileAutoregressiveModel & \hfuzz=500pt filename & \hfuzz=500pt file name of autoregressive model\\
\hline
\end{tabularx}

\clearpage
%==================================
\subsection{KalmanSmoother}\label{KalmanSmoother}
Apply the Rauch-Tung-Striebel smoother to a gravity field time series computed by \program{KalmanFilter}.
This is the implementation of the approach presented in

Kurtenbach, E., Eicker, A., Mayer-Gürr, T., Holschneider, M., Hayn, M., Fuhrmann, M., and Kusche, J. (2012).
Improved daily GRACE gravity field solutions using a Kalman smoother. Journal of Geodynamics, 59–60, 39–48.
\url{https://doi.org/10.1016/j.jog.2012.02.006}.

The result has zero phase and the squared magnitude response of \configFile{inputfileAutoregressiveModel}{matrix}
(see \reference{autoregressiveModel}{fundamentals.autoregressiveModel} for details).
\configFile{inputfileUpdatedState}{matrix} and \configFile{inputfileUpdatedStateCovariance}{matrix}
are the output of a \program{KalmanFilter} forward sweep.
The matrix files for\configFile{outputfileUpdatedState}{matrix}, \configFile{inputfileUpdatedState}{matrix}
and \configFile{inputfileUpdatedStateCovariance}{matrix} can also be specified using \configClass{loops}{loopType}.

See also \program{KalmanBuildNormals}, \program{KalmanFilter} and \program{KalmanSmootherLeastSquares}.


\keepXColumns
\begin{tabularx}{\textwidth}{N T A}
\hline
Name & Type & Annotation\\
\hline
\hfuzz=500pt\includegraphics[width=1em]{element-mustset-unbounded.pdf}~outputfileState & \hfuzz=500pt filename & \hfuzz=500pt estimated parameters (nx1-matrix)\\
\hfuzz=500pt\includegraphics[width=1em]{element-unbounded.pdf}~outputfileStateCovarianceMatrix & \hfuzz=500pt filename & \hfuzz=500pt estimated parameters' covariance matrix\\
\hfuzz=500pt\includegraphics[width=1em]{element-mustset-unbounded.pdf}~inputfileUpdatedState & \hfuzz=500pt filename & \hfuzz=500pt \\
\hfuzz=500pt\includegraphics[width=1em]{element-mustset-unbounded.pdf}~inputfileUpdatedStateCovarianceMatrix & \hfuzz=500pt filename & \hfuzz=500pt \\
\hfuzz=500pt\includegraphics[width=1em]{element-mustset.pdf}~inputfileAutoregressiveModel & \hfuzz=500pt filename & \hfuzz=500pt file name of autoregressive model\\
\hline
\end{tabularx}

\clearpage
%==================================
\subsection{KalmanSmootherLeastSquares}\label{KalmanSmootherLeastSquares}
This program estimates temporal gravity field variations with a constraint least squares adjustment.
Prior information is introduced by means of a \configClass{autoregressiveModelSequence}{autoregressiveModelSequenceType}
which represent a stationary random process (see the \reference{autoregressive model description}{fundamentals.autoregressiveModel}) for details.

The output files for the estimated gravity field (\configFile{outputfileSolution}{matrix}), the
corresponding standard deviations (\configFile{outputfileSigmax}{matrix}) and the full covariance matrix
(\configFile{outputfileCovariance}{matrix}) can be specified using \configClass{loops}{loopType}.
Similarly, the \configFile{inputfileNormalEquations}{normalEquation}
can also be specified using \configClass{loops}{loopType}.

See also \program{KalmanBuildNormals}, \program{KalmanFilter} and\program{KalmanSmoother}


\keepXColumns
\begin{tabularx}{\textwidth}{N T A}
\hline
Name & Type & Annotation\\
\hline
\hfuzz=500pt\includegraphics[width=1em]{element-mustset-unbounded.pdf}~outputfileSolution & \hfuzz=500pt filename & \hfuzz=500pt file name of solution vector (use time tags)\\
\hfuzz=500pt\includegraphics[width=1em]{element-unbounded.pdf}~outputfileSigmax & \hfuzz=500pt filename & \hfuzz=500pt file name of sigma vector (use time tags)\\
\hfuzz=500pt\includegraphics[width=1em]{element-unbounded.pdf}~outputfileCovariance & \hfuzz=500pt filename & \hfuzz=500pt file name of full covariance matrix (use time tags)\\
\hfuzz=500pt\includegraphics[width=1em]{element-mustset-unbounded.pdf}~inputfileNormalEquations & \hfuzz=500pt filename & \hfuzz=500pt input normal equations (loopTime will be expanded)\\
\hfuzz=500pt\includegraphics[width=1em]{element-mustset.pdf}~autoregressiveModelSequence & \hfuzz=500pt \hyperref[autoregressiveModelSequenceType]{autoregressiveModelSequence} & \hfuzz=500pt file containing AR model for spatiotemporal constraint\\
\hline
\end{tabularx}

This program is \reference{parallelized}{general.parallelization}.
\clearpage
%==================================
\section{Programs: Matrix}
\subsection{Matrix2EmpiricalCovariance}\label{Matrix2EmpiricalCovariance}
This program estimates a spatial and temporal covariance matrix from a time series of \configFile{matrix}{matrix} files according to
\begin{equation}
\M\Sigma(\Delta i) = \frac{1}{N}\sum_{i=1}^N \M x_i \M x_{i+\Delta i}^T
\end{equation}
where $\Delta i$ is given by \config{differenceStep}.



\keepXColumns
\begin{tabularx}{\textwidth}{N T A}
\hline
Name & Type & Annotation\\
\hline
\hfuzz=500pt\includegraphics[width=1em]{element-mustset.pdf}~outputfileCovarianceMatrix & \hfuzz=500pt filename & \hfuzz=500pt \\
\hfuzz=500pt\includegraphics[width=1em]{element-mustset-unbounded.pdf}~inputfileMatrix & \hfuzz=500pt filename & \hfuzz=500pt \\
\hfuzz=500pt\includegraphics[width=1em]{element.pdf}~removeMean & \hfuzz=500pt boolean & \hfuzz=500pt \\
\hfuzz=500pt\includegraphics[width=1em]{element.pdf}~differenceStep & \hfuzz=500pt uint & \hfuzz=500pt choose dt for: x,i(t) - x,j(t+dt)\\
\hfuzz=500pt\includegraphics[width=1em]{element.pdf}~regularize & \hfuzz=500pt sequence & \hfuzz=500pt add value to main diagonal\\
\hfuzz=500pt\includegraphics[width=1em]{connector.pdf}\includegraphics[width=1em]{element.pdf}~factorOfTrace & \hfuzz=500pt boolean & \hfuzz=500pt multiply factor with trace of matrix\\
\hfuzz=500pt\includegraphics[width=1em]{connector.pdf}\includegraphics[width=1em]{element.pdf}~regularizationFactor & \hfuzz=500pt double & \hfuzz=500pt add to main diagonal\\
\hline
\end{tabularx}

\clearpage
%==================================
\section{Programs: Misc}
\subsection{DigitalFilter2FrequencyResponse}\label{DigitalFilter2FrequencyResponse}
Compute amplitude-, phase-, group delay and frequency response of a \configClass{digitalFilter}{digitalFilterType} cascade.
The \configFile{outputfileResponse}{matrix} is a matrix with following columns:
freq $[Hz]$, ampl, phase $[rad]$, group delay $[-]$, real, imag.

When \config{unwrapPhase} is set to true, $2\pi$ jumps of the phase response are removed before writing the output to file.

The response of the filter cascade is given by the product of each individual frequency response:
\begin{equation}
  H(f) = \prod_f H_j(f).
\end{equation}
Amplitude and phase response are computed from the frequency response via
\begin{equation}
  A(f) = |H(f)| \hspace{5pt}\text{and}\hspace{5pt} \Phi(f) = \arctan \frac{\mathcal{I}(H(f))}{\mathcal{R}(H(f))}.
\end{equation}
The group delay is computed by numerically differentiating the phase response
\begin{equation}
  \tau_g(f_k) = \frac{1}{2} \left[\frac{\Phi(f_k) - \Phi(f_{k-1})}{2\pi(f_k-f_{k-1})} + \frac{\Phi(f_{k+1}) - \Phi(f_{k})}{2\pi(f_{k+1}-f_{k})}\right] \approx \frac{d\Phi}{df}\frac{df}{d\omega}.
\end{equation}
The frequency vector for a \config{length} $N$ and a \config{sampling} $\Delta t$ is given by
\begin{equation}
  f_k = \frac{k}{N \Delta t}, \hspace{15pt} k \in \{0, \dots, \left\lfloor\frac{N+2}{2}\right\rfloor-1\}.
\end{equation}

See also \program{DigitalFilter2ImpulseResponse}.


\keepXColumns
\begin{tabularx}{\textwidth}{N T A}
\hline
Name & Type & Annotation\\
\hline
\hfuzz=500pt\includegraphics[width=1em]{element-mustset.pdf}~outputfileResponse & \hfuzz=500pt filename & \hfuzz=500pt columns: freq [Hz], ampl, phase [rad], group delay [-], real, imag\\
\hfuzz=500pt\includegraphics[width=1em]{element-mustset-unbounded.pdf}~digitalFilter & \hfuzz=500pt \hyperref[digitalFilterType]{digitalFilter} & \hfuzz=500pt \\
\hfuzz=500pt\includegraphics[width=1em]{element.pdf}~length & \hfuzz=500pt uint & \hfuzz=500pt length of the data series in time domain\\
\hfuzz=500pt\includegraphics[width=1em]{element.pdf}~sampling & \hfuzz=500pt double & \hfuzz=500pt sampling to determine frequency [seconds]\\
\hfuzz=500pt\includegraphics[width=1em]{element.pdf}~skipZeroFrequency & \hfuzz=500pt boolean & \hfuzz=500pt omit zero frequency when writing to file\\
\hfuzz=500pt\includegraphics[width=1em]{element.pdf}~unwrapPhase & \hfuzz=500pt boolean & \hfuzz=500pt unwrap phase response\\
\hline
\end{tabularx}

\clearpage
%==================================
\subsection{DigitalFilter2ImpulseResponse}\label{DigitalFilter2ImpulseResponse}
Impulse response of a \configClass{digitalFilter}{digitalFilterType} cascade.
The impulse response is computed by filtering a sequence with \config{length} samples and a unit impulse at index \config{pulseLag}.

The \configFile{outputfileResponse}{matrix} is a matrix with the time stamp (zero at \config{pulseLag})
in the first column and the impulse response $h_k$ in the second column.

See also \program{DigitalFilter2FrequencyResponse}.


\keepXColumns
\begin{tabularx}{\textwidth}{N T A}
\hline
Name & Type & Annotation\\
\hline
\hfuzz=500pt\includegraphics[width=1em]{element-mustset.pdf}~outputfileResponse & \hfuzz=500pt filename & \hfuzz=500pt columns: time [seconds], response\\
\hfuzz=500pt\includegraphics[width=1em]{element-mustset-unbounded.pdf}~digitalFilter & \hfuzz=500pt \hyperref[digitalFilterType]{digitalFilter} & \hfuzz=500pt \\
\hfuzz=500pt\includegraphics[width=1em]{element.pdf}~length & \hfuzz=500pt uint & \hfuzz=500pt length of the impulse response\\
\hfuzz=500pt\includegraphics[width=1em]{element.pdf}~pulseLag & \hfuzz=500pt uint & \hfuzz=500pt start of the pulse in the data series\\
\hfuzz=500pt\includegraphics[width=1em]{element.pdf}~sampling & \hfuzz=500pt double & \hfuzz=500pt [seconds]\\
\hline
\end{tabularx}

\clearpage
%==================================
\subsection{EarthOrientationParameterTimeSeries}\label{EarthOrientationParameterTimeSeries}
Computes a \configClass{timeSeries}{timeSeriesType} (GPS time) of Earth Orientation Parameter (EOP).
The \file{instrument file}{instrument} (MISCVALUES) contains the elements at each epoch in the following order:
\begin{itemize}
\item $x_p$ [rad]
\item $y_p$ [rad]
\item $s_p$ [rad]
\item $UT1-UTC$ [seconds]
\item length of day (LOD) [seconds]
\item $X$ [rad]
\item $Y$ [rad]
\item $S$ [rad]
\end{itemize}
The values are in situ values with all corrections and models applied. The time series can be used to
precompute Earth rotation with a low temporal resolution (e.g. 10 min) and reuse the file in
\configClass{earthRotation:file}{earthRotationType:file} to interpolate the data to the needed epochs
(e.g. to rotate orbit data). As some Earth rotation models are quite slow this can accelerate the computation.


\keepXColumns
\begin{tabularx}{\textwidth}{N T A}
\hline
Name & Type & Annotation\\
\hline
\hfuzz=500pt\includegraphics[width=1em]{element-mustset.pdf}~outputfileEOP & \hfuzz=500pt filename & \hfuzz=500pt each row: mjd(GPS), xp, yp, sp, dUT1, LOD, X, Y, S\\
\hfuzz=500pt\includegraphics[width=1em]{element-mustset.pdf}~earthRotation & \hfuzz=500pt \hyperref[earthRotationType]{earthRotation} & \hfuzz=500pt \\
\hfuzz=500pt\includegraphics[width=1em]{element-mustset-unbounded.pdf}~timeSeries & \hfuzz=500pt \hyperref[timeSeriesType]{timeSeries} & \hfuzz=500pt \\
\hline
\end{tabularx}

This program is \reference{parallelized}{general.parallelization}.
\clearpage
%==================================
\subsection{EarthRotaryVectorTimeSeries}\label{EarthRotaryVectorTimeSeries}
Computes a \configFile{outputfileTimeSeries}{instrument} of Earth's rotary axis
and its temporal derivative at \configClass{timeSeries}{timeSeriesType} (GPS time).
The \file{instrument file}{instrument} (MISCVALUES) contains the elements at each epoch in the following order:
\begin{itemize}
\item $\omega_x [rad/s]$
\item $\omega_y [rad/s]$
\item $\omega_z [rad/s]$
\item $\dot{\omega}_x [rad/s^2]$
\item $\dot{\omega}_y [rad/s^2]$
\item $\dot{\omega}_z [rad/s^2]$.
\end{itemize}


\keepXColumns
\begin{tabularx}{\textwidth}{N T A}
\hline
Name & Type & Annotation\\
\hline
\hfuzz=500pt\includegraphics[width=1em]{element-mustset.pdf}~outputfileTimeSeries & \hfuzz=500pt filename & \hfuzz=500pt wx, wy, wz [rad], dwx, dwy, dwz [rad/s\textasciicircum{}2]\\
\hfuzz=500pt\includegraphics[width=1em]{element-mustset.pdf}~earthRotation & \hfuzz=500pt \hyperref[earthRotationType]{earthRotation} & \hfuzz=500pt \\
\hfuzz=500pt\includegraphics[width=1em]{element-mustset-unbounded.pdf}~timeSeries & \hfuzz=500pt \hyperref[timeSeriesType]{timeSeries} & \hfuzz=500pt \\
\hfuzz=500pt\includegraphics[width=1em]{element.pdf}~inTRF & \hfuzz=500pt boolean & \hfuzz=500pt terrestrial reference frame, otherwise celestial\\
\hline
\end{tabularx}

This program is \reference{parallelized}{general.parallelization}.
\clearpage
%==================================
\subsection{FilterMatrixWindowedPotentialCoefficients}\label{FilterMatrixWindowedPotentialCoefficients}
Create a spherical harmonic window matrix. The window matrix $\mathbf{W}$ is generated in space domain through
spherical harmonic synthesis and analysis matrices.
The resulting linear operator can be written as
\begin{equation}
\mathbf{W} = \mathbf{K} \mathbf{A} \mathbf{\Omega} \mathbf{S} \mathbf{K}^{-1}.
\end{equation}
Here, $\mathbf{K}$ is a diagonal matrix with the \configClass{kernel}{kernelType} coefficients on the main diagonal,
$\mathbf{S}$ is the spherical harmonic synthesis matrix, $\mathbf{\Omega}$ is defined by the values in
\file{inputfileGriddedData}{griddedData} and the
expression \config{value}, $\mathbf{A}$ is the spherical harmonic analysis matrix.
The resulting window matrix is written to a \file{matrix}{matrix} file.

The spherical harmonic degree range, and coefficient numbering are defined by
\config{minDegree}, \config{maxDegree}, and \configClass{numbering}{sphericalHarmonicsNumberingType}.

Note that a proper window function $\mathbf{\Omega}$ should contain values in the range [0, 1].
The window function $\mathbf{\Omega}$ can feature a smooth transition between 0 and 1 to avoid ringing effects.


\keepXColumns
\begin{tabularx}{\textwidth}{N T A}
\hline
Name & Type & Annotation\\
\hline
\hfuzz=500pt\includegraphics[width=1em]{element.pdf}~outputfileWindowMatrix & \hfuzz=500pt filename & \hfuzz=500pt \\
\hfuzz=500pt\includegraphics[width=1em]{element-mustset.pdf}~inputfileGriddedData & \hfuzz=500pt filename & \hfuzz=500pt gridded data which defines the window function in space domain\\
\hfuzz=500pt\includegraphics[width=1em]{element-mustset.pdf}~value & \hfuzz=500pt expression & \hfuzz=500pt expression to compute the window function (input columns are named data0, data1, ...)\\
\hfuzz=500pt\includegraphics[width=1em]{element-mustset.pdf}~kernel & \hfuzz=500pt \hyperref[kernelType]{kernel} & \hfuzz=500pt kernel for windowing\\
\hfuzz=500pt\includegraphics[width=1em]{element.pdf}~minDegree & \hfuzz=500pt uint & \hfuzz=500pt \\
\hfuzz=500pt\includegraphics[width=1em]{element-mustset.pdf}~maxDegree & \hfuzz=500pt uint & \hfuzz=500pt \\
\hfuzz=500pt\includegraphics[width=1em]{element.pdf}~GM & \hfuzz=500pt double & \hfuzz=500pt Geocentric gravitational constant\\
\hfuzz=500pt\includegraphics[width=1em]{element.pdf}~R & \hfuzz=500pt double & \hfuzz=500pt reference radius\\
\hfuzz=500pt\includegraphics[width=1em]{element-mustset.pdf}~numbering & \hfuzz=500pt \hyperref[sphericalHarmonicsNumberingType]{sphericalHarmonicsNumbering} & \hfuzz=500pt numbering scheme for solution vector\\
\hline
\end{tabularx}

This program is \reference{parallelized}{general.parallelization}.
\clearpage
%==================================
\subsection{FunctionsCalculate}\label{FunctionsCalculate}
This program manipulates \file{matrix files}{matrix} with data in columns.
If several \config{inputfile}s are given the data columns are copied side by side.
All \config{inputfile}s must contain the same number of rows.
The columns are enumerated by \verb|data0|,~\verb|data1|,~\ldots.

The content of \configFile{outputfile}{matrix} is controlled by \config{outColumn}.
The algorithm to compute the output is as follows:
The expressions in \config{outColumn} are evaluated once for each row of the input.
The variables \verb|data0|,~\verb|data1|,~\ldots are replaced by the according values from the input columns before.
Additional variables are available, e.g. \verb|index|, \verb|data0rms|, see~\reference{dataVariables}{general.parser:dataVariables}.
If no \config{outColumn} are specified all input columns are used instead directly.

For a simplified handling \config{constant}s can be defined by \verb|name=value|, e.g. \verb|annual=365.25|.
It is also possible to estimate \config{parameter}s in a least squares adjustment.
The \config{leastSquares} serves as template for observation equations for every row.
The expression \config{leastSquares} is evaluated for each row in the \config{inputfile}.
The variables \verb|data0|,~\verb|data1|,~\ldots are replaced by the according values from the input columns before.
In the next step the parameters are estimated in order to minimize the expressions in \config{leastSquares}
in the sense of least squares.

Afterwards complete rows are removed if one of the \config{removalCriteria} expressions for this row evaluates true (not zero).

An extra \config{statistics} file can be generated with one row of data. For the computation of the \config{outColumn} values
all~\reference{dataVariables}{general.parser:dataVariables} are available (e.g. \verb|data3mean|, \verb|data4std|)
inclusively the \config{constant}s and estimated \config{parameter}s but without the \verb|data0|,~\verb|data1|,~\ldots itself.
The variables and the numbering of the columns refers to the \configFile{outputfile}{matrix}.

First example: To calculate the mean of two values at each row set \config{outColumn} to \verb|0.5*(data1+data0)|.

Second example: An input file contain a column with times and a column with values.
To remove a trend from the values define the \config{parameter}s \verb|trend| and \verb|bias|.
The observation equation in \config{leastSquares} is \verb|data1 - (trend*data0+bias)|.
For output you can define the following columns for example:
\begin{itemize}
\item \config{outColumn}=\verb|data0|: points in time.
\item \config{outColumn}=\verb|data1|: the values itself.
\item \config{outColumn}=\verb|trend*data0+bias|: the linear fit.
\item \config{outColumn}=\verb|data1-trend*data0-bias|: the residuals.
\end{itemize}
The extra statistics file could contain in this case:
\begin{itemize}
\item \config{outColumn}=\verb|data0max-data0min|: time span.
\item \config{outColumn}=\verb|bias|: estimated parameter.
\item \config{outColumn}=\verb|trend|: estimated parameter.
\item \config{outColumn}=\verb|data3rms|: root mean square of the residuals.
\end{itemize}

See also \program{InstrumentArcCalculate}, \program{GriddedDataCalculate}, \program{MatrixCalculate}.


\keepXColumns
\begin{tabularx}{\textwidth}{N T A}
\hline
Name & Type & Annotation\\
\hline
\hfuzz=500pt\includegraphics[width=1em]{element.pdf}~outputfile & \hfuzz=500pt filename & \hfuzz=500pt \\
\hfuzz=500pt\includegraphics[width=1em]{element-mustset-unbounded.pdf}~inputfile & \hfuzz=500pt filename & \hfuzz=500pt \\
\hfuzz=500pt\includegraphics[width=1em]{element-unbounded.pdf}~constant & \hfuzz=500pt expression & \hfuzz=500pt define a constant by name=value\\
\hfuzz=500pt\includegraphics[width=1em]{element-unbounded.pdf}~parameter & \hfuzz=500pt expression & \hfuzz=500pt define a parameter by name[=value]\\
\hfuzz=500pt\includegraphics[width=1em]{element-unbounded.pdf}~leastSquares & \hfuzz=500pt expression & \hfuzz=500pt try to minimize the expression by adjustment of the parameters\\
\hfuzz=500pt\includegraphics[width=1em]{element-unbounded.pdf}~removalCriteria & \hfuzz=500pt expression & \hfuzz=500pt row is removed if one criterion evaluates true.\\
\hfuzz=500pt\includegraphics[width=1em]{element-unbounded.pdf}~outColumn & \hfuzz=500pt expression & \hfuzz=500pt expression to compute output columns (input columns are named data0, data1, ...)\\
\hfuzz=500pt\includegraphics[width=1em]{element.pdf}~statistics & \hfuzz=500pt sequence & \hfuzz=500pt \\
\hfuzz=500pt\includegraphics[width=1em]{connector.pdf}\includegraphics[width=1em]{element-mustset.pdf}~outputfile & \hfuzz=500pt filename & \hfuzz=500pt matrix with one row, columns are user defined\\
\hfuzz=500pt\includegraphics[width=1em]{connector.pdf}\includegraphics[width=1em]{element-mustset-unbounded.pdf}~outColumn & \hfuzz=500pt expression & \hfuzz=500pt expression to compute statistics columns, data* are the outputColumns\\
\hline
\end{tabularx}

\clearpage
%==================================
\subsection{Grs2PotentialCoefficients}\label{Grs2PotentialCoefficients}
This program creates potential coefficients from the defining constants
of a Geodetic Reference System (GRS). The potential coeffiencts excludes the centrifugal part.
The form of the reference ellipsoid is either determined by the dynamical form factor \config{J2},
or the geometric \config{inverseFlattening}. One of those form parameters must be specified.

The default values create the GRS80.


\keepXColumns
\begin{tabularx}{\textwidth}{N T A}
\hline
Name & Type & Annotation\\
\hline
\hfuzz=500pt\includegraphics[width=1em]{element-mustset.pdf}~outputfilePotentialCoefficients & \hfuzz=500pt filename & \hfuzz=500pt \\
\hfuzz=500pt\includegraphics[width=1em]{element-mustset.pdf}~maxDegree & \hfuzz=500pt uint & \hfuzz=500pt \\
\hfuzz=500pt\includegraphics[width=1em]{element.pdf}~GM & \hfuzz=500pt double & \hfuzz=500pt Geocentric gravitational constant\\
\hfuzz=500pt\includegraphics[width=1em]{element.pdf}~R & \hfuzz=500pt double & \hfuzz=500pt reference radius\\
\hfuzz=500pt\includegraphics[width=1em]{element.pdf}~omega & \hfuzz=500pt double & \hfuzz=500pt Angular velocity of rotation\\
\hfuzz=500pt\includegraphics[width=1em]{element.pdf}~J2 & \hfuzz=500pt double & \hfuzz=500pt Dynamical form factor\\
\hfuzz=500pt\includegraphics[width=1em]{element.pdf}~inverseFlattening & \hfuzz=500pt double & \hfuzz=500pt Geometric inverse flattening of reference ellipsoid (0: sphere, ignored when J2 is set)\\
\hline
\end{tabularx}

\clearpage
%==================================
\subsection{Kaula2SigmaPotentialCoefficients}\label{Kaula2SigmaPotentialCoefficients}
Create signal standard deviations of potential coefficients according Kaula's rule of thumb
\begin{equation}
  \sigma_n = \frac{f}{n^p},
\end{equation}
with the degree $n$, the \config{factor} $f$, and the \config{power} $p$.

The standard deviations are written as formal errors of
 \configFile{outputfilePotentialCoefficients}{potentialCoefficients}.


\keepXColumns
\begin{tabularx}{\textwidth}{N T A}
\hline
Name & Type & Annotation\\
\hline
\hfuzz=500pt\includegraphics[width=1em]{element-mustset.pdf}~outputfilePotentialCoefficients & \hfuzz=500pt filename & \hfuzz=500pt \\
\hfuzz=500pt\includegraphics[width=1em]{element.pdf}~minDegree & \hfuzz=500pt uint & \hfuzz=500pt \\
\hfuzz=500pt\includegraphics[width=1em]{element-mustset.pdf}~maxDegree & \hfuzz=500pt uint & \hfuzz=500pt \\
\hfuzz=500pt\includegraphics[width=1em]{element.pdf}~GM & \hfuzz=500pt double & \hfuzz=500pt Geocentric gravitational constant\\
\hfuzz=500pt\includegraphics[width=1em]{element.pdf}~R & \hfuzz=500pt double & \hfuzz=500pt reference radius\\
\hfuzz=500pt\includegraphics[width=1em]{element.pdf}~power & \hfuzz=500pt double & \hfuzz=500pt sigma = factor/degree\textasciicircum{}power\\
\hfuzz=500pt\includegraphics[width=1em]{element.pdf}~factor & \hfuzz=500pt double & \hfuzz=500pt sigma = factor/degree\textasciicircum{}power\\
\hline
\end{tabularx}

\clearpage
%==================================
\subsection{Kernel2Coefficients}\label{Kernel2Coefficients}
This program computes and returns the coefficients and inverse coefficients of a \configClass{kernel}{kernelType}
from from \config{minDegree} to \config{maxDegree} at a given \config{height}.

The main purpose is for visualization with \program{PlotGraph}.


\keepXColumns
\begin{tabularx}{\textwidth}{N T A}
\hline
Name & Type & Annotation\\
\hline
\hfuzz=500pt\includegraphics[width=1em]{element-mustset.pdf}~outputfileMatrix & \hfuzz=500pt filename & \hfuzz=500pt matrix with columns degree, coefficients and inverse coefficients\\
\hfuzz=500pt\includegraphics[width=1em]{element-mustset.pdf}~kernel & \hfuzz=500pt \hyperref[kernelType]{kernel} & \hfuzz=500pt \\
\hfuzz=500pt\includegraphics[width=1em]{element.pdf}~minDegree & \hfuzz=500pt uint & \hfuzz=500pt minimum degree of returned coefficients\\
\hfuzz=500pt\includegraphics[width=1em]{element-mustset.pdf}~maxDegre & \hfuzz=500pt uint & \hfuzz=500pt compute coefficients up to maxDegree\\
\hfuzz=500pt\includegraphics[width=1em]{element.pdf}~height & \hfuzz=500pt double & \hfuzz=500pt evaluate kernel at R+height [m]\\
\hfuzz=500pt\includegraphics[width=1em]{element.pdf}~R & \hfuzz=500pt double & \hfuzz=500pt reference radius\\
\hline
\end{tabularx}

\clearpage
%==================================
\subsection{Kernel2SigmaPotentialCoefficients}\label{Kernel2SigmaPotentialCoefficients}
Create variances of spherical harmonics by convolution a kernel with white noise,
e.g. to display filter coefficients of a Gaussian filter.
The coefficients are written as formal errors of \configFile{outputfilePotentialCoefficients}{potentialCoefficients}.


\keepXColumns
\begin{tabularx}{\textwidth}{N T A}
\hline
Name & Type & Annotation\\
\hline
\hfuzz=500pt\includegraphics[width=1em]{element-mustset.pdf}~outputfilePotentialCoefficients & \hfuzz=500pt filename & \hfuzz=500pt \\
\hfuzz=500pt\includegraphics[width=1em]{element-mustset.pdf}~kernel & \hfuzz=500pt \hyperref[kernelType]{kernel} & \hfuzz=500pt \\
\hfuzz=500pt\includegraphics[width=1em]{element.pdf}~minDegree & \hfuzz=500pt uint & \hfuzz=500pt \\
\hfuzz=500pt\includegraphics[width=1em]{element.pdf}~maxDegree & \hfuzz=500pt uint & \hfuzz=500pt \\
\hfuzz=500pt\includegraphics[width=1em]{element.pdf}~GM & \hfuzz=500pt double & \hfuzz=500pt Geocentric gravitational constant\\
\hfuzz=500pt\includegraphics[width=1em]{element.pdf}~R & \hfuzz=500pt double & \hfuzz=500pt reference radius\\
\hfuzz=500pt\includegraphics[width=1em]{element.pdf}~factor & \hfuzz=500pt double & \hfuzz=500pt \\
\hline
\end{tabularx}

\clearpage
%==================================
\subsection{KernelEvaluate}\label{KernelEvaluate}
Compute \configClass{kernel}{kernelType} values for distant angles.
The main purpose is for visualization with \program{PlotGraph}.


\keepXColumns
\begin{tabularx}{\textwidth}{N T A}
\hline
Name & Type & Annotation\\
\hline
\hfuzz=500pt\includegraphics[width=1em]{element-mustset.pdf}~outputfileMatrix & \hfuzz=500pt filename & \hfuzz=500pt matrix with first column the angle [degree], second the kernel value\\
\hfuzz=500pt\includegraphics[width=1em]{element-mustset.pdf}~kernel & \hfuzz=500pt \hyperref[kernelType]{kernel} & \hfuzz=500pt \\
\hfuzz=500pt\includegraphics[width=1em]{element.pdf}~minAngle & \hfuzz=500pt angle & \hfuzz=500pt [degree]\\
\hfuzz=500pt\includegraphics[width=1em]{element.pdf}~maxAngle & \hfuzz=500pt angle & \hfuzz=500pt [degree]\\
\hfuzz=500pt\includegraphics[width=1em]{element.pdf}~sampling & \hfuzz=500pt angle & \hfuzz=500pt [degree]\\
\hfuzz=500pt\includegraphics[width=1em]{element.pdf}~height & \hfuzz=500pt double & \hfuzz=500pt evaluate at R+height [m]\\
\hfuzz=500pt\includegraphics[width=1em]{element.pdf}~R & \hfuzz=500pt double & \hfuzz=500pt reference radius\\
\hline
\end{tabularx}

\clearpage
%==================================
\subsection{MagneticField2GriddedData}\label{MagneticField2GriddedData}
Computes x, y, z of the magentic field vector.


\keepXColumns
\begin{tabularx}{\textwidth}{N T A}
\hline
Name & Type & Annotation\\
\hline
\hfuzz=500pt\includegraphics[width=1em]{element-mustset.pdf}~outputfileGriddedData & \hfuzz=500pt filename & \hfuzz=500pt x, y, z [Tesla = kg/A/s**2]\\
\hfuzz=500pt\includegraphics[width=1em]{element-mustset.pdf}~magnetosphere & \hfuzz=500pt \hyperref[magnetosphereType]{magnetosphere} & \hfuzz=500pt \\
\hfuzz=500pt\includegraphics[width=1em]{element-mustset-unbounded.pdf}~grid & \hfuzz=500pt \hyperref[gridType]{grid} & \hfuzz=500pt \\
\hfuzz=500pt\includegraphics[width=1em]{element.pdf}~time & \hfuzz=500pt time & \hfuzz=500pt \\
\hfuzz=500pt\includegraphics[width=1em]{element.pdf}~localReferenceFrame & \hfuzz=500pt boolean & \hfuzz=500pt local left handed reference frame (north, east, up)\\
\hfuzz=500pt\includegraphics[width=1em]{element.pdf}~R & \hfuzz=500pt double & \hfuzz=500pt reference radius for ellipsoidal coordinates on output\\
\hfuzz=500pt\includegraphics[width=1em]{element.pdf}~inverseFlattening & \hfuzz=500pt double & \hfuzz=500pt reference flattening for ellipsoidal coordinates on output, 0: spherical coordinates\\
\hline
\end{tabularx}

This program is \reference{parallelized}{general.parallelization}.
\clearpage
%==================================
\subsection{MatrixCalculate}\label{MatrixCalculate}
This program creates a \file{matrix}{matrix} from multiple matrices.
All \configClass{matrices}{matrixGeneratorType} are summed up. The size of the resulting matrix is exandeded to fit all matrices.
The class \configClass{matrixGenerator}{matrixGeneratorType} allows complex matrix operations before.


\keepXColumns
\begin{tabularx}{\textwidth}{N T A}
\hline
Name & Type & Annotation\\
\hline
\hfuzz=500pt\includegraphics[width=1em]{element-mustset.pdf}~outputfileMatrix & \hfuzz=500pt filename & \hfuzz=500pt \\
\hfuzz=500pt\includegraphics[width=1em]{element-mustset-unbounded.pdf}~matrix & \hfuzz=500pt \hyperref[matrixGeneratorType]{matrixGenerator} & \hfuzz=500pt \\
\hline
\end{tabularx}

\clearpage
%==================================
\subsection{ObservationEquations2Files}\label{ObservationEquations2Files}
This program computes the linearized and decorrelated equation system for each arc $i$:
\begin{equation}
\M l_i  = \M A_i \M x + \M B_i \M y_i + \M e_i
\end{equation}
using class \configClass{observation}{observationType} and writes $\M A_i$, $\M B_i$ and $\M l_i$ as \file{matrix}{matrix} files.


\keepXColumns
\begin{tabularx}{\textwidth}{N T A}
\hline
Name & Type & Annotation\\
\hline
\hfuzz=500pt\includegraphics[width=1em]{element.pdf}~outputfileObservationVector & \hfuzz=500pt filename & \hfuzz=500pt one file for each arc\\
\hfuzz=500pt\includegraphics[width=1em]{element.pdf}~outputfileDesignMatrix & \hfuzz=500pt filename & \hfuzz=500pt one file for each arc, without arc related parameters\\
\hfuzz=500pt\includegraphics[width=1em]{element.pdf}~outputfileDesignMatrixArc & \hfuzz=500pt filename & \hfuzz=500pt one file for each arc, arc related parameters\\
\hfuzz=500pt\includegraphics[width=1em]{element.pdf}~variableArc & \hfuzz=500pt string & \hfuzz=500pt variable with arc number\\
\hfuzz=500pt\includegraphics[width=1em]{element.pdf}~outputfileParameterNames & \hfuzz=500pt filename & \hfuzz=500pt without arc related parameters\\
\hfuzz=500pt\includegraphics[width=1em]{element-mustset.pdf}~observation & \hfuzz=500pt \hyperref[observationType]{observation} & \hfuzz=500pt \\
\hline
\end{tabularx}

This program is \reference{parallelized}{general.parallelization}.
\clearpage
%==================================
\subsection{PotentialCoefficients2BlockMeanTimeSplines}\label{PotentialCoefficients2BlockMeanTimeSplines}
This program is a simplified version of \program{Gravityfield2TimeSplines}.
It reads a series of potential coefficient files (\configFile{inputfilePotentialCoefficients}{potentialCoefficients})
and creates a time splines file with spline degree 0 (temporal block means) or degree 1 (linear splines).
The time intervals in which the potential coefficients are valid are defined between adjacent
points in time given by \config{splineTimeSeries}. Therefore one more point in time is needed
than the number of potential coefficient files for degree 0.

The coefficients can be filtered with \configClass{filter}{sphericalHarmonicsFilterType}.
If set the expansion is limited in the range between \config{minDegree} and \config{maxDegree} inclusivly.
The coefficients are related to the reference radius~\config{R} and the Earth gravitational constant \config{GM}.

This program is useful e.g. to combine monthly GRACE solutions to one file.


\keepXColumns
\begin{tabularx}{\textwidth}{N T A}
\hline
Name & Type & Annotation\\
\hline
\hfuzz=500pt\includegraphics[width=1em]{element-mustset.pdf}~outputfileTimeSplines & \hfuzz=500pt filename & \hfuzz=500pt \\
\hfuzz=500pt\includegraphics[width=1em]{element.pdf}~outputfileTimeSplinesCovariance & \hfuzz=500pt filename & \hfuzz=500pt only the variances are saved\\
\hfuzz=500pt\includegraphics[width=1em]{element-mustset-unbounded.pdf}~inputfilePotentialCoefficients & \hfuzz=500pt filename & \hfuzz=500pt \\
\hfuzz=500pt\includegraphics[width=1em]{element-unbounded.pdf}~filter & \hfuzz=500pt \hyperref[sphericalHarmonicsFilterType]{sphericalHarmonicsFilter} & \hfuzz=500pt \\
\hfuzz=500pt\includegraphics[width=1em]{element.pdf}~minDegree & \hfuzz=500pt uint & \hfuzz=500pt \\
\hfuzz=500pt\includegraphics[width=1em]{element.pdf}~maxDegree & \hfuzz=500pt uint & \hfuzz=500pt \\
\hfuzz=500pt\includegraphics[width=1em]{element.pdf}~GM & \hfuzz=500pt double & \hfuzz=500pt Geocentric gravitational constant\\
\hfuzz=500pt\includegraphics[width=1em]{element.pdf}~R & \hfuzz=500pt double & \hfuzz=500pt reference radius\\
\hfuzz=500pt\includegraphics[width=1em]{element.pdf}~removeMean & \hfuzz=500pt boolean & \hfuzz=500pt remove the temporal mean of the series before estimating the splines\\
\hfuzz=500pt\includegraphics[width=1em]{element.pdf}~interpolate & \hfuzz=500pt boolean & \hfuzz=500pt interpolate missing files\\
\hfuzz=500pt\includegraphics[width=1em]{element-mustset-unbounded.pdf}~splineTimeSeries & \hfuzz=500pt \hyperref[timeSeriesType]{timeSeries} & \hfuzz=500pt input files must be between points in time\\
\hfuzz=500pt\includegraphics[width=1em]{element.pdf}~splineDegree & \hfuzz=500pt uint & \hfuzz=500pt degree of splines\\
\hline
\end{tabularx}

\clearpage
%==================================
\subsection{PotentialCoefficients2DegreeAmplitudes}\label{PotentialCoefficients2DegreeAmplitudes}
This program computes degree amplitudes from
\file{potentialCoefficients files}{potentialCoefficients}
and saves them to a \file{matrix}{matrix} file.

The coefficients can be filtered with \configClass{filter}{sphericalHarmonicsFilterType} and converted
to different functionals with \configClass{kernel}{kernelType}. The gravity field can be evaluated at
different altitudes by specifying \config{evaluationRadius}. Polar regions can be excluded
by setting \config{polarGap}. If set the expansion is limited in the range between \config{minDegree}
and \config{maxDegree} inclusivly. The coefficients are related to the reference radius~\config{R}
and the Earth gravitational constant \config{GM}.

The \configFile{outputfileMatrix}{matrix} contains in the first 3 columns the degree, the degree amplitude, and
the formal errors. For each additional \configFile{inputfilePotentialCoefficients}{potentialCoefficients} three columns
are appended: the degree amplitude, the formal errors, and the difference to the first file.

For example the data columns for 4 \configFile{inputfilePotentialCoefficients}{potentialCoefficients} are
\begin{itemize}
\item degree=\verb|data0|
\item PotentialCoefficients0: signal=\verb|data1|, error=\verb|data2|,
\item PotentialCoefficients1: signal=\verb|data3|, error=\verb|data4|,  difference=\verb|data5|,
\item PotentialCoefficients2: signal=\verb|data6|, error=\verb|data7|,  difference=\verb|data8|,
\item PotentialCoefficients3: signal=\verb|data9|, error=\verb|data10|, difference=\verb|data11|.
\end{itemize}

See also \program{Gravityfield2DegreeAmplitudes}.


\keepXColumns
\begin{tabularx}{\textwidth}{N T A}
\hline
Name & Type & Annotation\\
\hline
\hfuzz=500pt\includegraphics[width=1em]{element-mustset.pdf}~outputfileMatrix & \hfuzz=500pt filename & \hfuzz=500pt matrix with degree, signal amplitude, formal error, differences\\
\hfuzz=500pt\includegraphics[width=1em]{element-mustset-unbounded.pdf}~inputfilePotentialCoefficients & \hfuzz=500pt filename & \hfuzz=500pt \\
\hfuzz=500pt\includegraphics[width=1em]{element-mustset.pdf}~kernel & \hfuzz=500pt \hyperref[kernelType]{kernel} & \hfuzz=500pt \\
\hfuzz=500pt\includegraphics[width=1em]{element-unbounded.pdf}~filter & \hfuzz=500pt \hyperref[sphericalHarmonicsFilterType]{sphericalHarmonicsFilter} & \hfuzz=500pt filter the coefficients\\
\hfuzz=500pt\includegraphics[width=1em]{element-mustset.pdf}~type & \hfuzz=500pt choice & \hfuzz=500pt type of variances\\
\hfuzz=500pt\includegraphics[width=1em]{connector.pdf}\includegraphics[width=1em]{element-mustset.pdf}~rms & \hfuzz=500pt  & \hfuzz=500pt degree amplitudes (square root of degree variances)\\
\hfuzz=500pt\includegraphics[width=1em]{connector.pdf}\includegraphics[width=1em]{element-mustset.pdf}~accumulation & \hfuzz=500pt  & \hfuzz=500pt cumulate variances over degrees\\
\hfuzz=500pt\includegraphics[width=1em]{element.pdf}~evaluationRadius & \hfuzz=500pt double & \hfuzz=500pt evaluate the gravity field at this radius (default: evaluate at surface\\
\hfuzz=500pt\includegraphics[width=1em]{element.pdf}~polarGap & \hfuzz=500pt angle & \hfuzz=500pt exclude polar regions (aperture angle in degrees)\\
\hfuzz=500pt\includegraphics[width=1em]{element.pdf}~minDegree & \hfuzz=500pt uint & \hfuzz=500pt \\
\hfuzz=500pt\includegraphics[width=1em]{element.pdf}~maxDegree & \hfuzz=500pt uint & \hfuzz=500pt \\
\hfuzz=500pt\includegraphics[width=1em]{element.pdf}~GM & \hfuzz=500pt double & \hfuzz=500pt Geocentric gravitational constant\\
\hfuzz=500pt\includegraphics[width=1em]{element.pdf}~R & \hfuzz=500pt double & \hfuzz=500pt reference radius\\
\hline
\end{tabularx}

\clearpage
%==================================
\subsection{RadialBasisSplines2KernelCoefficients}\label{RadialBasisSplines2KernelCoefficients}
This program calculates the coefficients $k_n$ of a \configClass{kernel:coefficients}{kernelType:coefficients} according to
\begin{equation}
  k_n = \frac{GM}{4\pi R}\frac{\sigma_n}{\sqrt{2n+1}}.
\end{equation}
from a given \configClass{gravityfield}{gravityfieldType},
with \config{R} and \config{GM} describing the reference radius and the geocentric constant, respectively.
The $\sigma_n$
stand for the gravity field accuracies (from degree \config{minDegree} to \config{maxDegree}), if they are given.
If no accuracies are provided, the $\sigma_n$
represent the square root of the degree variances of the gravity field.
If \config{maxDegree} exceeds the maximum degree given by \configClass{gravityfield}{gravityfieldType},
the higher degrees are complemented by Kaula's rule
The output of the coefficients is given in the file  \configFile{outputfileCoefficients}{matrix}.


\keepXColumns
\begin{tabularx}{\textwidth}{N T A}
\hline
Name & Type & Annotation\\
\hline
\hfuzz=500pt\includegraphics[width=1em]{element-mustset.pdf}~outputfileCoefficients & \hfuzz=500pt filename & \hfuzz=500pt \\
\hfuzz=500pt\includegraphics[width=1em]{element-unbounded.pdf}~gravityfield & \hfuzz=500pt \hyperref[gravityfieldType]{gravityfield} & \hfuzz=500pt use sigmas, if not given use signal (cnm,snm), if not given use kaulas rule\\
\hfuzz=500pt\includegraphics[width=1em]{element-mustset.pdf}~minDegree & \hfuzz=500pt uint & \hfuzz=500pt \\
\hfuzz=500pt\includegraphics[width=1em]{element.pdf}~maxDegree & \hfuzz=500pt uint & \hfuzz=500pt \\
\hfuzz=500pt\includegraphics[width=1em]{element.pdf}~GM & \hfuzz=500pt double & \hfuzz=500pt Geocentric gravitational constant\\
\hfuzz=500pt\includegraphics[width=1em]{element.pdf}~R & \hfuzz=500pt double & \hfuzz=500pt reference radius\\
\hfuzz=500pt\includegraphics[width=1em]{element.pdf}~kaulaPower & \hfuzz=500pt double & \hfuzz=500pt sigma = kaulaFactor/degree\textasciicircum{}kaulaPower\\
\hfuzz=500pt\includegraphics[width=1em]{element.pdf}~kaulaFactor & \hfuzz=500pt double & \hfuzz=500pt sigma = kaulaFactor/degree\textasciicircum{}kaulaPower\\
\hline
\end{tabularx}

\clearpage
%==================================
\subsection{SatelliteModelCreate}\label{SatelliteModelCreate}
This program creates a satellite macro model for the estimation of non-gravitational accelerations acting on a satellite.
Mandatory input values are the \config{satelliteName}, \config{mass}, \config{coefficientDrag} and information
about the satellite \config{surfaces}. For low Earth orbiting satellites, like GRACE for instance, a good guess
for the drag coefficient could be 2.3. Apart from that, it is latter on possible to estimate a more precise variable drag coefficient
(e.g. \configClass{miscAccelerations:atmosphericDrag}{miscAccelerationsType:atmosphericDrag}), which will override this initial guess.
Concerning the satellite surfaces an external file must be imported which must contain information about each single
 satellite plate in terms of plate \config{area}, the associated plate normal and re-radiation properties
(reflexion, diffusion and absorption) properties in the visible and IR part. Examplarily, a description of
the macro model for GRACE can be found under:
\url{https://podaac-tools.jpl.nasa.gov/drive/files/allData/grace/docs/ProdSpecDoc_v4.6.pdf}
Additionally, it is possible to add further information like antennaThrust, solar panel, temporal mass changes and
massInstrument using the modules option.


\keepXColumns
\begin{tabularx}{\textwidth}{N T A}
\hline
Name & Type & Annotation\\
\hline
\hfuzz=500pt\includegraphics[width=1em]{element-mustset.pdf}~outputfileSatelliteModel & \hfuzz=500pt filename & \hfuzz=500pt \\
\hfuzz=500pt\includegraphics[width=1em]{element-mustset-unbounded.pdf}~satellite & \hfuzz=500pt sequence & \hfuzz=500pt \\
\hfuzz=500pt\includegraphics[width=1em]{connector.pdf}\includegraphics[width=1em]{element-mustset.pdf}~satelliteName & \hfuzz=500pt string & \hfuzz=500pt \\
\hfuzz=500pt\includegraphics[width=1em]{connector.pdf}\includegraphics[width=1em]{element-mustset.pdf}~mass & \hfuzz=500pt double & \hfuzz=500pt \\
\hfuzz=500pt\includegraphics[width=1em]{connector.pdf}\includegraphics[width=1em]{element-mustset.pdf}~coefficientDrag & \hfuzz=500pt double & \hfuzz=500pt \\
\hfuzz=500pt\includegraphics[width=1em]{connector.pdf}\includegraphics[width=1em]{element-mustset.pdf}~surfaces & \hfuzz=500pt sequence & \hfuzz=500pt \\
\hfuzz=500pt\quad\includegraphics[width=1em]{connector.pdf}\includegraphics[width=1em]{element-mustset.pdf}~inputfile & \hfuzz=500pt filename & \hfuzz=500pt each line must contain one surface element\\
\hfuzz=500pt\quad\includegraphics[width=1em]{connector.pdf}\includegraphics[width=1em]{element-mustset.pdf}~type & \hfuzz=500pt expression & \hfuzz=500pt 0: plate, 1: sphere, 2: cylinder\\
\hfuzz=500pt\quad\includegraphics[width=1em]{connector.pdf}\includegraphics[width=1em]{element-mustset.pdf}~area & \hfuzz=500pt expression & \hfuzz=500pt [m**2]\\
\hfuzz=500pt\quad\includegraphics[width=1em]{connector.pdf}\includegraphics[width=1em]{element-mustset.pdf}~normalX & \hfuzz=500pt expression & \hfuzz=500pt \\
\hfuzz=500pt\quad\includegraphics[width=1em]{connector.pdf}\includegraphics[width=1em]{element-mustset.pdf}~normalY & \hfuzz=500pt expression & \hfuzz=500pt \\
\hfuzz=500pt\quad\includegraphics[width=1em]{connector.pdf}\includegraphics[width=1em]{element-mustset.pdf}~normalZ & \hfuzz=500pt expression & \hfuzz=500pt \\
\hfuzz=500pt\quad\includegraphics[width=1em]{connector.pdf}\includegraphics[width=1em]{element-mustset.pdf}~reflexionVisible & \hfuzz=500pt expression & \hfuzz=500pt \\
\hfuzz=500pt\quad\includegraphics[width=1em]{connector.pdf}\includegraphics[width=1em]{element-mustset.pdf}~diffusionVisible & \hfuzz=500pt expression & \hfuzz=500pt \\
\hfuzz=500pt\quad\includegraphics[width=1em]{connector.pdf}\includegraphics[width=1em]{element-mustset.pdf}~absorptionVisible & \hfuzz=500pt expression & \hfuzz=500pt \\
\hfuzz=500pt\quad\includegraphics[width=1em]{connector.pdf}\includegraphics[width=1em]{element-mustset.pdf}~reflexionInfrared & \hfuzz=500pt expression & \hfuzz=500pt \\
\hfuzz=500pt\quad\includegraphics[width=1em]{connector.pdf}\includegraphics[width=1em]{element-mustset.pdf}~diffusionInfrared & \hfuzz=500pt expression & \hfuzz=500pt \\
\hfuzz=500pt\quad\includegraphics[width=1em]{connector.pdf}\includegraphics[width=1em]{element-mustset.pdf}~absorptionInfrared & \hfuzz=500pt expression & \hfuzz=500pt \\
\hfuzz=500pt\quad\includegraphics[width=1em]{connector.pdf}\includegraphics[width=1em]{element-mustset.pdf}~hasThermalReemission & \hfuzz=500pt expression & \hfuzz=500pt (0 = no, 1 = yes) spontaneous reemission of absorbed radiation\\
\hfuzz=500pt\includegraphics[width=1em]{connector.pdf}\includegraphics[width=1em]{element-unbounded.pdf}~module & \hfuzz=500pt choice & \hfuzz=500pt \\
\hfuzz=500pt\quad\includegraphics[width=1em]{connector.pdf}\includegraphics[width=1em]{element-mustset.pdf}~antennaThrust & \hfuzz=500pt sequence & \hfuzz=500pt \\
\hfuzz=500pt\quad\quad\includegraphics[width=1em]{connector.pdf}\includegraphics[width=1em]{element.pdf}~thrustX & \hfuzz=500pt double & \hfuzz=500pt \\
\hfuzz=500pt\quad\quad\includegraphics[width=1em]{connector.pdf}\includegraphics[width=1em]{element.pdf}~thrustY & \hfuzz=500pt double & \hfuzz=500pt \\
\hfuzz=500pt\quad\quad\includegraphics[width=1em]{connector.pdf}\includegraphics[width=1em]{element.pdf}~thrustZ & \hfuzz=500pt double & \hfuzz=500pt \\
\hfuzz=500pt\quad\includegraphics[width=1em]{connector.pdf}\includegraphics[width=1em]{element-mustset.pdf}~solarPanel & \hfuzz=500pt sequence & \hfuzz=500pt \\
\hfuzz=500pt\quad\quad\includegraphics[width=1em]{connector.pdf}\includegraphics[width=1em]{element.pdf}~rotationAxisX & \hfuzz=500pt double & \hfuzz=500pt \\
\hfuzz=500pt\quad\quad\includegraphics[width=1em]{connector.pdf}\includegraphics[width=1em]{element.pdf}~rotationAxisY & \hfuzz=500pt double & \hfuzz=500pt \\
\hfuzz=500pt\quad\quad\includegraphics[width=1em]{connector.pdf}\includegraphics[width=1em]{element.pdf}~rotationAxisZ & \hfuzz=500pt double & \hfuzz=500pt \\
\hfuzz=500pt\quad\quad\includegraphics[width=1em]{connector.pdf}\includegraphics[width=1em]{element.pdf}~normalX & \hfuzz=500pt double & \hfuzz=500pt Direction to sun\\
\hfuzz=500pt\quad\quad\includegraphics[width=1em]{connector.pdf}\includegraphics[width=1em]{element.pdf}~normalY & \hfuzz=500pt double & \hfuzz=500pt Direction to sun\\
\hfuzz=500pt\quad\quad\includegraphics[width=1em]{connector.pdf}\includegraphics[width=1em]{element.pdf}~normalZ & \hfuzz=500pt double & \hfuzz=500pt Direction to sun\\
\hfuzz=500pt\quad\quad\includegraphics[width=1em]{connector.pdf}\includegraphics[width=1em]{element-mustset-unbounded.pdf}~indexSurface & \hfuzz=500pt uint & \hfuzz=500pt index of solar panel surfaces\\
\hfuzz=500pt\quad\includegraphics[width=1em]{connector.pdf}\includegraphics[width=1em]{element-mustset.pdf}~massChange & \hfuzz=500pt sequence & \hfuzz=500pt \\
\hfuzz=500pt\quad\quad\includegraphics[width=1em]{connector.pdf}\includegraphics[width=1em]{element-mustset-unbounded.pdf}~time & \hfuzz=500pt time & \hfuzz=500pt \\
\hfuzz=500pt\quad\quad\includegraphics[width=1em]{connector.pdf}\includegraphics[width=1em]{element-mustset-unbounded.pdf}~mass & \hfuzz=500pt double & \hfuzz=500pt \\
\hfuzz=500pt\quad\includegraphics[width=1em]{connector.pdf}\includegraphics[width=1em]{element-mustset.pdf}~massInstrument & \hfuzz=500pt sequence & \hfuzz=500pt \\
\hfuzz=500pt\quad\quad\includegraphics[width=1em]{connector.pdf}\includegraphics[width=1em]{element-mustset-unbounded.pdf}~inputfileInstrument & \hfuzz=500pt filename & \hfuzz=500pt \\
\hline
\end{tabularx}

\clearpage
%==================================
\subsection{TemporalRepresentation2TimeSeries}\label{TemporalRepresentation2TimeSeries}
This program computes the design matrix of temporal representation at a given time series.
The output matrix contains the time steps in MJD in the first column, the other columns contain the design matrix.
The intention of this program is to visualize the parametrization together with \program{PlotGraph}.


\keepXColumns
\begin{tabularx}{\textwidth}{N T A}
\hline
Name & Type & Annotation\\
\hline
\hfuzz=500pt\includegraphics[width=1em]{element-mustset.pdf}~outputfileMatrix & \hfuzz=500pt filename & \hfuzz=500pt Time (MJD) in first column, design matrix follows\\
\hfuzz=500pt\includegraphics[width=1em]{element-mustset-unbounded.pdf}~timeSeries & \hfuzz=500pt \hyperref[timeSeriesType]{timeSeries} & \hfuzz=500pt \\
\hfuzz=500pt\includegraphics[width=1em]{element-mustset-unbounded.pdf}~temporal & \hfuzz=500pt \hyperref[parametrizationTemporalType]{parametrizationTemporal} & \hfuzz=500pt \\
\hline
\end{tabularx}

\clearpage
%==================================
\subsection{ThermosphericState2GriddedData}\label{ThermosphericState2GriddedData}
This program converts the output (neutral mass density,temperature) of an empirical thermosphere model (e.g. JB2008) on a given \configClass{grid}{gridType}.
Additionally, also the thermospheric winds estimated by using the horizontal wind model HWM 2014 can be assessed.
The time for the evaluation can be specified in \config{time}. The values will be saved together with points expressed as ellipsoidal coordinates
(longitude, latitude, height) based on a reference ellipsoid with parameters \config{R} and \config{inverseFlattening}.
\fig{!hb}{1.0}{thermosphericState2GriddedData}{fig:thermosphericState2GriddedData}{JB2008 model in 300 km height at 2003-07-01 12:00.}


\keepXColumns
\begin{tabularx}{\textwidth}{N T A}
\hline
Name & Type & Annotation\\
\hline
\hfuzz=500pt\includegraphics[width=1em]{element-mustset.pdf}~outputfileGriddedData & \hfuzz=500pt filename & \hfuzz=500pt density [kg/m**3], temperature [K], wind (x, y, z) [m/s**2]\\
\hfuzz=500pt\includegraphics[width=1em]{element-mustset.pdf}~thermosphere & \hfuzz=500pt \hyperref[thermosphereType]{thermosphere} & \hfuzz=500pt \\
\hfuzz=500pt\includegraphics[width=1em]{element-mustset-unbounded.pdf}~grid & \hfuzz=500pt \hyperref[gridType]{grid} & \hfuzz=500pt \\
\hfuzz=500pt\includegraphics[width=1em]{element-mustset.pdf}~time & \hfuzz=500pt time & \hfuzz=500pt \\
\hfuzz=500pt\includegraphics[width=1em]{element.pdf}~localReferenceFrame & \hfuzz=500pt boolean & \hfuzz=500pt wind in local north, east, up, otherwise global terrestrial\\
\hfuzz=500pt\includegraphics[width=1em]{element.pdf}~R & \hfuzz=500pt double & \hfuzz=500pt reference radius for ellipsoidal coordinates on output\\
\hfuzz=500pt\includegraphics[width=1em]{element.pdf}~inverseFlattening & \hfuzz=500pt double & \hfuzz=500pt reference flattening for ellipsoidal coordinates on output, 0: spherical coordinates\\
\hline
\end{tabularx}

This program is \reference{parallelized}{general.parallelization}.
\clearpage
%==================================
\subsection{TimeSeries2PotentialCoefficients}\label{TimeSeries2PotentialCoefficients}
Interpret the data columns of \configFile{inputfileTimeSeries}{instrument}
as potential coefficients. The sequence of coefficients is given by
\configClass{numbering}{sphericalHarmonicsNumberingType} starting from data column \config{startDataFields}.

For each epoch a \configFile{outputfilesPotentialCoefficients}{potentialCoefficients}
is written where the \config{variableLoopTime} and \config{variableLoopIndex} are expanded for
each point of the given time series to create the file name for this epoch,
see \reference{text parser}{general.parser:text}.

See also \program{Gravityfield2PotentialCoefficientsTimeSeries}.


\keepXColumns
\begin{tabularx}{\textwidth}{N T A}
\hline
Name & Type & Annotation\\
\hline
\hfuzz=500pt\includegraphics[width=1em]{element-mustset.pdf}~outputfilesPotentialCoefficients & \hfuzz=500pt filename & \hfuzz=500pt for each epoch\\
\hfuzz=500pt\includegraphics[width=1em]{element.pdf}~variableLoopTime & \hfuzz=500pt string & \hfuzz=500pt variable with time of each epoch\\
\hfuzz=500pt\includegraphics[width=1em]{element.pdf}~variableLoopIndex & \hfuzz=500pt string & \hfuzz=500pt variable with index of current epoch (starts with zero)\\
\hfuzz=500pt\includegraphics[width=1em]{element.pdf}~variableLoopCount & \hfuzz=500pt string & \hfuzz=500pt variable with total number of epochs\\
\hfuzz=500pt\includegraphics[width=1em]{element-mustset.pdf}~inputfileTimeSeries & \hfuzz=500pt filename & \hfuzz=500pt each epoch: multiple data for points (MISCVALUES)\\
\hfuzz=500pt\includegraphics[width=1em]{element.pdf}~startDataFields & \hfuzz=500pt uint & \hfuzz=500pt first data column\\
\hfuzz=500pt\includegraphics[width=1em]{element-mustset.pdf}~minDegree & \hfuzz=500pt uint & \hfuzz=500pt minimal degree\\
\hfuzz=500pt\includegraphics[width=1em]{element-mustset.pdf}~maxDegree & \hfuzz=500pt uint & \hfuzz=500pt maximal degree\\
\hfuzz=500pt\includegraphics[width=1em]{element.pdf}~GM & \hfuzz=500pt double & \hfuzz=500pt Geocentric gravitational constant\\
\hfuzz=500pt\includegraphics[width=1em]{element.pdf}~R & \hfuzz=500pt double & \hfuzz=500pt reference radius\\
\hfuzz=500pt\includegraphics[width=1em]{element-mustset.pdf}~numbering & \hfuzz=500pt \hyperref[sphericalHarmonicsNumberingType]{sphericalHarmonicsNumbering} & \hfuzz=500pt numbering scheme\\
\hline
\end{tabularx}

\clearpage
%==================================
\subsection{TimeSeriesCreate}\label{TimeSeriesCreate}
This program generates an \file{instrument file}{instrument},
containing a time series.


\keepXColumns
\begin{tabularx}{\textwidth}{N T A}
\hline
Name & Type & Annotation\\
\hline
\hfuzz=500pt\includegraphics[width=1em]{element-mustset.pdf}~outputfileTimeSeries & \hfuzz=500pt filename & \hfuzz=500pt instrument file\\
\hfuzz=500pt\includegraphics[width=1em]{element-mustset-unbounded.pdf}~timeSeries & \hfuzz=500pt \hyperref[timeSeriesType]{timeSeries} & \hfuzz=500pt time series to be created\\
\hfuzz=500pt\includegraphics[width=1em]{element-unbounded.pdf}~data & \hfuzz=500pt expression & \hfuzz=500pt expression of output columns, extra 'epoch' variable\\
\hline
\end{tabularx}

\clearpage
%==================================
\subsection{Variational2OrbitAndStarCamera}\label{Variational2OrbitAndStarCamera}
Extracts the reference \configFile{outputfileOrbit}{instrument}, \configFile{outputfileStarCamera}{instrument},
and \configFile{outputfileEarthRotation}{instrument} from \configFile{inputfileVariational}{variationalEquation}.


\keepXColumns
\begin{tabularx}{\textwidth}{N T A}
\hline
Name & Type & Annotation\\
\hline
\hfuzz=500pt\includegraphics[width=1em]{element.pdf}~outputfileOrbit & \hfuzz=500pt filename & \hfuzz=500pt output orbit (instrument) file\\
\hfuzz=500pt\includegraphics[width=1em]{element.pdf}~outputfileStarCamera & \hfuzz=500pt filename & \hfuzz=500pt output satellite attidude as star camera (instrument) file\\
\hfuzz=500pt\includegraphics[width=1em]{element.pdf}~outputfileEarthRotation & \hfuzz=500pt filename & \hfuzz=500pt output Earth rotation as star camera (instrument) file\\
\hfuzz=500pt\includegraphics[width=1em]{element-mustset.pdf}~inputfileVariational & \hfuzz=500pt filename & \hfuzz=500pt input variational file\\
\hline
\end{tabularx}

\clearpage
%==================================
\section{Programs: NormalEquation}
\subsection{NormalsAccumulate}\label{NormalsAccumulate}
This program accumulates normal equations and writes the total combined system to
\configFile{outputfileNormalequation}{normalEquation}.
The \configFile{inputfileNormalEquation}{normalEquation}s must have all the same size and the same block structure.
This program is the simplified and fast version of the more general program \program{NormalsBuild}.


\keepXColumns
\begin{tabularx}{\textwidth}{N T A}
\hline
Name & Type & Annotation\\
\hline
\hfuzz=500pt\includegraphics[width=1em]{element-mustset.pdf}~outputfileNormalEquation & \hfuzz=500pt filename & \hfuzz=500pt \\
\hfuzz=500pt\includegraphics[width=1em]{element-mustset-unbounded.pdf}~inputfileNormalEquation & \hfuzz=500pt filename & \hfuzz=500pt \\
\hline
\end{tabularx}

\clearpage
%==================================
\subsection{NormalsBuild}\label{NormalsBuild}
This program accumulates \configClass{normalEquation}{normalEquationType}s and
writes the total combined system to \configFile{outputfileNormalequation}{normalEquation}.
For a detailed description of the used algorithm see \configClass{normalEquation}{normalEquationType}.
Large normal equation systems can be divided into blocks with \config{normalsBlockSize}.

A simplifed and fast version of this program is \program{NormalsAccumulate}.
To solve the system of normal equations use \program{NormalsSolverVCE}.


\keepXColumns
\begin{tabularx}{\textwidth}{N T A}
\hline
Name & Type & Annotation\\
\hline
\hfuzz=500pt\includegraphics[width=1em]{element-mustset.pdf}~outputfileNormalEquation & \hfuzz=500pt filename & \hfuzz=500pt \\
\hfuzz=500pt\includegraphics[width=1em]{element-mustset-unbounded.pdf}~normalEquation & \hfuzz=500pt \hyperref[normalEquationType]{normalEquation} & \hfuzz=500pt \\
\hfuzz=500pt\includegraphics[width=1em]{element.pdf}~normalsBlockSize & \hfuzz=500pt uint & \hfuzz=500pt block size for distributing the normal equations, 0: one block\\
\hline
\end{tabularx}

This program is \reference{parallelized}{general.parallelization}.
\clearpage
%==================================
\subsection{NormalsBuildShortTimeStaticLongTime}\label{NormalsBuildShortTimeStaticLongTime}
This program sets up normal equations based on \configClass{observation}{observationType}.
Additionally short time and long time variations can be parametrized based on the static parameters
in \configClass{observation}{observationType} in an efficient way. The observation equations
are divided into time intervals $i \in \{1, ..., N\}$ (e.g. daily) as defined in
\configFile{inputfileArcList}{arcList}.

With \config{estimateLongTimeVariations} additional temporal variations can be co-estimated
for a subset of the parameters selected by \configClass{parameterSelection}{parameterSelectorType}.
These parameters might be spherical harmonic coefficients with a limited maximum degree.
The temporal variations are represented by base functions $\Phi_k(t_i)$ (e.g. trend and annual oscillation)
given by \configClass{parametrizationTemporal}{parametrizationTemporalType}.
The temporal base functions are evaluated at the mid time~$t_i$ of each interval~$i$, multiplicated
with the design matrix $\M A_i$ of the selected parameters, and the design matrix is extended
accordingly.

\fig{!hb}{0.8}{normalsBuildShortTimeStaticLongTime}{fig:normalsBuildShortTimeStaticLongTime}{Schema of the extended design matrix.}

With \config{estimateShortTimeVariations} short time variations of the gravity field can be co-estimated.
Their purpose is to mitigate temporal aliasing.
The short time parameters selected by \configClass{parameterSelection}{parameterSelectorType}
(e.g. daily constant or linear splines every 6 hour) are constrained by an
\configClass{autoregressiveModelSequence}{autoregressiveModelSequenceType}. If only a static parameter
set is selected the coressponding part of the design matrix is copied and modeled as a constant value
per interval in \configFile{inputfileArcList}{arcList} additionally so the corresponding temporal factor can be expressed as
\begin{equation}
  \Phi_i(t)  =
  \begin{cases}
    1 &\text{if} \hspace{5pt} t \in [t_i, t_{i+1}) \\
    0 & \text{otherwise}
  \end{cases}.
\end{equation}

Before writing the normal equations to \configFile{outputfileNormalEquation}{normalEquation}
short time gravity and satellite specific parameters can be eliminated with \config{eliminateParameter}.

Example: For the computation of the mean gravity field ITSG-Grace2018s with additional trend and annual signal
the normal equations are computed month by month and accumulated afterwards (see \program{NormalsAccumulate}).
The observations were divided into daily intervals with \configFile{inputfileArcList}{arcList}.
The static gravity field has been parametrized as spherical harmonics
up to degree $n=200$ in \configClass{observation:parametrizationGravity}{parametrizationGravityType}.
The trend and annual signals defined by
\configClass{estimateLongTimeVariations:parametrizationTemporal}{parametrizationTemporalType}
were estimated for selected parameters up to degree $n=120$.
To mitigate temporal aliasing daily gravity fields up to degree $n=40$ were setup and constrained
with an \configClass{autoregressiveModelSequence}{autoregressiveModelSequenceType} up to order three.

A detailed description of the approach is given in:
Kvas, A., Mayer-Gürr, T. GRACE gravity field recovery with background model uncertainties.
J Geod 93, 2543–2552 (2019). \url{https://doi.org/10.1007/s00190-019-01314-1}.


\keepXColumns
\begin{tabularx}{\textwidth}{N T A}
\hline
Name & Type & Annotation\\
\hline
\hfuzz=500pt\includegraphics[width=1em]{element-mustset.pdf}~outputfileNormalEquation & \hfuzz=500pt filename & \hfuzz=500pt outputfile for normal equations\\
\hfuzz=500pt\includegraphics[width=1em]{element-mustset.pdf}~observation & \hfuzz=500pt \hyperref[observationType]{observation} & \hfuzz=500pt \\
\hfuzz=500pt\includegraphics[width=1em]{element.pdf}~estimateShortTimeVariations & \hfuzz=500pt sequence & \hfuzz=500pt co-estimate short time gravity field variations\\
\hfuzz=500pt\includegraphics[width=1em]{connector.pdf}\includegraphics[width=1em]{element-mustset.pdf}~autoregressiveModelSequence & \hfuzz=500pt \hyperref[autoregressiveModelSequenceType]{autoregressiveModelSequence} & \hfuzz=500pt AR model sequence for constraining short time gravity variations\\
\hfuzz=500pt\includegraphics[width=1em]{connector.pdf}\includegraphics[width=1em]{element-mustset-unbounded.pdf}~parameterSelection & \hfuzz=500pt \hyperref[parameterSelectorType]{parameterSelector} & \hfuzz=500pt parameters describing the short time gravity field\\
\hfuzz=500pt\includegraphics[width=1em]{element.pdf}~estimateLongTimeVariations & \hfuzz=500pt sequence & \hfuzz=500pt co-estimate long time gravity field variations\\
\hfuzz=500pt\includegraphics[width=1em]{connector.pdf}\includegraphics[width=1em]{element-mustset-unbounded.pdf}~parametrizationTemporal & \hfuzz=500pt \hyperref[parametrizationTemporalType]{parametrizationTemporal} & \hfuzz=500pt parametrization of time variations (trend, annual, ...)\\
\hfuzz=500pt\includegraphics[width=1em]{connector.pdf}\includegraphics[width=1em]{element-mustset-unbounded.pdf}~parameterSelection & \hfuzz=500pt \hyperref[parameterSelectorType]{parameterSelector} & \hfuzz=500pt parameters describing the long time gravity field\\
\hfuzz=500pt\includegraphics[width=1em]{element-mustset.pdf}~inputfileArcList & \hfuzz=500pt filename & \hfuzz=500pt list to correspond points of time to arc numbers\\
\hfuzz=500pt\includegraphics[width=1em]{element.pdf}~defaultBlockSize & \hfuzz=500pt uint & \hfuzz=500pt block size for distributing the normal equations, 0: one block\\
\hfuzz=500pt\includegraphics[width=1em]{element.pdf}~eliminateParameter & \hfuzz=500pt boolean & \hfuzz=500pt eliminate short time and state parameter\\
\hline
\end{tabularx}

This program is \reference{parallelized}{general.parallelization}.
\clearpage
%==================================
\subsection{NormalsCreate}\label{NormalsCreate}
Create \file{normal equations}{normalEquation}
from calculated matrices (\configClass{matrixGenerator}{matrixGeneratorType}).

The \configFile{inputfileParameterNames}{parameterName} can be created with \program{ParameterNamesCreate}.

The \configClass{normalMatrix}{matrixGeneratorType} must be symmetric.
The \configClass{rightHandSide}{matrixGeneratorType} must have the same number of rows
and can contain multiple columns for multiple solutions.

The Vector $\M l^T\M P\M l$ is the quadratic sum of observations for each column of the right hand side.
It is used to determine the aposteriori accuracy
\begin{equation}
\hat{\sigma}^2 = \frac{\hat{\M e}^T\M P\hat{\M e}}{n-m} = \frac{\M l^T\M P\M l - \M n^T\hat{\M x}}{n-m}.
\end{equation}
If the vector is not given, it is automatically determined by assuming $\hat{\sigma}^2=1$.

The number of observations~$n$ is given by the expression \config{observationCount}.
The variable \verb|observationCount| can be used, if it is set by a normal equation file
\configFile{inputfileNormalEquationObsCount}{normalEquation}.


\keepXColumns
\begin{tabularx}{\textwidth}{N T A}
\hline
Name & Type & Annotation\\
\hline
\hfuzz=500pt\includegraphics[width=1em]{element-mustset.pdf}~outputfileNormalEquation & \hfuzz=500pt filename & \hfuzz=500pt \\
\hfuzz=500pt\includegraphics[width=1em]{element.pdf}~inputfileParameterNames & \hfuzz=500pt filename & \hfuzz=500pt \\
\hfuzz=500pt\includegraphics[width=1em]{element-mustset-unbounded.pdf}~normalMatrix & \hfuzz=500pt \hyperref[matrixGeneratorType]{matrixGenerator} & \hfuzz=500pt \\
\hfuzz=500pt\includegraphics[width=1em]{element-mustset-unbounded.pdf}~rightHandSide & \hfuzz=500pt \hyperref[matrixGeneratorType]{matrixGenerator} & \hfuzz=500pt \\
\hfuzz=500pt\includegraphics[width=1em]{element-unbounded.pdf}~lPl & \hfuzz=500pt \hyperref[matrixGeneratorType]{matrixGenerator} & \hfuzz=500pt vector with size of rhs columns\\
\hfuzz=500pt\includegraphics[width=1em]{element.pdf}~inputfileNormalEquationObsCount & \hfuzz=500pt filename & \hfuzz=500pt sets the variable observationCount\\
\hfuzz=500pt\includegraphics[width=1em]{element-mustset.pdf}~observationCount & \hfuzz=500pt expression & \hfuzz=500pt (variables: rows, columns (rhs), observationCount)\\
\hline
\end{tabularx}

\clearpage
%==================================
\subsection{NormalsEliminate}\label{NormalsEliminate}
This program eliminates parameters from a system of \configFile{inputfileNormalEquation}{normalEquation}s.
To just remove (cutting out) parameters use \program{NormalsReorder}.

The \configClass{remainingParameters}{parameterSelectorType} allows the selection
of parameters that will remain, all others will be eliminated. The order of remaining parameters
can be modified via the parameter selection. Block size of the output normal matrix can be adjusted with
\config{outBlockSize}. If it is set to zero, the \configFile{outputfileNormalEquation}{normalEquation}
is written to a single block file.

For example the normal equations are divided into two groups of
parameters $\hat{\M x}_1$ and $\hat{\M x}_2$ according to
\begin{equation}
\begin{pmatrix}
  \M N_{11} & \M N_{12} \\
  \M N_{21} & \M N_{22}
\end{pmatrix}
\begin{pmatrix} \hat{\M x}_1 \\ \hat{\M x}_2 \end{pmatrix}
=
\begin{pmatrix}
  \M n_1 \\
  \M n_2
\end{pmatrix}.
\end{equation}
and $\hat{\M x}_2$ shall be eliminated, the reduced system of normal equations is given by
\begin{equation}
\bar{\M N}\hat{\M x} = \bar{\M n}
\qquad\text{with}\qquad
\bar{\M N}=\M N_{11}-\M N_{12}\M N_{22}^{-1}\M N_{12}^T
\qquad\text{and}\qquad\bar{\M n} =  \M n_1 - \M N_{12}\M N_{22}^{-1}\M n_2.
\end{equation}

See also \program{NormalsReorder}.


\keepXColumns
\begin{tabularx}{\textwidth}{N T A}
\hline
Name & Type & Annotation\\
\hline
\hfuzz=500pt\includegraphics[width=1em]{element-mustset.pdf}~outputfileNormalEquation & \hfuzz=500pt filename & \hfuzz=500pt \\
\hfuzz=500pt\includegraphics[width=1em]{element-mustset.pdf}~inputfileNormalEquation & \hfuzz=500pt filename & \hfuzz=500pt \\
\hfuzz=500pt\includegraphics[width=1em]{element-mustset-unbounded.pdf}~remainingParameters & \hfuzz=500pt \hyperref[parameterSelectorType]{parameterSelector} & \hfuzz=500pt parameter order/selection of output normal equations\\
\hfuzz=500pt\includegraphics[width=1em]{element.pdf}~outBlockSize & \hfuzz=500pt uint & \hfuzz=500pt block size for distributing the normal equations, 0: one block\\
\hline
\end{tabularx}

This program is \reference{parallelized}{general.parallelization}.
\clearpage
%==================================
\subsection{NormalsMultiplyAdd}\label{NormalsMultiplyAdd}
This program modifies \configFile{inputfileNormalEquation}{normalEquation} in a way
that $\bar{\M x}$ is estimated instead of $\M x$.
\begin{equation}
 \bar{\M x} := \M x + \alpha\, \M x_0,
\end{equation}
where $\M x_0$ is \configFile{inputfileParameter}{matrix} and $\alpha$ is \config{factor}.
This can be used to re-add reduced reference fields before a combined estimation
at normal equation level.
Therefore the right hand side of the normal equations is modified by
\begin{equation}
 \bar{\M n} := \M n + \alpha\,\M N\M x_0,
\end{equation}
and the quadratic sum of observations by
\begin{equation}
 \bar{\M l^T\M P\M l} := \M l^T\M P\M l + \alpha^2\,\M x_0^T\M N\M x_0 + 2\alpha\,\M x_0^T\M n
\end{equation}

As the normal matrix itself is not modified, rewriting of the matrix can be disabled by setting
\config{writeNormalMatrix} to false.


\keepXColumns
\begin{tabularx}{\textwidth}{N T A}
\hline
Name & Type & Annotation\\
\hline
\hfuzz=500pt\includegraphics[width=1em]{element-mustset.pdf}~outputfileNormalEquation & \hfuzz=500pt filename & \hfuzz=500pt \\
\hfuzz=500pt\includegraphics[width=1em]{element-mustset.pdf}~inputfileNormalEquation & \hfuzz=500pt filename & \hfuzz=500pt \\
\hfuzz=500pt\includegraphics[width=1em]{element-mustset.pdf}~inputfileParameter & \hfuzz=500pt filename & \hfuzz=500pt x\\
\hfuzz=500pt\includegraphics[width=1em]{element.pdf}~factor & \hfuzz=500pt double & \hfuzz=500pt alpha\\
\hfuzz=500pt\includegraphics[width=1em]{element.pdf}~writeNormalMatrix & \hfuzz=500pt boolean & \hfuzz=500pt write full coefficient matrix, right hand sides and info files\\
\hline
\end{tabularx}

This program is \reference{parallelized}{general.parallelization}.
\clearpage
%==================================
\subsection{NormalsRegularizationBorders}\label{NormalsRegularizationBorders}
This program sets up two regularization matrices for two different regional areas.
For a given set of points defined by \configClass{grid}{gridType} it is evaluated, whether each point
(corresponding to an unknown parameter of a respective parameterization by space localizing basis functions)
is inside or outside a certain area given by \configClass{border}{borderType}.
Each regularization matrix is a diagonal matrix, one of them features a one if the
point is inside, and a zero if the point lies outside the area. The other matrix features
a zero if the point is inside, and a one if the point lies outside the area
This results in two regularization matrices with
\begin{equation}
\M R_1+\M R_2=\M I.
\end{equation}
The two matrices are provided as vectors of the diagonal
in the output files \configFile{outputfileOutside}{matrix} and \configFile{outputfileInside}{matrix}.
The regularization matrices are then used by \configClass{normalEquation:regularization}{normalEquationType:regularization}.
As an example, the two different areas could be oceanic regions on the one hand and continental areas on the other hand.


\keepXColumns
\begin{tabularx}{\textwidth}{N T A}
\hline
Name & Type & Annotation\\
\hline
\hfuzz=500pt\includegraphics[width=1em]{element-mustset.pdf}~outputfileInside & \hfuzz=500pt filename & \hfuzz=500pt \\
\hfuzz=500pt\includegraphics[width=1em]{element-mustset.pdf}~outputfileOutside & \hfuzz=500pt filename & \hfuzz=500pt \\
\hfuzz=500pt\includegraphics[width=1em]{element-mustset-unbounded.pdf}~grid & \hfuzz=500pt \hyperref[gridType]{grid} & \hfuzz=500pt nodal point distribution of parameters, e.g harmonics splines\\
\hfuzz=500pt\includegraphics[width=1em]{element-mustset-unbounded.pdf}~border & \hfuzz=500pt \hyperref[borderType]{border} & \hfuzz=500pt regularization areas, e.g land and ocean\\
\hline
\end{tabularx}

\clearpage
%==================================
\subsection{NormalsRegularizationSphericalHarmonics}\label{NormalsRegularizationSphericalHarmonics}
Diagonal regularization matrix from gravity field accuracies,
if not given from signal (cnm,snm), if not given from kaulas rule.
The inverse accuracies $1/\sigma_n^2$ are used as weights in the regularization matrix.
The diagonal is saved as Vector.

The corresponding pseudo observations can be computed with \program{Gravityfield2SphericalHarmonicsVector}.


\keepXColumns
\begin{tabularx}{\textwidth}{N T A}
\hline
Name & Type & Annotation\\
\hline
\hfuzz=500pt\includegraphics[width=1em]{element-mustset.pdf}~outputfileDiagonalmatrix & \hfuzz=500pt filename & \hfuzz=500pt \\
\hfuzz=500pt\includegraphics[width=1em]{element-unbounded.pdf}~gravityfield & \hfuzz=500pt \hyperref[gravityfieldType]{gravityfield} & \hfuzz=500pt use sigmas, if not given use signal (cnm,snm), if not given use kaulas rule\\
\hfuzz=500pt\includegraphics[width=1em]{element.pdf}~minRegularizationDegree & \hfuzz=500pt uint & \hfuzz=500pt \\
\hfuzz=500pt\includegraphics[width=1em]{element.pdf}~maxRegularizationDegree & \hfuzz=500pt uint & \hfuzz=500pt \\
\hfuzz=500pt\includegraphics[width=1em]{element-mustset.pdf}~minDegree & \hfuzz=500pt uint & \hfuzz=500pt \\
\hfuzz=500pt\includegraphics[width=1em]{element-mustset.pdf}~maxDegree & \hfuzz=500pt uint & \hfuzz=500pt \\
\hfuzz=500pt\includegraphics[width=1em]{element-mustset.pdf}~numbering & \hfuzz=500pt \hyperref[sphericalHarmonicsNumberingType]{sphericalHarmonicsNumbering} & \hfuzz=500pt numbering scheme for regul matrix\\
\hfuzz=500pt\includegraphics[width=1em]{element.pdf}~GM & \hfuzz=500pt double & \hfuzz=500pt Geocentric gravitational constant\\
\hfuzz=500pt\includegraphics[width=1em]{element.pdf}~R & \hfuzz=500pt double & \hfuzz=500pt reference radius\\
\hfuzz=500pt\includegraphics[width=1em]{element.pdf}~makeIsotropic & \hfuzz=500pt boolean & \hfuzz=500pt \\
\hfuzz=500pt\includegraphics[width=1em]{element.pdf}~kaulaPower & \hfuzz=500pt double & \hfuzz=500pt sigma = kaulaFactor*degree**kaulaPower\\
\hfuzz=500pt\includegraphics[width=1em]{element.pdf}~kaulaFactor & \hfuzz=500pt double & \hfuzz=500pt sigma = kaulaFactor*degree**kaulaPower\\
\hline
\end{tabularx}

\clearpage
%==================================
\subsection{NormalsReorder}\label{NormalsReorder}
Reorder \configFile{inputfileNormalEquation}{normalEquation} by selecting parameters in a specific order.
The \configClass{parameterSelection}{parameterSelectorType} also allows one to change dimension of the normal equations,
either by cutting parameters or by inserting zero rows/columns for additional parameters.
Without \configClass{parameterSelection}{parameterSelectorType} the order of parameters remains the same.
Additionally the block sizes of the files can be adjusted. If \config{outBlockSize} is set to zero,
the normal matrix is written to a single block file, which is needed by some programs.

To eliminate parameters without changing the result of the other parameters use \program{NormalsEliminate}.


\keepXColumns
\begin{tabularx}{\textwidth}{N T A}
\hline
Name & Type & Annotation\\
\hline
\hfuzz=500pt\includegraphics[width=1em]{element-mustset.pdf}~outputfileNormalEquation & \hfuzz=500pt filename & \hfuzz=500pt \\
\hfuzz=500pt\includegraphics[width=1em]{element-mustset.pdf}~inputfileNormalEquation & \hfuzz=500pt filename & \hfuzz=500pt \\
\hfuzz=500pt\includegraphics[width=1em]{element-unbounded.pdf}~parameterSelection & \hfuzz=500pt \hyperref[parameterSelectorType]{parameterSelector} & \hfuzz=500pt parameter order/selection of output normal equations\\
\hfuzz=500pt\includegraphics[width=1em]{element.pdf}~outBlockSize & \hfuzz=500pt uint & \hfuzz=500pt block size for distributing the normal equations, 0: one block\\
\hline
\end{tabularx}

This program is \reference{parallelized}{general.parallelization}.
\clearpage
%==================================
\subsection{NormalsScale}\label{NormalsScale}
Scales rows and columns of a system of \configFile{inputfileNormalEquation}{normalEquation}
given by a diagonal matrix \configFile{inputfileFactorVector}{matrix} $\M S$
\begin{equation}
  \bar{\M N} := \M S \M N \M S \qquad\text{and}\qquad \bar{\M n} := \M S \M n.
\end{equation}
The estimated solution is now
\begin{equation}
  \bar{\M x} := \M S^{-1} \M x.
\end{equation}
This is effectively the same as rescaling columns of the design matrix.
This program is useful when combining normal equations from different sources,
for example in case the units of certain parameters don't match.


\keepXColumns
\begin{tabularx}{\textwidth}{N T A}
\hline
Name & Type & Annotation\\
\hline
\hfuzz=500pt\includegraphics[width=1em]{element-mustset.pdf}~outputfileNormalEquation & \hfuzz=500pt filename & \hfuzz=500pt \\
\hfuzz=500pt\includegraphics[width=1em]{element-mustset.pdf}~inputfileNormalEquation & \hfuzz=500pt filename & \hfuzz=500pt \\
\hfuzz=500pt\includegraphics[width=1em]{element-mustset.pdf}~inputfileFactorVector & \hfuzz=500pt filename & \hfuzz=500pt Vector containing the factors\\
\hline
\end{tabularx}

This program is \reference{parallelized}{general.parallelization}.
\clearpage
%==================================
\subsection{NormalsSolverVCE}\label{NormalsSolverVCE}
This program accumulates \configClass{normalEquation}{normalEquationType}
and solves the total combined system.
The relative weigthing between the individual normals is determined iteratively
by means of variance component estimation (VCE). For a detailed description
of the used algorithm see \configClass{normalEquation}{normalEquationType}.

Besides the estimated parameter vector (\configFile{outputfileSolution}{matrix}) the
estimated accuracies (\configFile{outputfileSigmax}{matrix}) and the full covariance matrix
(\configFile{outputfileCovariance}{matrix}) can be saved. Also the combined normal system
can be written to \configFile{outputfileNormalEquation}{normalEquation}.

The \configFile{outputfileContribution}{matrix} is a matrix with rows for each estimated
parameter and columns for each \configClass{normalEquation}{normalEquationType}
and indicates the contribution of the individual normals to the estimated parameters.
Each row sum up to one.

See also \program{NormalsBuild}.


\keepXColumns
\begin{tabularx}{\textwidth}{N T A}
\hline
Name & Type & Annotation\\
\hline
\hfuzz=500pt\includegraphics[width=1em]{element.pdf}~outputfileSolution & \hfuzz=500pt filename & \hfuzz=500pt parameter vector\\
\hfuzz=500pt\includegraphics[width=1em]{element.pdf}~outputfileSigmax & \hfuzz=500pt filename & \hfuzz=500pt standard deviations of the parameters (sqrt of the diagonal of the inverse normal equation)\\
\hfuzz=500pt\includegraphics[width=1em]{element.pdf}~outputfileCovariance & \hfuzz=500pt filename & \hfuzz=500pt full covariance matrix\\
\hfuzz=500pt\includegraphics[width=1em]{element.pdf}~outputfileContribution & \hfuzz=500pt filename & \hfuzz=500pt contribution of normal system components to the solution vector\\
\hfuzz=500pt\includegraphics[width=1em]{element.pdf}~outputfileVarianceFactors & \hfuzz=500pt filename & \hfuzz=500pt estimated variance factors as vector\\
\hfuzz=500pt\includegraphics[width=1em]{element.pdf}~outputfileNormalEquation & \hfuzz=500pt filename & \hfuzz=500pt the combined normal equation system\\
\hfuzz=500pt\includegraphics[width=1em]{element-mustset-unbounded.pdf}~normalEquation & \hfuzz=500pt \hyperref[normalEquationType]{normalEquation} & \hfuzz=500pt \\
\hfuzz=500pt\includegraphics[width=1em]{element.pdf}~inputfileApproxSolution & \hfuzz=500pt filename & \hfuzz=500pt to accelerate convergence\\
\hfuzz=500pt\includegraphics[width=1em]{element.pdf}~rightHandSideNumberVCE & \hfuzz=500pt uint & \hfuzz=500pt the right hand side number for estimation of variance factors\\
\hfuzz=500pt\includegraphics[width=1em]{element.pdf}~normalsBlockSize & \hfuzz=500pt uint & \hfuzz=500pt block size for distributing the normal equations, 0: one block\\
\hfuzz=500pt\includegraphics[width=1em]{element.pdf}~maxIterationCount & \hfuzz=500pt uint & \hfuzz=500pt maximum number of iterations for variance component estimation\\
\hline
\end{tabularx}

This program is \reference{parallelized}{general.parallelization}.
\clearpage
%==================================
\subsection{NormalsTemporalCombination}\label{NormalsTemporalCombination}
This program reads a times series of \configFile{inputfileNormalequation}{normalEquation}
with asscociated \configClass{timeSeries}{timeSeriesType} and setup a new combined normal equation system.
For each parameter a \configClass{parametrizationTemporal}{parametrizationTemporalType} is used.

It can be used to estimate trend and annual spherical harmonic coefficients from monthly GRACE normal equations.


\keepXColumns
\begin{tabularx}{\textwidth}{N T A}
\hline
Name & Type & Annotation\\
\hline
\hfuzz=500pt\includegraphics[width=1em]{element-mustset.pdf}~outputfileNormalEquation & \hfuzz=500pt filename & \hfuzz=500pt \\
\hfuzz=500pt\includegraphics[width=1em]{element-mustset-unbounded.pdf}~inputfileNormalEquation & \hfuzz=500pt filename & \hfuzz=500pt normal equations for each point in time\\
\hfuzz=500pt\includegraphics[width=1em]{element-mustset-unbounded.pdf}~timeSeries & \hfuzz=500pt \hyperref[timeSeriesType]{timeSeries} & \hfuzz=500pt times of each normal equations\\
\hfuzz=500pt\includegraphics[width=1em]{element-mustset-unbounded.pdf}~parametrizationTemporal & \hfuzz=500pt \hyperref[parametrizationTemporalType]{parametrizationTemporal} & \hfuzz=500pt \\
\hline
\end{tabularx}

This program is \reference{parallelized}{general.parallelization}.
\clearpage
%==================================
\subsection{ParameterNamesCreate}\label{ParameterNamesCreate}
Generate a \configFile{outputfileParameterNames}{parameterName} by \configClass{parameterName}{parameterNamesType}.
This file can be used in \program{NormalsCreate} or in the class \configClass{parameterSelector}{parameterSelectorType}.


\keepXColumns
\begin{tabularx}{\textwidth}{N T A}
\hline
Name & Type & Annotation\\
\hline
\hfuzz=500pt\includegraphics[width=1em]{element-mustset.pdf}~outputfileParameterNames & \hfuzz=500pt filename & \hfuzz=500pt output parameter names file\\
\hfuzz=500pt\includegraphics[width=1em]{element-mustset-unbounded.pdf}~parameterName & \hfuzz=500pt \hyperref[parameterNamesType]{parameterNames} & \hfuzz=500pt \\
\hline
\end{tabularx}

\clearpage
%==================================
\subsection{ParameterSelection2IndexVector}\label{ParameterSelection2IndexVector}
Generate index vector from parameter selection in \file{matrix format}{matrix}.
This vector can be used in \program{MatrixCalculate}
with \configClass{matrix:reorder}{matrixGeneratorType:reorder}
to reorder arbitrary vectors and matrices similar to \program{NormalsReorder}.

The \configClass{parameterSelection}{parameterSelectorType} allows reordering and dimension changes,
either by cutting parameters or by inserting additional parameters.
\configFile{outputfileIndexVector}{matrix} contains indices of parameters in
\configFile{inputfileParameterNames}{parameterName} or -1 for newly added parameters.
\configFile{outputfileParameterNames}{parameterName} contains the selected parameter names.


\keepXColumns
\begin{tabularx}{\textwidth}{N T A}
\hline
Name & Type & Annotation\\
\hline
\hfuzz=500pt\includegraphics[width=1em]{element.pdf}~outputfileIndexVector & \hfuzz=500pt filename & \hfuzz=500pt indices of source parameters in target normal equations\\
\hfuzz=500pt\includegraphics[width=1em]{element.pdf}~outputfileParameterNames & \hfuzz=500pt filename & \hfuzz=500pt output parameter names file\\
\hfuzz=500pt\includegraphics[width=1em]{element-mustset.pdf}~inputfileParameterNames & \hfuzz=500pt filename & \hfuzz=500pt parameter names file of source normal equations\\
\hfuzz=500pt\includegraphics[width=1em]{element-mustset-unbounded.pdf}~parameterSelection & \hfuzz=500pt \hyperref[parameterSelectorType]{parameterSelector} & \hfuzz=500pt parameter order/selection of target normal equations\\
\hline
\end{tabularx}

\clearpage
%==================================
\section{Programs: Orbit}
\subsection{Orbit2ArgumentOfLatitude}\label{Orbit2ArgumentOfLatitude}
This program computes the argument of latitude of an \file{orbit}{instrument}
and writes it as \file{instrument file}{instrument} (MISCVALUE(S)).
The data of \configFile{inputfileInstrument}{instrument} are appended as values to each epoch.

\fig{!hb}{0.8}{instrumentOrbit2ArgumentOfLatitude}{fig:instrumentOrbit2ArgumentOfLatitude}{Derivation filtered GRACE range-rate residuals.}


\keepXColumns
\begin{tabularx}{\textwidth}{N T A}
\hline
Name & Type & Annotation\\
\hline
\hfuzz=500pt\includegraphics[width=1em]{element-mustset.pdf}~outputfileArgOfLatitude & \hfuzz=500pt filename & \hfuzz=500pt instrument file (MISCVALUE(S): argLat, ...)\\
\hfuzz=500pt\includegraphics[width=1em]{element-mustset.pdf}~inputfileOrbit & \hfuzz=500pt filename & \hfuzz=500pt \\
\hfuzz=500pt\includegraphics[width=1em]{element-unbounded.pdf}~inputfileInstrument & \hfuzz=500pt filename & \hfuzz=500pt data are appended\\
\hline
\end{tabularx}

This program is \reference{parallelized}{general.parallelization}.
\clearpage
%==================================
\subsection{Orbit2BetaPrimeAngle}\label{Orbit2BetaPrimeAngle}
This program computes the beta prime angle (between the orbital plane and earth-sun direction)
and writes it as MISCVALUE(S) \file{instrument file}{instrument}. The angle is calculated w.r.t the sun (per default),
but can be changed.
The data of \configFile{inputfileInstrument}{instrument} are appended as values to each epoch.


\keepXColumns
\begin{tabularx}{\textwidth}{N T A}
\hline
Name & Type & Annotation\\
\hline
\hfuzz=500pt\includegraphics[width=1em]{element-mustset.pdf}~outputfileBetaAngle & \hfuzz=500pt filename & \hfuzz=500pt instrument file (MISCVALUE(S): beta', ...)\\
\hfuzz=500pt\includegraphics[width=1em]{element-mustset.pdf}~inputfileOrbit & \hfuzz=500pt filename & \hfuzz=500pt \\
\hfuzz=500pt\includegraphics[width=1em]{element-unbounded.pdf}~inputfileInstrument & \hfuzz=500pt filename & \hfuzz=500pt data are appended\\
\hfuzz=500pt\includegraphics[width=1em]{element-mustset.pdf}~ephemerides & \hfuzz=500pt \hyperref[ephemeridesType]{ephemerides} & \hfuzz=500pt \\
\hfuzz=500pt\includegraphics[width=1em]{element.pdf}~planet & \hfuzz=500pt \hyperref[planetType]{planet} & \hfuzz=500pt \\
\hline
\end{tabularx}

This program is \reference{parallelized}{general.parallelization}.
\clearpage
%==================================
\subsection{Orbit2EclipseFactor}\label{Orbit2EclipseFactor}
This program generates an \file{instrument file}{instrument} (MISCVALUE(S)) containing the eclipse factor for a given set of orbit.
The data of \configFile{inputfileInstrument}{instrument} are appended as values to each epoch.


\keepXColumns
\begin{tabularx}{\textwidth}{N T A}
\hline
Name & Type & Annotation\\
\hline
\hfuzz=500pt\includegraphics[width=1em]{element-mustset.pdf}~outputfileEclipseFactor & \hfuzz=500pt filename & \hfuzz=500pt instrument file (MISCVALUE(S): eclipse, ...)\\
\hfuzz=500pt\includegraphics[width=1em]{element-mustset.pdf}~inputfileOrbit & \hfuzz=500pt filename & \hfuzz=500pt \\
\hfuzz=500pt\includegraphics[width=1em]{element-unbounded.pdf}~inputfileInstrument & \hfuzz=500pt filename & \hfuzz=500pt data are appended\\
\hfuzz=500pt\includegraphics[width=1em]{element-mustset.pdf}~ephemerides & \hfuzz=500pt \hyperref[ephemeridesType]{ephemerides} & \hfuzz=500pt \\
\hfuzz=500pt\includegraphics[width=1em]{element-mustset.pdf}~eclipse & \hfuzz=500pt \hyperref[eclipseType]{eclipse} & \hfuzz=500pt \\
\hline
\end{tabularx}

This program is \reference{parallelized}{general.parallelization}.
\clearpage
%==================================
\subsection{Orbit2Groundtracks}\label{Orbit2Groundtracks}
This program write \file{satellites positions}{instrument} as \file{gridded data}{griddedData}
(\config{outputfileTrackGriddedData}) in a terrestrial reference frame. The points are expressed as ellipsoidal coordinates
(longitude, latitude, height) based on a reference ellipsoid with parameters \config{R} and
\config{inverseFlattening}. The orbit data are given in the celestial frame so \configClass{earthRotation}{earthRotationType}
is needed to transform the data into the terrestrial frame.
The data of \configFile{inputfileInstrument}{instrument} are appended as values to each point.


\keepXColumns
\begin{tabularx}{\textwidth}{N T A}
\hline
Name & Type & Annotation\\
\hline
\hfuzz=500pt\includegraphics[width=1em]{element-mustset.pdf}~outputfileGriddedData & \hfuzz=500pt filename & \hfuzz=500pt positions as gridded data\\
\hfuzz=500pt\includegraphics[width=1em]{element-mustset.pdf}~inputfileOrbit & \hfuzz=500pt filename & \hfuzz=500pt \\
\hfuzz=500pt\includegraphics[width=1em]{element-unbounded.pdf}~inputfileInstrument & \hfuzz=500pt filename & \hfuzz=500pt values at grid points\\
\hfuzz=500pt\includegraphics[width=1em]{element-mustset.pdf}~earthRotation & \hfuzz=500pt \hyperref[earthRotationType]{earthRotation} & \hfuzz=500pt transformation from CRF to TRF\\
\hfuzz=500pt\includegraphics[width=1em]{element.pdf}~R & \hfuzz=500pt double & \hfuzz=500pt reference radius for ellipsoidal coordinates\\
\hfuzz=500pt\includegraphics[width=1em]{element.pdf}~inverseFlattening & \hfuzz=500pt double & \hfuzz=500pt reference flattening for ellipsoidal coordinates\\
\hline
\end{tabularx}

\clearpage
%==================================
\subsection{Orbit2Kepler}\label{Orbit2Kepler}
This program computes the osculating Keplerian elements from position and velocity
of a given \configFile{inputfileOrbit}{instrument}.
The \configFile{inputfileOrbit}{instrument} must contain positions and velocities (see \program{OrbitAddVelocityAndAcceleration}).

The \config{outputfileKepler} is an \file{instrument file}{instrument} (MISCVALUES)
with the Keplerian elements at each epoch in the following order
\begin{itemize}
\item Ascending Node $\Omega$ [degree]
\item Inclination $i$ [degree]
\item Argument of perigee $\omega$ [degree]
\item major axis $a$ [m]
\item eccentricity $e$
\item mean anomaly $M$ [degree]
\item transit time of perigee $\tau$ [mjd]
\end{itemize}
The data of \configFile{inputfileInstrument}{instrument} are appended as values to each epoch.


\keepXColumns
\begin{tabularx}{\textwidth}{N T A}
\hline
Name & Type & Annotation\\
\hline
\hfuzz=500pt\includegraphics[width=1em]{element-mustset.pdf}~outputfileKepler & \hfuzz=500pt filename & \hfuzz=500pt instrument file (MISCVALUES: Omega, i, omega [degree], a [m], e, M [degree], tau [mjd], ...)\\
\hfuzz=500pt\includegraphics[width=1em]{element-mustset.pdf}~inputfileOrbit & \hfuzz=500pt filename & \hfuzz=500pt position and velocity at each epoch define the kepler orbit\\
\hfuzz=500pt\includegraphics[width=1em]{element-unbounded.pdf}~inputfileInstrument & \hfuzz=500pt filename & \hfuzz=500pt data is appended\\
\hfuzz=500pt\includegraphics[width=1em]{element.pdf}~GM & \hfuzz=500pt double & \hfuzz=500pt Geocentric gravitational constant\\
\hline
\end{tabularx}

This program is \reference{parallelized}{general.parallelization}.
\clearpage
%==================================
\subsection{Orbit2MagneticField}\label{Orbit2MagneticField}
This program computes the magentic field vector($x, y, z$ $[Tesla = kg/A/s^2]$ in CRF))
along an \file{orbit}{instrument} and writes it as \file{instrument file}{instrument} (MISCVALUES).
The data of \configFile{inputfileInstrument}{instrument} are appended as data columns to each epoch.


\keepXColumns
\begin{tabularx}{\textwidth}{N T A}
\hline
Name & Type & Annotation\\
\hline
\hfuzz=500pt\includegraphics[width=1em]{element-mustset.pdf}~outputfileMagneticField & \hfuzz=500pt filename & \hfuzz=500pt instrument file (x,y,z in CRF [Tesla = kg/A/s\textasciicircum{}2]), ...)\\
\hfuzz=500pt\includegraphics[width=1em]{element-mustset.pdf}~inputfileOrbit & \hfuzz=500pt filename & \hfuzz=500pt \\
\hfuzz=500pt\includegraphics[width=1em]{element-unbounded.pdf}~inputfileInstrument & \hfuzz=500pt filename & \hfuzz=500pt data are appended to output file\\
\hfuzz=500pt\includegraphics[width=1em]{element-mustset.pdf}~magnetosphere & \hfuzz=500pt \hyperref[magnetosphereType]{magnetosphere} & \hfuzz=500pt \\
\hfuzz=500pt\includegraphics[width=1em]{element-mustset.pdf}~earthRotation & \hfuzz=500pt \hyperref[earthRotationType]{earthRotation} & \hfuzz=500pt \\
\hline
\end{tabularx}

This program is \reference{parallelized}{general.parallelization}.
\clearpage
%==================================
\subsection{Orbit2ThermosphericState}\label{Orbit2ThermosphericState}
This program computes the thermosperic state (density, temperature, wind (x,y,z in CRF))
based on emprical models along an \file{orbit}{instrument}
and writes it as \file{instrument file}{instrument} (MISCVALUES).
The wind is given in an celestial reference frame (CRF).
The data of \configFile{inputfileInstrument}{instrument} are appended as values to each epoch.


\keepXColumns
\begin{tabularx}{\textwidth}{N T A}
\hline
Name & Type & Annotation\\
\hline
\hfuzz=500pt\includegraphics[width=1em]{element-mustset.pdf}~outputfileThermosphericState & \hfuzz=500pt filename & \hfuzz=500pt instrument file (MISCVALUES: density, temperature, wind (x,y,z in CRF), ...)\\
\hfuzz=500pt\includegraphics[width=1em]{element-mustset.pdf}~inputfileOrbit & \hfuzz=500pt filename & \hfuzz=500pt \\
\hfuzz=500pt\includegraphics[width=1em]{element-unbounded.pdf}~inputfileInstrument & \hfuzz=500pt filename & \hfuzz=500pt data are appended to output file\\
\hfuzz=500pt\includegraphics[width=1em]{element-mustset.pdf}~thermosphere & \hfuzz=500pt \hyperref[thermosphereType]{thermosphere} & \hfuzz=500pt \\
\hfuzz=500pt\includegraphics[width=1em]{element-mustset.pdf}~earthRotation & \hfuzz=500pt \hyperref[earthRotationType]{earthRotation} & \hfuzz=500pt \\
\hline
\end{tabularx}

This program is \reference{parallelized}{general.parallelization}.
\clearpage
%==================================
\subsection{OrbitAddVelocityAndAcceleration}\label{OrbitAddVelocityAndAcceleration}
This program computes velocities and accelerations from a given \file{orbit}{instrument}
by differentiating a moving polynomial.
The values are saved in one output file which then contains orbit, velocity and acceleration.


\keepXColumns
\begin{tabularx}{\textwidth}{N T A}
\hline
Name & Type & Annotation\\
\hline
\hfuzz=500pt\includegraphics[width=1em]{element-mustset.pdf}~outputfileOrbit & \hfuzz=500pt filename & \hfuzz=500pt \\
\hfuzz=500pt\includegraphics[width=1em]{element-mustset.pdf}~inputfileOrbit & \hfuzz=500pt filename & \hfuzz=500pt \\
\hfuzz=500pt\includegraphics[width=1em]{element.pdf}~polynomialDegree & \hfuzz=500pt uint & \hfuzz=500pt Polynomial degree, must be even!\\
\hline
\end{tabularx}

This program is \reference{parallelized}{general.parallelization}.
\clearpage
%==================================
\subsection{PlanetOrbit}\label{PlanetOrbit}
Creates an \file{orbit file}{instrument} of sun, moon, or planets.
The orbit is given in the celestial reference frame (CRF)
or alternatively in the terrestrial reference frame (TRF)
if \configClass{earthRotation}{earthRotationType} is provided.


\keepXColumns
\begin{tabularx}{\textwidth}{N T A}
\hline
Name & Type & Annotation\\
\hline
\hfuzz=500pt\includegraphics[width=1em]{element-mustset.pdf}~outputfileOrbit & \hfuzz=500pt filename & \hfuzz=500pt \\
\hfuzz=500pt\includegraphics[width=1em]{element-mustset.pdf}~planet & \hfuzz=500pt \hyperref[planetType]{planet} & \hfuzz=500pt \\
\hfuzz=500pt\includegraphics[width=1em]{element-mustset-unbounded.pdf}~timeSeries & \hfuzz=500pt \hyperref[timeSeriesType]{timeSeries} & \hfuzz=500pt \\
\hfuzz=500pt\includegraphics[width=1em]{element-mustset.pdf}~ephemerides & \hfuzz=500pt \hyperref[ephemeridesType]{ephemerides} & \hfuzz=500pt \\
\hfuzz=500pt\includegraphics[width=1em]{element.pdf}~earthRotation & \hfuzz=500pt \hyperref[earthRotationType]{earthRotation} & \hfuzz=500pt transform orbits into TRF\\
\hline
\end{tabularx}

\clearpage
%==================================
\section{Programs: Plot}
\subsection{PlotDegreeAmplitudes}\label{PlotDegreeAmplitudes}
Plot degree amplitudes of potential coefficients computed by \program{Gravityfield2DegreeAmplitudes}
or \program{PotentialCoefficients2DegreeAmplitudes} using the GMT Generic Mapping Tools
(\url{https://www.generic-mapping-tools.org}).
A variety of image file formats are supported (e.g. png, jpg, eps) determined by the extension of \config{outputfile}.
This is a convenience program with meaningful default values. The same plots can be generated with the more general \program{PlotGraph}.

\fig{!hb}{0.8}{plotDegreeAmplitudes}{fig:plotDegreeAmplitudes}{Comparison of GRACE solutions (2008-06) with GOCO06s.}


\keepXColumns
\begin{tabularx}{\textwidth}{N T A}
\hline
Name & Type & Annotation\\
\hline
\hfuzz=500pt\includegraphics[width=1em]{element-mustset.pdf}~outputfile & \hfuzz=500pt filename & \hfuzz=500pt *.png, *.jpg, *.eps, ...\\
\hfuzz=500pt\includegraphics[width=1em]{element.pdf}~title & \hfuzz=500pt string & \hfuzz=500pt \\
\hfuzz=500pt\includegraphics[width=1em]{element-mustset-unbounded.pdf}~layer & \hfuzz=500pt \hyperref[plotGraphLayerType]{plotGraphLayer} & \hfuzz=500pt \\
\hfuzz=500pt\includegraphics[width=1em]{element.pdf}~minDegree & \hfuzz=500pt uint & \hfuzz=500pt \\
\hfuzz=500pt\includegraphics[width=1em]{element.pdf}~maxDegree & \hfuzz=500pt uint & \hfuzz=500pt \\
\hfuzz=500pt\includegraphics[width=1em]{element.pdf}~majorTickSpacingDegree & \hfuzz=500pt double & \hfuzz=500pt boundary annotation\\
\hfuzz=500pt\includegraphics[width=1em]{element.pdf}~minorTickSpacingDegree & \hfuzz=500pt double & \hfuzz=500pt frame tick spacing\\
\hfuzz=500pt\includegraphics[width=1em]{element.pdf}~gridLineSpacingDegree & \hfuzz=500pt double & \hfuzz=500pt gridline spacing\\
\hfuzz=500pt\includegraphics[width=1em]{element.pdf}~labelDegree & \hfuzz=500pt string & \hfuzz=500pt description of the x-axis\\
\hfuzz=500pt\includegraphics[width=1em]{element.pdf}~logarithmicDegree & \hfuzz=500pt boolean & \hfuzz=500pt use logarithmic scale for the x-axis\\
\hfuzz=500pt\includegraphics[width=1em]{element.pdf}~minY & \hfuzz=500pt double & \hfuzz=500pt \\
\hfuzz=500pt\includegraphics[width=1em]{element.pdf}~maxY & \hfuzz=500pt double & \hfuzz=500pt \\
\hfuzz=500pt\includegraphics[width=1em]{element.pdf}~majorTickSpacingY & \hfuzz=500pt double & \hfuzz=500pt boundary annotation\\
\hfuzz=500pt\includegraphics[width=1em]{element.pdf}~minorTickSpacingY & \hfuzz=500pt double & \hfuzz=500pt frame tick spacing\\
\hfuzz=500pt\includegraphics[width=1em]{element.pdf}~gridLineSpacingY & \hfuzz=500pt double & \hfuzz=500pt gridline spacing\\
\hfuzz=500pt\includegraphics[width=1em]{element.pdf}~unitY & \hfuzz=500pt string & \hfuzz=500pt appended to axis values\\
\hfuzz=500pt\includegraphics[width=1em]{element.pdf}~labelY & \hfuzz=500pt string & \hfuzz=500pt description of the y-axis\\
\hfuzz=500pt\includegraphics[width=1em]{element.pdf}~logarithmicY & \hfuzz=500pt boolean & \hfuzz=500pt use logarithmic scale for the y-axis\\
\hfuzz=500pt\includegraphics[width=1em]{element.pdf}~gridLine & \hfuzz=500pt \hyperref[plotLineType]{plotLine} & \hfuzz=500pt The style of the grid lines.\\
\hfuzz=500pt\includegraphics[width=1em]{element.pdf}~legend & \hfuzz=500pt \hyperref[plotLegendType]{plotLegend} & \hfuzz=500pt \\
\hfuzz=500pt\includegraphics[width=1em]{element-mustset.pdf}~options & \hfuzz=500pt sequence & \hfuzz=500pt further options...\\
\hfuzz=500pt\includegraphics[width=1em]{connector.pdf}\includegraphics[width=1em]{element.pdf}~width & \hfuzz=500pt double & \hfuzz=500pt in cm\\
\hfuzz=500pt\includegraphics[width=1em]{connector.pdf}\includegraphics[width=1em]{element.pdf}~height & \hfuzz=500pt double & \hfuzz=500pt in cm\\
\hfuzz=500pt\includegraphics[width=1em]{connector.pdf}\includegraphics[width=1em]{element.pdf}~titleFontSize & \hfuzz=500pt uint & \hfuzz=500pt in pt\\
\hfuzz=500pt\includegraphics[width=1em]{connector.pdf}\includegraphics[width=1em]{element.pdf}~marginTitle & \hfuzz=500pt double & \hfuzz=500pt between title and figure [cm]\\
\hfuzz=500pt\includegraphics[width=1em]{connector.pdf}\includegraphics[width=1em]{element.pdf}~drawGridOnTop & \hfuzz=500pt boolean & \hfuzz=500pt grid lines above all other lines/points\\
\hfuzz=500pt\includegraphics[width=1em]{connector.pdf}\includegraphics[width=1em]{element-unbounded.pdf}~options & \hfuzz=500pt string & \hfuzz=500pt \\
\hfuzz=500pt\includegraphics[width=1em]{connector.pdf}\includegraphics[width=1em]{element.pdf}~transparent & \hfuzz=500pt boolean & \hfuzz=500pt make background transparent\\
\hfuzz=500pt\includegraphics[width=1em]{connector.pdf}\includegraphics[width=1em]{element.pdf}~dpi & \hfuzz=500pt uint & \hfuzz=500pt use this resolution when rasterizing postscript file\\
\hfuzz=500pt\includegraphics[width=1em]{connector.pdf}\includegraphics[width=1em]{element.pdf}~removeFiles & \hfuzz=500pt boolean & \hfuzz=500pt remove .gmt and script files\\
\hline
\end{tabularx}

\clearpage
%==================================
\subsection{PlotGraph}\label{PlotGraph}
Generates a two dimensional xy plot using the GMT Generic Mapping Tools (\url{https://www.generic-mapping-tools.org}).
A variety of image file formats are supported (e.g. png, jpg, eps) determined by the extension of \config{outputfile}.

The plotting area is defined by the two axes \configClass{axisX/Y}{plotAxisType}. An alternative \configClass{axisY2}{plotAxisType}
on the right hand side can be added. The content of the graph itself is defined
by one or more \configClass{layer}{plotGraphLayerType}s.

The plot programs create a temporary directory in the path of \config{outputfile}, writes all needed data into it,
generates a batch/shell script with the GMT commands, execute it, and remove the temporary directory.
With setting \config{options:removeFiles}=false the last step is skipped and it is possible to adjust the plot manually
to specific publication needs. Individual GMT settings are adjusted with \config{options:options}="\verb|FORMAT=value|",
see \url{https://docs.generic-mapping-tools.org/latest/gmt.conf.html}.

See also: \program{PlotDegreeAmplitudes}, \program{PlotMap}, \program{PlotMatrix}, \program{PlotSphericalHarmonicsTriangle}.


\keepXColumns
\begin{tabularx}{\textwidth}{N T A}
\hline
Name & Type & Annotation\\
\hline
\hfuzz=500pt\includegraphics[width=1em]{element-mustset.pdf}~outputfile & \hfuzz=500pt filename & \hfuzz=500pt *.png, *.jpg, *.eps, ...\\
\hfuzz=500pt\includegraphics[width=1em]{element.pdf}~title & \hfuzz=500pt string & \hfuzz=500pt \\
\hfuzz=500pt\includegraphics[width=1em]{element-mustset-unbounded.pdf}~layer & \hfuzz=500pt \hyperref[plotGraphLayerType]{plotGraphLayer} & \hfuzz=500pt \\
\hfuzz=500pt\includegraphics[width=1em]{element-mustset.pdf}~axisX & \hfuzz=500pt \hyperref[plotAxisType]{plotAxis} & \hfuzz=500pt \\
\hfuzz=500pt\includegraphics[width=1em]{element-mustset.pdf}~axisY & \hfuzz=500pt \hyperref[plotAxisType]{plotAxis} & \hfuzz=500pt \\
\hfuzz=500pt\includegraphics[width=1em]{element.pdf}~axisY2 & \hfuzz=500pt \hyperref[plotAxisType]{plotAxis} & \hfuzz=500pt Second y-axis on right hand side\\
\hfuzz=500pt\includegraphics[width=1em]{element.pdf}~colorbar & \hfuzz=500pt \hyperref[plotColorbarType]{plotColorbar} & \hfuzz=500pt \\
\hfuzz=500pt\includegraphics[width=1em]{element.pdf}~legend & \hfuzz=500pt \hyperref[plotLegendType]{plotLegend} & \hfuzz=500pt \\
\hfuzz=500pt\includegraphics[width=1em]{element-mustset.pdf}~options & \hfuzz=500pt sequence & \hfuzz=500pt further options...\\
\hfuzz=500pt\includegraphics[width=1em]{connector.pdf}\includegraphics[width=1em]{element.pdf}~width & \hfuzz=500pt double & \hfuzz=500pt in cm\\
\hfuzz=500pt\includegraphics[width=1em]{connector.pdf}\includegraphics[width=1em]{element.pdf}~height & \hfuzz=500pt double & \hfuzz=500pt in cm\\
\hfuzz=500pt\includegraphics[width=1em]{connector.pdf}\includegraphics[width=1em]{element.pdf}~titleFontSize & \hfuzz=500pt uint & \hfuzz=500pt in pt\\
\hfuzz=500pt\includegraphics[width=1em]{connector.pdf}\includegraphics[width=1em]{element.pdf}~marginTitle & \hfuzz=500pt double & \hfuzz=500pt between title and figure [cm]\\
\hfuzz=500pt\includegraphics[width=1em]{connector.pdf}\includegraphics[width=1em]{element.pdf}~drawGridOnTop & \hfuzz=500pt boolean & \hfuzz=500pt grid lines above all other lines/points\\
\hfuzz=500pt\includegraphics[width=1em]{connector.pdf}\includegraphics[width=1em]{element-unbounded.pdf}~options & \hfuzz=500pt string & \hfuzz=500pt \\
\hfuzz=500pt\includegraphics[width=1em]{connector.pdf}\includegraphics[width=1em]{element.pdf}~transparent & \hfuzz=500pt boolean & \hfuzz=500pt make background transparent\\
\hfuzz=500pt\includegraphics[width=1em]{connector.pdf}\includegraphics[width=1em]{element.pdf}~dpi & \hfuzz=500pt uint & \hfuzz=500pt use this resolution when rasterizing postscript file\\
\hfuzz=500pt\includegraphics[width=1em]{connector.pdf}\includegraphics[width=1em]{element.pdf}~removeFiles & \hfuzz=500pt boolean & \hfuzz=500pt remove .gmt and script files\\
\hline
\end{tabularx}

\clearpage
%==================================
\subsection{PlotMap}\label{PlotMap}
Generates a map using the GMT Generic Mapping Tools (\url{https://www.generic-mapping-tools.org}).
A variety of image file formats are supported (e.g. png, jpg, eps) determined by the extension of \config{outputfile}.

The base map is defined by a \configClass{projection}{plotMapProjectionType} of an ellipsoid (\config{R}, \config{inverseFlattening}).
The content of the map itself is defined by one or more \configClass{layer}{plotMapLayerType}s.

The plot programs create a temporary directory in the path of \config{outputfile}, writes all needed data into it,
generates a batch/shell script with the GMT commands, execute it, and remove the temporary directory.
With setting \config{options:removeFiles}=false the last step is skipped and it is possible to adjust the plot manually
to specific publication needs. Individual GMT settings are adjusted with \config{options:options}="\verb|FORMAT=value|",
see \url{https://docs.generic-mapping-tools.org/latest/gmt.conf.html}.

See also: \program{PlotDegreeAmplitudes}, \program{PlotGraph}, \program{PlotMatrix}, \program{PlotSphericalHarmonicsTriangle}.

\fig{!hb}{0.8}{plotMap}{fig:plotMap}{A Robinson projection with griddedData (geoid), coast, polygon (amazon), and points (IGS stations) layer.}


\keepXColumns
\begin{tabularx}{\textwidth}{N T A}
\hline
Name & Type & Annotation\\
\hline
\hfuzz=500pt\includegraphics[width=1em]{element-mustset.pdf}~outputfile & \hfuzz=500pt filename & \hfuzz=500pt *.png, *.jpg, *.eps, ...\\
\hfuzz=500pt\includegraphics[width=1em]{element.pdf}~title & \hfuzz=500pt string & \hfuzz=500pt \\
\hfuzz=500pt\includegraphics[width=1em]{element.pdf}~statisticInfos & \hfuzz=500pt boolean & \hfuzz=500pt \\
\hfuzz=500pt\includegraphics[width=1em]{element-mustset-unbounded.pdf}~layer & \hfuzz=500pt \hyperref[plotMapLayerType]{plotMapLayer} & \hfuzz=500pt \\
\hfuzz=500pt\includegraphics[width=1em]{element.pdf}~R & \hfuzz=500pt double & \hfuzz=500pt reference radius for ellipsoidal coordinates on output\\
\hfuzz=500pt\includegraphics[width=1em]{element.pdf}~inverseFlattening & \hfuzz=500pt double & \hfuzz=500pt reference flattening for ellipsoidal coordinates on output, 0: spherical coordinates\\
\hfuzz=500pt\includegraphics[width=1em]{element.pdf}~minLambda & \hfuzz=500pt angle & \hfuzz=500pt min. longitude (default: compute from input data)\\
\hfuzz=500pt\includegraphics[width=1em]{element.pdf}~maxLambda & \hfuzz=500pt angle & \hfuzz=500pt max. longitude (default: compute from input data)\\
\hfuzz=500pt\includegraphics[width=1em]{element.pdf}~minPhi & \hfuzz=500pt angle & \hfuzz=500pt min. latitude (default: compute from input data)\\
\hfuzz=500pt\includegraphics[width=1em]{element.pdf}~maxPhi & \hfuzz=500pt angle & \hfuzz=500pt max. latitude (default: compute from input data)\\
\hfuzz=500pt\includegraphics[width=1em]{element.pdf}~majorTickSpacing & \hfuzz=500pt angle & \hfuzz=500pt boundary annotation\\
\hfuzz=500pt\includegraphics[width=1em]{element.pdf}~minorTickSpacing & \hfuzz=500pt angle & \hfuzz=500pt frame tick spacing\\
\hfuzz=500pt\includegraphics[width=1em]{element.pdf}~gridLineSpacing & \hfuzz=500pt angle & \hfuzz=500pt gridline spacing\\
\hfuzz=500pt\includegraphics[width=1em]{element.pdf}~colorbar & \hfuzz=500pt \hyperref[plotColorbarType]{plotColorbar} & \hfuzz=500pt \\
\hfuzz=500pt\includegraphics[width=1em]{element-mustset.pdf}~projection & \hfuzz=500pt \hyperref[plotMapProjectionType]{plotMapProjection} & \hfuzz=500pt map projection\\
\hfuzz=500pt\includegraphics[width=1em]{element-mustset.pdf}~options & \hfuzz=500pt sequence & \hfuzz=500pt further options...\\
\hfuzz=500pt\includegraphics[width=1em]{connector.pdf}\includegraphics[width=1em]{element.pdf}~width & \hfuzz=500pt double & \hfuzz=500pt in cm\\
\hfuzz=500pt\includegraphics[width=1em]{connector.pdf}\includegraphics[width=1em]{element.pdf}~height & \hfuzz=500pt double & \hfuzz=500pt in cm\\
\hfuzz=500pt\includegraphics[width=1em]{connector.pdf}\includegraphics[width=1em]{element.pdf}~titleFontSize & \hfuzz=500pt uint & \hfuzz=500pt in pt\\
\hfuzz=500pt\includegraphics[width=1em]{connector.pdf}\includegraphics[width=1em]{element.pdf}~marginTitle & \hfuzz=500pt double & \hfuzz=500pt between title and figure [cm]\\
\hfuzz=500pt\includegraphics[width=1em]{connector.pdf}\includegraphics[width=1em]{element.pdf}~drawGridOnTop & \hfuzz=500pt boolean & \hfuzz=500pt grid lines above all other lines/points\\
\hfuzz=500pt\includegraphics[width=1em]{connector.pdf}\includegraphics[width=1em]{element-unbounded.pdf}~options & \hfuzz=500pt string & \hfuzz=500pt \\
\hfuzz=500pt\includegraphics[width=1em]{connector.pdf}\includegraphics[width=1em]{element.pdf}~transparent & \hfuzz=500pt boolean & \hfuzz=500pt make background transparent\\
\hfuzz=500pt\includegraphics[width=1em]{connector.pdf}\includegraphics[width=1em]{element.pdf}~dpi & \hfuzz=500pt uint & \hfuzz=500pt use this resolution when rasterizing postscript file\\
\hfuzz=500pt\includegraphics[width=1em]{connector.pdf}\includegraphics[width=1em]{element.pdf}~removeFiles & \hfuzz=500pt boolean & \hfuzz=500pt remove .gmt and script files\\
\hline
\end{tabularx}

\clearpage
%==================================
\subsection{PlotMatrix}\label{PlotMatrix}
Plot the coefficients of a \configFile{inputfileMatrix}{matrix}
using the GMT Generic Mapping Tools (\url{https://www.generic-mapping-tools.org}).
A variety of image file formats are supported (e.g. png, jpg, eps) determined by the extension of \config{outputfile}.

The plot programs create a temporary directory in the path of \config{outputfile}, writes all needed data into it,
generates a batch/shell script with the GMT commands, execute it, and remove the temporary directory.
With setting \config{options:removeFiles}=false the last step is skipped and it is possible to adjust the plot manually
to specific publication needs. Individual GMT settings are adjusted with \config{options:options}="\verb|FORMAT=value|",
see \url{https://docs.generic-mapping-tools.org/latest/gmt.conf.html}.

\fig{!hb}{0.6}{plotMatrix}{fig:plotMatrix}{Upper left part of the DDK filter matrix.}


\keepXColumns
\begin{tabularx}{\textwidth}{N T A}
\hline
Name & Type & Annotation\\
\hline
\hfuzz=500pt\includegraphics[width=1em]{element-mustset.pdf}~outputfile & \hfuzz=500pt filename & \hfuzz=500pt *.png, *.jpg, *.eps, ...\\
\hfuzz=500pt\includegraphics[width=1em]{element.pdf}~title & \hfuzz=500pt string & \hfuzz=500pt \\
\hfuzz=500pt\includegraphics[width=1em]{element-mustset.pdf}~inputfileMatrix & \hfuzz=500pt filename & \hfuzz=500pt \\
\hfuzz=500pt\includegraphics[width=1em]{element.pdf}~minColumn & \hfuzz=500pt uint & \hfuzz=500pt minimum column index to plot\\
\hfuzz=500pt\includegraphics[width=1em]{element.pdf}~maxColumn & \hfuzz=500pt uint & \hfuzz=500pt maximum column index to plot\\
\hfuzz=500pt\includegraphics[width=1em]{element.pdf}~majorTickSpacingX & \hfuzz=500pt double & \hfuzz=500pt boundary annotation\\
\hfuzz=500pt\includegraphics[width=1em]{element.pdf}~minorTickSpacingX & \hfuzz=500pt double & \hfuzz=500pt frame tick spacing\\
\hfuzz=500pt\includegraphics[width=1em]{element.pdf}~gridLineSpacingX & \hfuzz=500pt double & \hfuzz=500pt gridline spacing\\
\hfuzz=500pt\includegraphics[width=1em]{element.pdf}~minRow & \hfuzz=500pt uint & \hfuzz=500pt minimum row index to plot\\
\hfuzz=500pt\includegraphics[width=1em]{element.pdf}~maxRow & \hfuzz=500pt uint & \hfuzz=500pt maximum row index to plot\\
\hfuzz=500pt\includegraphics[width=1em]{element.pdf}~majorTickSpacingY & \hfuzz=500pt double & \hfuzz=500pt boundary annotation\\
\hfuzz=500pt\includegraphics[width=1em]{element.pdf}~minorTickSpacingY & \hfuzz=500pt double & \hfuzz=500pt frame tick spacing\\
\hfuzz=500pt\includegraphics[width=1em]{element.pdf}~gridLineSpacingY & \hfuzz=500pt double & \hfuzz=500pt gridline spacing\\
\hfuzz=500pt\includegraphics[width=1em]{element.pdf}~gridLine & \hfuzz=500pt \hyperref[plotLineType]{plotLine} & \hfuzz=500pt The style of the grid lines.\\
\hfuzz=500pt\includegraphics[width=1em]{element-mustset.pdf}~colorbar & \hfuzz=500pt \hyperref[plotColorbarType]{plotColorbar} & \hfuzz=500pt \\
\hfuzz=500pt\includegraphics[width=1em]{element-mustset.pdf}~options & \hfuzz=500pt sequence & \hfuzz=500pt further options...\\
\hfuzz=500pt\includegraphics[width=1em]{connector.pdf}\includegraphics[width=1em]{element.pdf}~width & \hfuzz=500pt double & \hfuzz=500pt in cm\\
\hfuzz=500pt\includegraphics[width=1em]{connector.pdf}\includegraphics[width=1em]{element.pdf}~height & \hfuzz=500pt double & \hfuzz=500pt in cm\\
\hfuzz=500pt\includegraphics[width=1em]{connector.pdf}\includegraphics[width=1em]{element.pdf}~titleFontSize & \hfuzz=500pt uint & \hfuzz=500pt in pt\\
\hfuzz=500pt\includegraphics[width=1em]{connector.pdf}\includegraphics[width=1em]{element.pdf}~marginTitle & \hfuzz=500pt double & \hfuzz=500pt between title and figure [cm]\\
\hfuzz=500pt\includegraphics[width=1em]{connector.pdf}\includegraphics[width=1em]{element.pdf}~drawGridOnTop & \hfuzz=500pt boolean & \hfuzz=500pt grid lines above all other lines/points\\
\hfuzz=500pt\includegraphics[width=1em]{connector.pdf}\includegraphics[width=1em]{element-unbounded.pdf}~options & \hfuzz=500pt string & \hfuzz=500pt \\
\hfuzz=500pt\includegraphics[width=1em]{connector.pdf}\includegraphics[width=1em]{element.pdf}~transparent & \hfuzz=500pt boolean & \hfuzz=500pt make background transparent\\
\hfuzz=500pt\includegraphics[width=1em]{connector.pdf}\includegraphics[width=1em]{element.pdf}~dpi & \hfuzz=500pt uint & \hfuzz=500pt use this resolution when rasterizing postscript file\\
\hfuzz=500pt\includegraphics[width=1em]{connector.pdf}\includegraphics[width=1em]{element.pdf}~removeFiles & \hfuzz=500pt boolean & \hfuzz=500pt remove .gmt and script files\\
\hline
\end{tabularx}

\clearpage
%==================================
\subsection{PlotSphericalHarmonicsTriangle}\label{PlotSphericalHarmonicsTriangle}
Plot the potential coefficients of a spherical harmonic expansion
using the GMT Generic Mapping Tools (\url{https://www.generic-mapping-tools.org}).
A variety of image file formats are supported (e.g. png, jpg, eps) determined by the extension of \config{outputfile}.

This program plots the formal errors (sigmas).
If \configClass{gravityfield}{gravityfieldType} provides no sigmas
e.g. with \config{setSigmasToZero} in \configClass{gravityfield:potentialCoefficients}{gravityfieldType:potentialCoefficients}
the coefficients itself are plotted instead.

The plot programs create a temporary directory in the path of \config{outputfile}, writes all needed data into it,
generates a batch/shell script with the GMT commands, execute it, and remove the temporary directory.
With setting \config{options:removeFiles}=false the last step is skipped and it is possible to adjust the plot manually
to specific publication needs. Individual GMT settings are adjusted with \config{options:options}="\verb|FORMAT=value|",
see \url{https://docs.generic-mapping-tools.org/latest/gmt.conf.html}.

\fig{!hb}{0.8}{plotSphericalHarmonicsTriangle}{fig:plotSphericalHarmonicsTriangle}{Formal errors of GOCO06s.}


\keepXColumns
\begin{tabularx}{\textwidth}{N T A}
\hline
Name & Type & Annotation\\
\hline
\hfuzz=500pt\includegraphics[width=1em]{element-mustset.pdf}~outputfile & \hfuzz=500pt filename & \hfuzz=500pt *.png, *.jpg, *.eps, ...\\
\hfuzz=500pt\includegraphics[width=1em]{element.pdf}~title & \hfuzz=500pt string & \hfuzz=500pt \\
\hfuzz=500pt\includegraphics[width=1em]{element-mustset-unbounded.pdf}~gravityfield & \hfuzz=500pt \hyperref[gravityfieldType]{gravityfield} & \hfuzz=500pt use sigmas, if not given use signal (cnm,snm)\\
\hfuzz=500pt\includegraphics[width=1em]{element.pdf}~time & \hfuzz=500pt time & \hfuzz=500pt at this time the gravity field will be evaluated\\
\hfuzz=500pt\includegraphics[width=1em]{element.pdf}~minDegree & \hfuzz=500pt uint & \hfuzz=500pt \\
\hfuzz=500pt\includegraphics[width=1em]{element.pdf}~maxDegree & \hfuzz=500pt uint & \hfuzz=500pt \\
\hfuzz=500pt\includegraphics[width=1em]{element.pdf}~majorTickSpacing & \hfuzz=500pt double & \hfuzz=500pt boundary annotation\\
\hfuzz=500pt\includegraphics[width=1em]{element.pdf}~minorTickSpacing & \hfuzz=500pt double & \hfuzz=500pt frame tick spacing\\
\hfuzz=500pt\includegraphics[width=1em]{element.pdf}~gridLineSpacing & \hfuzz=500pt double & \hfuzz=500pt gridline spacing\\
\hfuzz=500pt\includegraphics[width=1em]{element.pdf}~gridLine & \hfuzz=500pt \hyperref[plotLineType]{plotLine} & \hfuzz=500pt The style of the grid lines.\\
\hfuzz=500pt\includegraphics[width=1em]{element-mustset.pdf}~colorbar & \hfuzz=500pt \hyperref[plotColorbarType]{plotColorbar} & \hfuzz=500pt \\
\hfuzz=500pt\includegraphics[width=1em]{element-mustset.pdf}~options & \hfuzz=500pt sequence & \hfuzz=500pt further options...\\
\hfuzz=500pt\includegraphics[width=1em]{connector.pdf}\includegraphics[width=1em]{element.pdf}~width & \hfuzz=500pt double & \hfuzz=500pt in cm\\
\hfuzz=500pt\includegraphics[width=1em]{connector.pdf}\includegraphics[width=1em]{element.pdf}~height & \hfuzz=500pt double & \hfuzz=500pt in cm\\
\hfuzz=500pt\includegraphics[width=1em]{connector.pdf}\includegraphics[width=1em]{element.pdf}~titleFontSize & \hfuzz=500pt uint & \hfuzz=500pt in pt\\
\hfuzz=500pt\includegraphics[width=1em]{connector.pdf}\includegraphics[width=1em]{element.pdf}~marginTitle & \hfuzz=500pt double & \hfuzz=500pt between title and figure [cm]\\
\hfuzz=500pt\includegraphics[width=1em]{connector.pdf}\includegraphics[width=1em]{element.pdf}~drawGridOnTop & \hfuzz=500pt boolean & \hfuzz=500pt grid lines above all other lines/points\\
\hfuzz=500pt\includegraphics[width=1em]{connector.pdf}\includegraphics[width=1em]{element-unbounded.pdf}~options & \hfuzz=500pt string & \hfuzz=500pt \\
\hfuzz=500pt\includegraphics[width=1em]{connector.pdf}\includegraphics[width=1em]{element.pdf}~transparent & \hfuzz=500pt boolean & \hfuzz=500pt make background transparent\\
\hfuzz=500pt\includegraphics[width=1em]{connector.pdf}\includegraphics[width=1em]{element.pdf}~dpi & \hfuzz=500pt uint & \hfuzz=500pt use this resolution when rasterizing postscript file\\
\hfuzz=500pt\includegraphics[width=1em]{connector.pdf}\includegraphics[width=1em]{element.pdf}~removeFiles & \hfuzz=500pt boolean & \hfuzz=500pt remove .gmt and script files\\
\hline
\end{tabularx}

\clearpage
%==================================
\section{Programs: Preprocessing}
\subsection{PreprocessingDualSst}\label{PreprocessingDualSst}
This programs processes satellite-to-satellite-tracking (SST) and orbit observations in a GRACE like configuration.
Four different observation groups are considered separately: two types of SST and POD1/POD2 for the two satellites.
This program works similar to \program{PreprocessingSst}, see there for details. Here only the settings explained,
which are different.

Both SST observation types are reduced by the same background models and the same impact
of accelerometer measurements. The covariance matrix of the reduced observations should not consider
the the instrument noise only (\configClass{covarianceSst1/2}{covarianceSstType}) but must
take the cross correlations \configClass{covarianceAcc}{covarianceSstType} into account.
The covariance matrix of the reduced observations is given by
\begin{equation}
  \M\Sigma(\begin{bmatrix} \Delta l_{SST1} \\ \Delta l_{SST2} \end{bmatrix})
  = \begin{bmatrix} \M\Sigma_{SST1} + \M\Sigma_{ACC} & \M\Sigma_{ACC} \\
                   \M\Sigma_{ACC} & \M\Sigma_{SST2} + \M\Sigma_{ACC}
    \end{bmatrix}.
\end{equation}


\keepXColumns
\begin{tabularx}{\textwidth}{N T A}
\hline
Name & Type & Annotation\\
\hline
\hfuzz=500pt\includegraphics[width=1em]{element.pdf}~outputfileSolution & \hfuzz=500pt filename & \hfuzz=500pt estimated parameter vector (static part only)\\
\hfuzz=500pt\includegraphics[width=1em]{element.pdf}~outputfileSigmax & \hfuzz=500pt filename & \hfuzz=500pt standard deviations of the parameters (sqrt of the diagonal of the inverse normal equation)\\
\hfuzz=500pt\includegraphics[width=1em]{element.pdf}~outputfileParameterName & \hfuzz=500pt filename & \hfuzz=500pt estimated signal parameters (index is appended)\\
\hfuzz=500pt\includegraphics[width=1em]{element.pdf}~estimateArcSigmas & \hfuzz=500pt sequence & \hfuzz=500pt \\
\hfuzz=500pt\includegraphics[width=1em]{connector.pdf}\includegraphics[width=1em]{element.pdf}~outputfileSigmasPerArcSst1 & \hfuzz=500pt filename & \hfuzz=500pt accuracies of each arc (SST1)\\
\hfuzz=500pt\includegraphics[width=1em]{connector.pdf}\includegraphics[width=1em]{element.pdf}~outputfileSigmasPerArcSst2 & \hfuzz=500pt filename & \hfuzz=500pt accuracies of each arc (SST2)\\
\hfuzz=500pt\includegraphics[width=1em]{connector.pdf}\includegraphics[width=1em]{element.pdf}~outputfileSigmasPerArcAcc & \hfuzz=500pt filename & \hfuzz=500pt accuracies of each arc (ACC)\\
\hfuzz=500pt\includegraphics[width=1em]{connector.pdf}\includegraphics[width=1em]{element.pdf}~outputfileSigmasPerArcPod1 & \hfuzz=500pt filename & \hfuzz=500pt accuracies of each arc (POD1)\\
\hfuzz=500pt\includegraphics[width=1em]{connector.pdf}\includegraphics[width=1em]{element.pdf}~outputfileSigmasPerArcPod2 & \hfuzz=500pt filename & \hfuzz=500pt accuracies of each arc (POD2)\\
\hfuzz=500pt\includegraphics[width=1em]{element.pdf}~estimateEpochSigmas & \hfuzz=500pt sequence & \hfuzz=500pt \\
\hfuzz=500pt\includegraphics[width=1em]{connector.pdf}\includegraphics[width=1em]{element.pdf}~outputfileSigmasPerEpochSst1 & \hfuzz=500pt filename & \hfuzz=500pt accuracies of each epoch (SST1)\\
\hfuzz=500pt\includegraphics[width=1em]{connector.pdf}\includegraphics[width=1em]{element.pdf}~outputfileSigmasPerEpochSst2 & \hfuzz=500pt filename & \hfuzz=500pt accuracies of each epoch (SST2)\\
\hfuzz=500pt\includegraphics[width=1em]{connector.pdf}\includegraphics[width=1em]{element.pdf}~outputfileSigmasPerEpochAcc & \hfuzz=500pt filename & \hfuzz=500pt accuracies of each epoch (ACC)\\
\hfuzz=500pt\includegraphics[width=1em]{connector.pdf}\includegraphics[width=1em]{element.pdf}~outputfileSigmasPerEpochPod1 & \hfuzz=500pt filename & \hfuzz=500pt accuracies of each epoch (POD1)\\
\hfuzz=500pt\includegraphics[width=1em]{connector.pdf}\includegraphics[width=1em]{element.pdf}~outputfileSigmasPerEpochPod2 & \hfuzz=500pt filename & \hfuzz=500pt accuracies of each epoch (POD2)\\
\hfuzz=500pt\includegraphics[width=1em]{element.pdf}~estimateCovarianceFunctions & \hfuzz=500pt sequence & \hfuzz=500pt \\
\hfuzz=500pt\includegraphics[width=1em]{connector.pdf}\includegraphics[width=1em]{element.pdf}~outputfileCovarianceFunctionSst1 & \hfuzz=500pt filename & \hfuzz=500pt covariance function\\
\hfuzz=500pt\includegraphics[width=1em]{connector.pdf}\includegraphics[width=1em]{element.pdf}~outputfileCovarianceFunctionSst2 & \hfuzz=500pt filename & \hfuzz=500pt covariance function\\
\hfuzz=500pt\includegraphics[width=1em]{connector.pdf}\includegraphics[width=1em]{element.pdf}~outputfileCovarianceFunctionAcc & \hfuzz=500pt filename & \hfuzz=500pt covariance function\\
\hfuzz=500pt\includegraphics[width=1em]{connector.pdf}\includegraphics[width=1em]{element.pdf}~outputfileCovarianceFunctionPod1 & \hfuzz=500pt filename & \hfuzz=500pt covariance functions for along, cross, radial direction\\
\hfuzz=500pt\includegraphics[width=1em]{connector.pdf}\includegraphics[width=1em]{element.pdf}~outputfileCovarianceFunctionPod2 & \hfuzz=500pt filename & \hfuzz=500pt covariance functions for along, cross, radial direction\\
\hfuzz=500pt\includegraphics[width=1em]{element.pdf}~computeResiduals & \hfuzz=500pt sequence & \hfuzz=500pt \\
\hfuzz=500pt\includegraphics[width=1em]{connector.pdf}\includegraphics[width=1em]{element.pdf}~outputfileSst1Residuals & \hfuzz=500pt filename & \hfuzz=500pt \\
\hfuzz=500pt\includegraphics[width=1em]{connector.pdf}\includegraphics[width=1em]{element.pdf}~outputfileSst2Residuals & \hfuzz=500pt filename & \hfuzz=500pt \\
\hfuzz=500pt\includegraphics[width=1em]{connector.pdf}\includegraphics[width=1em]{element.pdf}~outputfileAccResiduals & \hfuzz=500pt filename & \hfuzz=500pt \\
\hfuzz=500pt\includegraphics[width=1em]{connector.pdf}\includegraphics[width=1em]{element.pdf}~outputfilePod1Residuals & \hfuzz=500pt filename & \hfuzz=500pt \\
\hfuzz=500pt\includegraphics[width=1em]{connector.pdf}\includegraphics[width=1em]{element.pdf}~outputfilePod2Residuals & \hfuzz=500pt filename & \hfuzz=500pt \\
\hfuzz=500pt\includegraphics[width=1em]{element-mustset.pdf}~observation & \hfuzz=500pt choice & \hfuzz=500pt obervation equations (Sst)\\
\hfuzz=500pt\includegraphics[width=1em]{connector.pdf}\includegraphics[width=1em]{element-mustset.pdf}~dualSstVariational & \hfuzz=500pt sequence & \hfuzz=500pt two SST observations\\
\hfuzz=500pt\quad\includegraphics[width=1em]{connector.pdf}\includegraphics[width=1em]{element-mustset.pdf}~rightHandSide & \hfuzz=500pt sequence & \hfuzz=500pt input for observation vectors\\
\hfuzz=500pt\quad\quad\includegraphics[width=1em]{connector.pdf}\includegraphics[width=1em]{element-mustset-unbounded.pdf}~inputfileSatelliteTracking1 & \hfuzz=500pt filename & \hfuzz=500pt ranging observations and corrections\\
\hfuzz=500pt\quad\quad\includegraphics[width=1em]{connector.pdf}\includegraphics[width=1em]{element-mustset-unbounded.pdf}~inputfileSatelliteTracking2 & \hfuzz=500pt filename & \hfuzz=500pt ranging observations and corrections\\
\hfuzz=500pt\quad\quad\includegraphics[width=1em]{connector.pdf}\includegraphics[width=1em]{element.pdf}~inputfileOrbit1 & \hfuzz=500pt filename & \hfuzz=500pt kinematic positions of satellite A as observations\\
\hfuzz=500pt\quad\quad\includegraphics[width=1em]{connector.pdf}\includegraphics[width=1em]{element.pdf}~inputfileOrbit2 & \hfuzz=500pt filename & \hfuzz=500pt kinematic positions of satellite B as observations\\
\hfuzz=500pt\quad\includegraphics[width=1em]{connector.pdf}\includegraphics[width=1em]{element-mustset.pdf}~sstType & \hfuzz=500pt choice & \hfuzz=500pt \\
\hfuzz=500pt\quad\quad\includegraphics[width=1em]{connector.pdf}\includegraphics[width=1em]{element-mustset.pdf}~range & \hfuzz=500pt  & \hfuzz=500pt \\
\hfuzz=500pt\quad\quad\includegraphics[width=1em]{connector.pdf}\includegraphics[width=1em]{element-mustset.pdf}~rangeRate & \hfuzz=500pt  & \hfuzz=500pt \\
\hfuzz=500pt\quad\quad\includegraphics[width=1em]{connector.pdf}\includegraphics[width=1em]{element-mustset.pdf}~none & \hfuzz=500pt  & \hfuzz=500pt \\
\hfuzz=500pt\quad\includegraphics[width=1em]{connector.pdf}\includegraphics[width=1em]{element-mustset.pdf}~inputfileVariational1 & \hfuzz=500pt filename & \hfuzz=500pt approximate position and integrated state matrix\\
\hfuzz=500pt\quad\includegraphics[width=1em]{connector.pdf}\includegraphics[width=1em]{element-mustset.pdf}~inputfileVariational2 & \hfuzz=500pt filename & \hfuzz=500pt approximate position and integrated state matrix\\
\hfuzz=500pt\quad\includegraphics[width=1em]{connector.pdf}\includegraphics[width=1em]{element.pdf}~ephemerides & \hfuzz=500pt \hyperref[ephemeridesType]{ephemerides} & \hfuzz=500pt \\
\hfuzz=500pt\quad\includegraphics[width=1em]{connector.pdf}\includegraphics[width=1em]{element-unbounded.pdf}~parametrizationGravity & \hfuzz=500pt \hyperref[parametrizationGravityType]{parametrizationGravity} & \hfuzz=500pt gravity field parametrization\\
\hfuzz=500pt\quad\includegraphics[width=1em]{connector.pdf}\includegraphics[width=1em]{element-unbounded.pdf}~parametrizationAcceleration1 & \hfuzz=500pt \hyperref[parametrizationAccelerationType]{parametrizationAcceleration} & \hfuzz=500pt orbit1 force parameters\\
\hfuzz=500pt\quad\includegraphics[width=1em]{connector.pdf}\includegraphics[width=1em]{element-unbounded.pdf}~parametrizationAcceleration2 & \hfuzz=500pt \hyperref[parametrizationAccelerationType]{parametrizationAcceleration} & \hfuzz=500pt orbit2 force parameters\\
\hfuzz=500pt\quad\includegraphics[width=1em]{connector.pdf}\includegraphics[width=1em]{element-unbounded.pdf}~parametrizationSst1 & \hfuzz=500pt \hyperref[parametrizationSatelliteTrackingType]{parametrizationSatelliteTracking} & \hfuzz=500pt satellite tracking parameter for first ranging observations\\
\hfuzz=500pt\quad\includegraphics[width=1em]{connector.pdf}\includegraphics[width=1em]{element-unbounded.pdf}~parametrizationSst2 & \hfuzz=500pt \hyperref[parametrizationSatelliteTrackingType]{parametrizationSatelliteTracking} & \hfuzz=500pt satellite tracking parameter for second ranging observations\\
\hfuzz=500pt\quad\includegraphics[width=1em]{connector.pdf}\includegraphics[width=1em]{element.pdf}~integrationDegree & \hfuzz=500pt uint & \hfuzz=500pt integration of forces by polynomial approximation of degree n\\
\hfuzz=500pt\quad\includegraphics[width=1em]{connector.pdf}\includegraphics[width=1em]{element.pdf}~interpolationDegree & \hfuzz=500pt uint & \hfuzz=500pt orbit interpolation by polynomial approximation of degree n\\
\hfuzz=500pt\includegraphics[width=1em]{element-mustset.pdf}~covarianceSst1 & \hfuzz=500pt sequence & \hfuzz=500pt \\
\hfuzz=500pt\includegraphics[width=1em]{connector.pdf}\includegraphics[width=1em]{element.pdf}~sigma & \hfuzz=500pt double & \hfuzz=500pt apriori factor of covariance function\\
\hfuzz=500pt\includegraphics[width=1em]{connector.pdf}\includegraphics[width=1em]{element.pdf}~inputfileSigmasPerArc & \hfuzz=500pt filename & \hfuzz=500pt apriori different accuaries for each arc (multiplicated with sigma)\\
\hfuzz=500pt\includegraphics[width=1em]{connector.pdf}\includegraphics[width=1em]{element.pdf}~inputfileSigmasPerEpoch & \hfuzz=500pt filename & \hfuzz=500pt apriori different accuaries for each epoch\\
\hfuzz=500pt\includegraphics[width=1em]{connector.pdf}\includegraphics[width=1em]{element.pdf}~inputfileCovarianceFunction & \hfuzz=500pt filename & \hfuzz=500pt approximate covariances in time\\
\hfuzz=500pt\includegraphics[width=1em]{connector.pdf}\includegraphics[width=1em]{element-unbounded.pdf}~inputfileCovarianceMatrixArc & \hfuzz=500pt filename & \hfuzz=500pt Must be given per sst arc with correct dimensions.\\
\hfuzz=500pt\includegraphics[width=1em]{connector.pdf}\includegraphics[width=1em]{element.pdf}~inputfileSigmasCovarianceMatrixArc & \hfuzz=500pt filename & \hfuzz=500pt Vector with one sigma for each \$<\$inputfileCovarianceMatrixArc\$>\$\\
\hfuzz=500pt\includegraphics[width=1em]{connector.pdf}\includegraphics[width=1em]{element.pdf}~sampling & \hfuzz=500pt double & \hfuzz=500pt [seconds] sampling of the covariance function\\
\hfuzz=500pt\includegraphics[width=1em]{element-mustset.pdf}~covarianceSst2 & \hfuzz=500pt sequence & \hfuzz=500pt \\
\hfuzz=500pt\includegraphics[width=1em]{connector.pdf}\includegraphics[width=1em]{element.pdf}~sigma & \hfuzz=500pt double & \hfuzz=500pt apriori factor of covariance function\\
\hfuzz=500pt\includegraphics[width=1em]{connector.pdf}\includegraphics[width=1em]{element.pdf}~inputfileSigmasPerArc & \hfuzz=500pt filename & \hfuzz=500pt apriori different accuaries for each arc (multiplicated with sigma)\\
\hfuzz=500pt\includegraphics[width=1em]{connector.pdf}\includegraphics[width=1em]{element.pdf}~inputfileSigmasPerEpoch & \hfuzz=500pt filename & \hfuzz=500pt apriori different accuaries for each epoch\\
\hfuzz=500pt\includegraphics[width=1em]{connector.pdf}\includegraphics[width=1em]{element.pdf}~inputfileCovarianceFunction & \hfuzz=500pt filename & \hfuzz=500pt approximate covariances in time\\
\hfuzz=500pt\includegraphics[width=1em]{connector.pdf}\includegraphics[width=1em]{element-unbounded.pdf}~inputfileCovarianceMatrixArc & \hfuzz=500pt filename & \hfuzz=500pt Must be given per sst arc with correct dimensions.\\
\hfuzz=500pt\includegraphics[width=1em]{connector.pdf}\includegraphics[width=1em]{element.pdf}~inputfileSigmasCovarianceMatrixArc & \hfuzz=500pt filename & \hfuzz=500pt Vector with one sigma for each \$<\$inputfileCovarianceMatrixArc\$>\$\\
\hfuzz=500pt\includegraphics[width=1em]{connector.pdf}\includegraphics[width=1em]{element.pdf}~sampling & \hfuzz=500pt double & \hfuzz=500pt [seconds] sampling of the covariance function\\
\hfuzz=500pt\includegraphics[width=1em]{element-mustset.pdf}~covarianceAcc & \hfuzz=500pt sequence & \hfuzz=500pt \\
\hfuzz=500pt\includegraphics[width=1em]{connector.pdf}\includegraphics[width=1em]{element.pdf}~sigma & \hfuzz=500pt double & \hfuzz=500pt apriori factor of covariance function\\
\hfuzz=500pt\includegraphics[width=1em]{connector.pdf}\includegraphics[width=1em]{element.pdf}~inputfileSigmasPerArc & \hfuzz=500pt filename & \hfuzz=500pt apriori different accuaries for each arc (multiplicated with sigma)\\
\hfuzz=500pt\includegraphics[width=1em]{connector.pdf}\includegraphics[width=1em]{element.pdf}~inputfileSigmasPerEpoch & \hfuzz=500pt filename & \hfuzz=500pt apriori different accuaries for each epoch\\
\hfuzz=500pt\includegraphics[width=1em]{connector.pdf}\includegraphics[width=1em]{element.pdf}~inputfileCovarianceFunction & \hfuzz=500pt filename & \hfuzz=500pt approximate covariances in time\\
\hfuzz=500pt\includegraphics[width=1em]{connector.pdf}\includegraphics[width=1em]{element-unbounded.pdf}~inputfileCovarianceMatrixArc & \hfuzz=500pt filename & \hfuzz=500pt Must be given per sst arc with correct dimensions.\\
\hfuzz=500pt\includegraphics[width=1em]{connector.pdf}\includegraphics[width=1em]{element.pdf}~inputfileSigmasCovarianceMatrixArc & \hfuzz=500pt filename & \hfuzz=500pt Vector with one sigma for each \$<\$inputfileCovarianceMatrixArc\$>\$\\
\hfuzz=500pt\includegraphics[width=1em]{connector.pdf}\includegraphics[width=1em]{element.pdf}~sampling & \hfuzz=500pt double & \hfuzz=500pt [seconds] sampling of the covariance function\\
\hfuzz=500pt\includegraphics[width=1em]{element-mustset.pdf}~covariancePod1 & \hfuzz=500pt sequence & \hfuzz=500pt \\
\hfuzz=500pt\includegraphics[width=1em]{connector.pdf}\includegraphics[width=1em]{element.pdf}~sigma & \hfuzz=500pt double & \hfuzz=500pt apriori factor of covariance function\\
\hfuzz=500pt\includegraphics[width=1em]{connector.pdf}\includegraphics[width=1em]{element.pdf}~inputfileSigmasPerArc & \hfuzz=500pt filename & \hfuzz=500pt apriori different accuaries for each arc (multiplicated with sigma)\\
\hfuzz=500pt\includegraphics[width=1em]{connector.pdf}\includegraphics[width=1em]{element.pdf}~inputfileSigmasPerEpoch & \hfuzz=500pt filename & \hfuzz=500pt apriori different accuaries for each epoch\\
\hfuzz=500pt\includegraphics[width=1em]{connector.pdf}\includegraphics[width=1em]{element.pdf}~inputfileCovarianceFunction & \hfuzz=500pt filename & \hfuzz=500pt approximate covariances in time\\
\hfuzz=500pt\includegraphics[width=1em]{connector.pdf}\includegraphics[width=1em]{element.pdf}~inputfileCovariancePodEpoch & \hfuzz=500pt filename & \hfuzz=500pt 3x3 epoch covariances\\
\hfuzz=500pt\includegraphics[width=1em]{connector.pdf}\includegraphics[width=1em]{element.pdf}~sampling & \hfuzz=500pt double & \hfuzz=500pt [seconds] sampling of the covariance function\\
\hfuzz=500pt\includegraphics[width=1em]{element-mustset.pdf}~covariancePod2 & \hfuzz=500pt sequence & \hfuzz=500pt \\
\hfuzz=500pt\includegraphics[width=1em]{connector.pdf}\includegraphics[width=1em]{element.pdf}~sigma & \hfuzz=500pt double & \hfuzz=500pt apriori factor of covariance function\\
\hfuzz=500pt\includegraphics[width=1em]{connector.pdf}\includegraphics[width=1em]{element.pdf}~inputfileSigmasPerArc & \hfuzz=500pt filename & \hfuzz=500pt apriori different accuaries for each arc (multiplicated with sigma)\\
\hfuzz=500pt\includegraphics[width=1em]{connector.pdf}\includegraphics[width=1em]{element.pdf}~inputfileSigmasPerEpoch & \hfuzz=500pt filename & \hfuzz=500pt apriori different accuaries for each epoch\\
\hfuzz=500pt\includegraphics[width=1em]{connector.pdf}\includegraphics[width=1em]{element.pdf}~inputfileCovarianceFunction & \hfuzz=500pt filename & \hfuzz=500pt approximate covariances in time\\
\hfuzz=500pt\includegraphics[width=1em]{connector.pdf}\includegraphics[width=1em]{element.pdf}~inputfileCovariancePodEpoch & \hfuzz=500pt filename & \hfuzz=500pt 3x3 epoch covariances\\
\hfuzz=500pt\includegraphics[width=1em]{connector.pdf}\includegraphics[width=1em]{element.pdf}~sampling & \hfuzz=500pt double & \hfuzz=500pt [seconds] sampling of the covariance function\\
\hfuzz=500pt\includegraphics[width=1em]{element.pdf}~estimateShortTimeVariations & \hfuzz=500pt sequence & \hfuzz=500pt co-estimate short time gravity field variations\\
\hfuzz=500pt\includegraphics[width=1em]{connector.pdf}\includegraphics[width=1em]{element.pdf}~estimateSigma & \hfuzz=500pt boolean & \hfuzz=500pt estimate standard deviation via VCE\\
\hfuzz=500pt\includegraphics[width=1em]{connector.pdf}\includegraphics[width=1em]{element-mustset.pdf}~autoregressiveModelSequence & \hfuzz=500pt \hyperref[autoregressiveModelSequenceType]{autoregressiveModelSequence} & \hfuzz=500pt AR model sequence for constraining short time gravity variations\\
\hfuzz=500pt\includegraphics[width=1em]{connector.pdf}\includegraphics[width=1em]{element-mustset-unbounded.pdf}~parameterSelection & \hfuzz=500pt \hyperref[parameterSelectorType]{parameterSelector} & \hfuzz=500pt parameters describing the short time gravity field\\
\hfuzz=500pt\includegraphics[width=1em]{element.pdf}~downweightPod & \hfuzz=500pt double & \hfuzz=500pt downweight factor for POD\\
\hfuzz=500pt\includegraphics[width=1em]{element.pdf}~inputfileArcList & \hfuzz=500pt filename & \hfuzz=500pt list to correspond points of time to arc numbers\\
\hfuzz=500pt\includegraphics[width=1em]{element.pdf}~iterationCount & \hfuzz=500pt uint & \hfuzz=500pt (maximum) number of iterations for the estimation of calibration parameter and error PSD\\
\hfuzz=500pt\includegraphics[width=1em]{element.pdf}~variableNameIterations & \hfuzz=500pt string & \hfuzz=500pt All output fileNames in preprocessing iteration are expanded with this variable prior to writing to disk\\
\hfuzz=500pt\includegraphics[width=1em]{element.pdf}~defaultBlockSize & \hfuzz=500pt uint & \hfuzz=500pt block size of static normal equation blocks\\
\hline
\end{tabularx}

This program is \reference{parallelized}{general.parallelization}.
\clearpage
%==================================
\subsection{PreprocessingGradiometer}\label{PreprocessingGradiometer}
This program estimates empirical covariance functions of the gradiometer instrument noise and determine arc wise variances to
downweight arcs with outliers. This program works similar to \program{PreprocessingPod}, see there for details.
Here only the settings explained, which are different.

...


\keepXColumns
\begin{tabularx}{\textwidth}{N T A}
\hline
Name & Type & Annotation\\
\hline
\hfuzz=500pt\includegraphics[width=1em]{element.pdf}~outputfileCovarianceFunction & \hfuzz=500pt filename & \hfuzz=500pt \\
\hfuzz=500pt\includegraphics[width=1em]{element.pdf}~outputfileSigmasPerArc & \hfuzz=500pt filename & \hfuzz=500pt accuracies of each arc\\
\hfuzz=500pt\includegraphics[width=1em]{element.pdf}~outputfileSggResiduals & \hfuzz=500pt filename & \hfuzz=500pt \\
\hfuzz=500pt\includegraphics[width=1em]{element-mustset.pdf}~rightHandSide & \hfuzz=500pt \hyperref[sggRightSideType]{sggRightSide} & \hfuzz=500pt input for the observation vector\\
\hfuzz=500pt\includegraphics[width=1em]{element-mustset.pdf}~inputfileOrbit & \hfuzz=500pt filename & \hfuzz=500pt \\
\hfuzz=500pt\includegraphics[width=1em]{element-mustset.pdf}~inputfileStarCamera & \hfuzz=500pt filename & \hfuzz=500pt \\
\hfuzz=500pt\includegraphics[width=1em]{element-mustset.pdf}~earthRotation & \hfuzz=500pt \hyperref[earthRotationType]{earthRotation} & \hfuzz=500pt \\
\hfuzz=500pt\includegraphics[width=1em]{element.pdf}~ephemerides & \hfuzz=500pt \hyperref[ephemeridesType]{ephemerides} & \hfuzz=500pt \\
\hfuzz=500pt\includegraphics[width=1em]{element-unbounded.pdf}~parametrizationBias & \hfuzz=500pt \hyperref[parametrizationTemporalType]{parametrizationTemporal} & \hfuzz=500pt per arc\\
\hfuzz=500pt\includegraphics[width=1em]{element-mustset.pdf}~covarianceSgg & \hfuzz=500pt sequence & \hfuzz=500pt \\
\hfuzz=500pt\includegraphics[width=1em]{connector.pdf}\includegraphics[width=1em]{element.pdf}~inputfileCovarianceFunction & \hfuzz=500pt filename & \hfuzz=500pt approximate covariances in time\\
\hfuzz=500pt\includegraphics[width=1em]{connector.pdf}\includegraphics[width=1em]{element-mustset.pdf}~covarianceLength & \hfuzz=500pt uint & \hfuzz=500pt counts observation epochs\\
\hfuzz=500pt\includegraphics[width=1em]{connector.pdf}\includegraphics[width=1em]{element-mustset.pdf}~sampling & \hfuzz=500pt double & \hfuzz=500pt [seconds] sampling of the covariance function\\
\hfuzz=500pt\includegraphics[width=1em]{element.pdf}~iterationCount & \hfuzz=500pt uint & \hfuzz=500pt for the estimation of calibration parameter and error PSD\\
\hline
\end{tabularx}

This program is \reference{parallelized}{general.parallelization}.
\clearpage
%==================================
\subsection{PreprocessingPod}\label{PreprocessingPod}
This program estimates empirical covariance functions of the instrument noise and determine arc wise variances to
downweight arc with outliers.

A complete least squares adjustment for gravity field determination is performed by computing the \config{observation}
equations, see \configClass{observation:podIntegral}{observationType:podIntegral} or
\configClass{observation:podVariational}{observationType:podVariational} for details. The normal equations
are accumulated and solved to \configFile{outputfileSolution}{matrix} together with the estimated accuracies
\configFile{outputfileSigmax}{matrix}. The estimated residuals~$\hat{\M e}=\M l-\M A\hat{\M x}$ can be computed with
\config{computeResiduals}.

For each component (along, cross, radial) of the kinematic orbit positions a noise covariance function is estimated
\begin{equation}
  \text{cov}(\Delta t_i) = \sum_{n=0}^{N-1} a_n^2 \cos\left(\frac{\pi}{T} n\Delta t_i\right).
\end{equation}
The covariance matrix is composed of the sum of matrices $F_n$ and unknown variance factors
\begin{equation}
  \M\Sigma = a_1^2\M F_1 + a_2^2 \M F_2 + \cdots + a_N^2\M F_N,
\end{equation}
with the cosine transformation matrices
\begin{equation}
  \M F_n = \left(\cos\left(\frac{\pi}{T} n(t_i-t_k)\right)\right)_{ik}.
\end{equation}

An additional variance factor can be computed (\config{estimateArcSigmas}) for each arc~$k$  according to
\begin{equation}
  \hat{\sigma}_k^2 = \frac{\hat{\M e}_k^T\M\Sigma^{-1}\hat{\M e}_k}{r_k},
\end{equation}
where $r_k$ is the redundancy. This variance factor should be around one for normal behaving arcs
as the noise characteristics is already considered by the covariance matrix but bad arcs get a much larger variance.
By appling this factor bad arcs or arcs with large outliers are downweighted.


\keepXColumns
\begin{tabularx}{\textwidth}{N T A}
\hline
Name & Type & Annotation\\
\hline
\hfuzz=500pt\includegraphics[width=1em]{element.pdf}~outputfileSolution & \hfuzz=500pt filename & \hfuzz=500pt estimated parameter vector (static part only)\\
\hfuzz=500pt\includegraphics[width=1em]{element.pdf}~outputfileSigmax & \hfuzz=500pt filename & \hfuzz=500pt standard deviations of the parameters (sqrt of the diagonal of the inverse normal equation)\\
\hfuzz=500pt\includegraphics[width=1em]{element.pdf}~outputfileParameterName & \hfuzz=500pt filename & \hfuzz=500pt names of estimated parameters (static part only)\\
\hfuzz=500pt\includegraphics[width=1em]{element.pdf}~estimateArcSigmas & \hfuzz=500pt sequence & \hfuzz=500pt \\
\hfuzz=500pt\includegraphics[width=1em]{connector.pdf}\includegraphics[width=1em]{element.pdf}~outputfileSigmasPerArcPod & \hfuzz=500pt filename & \hfuzz=500pt accuracies of each arc (POD2)\\
\hfuzz=500pt\includegraphics[width=1em]{element.pdf}~estimateCovarianceFunctions & \hfuzz=500pt sequence & \hfuzz=500pt \\
\hfuzz=500pt\includegraphics[width=1em]{connector.pdf}\includegraphics[width=1em]{element.pdf}~outputfileCovarianceFunctionPod & \hfuzz=500pt filename & \hfuzz=500pt covariance functions for along, cross, radial direction\\
\hfuzz=500pt\includegraphics[width=1em]{element.pdf}~computeResiduals & \hfuzz=500pt sequence & \hfuzz=500pt \\
\hfuzz=500pt\includegraphics[width=1em]{connector.pdf}\includegraphics[width=1em]{element.pdf}~outputfilePodResiduals & \hfuzz=500pt filename & \hfuzz=500pt \\
\hfuzz=500pt\includegraphics[width=1em]{element-mustset.pdf}~observation & \hfuzz=500pt choice & \hfuzz=500pt obervation equations (POD)\\
\hfuzz=500pt\includegraphics[width=1em]{connector.pdf}\includegraphics[width=1em]{element-mustset.pdf}~podIntegral & \hfuzz=500pt sequence & \hfuzz=500pt precise orbit data (integral approach)\\
\hfuzz=500pt\quad\includegraphics[width=1em]{connector.pdf}\includegraphics[width=1em]{element.pdf}~inputfileSatelliteModel & \hfuzz=500pt filename & \hfuzz=500pt satellite macro model\\
\hfuzz=500pt\quad\includegraphics[width=1em]{connector.pdf}\includegraphics[width=1em]{element-mustset-unbounded.pdf}~rightHandSide & \hfuzz=500pt \hyperref[podRightSideType]{podRightSide} & \hfuzz=500pt input for the reduced observation vector\\
\hfuzz=500pt\quad\includegraphics[width=1em]{connector.pdf}\includegraphics[width=1em]{element-mustset.pdf}~inputfileOrbit & \hfuzz=500pt filename & \hfuzz=500pt used to evaluate the observation equations, not used as observations\\
\hfuzz=500pt\quad\includegraphics[width=1em]{connector.pdf}\includegraphics[width=1em]{element-mustset.pdf}~inputfileStarCamera & \hfuzz=500pt filename & \hfuzz=500pt \\
\hfuzz=500pt\quad\includegraphics[width=1em]{connector.pdf}\includegraphics[width=1em]{element-mustset.pdf}~earthRotation & \hfuzz=500pt \hyperref[earthRotationType]{earthRotation} & \hfuzz=500pt \\
\hfuzz=500pt\quad\includegraphics[width=1em]{connector.pdf}\includegraphics[width=1em]{element.pdf}~ephemerides & \hfuzz=500pt \hyperref[ephemeridesType]{ephemerides} & \hfuzz=500pt \\
\hfuzz=500pt\quad\includegraphics[width=1em]{connector.pdf}\includegraphics[width=1em]{element-unbounded.pdf}~gradientfield & \hfuzz=500pt \hyperref[gravityfieldType]{gravityfield} & \hfuzz=500pt low order field to estimate the change of the gravity by position adjustement\\
\hfuzz=500pt\quad\includegraphics[width=1em]{connector.pdf}\includegraphics[width=1em]{element-unbounded.pdf}~parametrizationGravity & \hfuzz=500pt \hyperref[parametrizationGravityType]{parametrizationGravity} & \hfuzz=500pt gravity field parametrization\\
\hfuzz=500pt\quad\includegraphics[width=1em]{connector.pdf}\includegraphics[width=1em]{element-unbounded.pdf}~parametrizationAcceleration & \hfuzz=500pt \hyperref[parametrizationAccelerationType]{parametrizationAcceleration} & \hfuzz=500pt orbit force parameters\\
\hfuzz=500pt\quad\includegraphics[width=1em]{connector.pdf}\includegraphics[width=1em]{element.pdf}~keepSatelliteStates & \hfuzz=500pt boolean & \hfuzz=500pt set boundary values of each arc global\\
\hfuzz=500pt\quad\includegraphics[width=1em]{connector.pdf}\includegraphics[width=1em]{element.pdf}~integrationDegree & \hfuzz=500pt uint & \hfuzz=500pt integration of forces by polynomial approximation of degree n\\
\hfuzz=500pt\quad\includegraphics[width=1em]{connector.pdf}\includegraphics[width=1em]{element.pdf}~interpolationDegree & \hfuzz=500pt uint & \hfuzz=500pt orbit interpolation by polynomial approximation of degree n\\
\hfuzz=500pt\quad\includegraphics[width=1em]{connector.pdf}\includegraphics[width=1em]{element.pdf}~accelerateComputation & \hfuzz=500pt boolean & \hfuzz=500pt acceleration of computation by transforming the observations\\
\hfuzz=500pt\includegraphics[width=1em]{connector.pdf}\includegraphics[width=1em]{element-mustset.pdf}~podVariational & \hfuzz=500pt sequence & \hfuzz=500pt precise orbit data (variational equations)\\
\hfuzz=500pt\quad\includegraphics[width=1em]{connector.pdf}\includegraphics[width=1em]{element-mustset.pdf}~rightHandSide & \hfuzz=500pt sequence & \hfuzz=500pt input for observation vectors\\
\hfuzz=500pt\quad\quad\includegraphics[width=1em]{connector.pdf}\includegraphics[width=1em]{element-mustset.pdf}~inputfileOrbit & \hfuzz=500pt filename & \hfuzz=500pt kinematic positions as observations\\
\hfuzz=500pt\quad\includegraphics[width=1em]{connector.pdf}\includegraphics[width=1em]{element-mustset.pdf}~inputfileVariational & \hfuzz=500pt filename & \hfuzz=500pt approximate position and integrated state matrix\\
\hfuzz=500pt\quad\includegraphics[width=1em]{connector.pdf}\includegraphics[width=1em]{element.pdf}~ephemerides & \hfuzz=500pt \hyperref[ephemeridesType]{ephemerides} & \hfuzz=500pt \\
\hfuzz=500pt\quad\includegraphics[width=1em]{connector.pdf}\includegraphics[width=1em]{element-unbounded.pdf}~parametrizationGravity & \hfuzz=500pt \hyperref[parametrizationGravityType]{parametrizationGravity} & \hfuzz=500pt gravity field parametrization\\
\hfuzz=500pt\quad\includegraphics[width=1em]{connector.pdf}\includegraphics[width=1em]{element-unbounded.pdf}~parametrizationAcceleration & \hfuzz=500pt \hyperref[parametrizationAccelerationType]{parametrizationAcceleration} & \hfuzz=500pt orbit force parameters\\
\hfuzz=500pt\quad\includegraphics[width=1em]{connector.pdf}\includegraphics[width=1em]{element.pdf}~integrationDegree & \hfuzz=500pt uint & \hfuzz=500pt integration of forces by polynomial approximation of degree n\\
\hfuzz=500pt\quad\includegraphics[width=1em]{connector.pdf}\includegraphics[width=1em]{element.pdf}~interpolationDegree & \hfuzz=500pt uint & \hfuzz=500pt orbit interpolation by polynomial approximation of degree n\\
\hfuzz=500pt\quad\includegraphics[width=1em]{connector.pdf}\includegraphics[width=1em]{element.pdf}~accelerateComputation & \hfuzz=500pt boolean & \hfuzz=500pt acceleration of computation by transforming the observations\\
\hfuzz=500pt\includegraphics[width=1em]{element-mustset.pdf}~covariancePod & \hfuzz=500pt sequence & \hfuzz=500pt \\
\hfuzz=500pt\includegraphics[width=1em]{connector.pdf}\includegraphics[width=1em]{element.pdf}~sigma & \hfuzz=500pt double & \hfuzz=500pt apriori factor of covariance function\\
\hfuzz=500pt\includegraphics[width=1em]{connector.pdf}\includegraphics[width=1em]{element.pdf}~inputfileSigmasPerArc & \hfuzz=500pt filename & \hfuzz=500pt apriori different accuaries for each arc (multiplicated with sigma)\\
\hfuzz=500pt\includegraphics[width=1em]{connector.pdf}\includegraphics[width=1em]{element.pdf}~inputfileCovarianceFunction & \hfuzz=500pt filename & \hfuzz=500pt approximate covariances in time\\
\hfuzz=500pt\includegraphics[width=1em]{connector.pdf}\includegraphics[width=1em]{element.pdf}~inputfileCovariancePodEpoch & \hfuzz=500pt filename & \hfuzz=500pt 3x3 epoch covariances\\
\hfuzz=500pt\includegraphics[width=1em]{connector.pdf}\includegraphics[width=1em]{element.pdf}~sampling & \hfuzz=500pt double & \hfuzz=500pt [seconds] sampling of the covariance function\\
\hfuzz=500pt\includegraphics[width=1em]{element.pdf}~inputfileArcList & \hfuzz=500pt filename & \hfuzz=500pt list to correspond points of time to arc numbers\\
\hfuzz=500pt\includegraphics[width=1em]{element.pdf}~adjustmentThreshold & \hfuzz=500pt double & \hfuzz=500pt Adjustment factor threshold: Iteration will be stopped once both SST and POD adjustment factors are under this threshold\\
\hfuzz=500pt\includegraphics[width=1em]{element.pdf}~iterationCount & \hfuzz=500pt uint & \hfuzz=500pt (maximum) number of iterations for the estimation of calibration parameter and error PSD\\
\hline
\end{tabularx}

This program is \reference{parallelized}{general.parallelization}.
\clearpage
%==================================
\subsection{PreprocessingSst}\label{PreprocessingSst}
This program processes satellite-to-satellite-tracking (SST) and kinematic orbit observations in a GRACE like configuration.
Three different observation groups are considered separately: SST and POD1/POD2 for the two satellites.
This program works similar to \program{PreprocessingPod}, see there for details. Here only deviations
in the settings are explained.

Precise orbit data (POD) often contains systematic errors in addition to stochastic noise. In this case the
variance component estimation fails and assigns too much weight to the POD data. Therefore an additional
\config{downweightPod} factor can be applied to the standard deviation of POD for the next least squares adjustment
in the iteration. This factor should also applied as \config{sigma} in \configClass{observation}{observationType}
for computation of the final solution e.g. with \program{NormalsSolverVCE}.

Short time variations of the gravity field can be co-estimated together with the static/monthly
mean gravity field. The short time parameters must also be set in \configClass{observation:parametrizationGravity}{parametrizationGravityType} and
can then be selected by \configClass{estimateShortTimeVariations:parameterSelection}{parameterSelectorType}.
If these parameters are not time variable, for example when a range of static parameters is selected,
they are set up as constant for each time interval defined in \config{inputfileArcList}. The parameters are constrained by an
\configClass{estimateShortTimeVariations:autoregressiveModelSequence}{autoregressiveModelSequenceType}. The weight of
the constrain equations in terms of the standard deviation can be estimated by means of
Variance Component Estimation (VCE) if \config{estimateShortTimeVariations:estimateSigma} is set.
The mathematical background of this co-estimation can be found in:

Kvas, A., Mayer-Gürr, T. GRACE gravity field recovery with background model uncertainties.
J Geod 93, 2543–2552 (2019). \url{https://doi.org/10.1007/s00190-019-01314-1}.


\keepXColumns
\begin{tabularx}{\textwidth}{N T A}
\hline
Name & Type & Annotation\\
\hline
\hfuzz=500pt\includegraphics[width=1em]{element.pdf}~outputfileSolution & \hfuzz=500pt filename & \hfuzz=500pt estimated parameter vector (static part only)\\
\hfuzz=500pt\includegraphics[width=1em]{element.pdf}~outputfileSigmax & \hfuzz=500pt filename & \hfuzz=500pt standard deviations of the parameters (sqrt of the diagonal of the inverse normal equation)\\
\hfuzz=500pt\includegraphics[width=1em]{element.pdf}~outputfileParameterName & \hfuzz=500pt filename & \hfuzz=500pt estimated signal parameters (index is appended)\\
\hfuzz=500pt\includegraphics[width=1em]{element.pdf}~estimateArcSigmas & \hfuzz=500pt sequence & \hfuzz=500pt \\
\hfuzz=500pt\includegraphics[width=1em]{connector.pdf}\includegraphics[width=1em]{element.pdf}~outputfileSigmasPerArcSst & \hfuzz=500pt filename & \hfuzz=500pt accuracies of each arc (SST)\\
\hfuzz=500pt\includegraphics[width=1em]{connector.pdf}\includegraphics[width=1em]{element.pdf}~outputfileSigmasPerArcPod1 & \hfuzz=500pt filename & \hfuzz=500pt accuracies of each arc (POD1)\\
\hfuzz=500pt\includegraphics[width=1em]{connector.pdf}\includegraphics[width=1em]{element.pdf}~outputfileSigmasPerArcPod2 & \hfuzz=500pt filename & \hfuzz=500pt accuracies of each arc (POD2)\\
\hfuzz=500pt\includegraphics[width=1em]{element.pdf}~estimateEpochSigmas & \hfuzz=500pt sequence & \hfuzz=500pt \\
\hfuzz=500pt\includegraphics[width=1em]{connector.pdf}\includegraphics[width=1em]{element.pdf}~outputfileSigmasPerEpochSst & \hfuzz=500pt filename & \hfuzz=500pt accuracies of each epoch (SST)\\
\hfuzz=500pt\includegraphics[width=1em]{connector.pdf}\includegraphics[width=1em]{element.pdf}~outputfileSigmasPerEpochPod1 & \hfuzz=500pt filename & \hfuzz=500pt accuracies of each epoch (POD1)\\
\hfuzz=500pt\includegraphics[width=1em]{connector.pdf}\includegraphics[width=1em]{element.pdf}~outputfileSigmasPerEpochPod2 & \hfuzz=500pt filename & \hfuzz=500pt accuracies of each epoch (POD2)\\
\hfuzz=500pt\includegraphics[width=1em]{element.pdf}~estimateCovarianceFunctions & \hfuzz=500pt sequence & \hfuzz=500pt \\
\hfuzz=500pt\includegraphics[width=1em]{connector.pdf}\includegraphics[width=1em]{element.pdf}~outputfileCovarianceFunctionSst & \hfuzz=500pt filename & \hfuzz=500pt covariance function\\
\hfuzz=500pt\includegraphics[width=1em]{connector.pdf}\includegraphics[width=1em]{element.pdf}~outputfileCovarianceFunctionPod1 & \hfuzz=500pt filename & \hfuzz=500pt covariance functions for along, cross, radial direction\\
\hfuzz=500pt\includegraphics[width=1em]{connector.pdf}\includegraphics[width=1em]{element.pdf}~outputfileCovarianceFunctionPod2 & \hfuzz=500pt filename & \hfuzz=500pt covariance functions for along, cross, radial direction\\
\hfuzz=500pt\includegraphics[width=1em]{element.pdf}~estimateSstArcCovarianceSigmas & \hfuzz=500pt sequence & \hfuzz=500pt \\
\hfuzz=500pt\includegraphics[width=1em]{connector.pdf}\includegraphics[width=1em]{element.pdf}~outputfileSigmasCovarianceMatrixArc & \hfuzz=500pt filename & \hfuzz=500pt one variance factor per matrix\\
\hfuzz=500pt\includegraphics[width=1em]{element.pdf}~computeResiduals & \hfuzz=500pt sequence & \hfuzz=500pt \\
\hfuzz=500pt\includegraphics[width=1em]{connector.pdf}\includegraphics[width=1em]{element.pdf}~outputfileSstResiduals & \hfuzz=500pt filename & \hfuzz=500pt \\
\hfuzz=500pt\includegraphics[width=1em]{connector.pdf}\includegraphics[width=1em]{element.pdf}~outputfilePod1Residuals & \hfuzz=500pt filename & \hfuzz=500pt \\
\hfuzz=500pt\includegraphics[width=1em]{connector.pdf}\includegraphics[width=1em]{element.pdf}~outputfilePod2Residuals & \hfuzz=500pt filename & \hfuzz=500pt \\
\hfuzz=500pt\includegraphics[width=1em]{element-mustset.pdf}~observation & \hfuzz=500pt choice & \hfuzz=500pt obervation equations (Sst)\\
\hfuzz=500pt\includegraphics[width=1em]{connector.pdf}\includegraphics[width=1em]{element-mustset.pdf}~sstIntegral & \hfuzz=500pt sequence & \hfuzz=500pt integral approach\\
\hfuzz=500pt\quad\includegraphics[width=1em]{connector.pdf}\includegraphics[width=1em]{element.pdf}~inputfileSatelliteModel1 & \hfuzz=500pt filename & \hfuzz=500pt satellite macro model\\
\hfuzz=500pt\quad\includegraphics[width=1em]{connector.pdf}\includegraphics[width=1em]{element.pdf}~inputfileSatelliteModel2 & \hfuzz=500pt filename & \hfuzz=500pt satellite macro model\\
\hfuzz=500pt\quad\includegraphics[width=1em]{connector.pdf}\includegraphics[width=1em]{element-mustset-unbounded.pdf}~rightHandSide & \hfuzz=500pt \hyperref[sstRightSideType]{sstRightSide} & \hfuzz=500pt input for the reduced observation vector\\
\hfuzz=500pt\quad\includegraphics[width=1em]{connector.pdf}\includegraphics[width=1em]{element-mustset.pdf}~sstType & \hfuzz=500pt choice & \hfuzz=500pt \\
\hfuzz=500pt\quad\quad\includegraphics[width=1em]{connector.pdf}\includegraphics[width=1em]{element-mustset.pdf}~range & \hfuzz=500pt  & \hfuzz=500pt \\
\hfuzz=500pt\quad\quad\includegraphics[width=1em]{connector.pdf}\includegraphics[width=1em]{element-mustset.pdf}~rangeRate & \hfuzz=500pt  & \hfuzz=500pt \\
\hfuzz=500pt\quad\quad\includegraphics[width=1em]{connector.pdf}\includegraphics[width=1em]{element-mustset.pdf}~rangeAcceleration & \hfuzz=500pt  & \hfuzz=500pt \\
\hfuzz=500pt\quad\quad\includegraphics[width=1em]{connector.pdf}\includegraphics[width=1em]{element-mustset.pdf}~none & \hfuzz=500pt  & \hfuzz=500pt \\
\hfuzz=500pt\quad\includegraphics[width=1em]{connector.pdf}\includegraphics[width=1em]{element-mustset.pdf}~inputfileOrbit1 & \hfuzz=500pt filename & \hfuzz=500pt used to evaluate the observation equations, not used as observations\\
\hfuzz=500pt\quad\includegraphics[width=1em]{connector.pdf}\includegraphics[width=1em]{element-mustset.pdf}~inputfileOrbit2 & \hfuzz=500pt filename & \hfuzz=500pt used to evaluate the observation equations, not used as observations\\
\hfuzz=500pt\quad\includegraphics[width=1em]{connector.pdf}\includegraphics[width=1em]{element-mustset.pdf}~inputfileStarCamera1 & \hfuzz=500pt filename & \hfuzz=500pt \\
\hfuzz=500pt\quad\includegraphics[width=1em]{connector.pdf}\includegraphics[width=1em]{element-mustset.pdf}~inputfileStarCamera2 & \hfuzz=500pt filename & \hfuzz=500pt \\
\hfuzz=500pt\quad\includegraphics[width=1em]{connector.pdf}\includegraphics[width=1em]{element-mustset.pdf}~earthRotation & \hfuzz=500pt \hyperref[earthRotationType]{earthRotation} & \hfuzz=500pt \\
\hfuzz=500pt\quad\includegraphics[width=1em]{connector.pdf}\includegraphics[width=1em]{element.pdf}~ephemerides & \hfuzz=500pt \hyperref[ephemeridesType]{ephemerides} & \hfuzz=500pt \\
\hfuzz=500pt\quad\includegraphics[width=1em]{connector.pdf}\includegraphics[width=1em]{element-unbounded.pdf}~gradientfield & \hfuzz=500pt \hyperref[gravityfieldType]{gravityfield} & \hfuzz=500pt low order field to estimate the change of the gravity by position adjustement\\
\hfuzz=500pt\quad\includegraphics[width=1em]{connector.pdf}\includegraphics[width=1em]{element-unbounded.pdf}~parametrizationGravity & \hfuzz=500pt \hyperref[parametrizationGravityType]{parametrizationGravity} & \hfuzz=500pt gravity field parametrization\\
\hfuzz=500pt\quad\includegraphics[width=1em]{connector.pdf}\includegraphics[width=1em]{element-unbounded.pdf}~parametrizationAcceleration1 & \hfuzz=500pt \hyperref[parametrizationAccelerationType]{parametrizationAcceleration} & \hfuzz=500pt orbit1 force parameters\\
\hfuzz=500pt\quad\includegraphics[width=1em]{connector.pdf}\includegraphics[width=1em]{element-unbounded.pdf}~parametrizationAcceleration2 & \hfuzz=500pt \hyperref[parametrizationAccelerationType]{parametrizationAcceleration} & \hfuzz=500pt orbit2 force parameters\\
\hfuzz=500pt\quad\includegraphics[width=1em]{connector.pdf}\includegraphics[width=1em]{element-unbounded.pdf}~parametrizationSst & \hfuzz=500pt \hyperref[parametrizationSatelliteTrackingType]{parametrizationSatelliteTracking} & \hfuzz=500pt satellite tracking parameter\\
\hfuzz=500pt\quad\includegraphics[width=1em]{connector.pdf}\includegraphics[width=1em]{element.pdf}~keepSatelliteStates & \hfuzz=500pt boolean & \hfuzz=500pt set boundary values of each arc global\\
\hfuzz=500pt\quad\includegraphics[width=1em]{connector.pdf}\includegraphics[width=1em]{element.pdf}~integrationDegree & \hfuzz=500pt uint & \hfuzz=500pt integration of forces by polynomial approximation of degree n\\
\hfuzz=500pt\quad\includegraphics[width=1em]{connector.pdf}\includegraphics[width=1em]{element.pdf}~interpolationDegree & \hfuzz=500pt uint & \hfuzz=500pt orbit interpolation by polynomial approximation of degree n\\
\hfuzz=500pt\includegraphics[width=1em]{connector.pdf}\includegraphics[width=1em]{element-mustset.pdf}~sstVariational & \hfuzz=500pt sequence & \hfuzz=500pt variational equations\\
\hfuzz=500pt\quad\includegraphics[width=1em]{connector.pdf}\includegraphics[width=1em]{element-mustset.pdf}~rightHandSide & \hfuzz=500pt sequence & \hfuzz=500pt input for observation vectors\\
\hfuzz=500pt\quad\quad\includegraphics[width=1em]{connector.pdf}\includegraphics[width=1em]{element-mustset-unbounded.pdf}~inputfileSatelliteTracking & \hfuzz=500pt filename & \hfuzz=500pt ranging observations and corrections\\
\hfuzz=500pt\quad\quad\includegraphics[width=1em]{connector.pdf}\includegraphics[width=1em]{element.pdf}~inputfileOrbit1 & \hfuzz=500pt filename & \hfuzz=500pt kinematic positions of satellite A as observations\\
\hfuzz=500pt\quad\quad\includegraphics[width=1em]{connector.pdf}\includegraphics[width=1em]{element.pdf}~inputfileOrbit2 & \hfuzz=500pt filename & \hfuzz=500pt kinematic positions of satellite B as observations\\
\hfuzz=500pt\quad\includegraphics[width=1em]{connector.pdf}\includegraphics[width=1em]{element-mustset.pdf}~sstType & \hfuzz=500pt choice & \hfuzz=500pt \\
\hfuzz=500pt\quad\quad\includegraphics[width=1em]{connector.pdf}\includegraphics[width=1em]{element-mustset.pdf}~range & \hfuzz=500pt  & \hfuzz=500pt \\
\hfuzz=500pt\quad\quad\includegraphics[width=1em]{connector.pdf}\includegraphics[width=1em]{element-mustset.pdf}~rangeRate & \hfuzz=500pt  & \hfuzz=500pt \\
\hfuzz=500pt\quad\quad\includegraphics[width=1em]{connector.pdf}\includegraphics[width=1em]{element-mustset.pdf}~none & \hfuzz=500pt  & \hfuzz=500pt \\
\hfuzz=500pt\quad\includegraphics[width=1em]{connector.pdf}\includegraphics[width=1em]{element-mustset.pdf}~inputfileVariational1 & \hfuzz=500pt filename & \hfuzz=500pt approximate position and integrated state matrix\\
\hfuzz=500pt\quad\includegraphics[width=1em]{connector.pdf}\includegraphics[width=1em]{element-mustset.pdf}~inputfileVariational2 & \hfuzz=500pt filename & \hfuzz=500pt approximate position and integrated state matrix\\
\hfuzz=500pt\quad\includegraphics[width=1em]{connector.pdf}\includegraphics[width=1em]{element.pdf}~ephemerides & \hfuzz=500pt \hyperref[ephemeridesType]{ephemerides} & \hfuzz=500pt \\
\hfuzz=500pt\quad\includegraphics[width=1em]{connector.pdf}\includegraphics[width=1em]{element-unbounded.pdf}~parametrizationGravity & \hfuzz=500pt \hyperref[parametrizationGravityType]{parametrizationGravity} & \hfuzz=500pt gravity field parametrization\\
\hfuzz=500pt\quad\includegraphics[width=1em]{connector.pdf}\includegraphics[width=1em]{element-unbounded.pdf}~parametrizationAcceleration1 & \hfuzz=500pt \hyperref[parametrizationAccelerationType]{parametrizationAcceleration} & \hfuzz=500pt orbit1 force parameters\\
\hfuzz=500pt\quad\includegraphics[width=1em]{connector.pdf}\includegraphics[width=1em]{element-unbounded.pdf}~parametrizationAcceleration2 & \hfuzz=500pt \hyperref[parametrizationAccelerationType]{parametrizationAcceleration} & \hfuzz=500pt orbit2 force parameters\\
\hfuzz=500pt\quad\includegraphics[width=1em]{connector.pdf}\includegraphics[width=1em]{element-unbounded.pdf}~parametrizationSst & \hfuzz=500pt \hyperref[parametrizationSatelliteTrackingType]{parametrizationSatelliteTracking} & \hfuzz=500pt satellite tracking parameter\\
\hfuzz=500pt\quad\includegraphics[width=1em]{connector.pdf}\includegraphics[width=1em]{element.pdf}~integrationDegree & \hfuzz=500pt uint & \hfuzz=500pt integration of forces by polynomial approximation of degree n\\
\hfuzz=500pt\quad\includegraphics[width=1em]{connector.pdf}\includegraphics[width=1em]{element.pdf}~interpolationDegree & \hfuzz=500pt uint & \hfuzz=500pt orbit interpolation by polynomial approximation of degree n\\
\hfuzz=500pt\includegraphics[width=1em]{element-mustset.pdf}~covarianceSst & \hfuzz=500pt sequence & \hfuzz=500pt \\
\hfuzz=500pt\includegraphics[width=1em]{connector.pdf}\includegraphics[width=1em]{element.pdf}~sigma & \hfuzz=500pt double & \hfuzz=500pt apriori factor of covariance function\\
\hfuzz=500pt\includegraphics[width=1em]{connector.pdf}\includegraphics[width=1em]{element.pdf}~inputfileSigmasPerArc & \hfuzz=500pt filename & \hfuzz=500pt apriori different accuaries for each arc (multiplicated with sigma)\\
\hfuzz=500pt\includegraphics[width=1em]{connector.pdf}\includegraphics[width=1em]{element.pdf}~inputfileSigmasPerEpoch & \hfuzz=500pt filename & \hfuzz=500pt apriori different accuaries for each epoch\\
\hfuzz=500pt\includegraphics[width=1em]{connector.pdf}\includegraphics[width=1em]{element.pdf}~inputfileCovarianceFunction & \hfuzz=500pt filename & \hfuzz=500pt approximate covariances in time\\
\hfuzz=500pt\includegraphics[width=1em]{connector.pdf}\includegraphics[width=1em]{element-unbounded.pdf}~inputfileCovarianceMatrixArc & \hfuzz=500pt filename & \hfuzz=500pt Must be given per sst arc with correct dimensions.\\
\hfuzz=500pt\includegraphics[width=1em]{connector.pdf}\includegraphics[width=1em]{element.pdf}~inputfileSigmasCovarianceMatrixArc & \hfuzz=500pt filename & \hfuzz=500pt Vector with one sigma for each \$<\$inputfileCovarianceMatrixArc\$>\$\\
\hfuzz=500pt\includegraphics[width=1em]{connector.pdf}\includegraphics[width=1em]{element.pdf}~sampling & \hfuzz=500pt double & \hfuzz=500pt [seconds] sampling of the covariance function\\
\hfuzz=500pt\includegraphics[width=1em]{element-mustset.pdf}~covariancePod1 & \hfuzz=500pt sequence & \hfuzz=500pt \\
\hfuzz=500pt\includegraphics[width=1em]{connector.pdf}\includegraphics[width=1em]{element.pdf}~sigma & \hfuzz=500pt double & \hfuzz=500pt apriori factor of covariance function\\
\hfuzz=500pt\includegraphics[width=1em]{connector.pdf}\includegraphics[width=1em]{element.pdf}~inputfileSigmasPerArc & \hfuzz=500pt filename & \hfuzz=500pt apriori different accuaries for each arc (multiplicated with sigma)\\
\hfuzz=500pt\includegraphics[width=1em]{connector.pdf}\includegraphics[width=1em]{element.pdf}~inputfileSigmasPerEpoch & \hfuzz=500pt filename & \hfuzz=500pt apriori different accuaries for each epoch\\
\hfuzz=500pt\includegraphics[width=1em]{connector.pdf}\includegraphics[width=1em]{element.pdf}~inputfileCovarianceFunction & \hfuzz=500pt filename & \hfuzz=500pt approximate covariances in time\\
\hfuzz=500pt\includegraphics[width=1em]{connector.pdf}\includegraphics[width=1em]{element.pdf}~inputfileCovariancePodEpoch & \hfuzz=500pt filename & \hfuzz=500pt 3x3 epoch covariances\\
\hfuzz=500pt\includegraphics[width=1em]{connector.pdf}\includegraphics[width=1em]{element.pdf}~sampling & \hfuzz=500pt double & \hfuzz=500pt [seconds] sampling of the covariance function\\
\hfuzz=500pt\includegraphics[width=1em]{element-mustset.pdf}~covariancePod2 & \hfuzz=500pt sequence & \hfuzz=500pt \\
\hfuzz=500pt\includegraphics[width=1em]{connector.pdf}\includegraphics[width=1em]{element.pdf}~sigma & \hfuzz=500pt double & \hfuzz=500pt apriori factor of covariance function\\
\hfuzz=500pt\includegraphics[width=1em]{connector.pdf}\includegraphics[width=1em]{element.pdf}~inputfileSigmasPerArc & \hfuzz=500pt filename & \hfuzz=500pt apriori different accuaries for each arc (multiplicated with sigma)\\
\hfuzz=500pt\includegraphics[width=1em]{connector.pdf}\includegraphics[width=1em]{element.pdf}~inputfileSigmasPerEpoch & \hfuzz=500pt filename & \hfuzz=500pt apriori different accuaries for each epoch\\
\hfuzz=500pt\includegraphics[width=1em]{connector.pdf}\includegraphics[width=1em]{element.pdf}~inputfileCovarianceFunction & \hfuzz=500pt filename & \hfuzz=500pt approximate covariances in time\\
\hfuzz=500pt\includegraphics[width=1em]{connector.pdf}\includegraphics[width=1em]{element.pdf}~inputfileCovariancePodEpoch & \hfuzz=500pt filename & \hfuzz=500pt 3x3 epoch covariances\\
\hfuzz=500pt\includegraphics[width=1em]{connector.pdf}\includegraphics[width=1em]{element.pdf}~sampling & \hfuzz=500pt double & \hfuzz=500pt [seconds] sampling of the covariance function\\
\hfuzz=500pt\includegraphics[width=1em]{element.pdf}~estimateShortTimeVariations & \hfuzz=500pt sequence & \hfuzz=500pt co-estimate short time gravity field variations\\
\hfuzz=500pt\includegraphics[width=1em]{connector.pdf}\includegraphics[width=1em]{element.pdf}~estimateSigma & \hfuzz=500pt boolean & \hfuzz=500pt estimate standard deviation via VCE\\
\hfuzz=500pt\includegraphics[width=1em]{connector.pdf}\includegraphics[width=1em]{element-mustset.pdf}~autoregressiveModelSequence & \hfuzz=500pt \hyperref[autoregressiveModelSequenceType]{autoregressiveModelSequence} & \hfuzz=500pt AR model sequence for constraining short time gravity variations\\
\hfuzz=500pt\includegraphics[width=1em]{connector.pdf}\includegraphics[width=1em]{element-mustset-unbounded.pdf}~parameterSelection & \hfuzz=500pt \hyperref[parameterSelectorType]{parameterSelector} & \hfuzz=500pt parameters describing the short time gravity field\\
\hfuzz=500pt\includegraphics[width=1em]{element.pdf}~downweightPod & \hfuzz=500pt double & \hfuzz=500pt downweight factor for POD\\
\hfuzz=500pt\includegraphics[width=1em]{element.pdf}~inputfileArcList & \hfuzz=500pt filename & \hfuzz=500pt list to correspond points of time to arc numbers\\
\hfuzz=500pt\includegraphics[width=1em]{element.pdf}~iterationCount & \hfuzz=500pt uint & \hfuzz=500pt (maximum) number of iterations for the estimation of calibration parameter and error PSD\\
\hfuzz=500pt\includegraphics[width=1em]{element.pdf}~variableNameIterations & \hfuzz=500pt string & \hfuzz=500pt All output fileNames in preprocessing iteration are expanded with this variable prior to writing to disk\\
\hfuzz=500pt\includegraphics[width=1em]{element.pdf}~defaultBlockSize & \hfuzz=500pt uint & \hfuzz=500pt block size of static normal equation blocks\\
\hline
\end{tabularx}

This program is \reference{parallelized}{general.parallelization}.
\clearpage
%==================================
\subsection{PreprocessingVariationalEquation}\label{PreprocessingVariationalEquation}
This program integrates an orbit dynamically using the given forces and setup the state transition matrix
for each time step. These are the prerequisite for a least squares adjustement (e.g. gravity field determination) using
the variational equation approach. The variational equations are computed arc wise as defined by \configFile{inputfileOrbit}{instrument}.
This means for each arc new initial state parameters are setup.

In a first step the \configClass{forces}{forcesType} acting on the satellite are evaluated at the apriori positions given
by \configFile{inputfileOrbit}{instrument}. Non-conservative forces like solar radiation pressure needs the orientation of the
satellite (\configFile{inputfileStarCamera}{instrument}) and additional a satellite macro model (\config{satelliteModel})
with the surface properties. Furthermore \configFile{inputfileAccelerometer}{instrument} observations are also considered.

In a second step the accelerations are integrated twice to an dynamic orbit unsing a moving polynomial with the degree
\config{integrationDegree}. The orbit is corrected to be self-consistent. This means the forces should be evaluated
at the new integrated positions instead of the apriori ones. This correction is computed in a linear approximation
using the gradient of the forces with respect to the positions (\config{gradientfield}). As this term is small generally
only the largest force components has to be considered. A low degree spherical harmonic expansion of the static gravity
field (about up to degree 5) is sufficient in almost all cases. In this step also the state transition matrix (the partial
derivatices of the current state, position and velocity) with respect to the initial state is computed.
The integrated orbit together with the state transitions are stored in \configFile{outputfileVariational}{variationalEquation},
the integrated orbit only in \configFile{outputfileOrbit}{instrument}.

To improve the numerical stability a reference ellipse can be reduced beforehand using Enke's method (\config{useEnke}).
Mathematically the result is the same, but as the large central term is removed before and restored
afterwards more digits are available for the computation.

The integrated orbit should be fitted to observations afterwards by the programs
\program{PreprocessingVariationalEquationOrbitFit} and/or \program{PreprocessingVariationalEquationSstFit}.
They apply a least squares adjustment by estimating some satellite parameters (e.g. an accelerometer bias).
If the fitted orbit is to far away from the original \configFile{inputfileOrbit}{instrument} the linearization may not be
accurate enough. In this case \program{PreprocessingVariationalEquation} should be run again with the fitted orbit
as \configFile{inputfileOrbit}{instrument} and introducing the \config{estimatedParameters} as additional forces.


\keepXColumns
\begin{tabularx}{\textwidth}{N T A}
\hline
Name & Type & Annotation\\
\hline
\hfuzz=500pt\includegraphics[width=1em]{element-mustset.pdf}~outputfileVariational & \hfuzz=500pt filename & \hfuzz=500pt approximate position and integrated state matrix\\
\hfuzz=500pt\includegraphics[width=1em]{element.pdf}~outputfileOrbit & \hfuzz=500pt filename & \hfuzz=500pt integrated orbit\\
\hfuzz=500pt\includegraphics[width=1em]{element.pdf}~inputfileSatelliteModel & \hfuzz=500pt filename & \hfuzz=500pt satellite macro model\\
\hfuzz=500pt\includegraphics[width=1em]{element-mustset.pdf}~inputfileOrbit & \hfuzz=500pt filename & \hfuzz=500pt approximate position, used to evaluate the force\\
\hfuzz=500pt\includegraphics[width=1em]{element-mustset.pdf}~inputfileStarCamera & \hfuzz=500pt filename & \hfuzz=500pt rotation from body frame to CRF\\
\hfuzz=500pt\includegraphics[width=1em]{element.pdf}~inputfileAccelerometer & \hfuzz=500pt filename & \hfuzz=500pt non-gravitational forces in satellite reference frame\\
\hfuzz=500pt\includegraphics[width=1em]{element-mustset.pdf}~forces & \hfuzz=500pt \hyperref[forcesType]{forces} & \hfuzz=500pt \\
\hfuzz=500pt\includegraphics[width=1em]{element.pdf}~estimatedParameters & \hfuzz=500pt sequence & \hfuzz=500pt satellite parameters e.g. from orbit fit\\
\hfuzz=500pt\includegraphics[width=1em]{connector.pdf}\includegraphics[width=1em]{element-mustset-unbounded.pdf}~parametrizationAcceleration & \hfuzz=500pt \hyperref[parametrizationAccelerationType]{parametrizationAcceleration} & \hfuzz=500pt orbit force parameters\\
\hfuzz=500pt\includegraphics[width=1em]{connector.pdf}\includegraphics[width=1em]{element-mustset.pdf}~inputfileParameter & \hfuzz=500pt filename & \hfuzz=500pt estimated orbit force parameters\\
\hfuzz=500pt\includegraphics[width=1em]{element-mustset.pdf}~earthRotation & \hfuzz=500pt \hyperref[earthRotationType]{earthRotation} & \hfuzz=500pt \\
\hfuzz=500pt\includegraphics[width=1em]{element.pdf}~ephemerides & \hfuzz=500pt \hyperref[ephemeridesType]{ephemerides} & \hfuzz=500pt \\
\hfuzz=500pt\includegraphics[width=1em]{element-mustset-unbounded.pdf}~gradientfield & \hfuzz=500pt \hyperref[gravityfieldType]{gravityfield} & \hfuzz=500pt low order field to estimate the change of the gravity by position adjustement\\
\hfuzz=500pt\includegraphics[width=1em]{element.pdf}~integrationDegree & \hfuzz=500pt uint & \hfuzz=500pt integration of forces by polynomial approximation of degree n\\
\hfuzz=500pt\includegraphics[width=1em]{element.pdf}~useEnke & \hfuzz=500pt sequence & \hfuzz=500pt integrate differential forces to an elliptical reference trajectory\\
\hfuzz=500pt\includegraphics[width=1em]{connector.pdf}\includegraphics[width=1em]{element.pdf}~GM & \hfuzz=500pt double & \hfuzz=500pt geocentric gravitational constant used for elliptical reference orbit\\
\hline
\end{tabularx}

This program is \reference{parallelized}{general.parallelization}.
\clearpage
%==================================
\subsection{PreprocessingVariationalEquationOrbitFit}\label{PreprocessingVariationalEquationOrbitFit}
This program fits an \configFile{inputfileVariational}{variationalEquation} to an observed \configFile{inputfileOrbit}{instrument} by estimating parameters
in a least squares adjustment. Additional to the initial satellite state for each arc, these parameters can be
\configClass{parametrizationGravity}{parametrizationGravityType}, satellite \configClass{parametrizationAcceleration}{parametrizationAccelerationType}
and stochastic pulses (velocity jumps) at given times, \configClass{stochasticPulse}{timeSeriesType}. The estimated parameters can be stored with
\configFile{outputfileSolution}{matrix} and an extra file with the parameter names is created. The fitted orbit is written
as new reference in \configFile{outputfileVariational}{variationalEquation} and additionally in \configFile{outputfileOrbit}{instrument}.

The observed orbit positions (\configFile{inputfileOrbit}{instrument}) together with the epoch wise covariance matrix
(\configFile{inputfileCovariancePodEpoch}{instrument}) must be splitted in the same arcs as the variational equations but not
necessarily uniform distributed (use irregularData in \program{InstrumentSynchronize}). An iterative downweighting of
outliers is performed by M-Huber method.

The observation equations (parameter sensitity matrix) are computed by integration of the variational equations
(\configFile{inputfileVariational}{variationalEquation}) using a polynomial with \config{integrationDegree} and interpolated to the
observation epochs using a polynomial with \config{interpolationDegree}.

All parameters used here must be reestimated in the full least squares adjustment
for the gravity field determination to get a solution which is not biased towards the reference field.
The solution of additional estimations are relative (deltas) as the parameters are already used as Taylor point
in the reference orbit.

See also \program{PreprocessingVariationalEquation}.


\keepXColumns
\begin{tabularx}{\textwidth}{N T A}
\hline
Name & Type & Annotation\\
\hline
\hfuzz=500pt\includegraphics[width=1em]{element-mustset.pdf}~outputfileVariational & \hfuzz=500pt filename & \hfuzz=500pt approximate position and integrated state matrix\\
\hfuzz=500pt\includegraphics[width=1em]{element.pdf}~outputfileOrbit & \hfuzz=500pt filename & \hfuzz=500pt integrated orbit\\
\hfuzz=500pt\includegraphics[width=1em]{element.pdf}~outputfileSolution & \hfuzz=500pt filename & \hfuzz=500pt estimated calibration and state parameters\\
\hfuzz=500pt\includegraphics[width=1em]{element-mustset.pdf}~inputfileVariational & \hfuzz=500pt filename & \hfuzz=500pt approximate position and integrated state matrix\\
\hfuzz=500pt\includegraphics[width=1em]{element-mustset.pdf}~inputfileOrbit & \hfuzz=500pt filename & \hfuzz=500pt kinematic positions of satellite as observations\\
\hfuzz=500pt\includegraphics[width=1em]{element.pdf}~inputfileCovariancePodEpoch & \hfuzz=500pt filename & \hfuzz=500pt 3x3 epoch wise covariances\\
\hfuzz=500pt\includegraphics[width=1em]{element.pdf}~ephemerides & \hfuzz=500pt \hyperref[ephemeridesType]{ephemerides} & \hfuzz=500pt may be needed by parametrizationAcceleration\\
\hfuzz=500pt\includegraphics[width=1em]{element-unbounded.pdf}~parametrizationGravity & \hfuzz=500pt \hyperref[parametrizationGravityType]{parametrizationGravity} & \hfuzz=500pt gravity field parametrization\\
\hfuzz=500pt\includegraphics[width=1em]{element-unbounded.pdf}~parametrizationAcceleration & \hfuzz=500pt \hyperref[parametrizationAccelerationType]{parametrizationAcceleration} & \hfuzz=500pt orbit force parameters\\
\hfuzz=500pt\includegraphics[width=1em]{element-unbounded.pdf}~stochasticPulse & \hfuzz=500pt \hyperref[timeSeriesType]{timeSeries} & \hfuzz=500pt \\
\hfuzz=500pt\includegraphics[width=1em]{element.pdf}~integrationDegree & \hfuzz=500pt uint & \hfuzz=500pt integration of forces by polynomial approximation of degree n\\
\hfuzz=500pt\includegraphics[width=1em]{element.pdf}~interpolationDegree & \hfuzz=500pt uint & \hfuzz=500pt orbit interpolation by polynomial approximation of degree n\\
\hfuzz=500pt\includegraphics[width=1em]{element.pdf}~iterationCount & \hfuzz=500pt uint & \hfuzz=500pt for the estimation of calibration parameter and error PSD\\
\hline
\end{tabularx}

This program is \reference{parallelized}{general.parallelization}.
\clearpage
%==================================
\subsection{PreprocessingVariationalEquationSstFit}\label{PreprocessingVariationalEquationSstFit}
This program fits two \configFile{inputfileVariational1/2}{variationalEquation} to satellite-to-satellite-tracking (SST) and orbit
observations in a GRACE like configuration. It works similar to \program{PreprocessingVariationalEquationOrbitFit},
see there for details.

As the relative weighting of the observation types is important complex description of the covariances can be set with
\configClass{covarianceSst}{covarianceSstType}, \configClass{covariancePod1}{covariancePodType}, \configClass{covariancePod2}{covariancePodType}.


\keepXColumns
\begin{tabularx}{\textwidth}{N T A}
\hline
Name & Type & Annotation\\
\hline
\hfuzz=500pt\includegraphics[width=1em]{element-mustset.pdf}~outputfileVariational1 & \hfuzz=500pt filename & \hfuzz=500pt approximate position and integrated state matrix\\
\hfuzz=500pt\includegraphics[width=1em]{element-mustset.pdf}~outputfileVariational2 & \hfuzz=500pt filename & \hfuzz=500pt approximate position and integrated state matrix\\
\hfuzz=500pt\includegraphics[width=1em]{element.pdf}~outputfileOrbit1 & \hfuzz=500pt filename & \hfuzz=500pt integrated orbit\\
\hfuzz=500pt\includegraphics[width=1em]{element.pdf}~outputfileOrbit2 & \hfuzz=500pt filename & \hfuzz=500pt integrated orbit\\
\hfuzz=500pt\includegraphics[width=1em]{element.pdf}~outputfileSolution1 & \hfuzz=500pt filename & \hfuzz=500pt estimated calibration and state parameters\\
\hfuzz=500pt\includegraphics[width=1em]{element.pdf}~outputfileSolution2 & \hfuzz=500pt filename & \hfuzz=500pt estimated calibration and state parameters\\
\hfuzz=500pt\includegraphics[width=1em]{element-mustset.pdf}~rightHandSide & \hfuzz=500pt sequence & \hfuzz=500pt input for observation vectors\\
\hfuzz=500pt\includegraphics[width=1em]{connector.pdf}\includegraphics[width=1em]{element-mustset-unbounded.pdf}~inputfileSatelliteTracking & \hfuzz=500pt filename & \hfuzz=500pt ranging observations and corrections\\
\hfuzz=500pt\includegraphics[width=1em]{connector.pdf}\includegraphics[width=1em]{element.pdf}~inputfileOrbit1 & \hfuzz=500pt filename & \hfuzz=500pt kinematic positions of satellite A as observations\\
\hfuzz=500pt\includegraphics[width=1em]{connector.pdf}\includegraphics[width=1em]{element.pdf}~inputfileOrbit2 & \hfuzz=500pt filename & \hfuzz=500pt kinematic positions of satellite B as observations\\
\hfuzz=500pt\includegraphics[width=1em]{element-mustset.pdf}~sstType & \hfuzz=500pt choice & \hfuzz=500pt \\
\hfuzz=500pt\includegraphics[width=1em]{connector.pdf}\includegraphics[width=1em]{element-mustset.pdf}~range & \hfuzz=500pt  & \hfuzz=500pt \\
\hfuzz=500pt\includegraphics[width=1em]{connector.pdf}\includegraphics[width=1em]{element-mustset.pdf}~rangeRate & \hfuzz=500pt  & \hfuzz=500pt \\
\hfuzz=500pt\includegraphics[width=1em]{connector.pdf}\includegraphics[width=1em]{element-mustset.pdf}~none & \hfuzz=500pt  & \hfuzz=500pt \\
\hfuzz=500pt\includegraphics[width=1em]{element-mustset.pdf}~inputfileVariational1 & \hfuzz=500pt filename & \hfuzz=500pt approximate position and integrated state matrix\\
\hfuzz=500pt\includegraphics[width=1em]{element-mustset.pdf}~inputfileVariational2 & \hfuzz=500pt filename & \hfuzz=500pt approximate position and integrated state matrix\\
\hfuzz=500pt\includegraphics[width=1em]{element.pdf}~ephemerides & \hfuzz=500pt \hyperref[ephemeridesType]{ephemerides} & \hfuzz=500pt \\
\hfuzz=500pt\includegraphics[width=1em]{element-unbounded.pdf}~parametrizationGravity & \hfuzz=500pt \hyperref[parametrizationGravityType]{parametrizationGravity} & \hfuzz=500pt gravity field parametrization\\
\hfuzz=500pt\includegraphics[width=1em]{element-unbounded.pdf}~parametrizationAcceleration1 & \hfuzz=500pt \hyperref[parametrizationAccelerationType]{parametrizationAcceleration} & \hfuzz=500pt orbit1 force parameters\\
\hfuzz=500pt\includegraphics[width=1em]{element-unbounded.pdf}~parametrizationAcceleration2 & \hfuzz=500pt \hyperref[parametrizationAccelerationType]{parametrizationAcceleration} & \hfuzz=500pt orbit2 force parameters\\
\hfuzz=500pt\includegraphics[width=1em]{element-unbounded.pdf}~parametrizationSst & \hfuzz=500pt \hyperref[parametrizationSatelliteTrackingType]{parametrizationSatelliteTracking} & \hfuzz=500pt satellite tracking parameter\\
\hfuzz=500pt\includegraphics[width=1em]{element.pdf}~integrationDegree & \hfuzz=500pt uint & \hfuzz=500pt integration of forces by polynomial approximation of degree n\\
\hfuzz=500pt\includegraphics[width=1em]{element.pdf}~interpolationDegree & \hfuzz=500pt uint & \hfuzz=500pt orbit interpolation by polynomial approximation of degree n\\
\hfuzz=500pt\includegraphics[width=1em]{element-mustset.pdf}~covarianceSst & \hfuzz=500pt \hyperref[covarianceSstType]{covarianceSst} & \hfuzz=500pt covariance matrix of satellite to satellite tracking observations\\
\hfuzz=500pt\includegraphics[width=1em]{element-mustset.pdf}~covariancePod1 & \hfuzz=500pt \hyperref[covariancePodType]{covariancePod} & \hfuzz=500pt covariance matrix of kinematic orbits (satellite 1)\\
\hfuzz=500pt\includegraphics[width=1em]{element-mustset.pdf}~covariancePod2 & \hfuzz=500pt \hyperref[covariancePodType]{covariancePod} & \hfuzz=500pt covariance matrix of kinematic orbits (satellite 2)\\
\hfuzz=500pt\includegraphics[width=1em]{element.pdf}~iterationCount & \hfuzz=500pt uint & \hfuzz=500pt for the estimation of calibration parameter and error PSD\\
\hline
\end{tabularx}

This program is \reference{parallelized}{general.parallelization}.
\clearpage
%==================================
\section{Programs: Simulation}
\subsection{NoiseAccelerometer}\label{NoiseAccelerometer}
This program adds noise and biases to simulated \file{accelerometer data}{instrument}
generated by \program{SimulateAccelerometer}.
See \configClass{noiseGenerator}{noiseGeneratorType} for details on noise generation.


\keepXColumns
\begin{tabularx}{\textwidth}{N T A}
\hline
Name & Type & Annotation\\
\hline
\hfuzz=500pt\includegraphics[width=1em]{element-mustset.pdf}~outputfileAccelerometer & \hfuzz=500pt filename & \hfuzz=500pt \\
\hfuzz=500pt\includegraphics[width=1em]{element-mustset.pdf}~inputfileAccelerometer & \hfuzz=500pt filename & \hfuzz=500pt \\
\hfuzz=500pt\includegraphics[width=1em]{element.pdf}~biasAlong & \hfuzz=500pt double & \hfuzz=500pt [m/s**2]\\
\hfuzz=500pt\includegraphics[width=1em]{element.pdf}~biasCross & \hfuzz=500pt double & \hfuzz=500pt [m/s**2]\\
\hfuzz=500pt\includegraphics[width=1em]{element.pdf}~biasRadial & \hfuzz=500pt double & \hfuzz=500pt [m/s**2]\\
\hfuzz=500pt\includegraphics[width=1em]{element-unbounded.pdf}~noiseAlong & \hfuzz=500pt \hyperref[noiseGeneratorType]{noiseGenerator} & \hfuzz=500pt [m/s**2]\\
\hfuzz=500pt\includegraphics[width=1em]{element-unbounded.pdf}~noiseCross & \hfuzz=500pt \hyperref[noiseGeneratorType]{noiseGenerator} & \hfuzz=500pt [m/s**2]\\
\hfuzz=500pt\includegraphics[width=1em]{element-unbounded.pdf}~noiseRadial & \hfuzz=500pt \hyperref[noiseGeneratorType]{noiseGenerator} & \hfuzz=500pt [m/s**2]\\
\hline
\end{tabularx}

This program is \reference{parallelized}{general.parallelization}.
\clearpage
%==================================
\subsection{NoiseGriddedData}\label{NoiseGriddedData}
This program adds noise to \file{gridded data data}{griddedData}.
See \configClass{noiseGenerator}{noiseGeneratorType} for details on noise generation.


\keepXColumns
\begin{tabularx}{\textwidth}{N T A}
\hline
Name & Type & Annotation\\
\hline
\hfuzz=500pt\includegraphics[width=1em]{element-mustset.pdf}~outputfileGriddedData & \hfuzz=500pt filename & \hfuzz=500pt \\
\hfuzz=500pt\includegraphics[width=1em]{element-mustset.pdf}~inputfileGriddedData & \hfuzz=500pt filename & \hfuzz=500pt \\
\hfuzz=500pt\includegraphics[width=1em]{element-mustset-unbounded.pdf}~noise & \hfuzz=500pt \hyperref[noiseGeneratorType]{noiseGenerator} & \hfuzz=500pt \\
\hfuzz=500pt\includegraphics[width=1em]{element.pdf}~startDataFields & \hfuzz=500pt uint & \hfuzz=500pt start\\
\hfuzz=500pt\includegraphics[width=1em]{element.pdf}~countDataFields & \hfuzz=500pt uint & \hfuzz=500pt number of data fields (default: all after start)\\
\hline
\end{tabularx}

\clearpage
%==================================
\subsection{NoiseInstrument}\label{NoiseInstrument}
This program adds noise to \file{instrument data}{instrument}.
See \configClass{noiseGenerator}{noiseGeneratorType} for details on noise generation.


\keepXColumns
\begin{tabularx}{\textwidth}{N T A}
\hline
Name & Type & Annotation\\
\hline
\hfuzz=500pt\includegraphics[width=1em]{element-mustset.pdf}~outputfileInstrument & \hfuzz=500pt filename & \hfuzz=500pt \\
\hfuzz=500pt\includegraphics[width=1em]{element-mustset.pdf}~inputfileInstrument & \hfuzz=500pt filename & \hfuzz=500pt \\
\hfuzz=500pt\includegraphics[width=1em]{element-mustset-unbounded.pdf}~noise & \hfuzz=500pt \hyperref[noiseGeneratorType]{noiseGenerator} & \hfuzz=500pt \\
\hfuzz=500pt\includegraphics[width=1em]{element.pdf}~startDataFields & \hfuzz=500pt uint & \hfuzz=500pt start\\
\hfuzz=500pt\includegraphics[width=1em]{element.pdf}~countDataFields & \hfuzz=500pt uint & \hfuzz=500pt number of data fields (default: all after start)\\
\hline
\end{tabularx}

This program is \reference{parallelized}{general.parallelization}.
\clearpage
%==================================
\subsection{NoiseNormalsSolution}\label{NoiseNormalsSolution}
The inverse of the normal matrix of \configFile{inputfileNormalEquation}{normalEquation}
represents the covariance matrix of the estimated parameters. This program generates a noise vector with
\begin{equation}
\M\Sigma(\M e) = \M N^{-1},
\end{equation}
if generated input noise is standard white noise.

The noise vector is computed with
\begin{equation}
\M e = \M W^{-T} \M z,
\end{equation}
where $\M z$ is the generated \configClass{noise}{noiseGeneratorType} and
$\M W$ is the cholesky upper triangle matrix of the normal matrix $\M N=\M W^T\M W$.


\keepXColumns
\begin{tabularx}{\textwidth}{N T A}
\hline
Name & Type & Annotation\\
\hline
\hfuzz=500pt\includegraphics[width=1em]{element-mustset.pdf}~outputfileNoise & \hfuzz=500pt filename & \hfuzz=500pt generated noise as matrix: parameterCount x sampleCount\\
\hfuzz=500pt\includegraphics[width=1em]{element-mustset.pdf}~inputfileNormalEquation & \hfuzz=500pt filename & \hfuzz=500pt \\
\hfuzz=500pt\includegraphics[width=1em]{element-mustset-unbounded.pdf}~noise & \hfuzz=500pt \hyperref[noiseGeneratorType]{noiseGenerator} & \hfuzz=500pt \\
\hfuzz=500pt\includegraphics[width=1em]{element.pdf}~sampleCount & \hfuzz=500pt uint & \hfuzz=500pt number of samples to be generated\\
\hfuzz=500pt\includegraphics[width=1em]{element.pdf}~useEigenvalueDecomposition & \hfuzz=500pt boolean & \hfuzz=500pt use eigenvalue decomposition\\
\hline
\end{tabularx}

This program is \reference{parallelized}{general.parallelization}.
\clearpage
%==================================
\subsection{NoiseOrbit}\label{NoiseOrbit}
This program adds noise to simulated \file{satellite}{instrument}'s positions
and velocities generated by \program{SimulateOrbit} (along, cross, radial).
See \configClass{noiseGenerator}{noiseGeneratorType} for details on noise options.


\keepXColumns
\begin{tabularx}{\textwidth}{N T A}
\hline
Name & Type & Annotation\\
\hline
\hfuzz=500pt\includegraphics[width=1em]{element-mustset.pdf}~outputfileOrbit & \hfuzz=500pt filename & \hfuzz=500pt \\
\hfuzz=500pt\includegraphics[width=1em]{element-mustset.pdf}~inputfileOrbit & \hfuzz=500pt filename & \hfuzz=500pt \\
\hfuzz=500pt\includegraphics[width=1em]{element-unbounded.pdf}~noisePosition & \hfuzz=500pt \hyperref[noiseGeneratorType]{noiseGenerator} & \hfuzz=500pt along, cross, radial [m]\\
\hfuzz=500pt\includegraphics[width=1em]{element-unbounded.pdf}~noiseVelocity & \hfuzz=500pt \hyperref[noiseGeneratorType]{noiseGenerator} & \hfuzz=500pt along, cross, radial [m/s]\\
\hline
\end{tabularx}

This program is \reference{parallelized}{general.parallelization}.
\clearpage
%==================================
\subsection{NoiseSatelliteTracking}\label{NoiseSatelliteTracking}
This program adds noise to simulated satellite tracking data generated by \program{SimulateSatelliteTracking}.
See \configClass{noiseGenerator}{noiseGeneratorType} for details on noise generation.


\keepXColumns
\begin{tabularx}{\textwidth}{N T A}
\hline
Name & Type & Annotation\\
\hline
\hfuzz=500pt\includegraphics[width=1em]{element-mustset.pdf}~outputfileSatelliteTracking & \hfuzz=500pt filename & \hfuzz=500pt \\
\hfuzz=500pt\includegraphics[width=1em]{element-mustset.pdf}~inputfileSatelliteTracking & \hfuzz=500pt filename & \hfuzz=500pt \\
\hfuzz=500pt\includegraphics[width=1em]{element-unbounded.pdf}~noiseRange & \hfuzz=500pt \hyperref[noiseGeneratorType]{noiseGenerator} & \hfuzz=500pt [m]\\
\hfuzz=500pt\includegraphics[width=1em]{element-unbounded.pdf}~noiseRangeRate & \hfuzz=500pt \hyperref[noiseGeneratorType]{noiseGenerator} & \hfuzz=500pt [m/s]\\
\hfuzz=500pt\includegraphics[width=1em]{element-unbounded.pdf}~noiseRangeAcceleration & \hfuzz=500pt \hyperref[noiseGeneratorType]{noiseGenerator} & \hfuzz=500pt [m/s\textasciicircum{}2]\\
\hline
\end{tabularx}

This program is \reference{parallelized}{general.parallelization}.
\clearpage
%==================================
\subsection{NoiseStarCamera}\label{NoiseStarCamera}
This program adds noise to rotation observations. The noise is computed via a pseudo random sequence.
See \configClass{noiseGenerator}{noiseGeneratorType} for details on noise options.


\keepXColumns
\begin{tabularx}{\textwidth}{N T A}
\hline
Name & Type & Annotation\\
\hline
\hfuzz=500pt\includegraphics[width=1em]{element-mustset.pdf}~outputfileStarCamera & \hfuzz=500pt filename & \hfuzz=500pt \\
\hfuzz=500pt\includegraphics[width=1em]{element-mustset.pdf}~inputfileStarCamera & \hfuzz=500pt filename & \hfuzz=500pt \\
\hfuzz=500pt\includegraphics[width=1em]{element-mustset-unbounded.pdf}~noiseRoll & \hfuzz=500pt \hyperref[noiseGeneratorType]{noiseGenerator} & \hfuzz=500pt [rad]\\
\hfuzz=500pt\includegraphics[width=1em]{element-mustset-unbounded.pdf}~noisePitch & \hfuzz=500pt \hyperref[noiseGeneratorType]{noiseGenerator} & \hfuzz=500pt [rad]\\
\hfuzz=500pt\includegraphics[width=1em]{element-mustset-unbounded.pdf}~noiseYaw & \hfuzz=500pt \hyperref[noiseGeneratorType]{noiseGenerator} & \hfuzz=500pt [rad]\\
\hline
\end{tabularx}

This program is \reference{parallelized}{general.parallelization}.
\clearpage
%==================================
\subsection{NoiseTimeSeries}\label{NoiseTimeSeries}
This program generates \configFile{outputfileNoise}{instrument} with the requested characteristics.
See \configClass{noiseGenerator}{noiseGeneratorType} for details on noise options.


\keepXColumns
\begin{tabularx}{\textwidth}{N T A}
\hline
Name & Type & Annotation\\
\hline
\hfuzz=500pt\includegraphics[width=1em]{element-mustset.pdf}~outputfileNoise & \hfuzz=500pt filename & \hfuzz=500pt \\
\hfuzz=500pt\includegraphics[width=1em]{element.pdf}~outputfileCovarianceFunction & \hfuzz=500pt filename & \hfuzz=500pt \\
\hfuzz=500pt\includegraphics[width=1em]{element-mustset-unbounded.pdf}~noise & \hfuzz=500pt \hyperref[noiseGeneratorType]{noiseGenerator} & \hfuzz=500pt \\
\hfuzz=500pt\includegraphics[width=1em]{element-mustset-unbounded.pdf}~timeSeries & \hfuzz=500pt \hyperref[timeSeriesType]{timeSeries} & \hfuzz=500pt \\
\hfuzz=500pt\includegraphics[width=1em]{element-mustset.pdf}~columns & \hfuzz=500pt uint & \hfuzz=500pt number of noise series (columns)\\
\hline
\end{tabularx}

\clearpage
%==================================
\subsection{SimulateAccelerometer}\label{SimulateAccelerometer}
This program simulate \file{accelerometer data}{instrument}. The orientation of the accelerometer
is given by \configFile{inputfileStarCamera}{instrument} otherwise the celestial reference frame (CRF) is used.
For computation of non-conservative forces a \configFile{satelliteModel}{satelliteModel} is needed.


\keepXColumns
\begin{tabularx}{\textwidth}{N T A}
\hline
Name & Type & Annotation\\
\hline
\hfuzz=500pt\includegraphics[width=1em]{element-mustset.pdf}~outputfileAccelerometer & \hfuzz=500pt filename & \hfuzz=500pt \\
\hfuzz=500pt\includegraphics[width=1em]{element.pdf}~inputfileSatelliteModel & \hfuzz=500pt filename & \hfuzz=500pt satellite macro model\\
\hfuzz=500pt\includegraphics[width=1em]{element-mustset.pdf}~inputfileOrbit & \hfuzz=500pt filename & \hfuzz=500pt \\
\hfuzz=500pt\includegraphics[width=1em]{element.pdf}~inputfileStarCamera & \hfuzz=500pt filename & \hfuzz=500pt \\
\hfuzz=500pt\includegraphics[width=1em]{element-mustset.pdf}~earthRotation & \hfuzz=500pt \hyperref[earthRotationType]{earthRotation} & \hfuzz=500pt \\
\hfuzz=500pt\includegraphics[width=1em]{element.pdf}~ephemerides & \hfuzz=500pt \hyperref[ephemeridesType]{ephemerides} & \hfuzz=500pt \\
\hfuzz=500pt\includegraphics[width=1em]{element-mustset.pdf}~forces & \hfuzz=500pt \hyperref[forcesType]{forces} & \hfuzz=500pt \\
\hline
\end{tabularx}

This program is \reference{parallelized}{general.parallelization}.
\clearpage
%==================================
\subsection{SimulateAccelerometerCoMOffset}\label{SimulateAccelerometerCoMOffset}
This program generates an \file{accelerometer file}{instrument} containing perturbing accelerations
due to a given center of mass (CoM) offset. This includes centrifugal effects,
Euler forces and the effect of gravity gradients.


\keepXColumns
\begin{tabularx}{\textwidth}{N T A}
\hline
Name & Type & Annotation\\
\hline
\hfuzz=500pt\includegraphics[width=1em]{element-mustset.pdf}~outputfileAccelerometer & \hfuzz=500pt filename & \hfuzz=500pt effect of offset\\
\hfuzz=500pt\includegraphics[width=1em]{element-mustset.pdf}~inputfileOrbit & \hfuzz=500pt filename & \hfuzz=500pt \\
\hfuzz=500pt\includegraphics[width=1em]{element-mustset.pdf}~inputfileStarCamera & \hfuzz=500pt filename & \hfuzz=500pt \\
\hfuzz=500pt\includegraphics[width=1em]{element.pdf}~applyAngularRate & \hfuzz=500pt boolean & \hfuzz=500pt compute effect of centrifugal forces\\
\hfuzz=500pt\includegraphics[width=1em]{element.pdf}~applyAngularAccelerations & \hfuzz=500pt boolean & \hfuzz=500pt compute effect of Euler forces\\
\hfuzz=500pt\includegraphics[width=1em]{element-unbounded.pdf}~gradientfield & \hfuzz=500pt \hyperref[gravityfieldType]{gravityfield} & \hfuzz=500pt low order field to estimate the change of the gravity by position adjustement\\
\hfuzz=500pt\includegraphics[width=1em]{element-mustset.pdf}~earthRotation & \hfuzz=500pt \hyperref[earthRotationType]{earthRotation} & \hfuzz=500pt \\
\hfuzz=500pt\includegraphics[width=1em]{element.pdf}~interpolationDegree & \hfuzz=500pt uint & \hfuzz=500pt derivation of quaternions by polynomial interpolation of degree n\\
\hfuzz=500pt\includegraphics[width=1em]{element-mustset.pdf}~CoMOffsetX & \hfuzz=500pt double & \hfuzz=500pt offset [m]\\
\hfuzz=500pt\includegraphics[width=1em]{element-mustset.pdf}~CoMOffsetY & \hfuzz=500pt double & \hfuzz=500pt offset [m]\\
\hfuzz=500pt\includegraphics[width=1em]{element-mustset.pdf}~CoMOffsetZ & \hfuzz=500pt double & \hfuzz=500pt offset [m]\\
\hline
\end{tabularx}

This program is \reference{parallelized}{general.parallelization}.
\clearpage
%==================================
\subsection{SimulateGradiometer}\label{SimulateGradiometer}
This program simulates error free \file{gradiometer data}{instrument} along a satellite's orbit.
The orientation of the full tensor gradiometer is given by \configFile{inputfileStarCamera}{instrument}
otherwise the celestial reference frame (CRF) is used.
The gravity gradients are given by \configClass{gravityfield}{gravityfieldType} and
\configClass{tides}{tidesType}.


\keepXColumns
\begin{tabularx}{\textwidth}{N T A}
\hline
Name & Type & Annotation\\
\hline
\hfuzz=500pt\includegraphics[width=1em]{element-mustset.pdf}~outputfileGradiometer & \hfuzz=500pt filename & \hfuzz=500pt \\
\hfuzz=500pt\includegraphics[width=1em]{element-mustset.pdf}~inputfileOrbit & \hfuzz=500pt filename & \hfuzz=500pt \\
\hfuzz=500pt\includegraphics[width=1em]{element.pdf}~inputfileStarCamera & \hfuzz=500pt filename & \hfuzz=500pt \\
\hfuzz=500pt\includegraphics[width=1em]{element-mustset.pdf}~earthRotation & \hfuzz=500pt \hyperref[earthRotationType]{earthRotation} & \hfuzz=500pt \\
\hfuzz=500pt\includegraphics[width=1em]{element.pdf}~ephemerides & \hfuzz=500pt \hyperref[ephemeridesType]{ephemerides} & \hfuzz=500pt \\
\hfuzz=500pt\includegraphics[width=1em]{element-unbounded.pdf}~gravityfield & \hfuzz=500pt \hyperref[gravityfieldType]{gravityfield} & \hfuzz=500pt \\
\hfuzz=500pt\includegraphics[width=1em]{element-unbounded.pdf}~tides & \hfuzz=500pt \hyperref[tidesType]{tides} & \hfuzz=500pt \\
\hline
\end{tabularx}

This program is \reference{parallelized}{general.parallelization}.
\clearpage
%==================================
\subsection{SimulateKeplerOrbit}\label{SimulateKeplerOrbit}
This program simulates a Keplerian \file{orbit}{instrument} at a given \config{timeSeries}
starting from the given \config{integrationConstants}.


\keepXColumns
\begin{tabularx}{\textwidth}{N T A}
\hline
Name & Type & Annotation\\
\hline
\hfuzz=500pt\includegraphics[width=1em]{element-mustset.pdf}~outputfileOrbit & \hfuzz=500pt filename & \hfuzz=500pt \\
\hfuzz=500pt\includegraphics[width=1em]{element-mustset-unbounded.pdf}~timeSeries & \hfuzz=500pt \hyperref[timeSeriesType]{timeSeries} & \hfuzz=500pt \\
\hfuzz=500pt\includegraphics[width=1em]{element.pdf}~GM & \hfuzz=500pt double & \hfuzz=500pt Geocentric gravitational constant\\
\hfuzz=500pt\includegraphics[width=1em]{element-mustset.pdf}~integrationConstants & \hfuzz=500pt choice & \hfuzz=500pt \\
\hfuzz=500pt\includegraphics[width=1em]{connector.pdf}\includegraphics[width=1em]{element-mustset.pdf}~kepler & \hfuzz=500pt sequence & \hfuzz=500pt \\
\hfuzz=500pt\quad\includegraphics[width=1em]{connector.pdf}\includegraphics[width=1em]{element-mustset.pdf}~majorAxis & \hfuzz=500pt double & \hfuzz=500pt [m]\\
\hfuzz=500pt\quad\includegraphics[width=1em]{connector.pdf}\includegraphics[width=1em]{element-mustset.pdf}~eccentricity & \hfuzz=500pt double & \hfuzz=500pt [-]\\
\hfuzz=500pt\quad\includegraphics[width=1em]{connector.pdf}\includegraphics[width=1em]{element-mustset.pdf}~inclination & \hfuzz=500pt angle & \hfuzz=500pt [degree]\\
\hfuzz=500pt\quad\includegraphics[width=1em]{connector.pdf}\includegraphics[width=1em]{element-mustset.pdf}~ascendingNode & \hfuzz=500pt angle & \hfuzz=500pt [degree]\\
\hfuzz=500pt\quad\includegraphics[width=1em]{connector.pdf}\includegraphics[width=1em]{element-mustset.pdf}~argumentOfPerigee & \hfuzz=500pt angle & \hfuzz=500pt [degree]\\
\hfuzz=500pt\quad\includegraphics[width=1em]{connector.pdf}\includegraphics[width=1em]{element-mustset.pdf}~meanAnomaly & \hfuzz=500pt angle & \hfuzz=500pt [degree]\\
\hfuzz=500pt\quad\includegraphics[width=1em]{connector.pdf}\includegraphics[width=1em]{element-mustset.pdf}~time & \hfuzz=500pt time & \hfuzz=500pt integration constants are valid at this epoch\\
\hfuzz=500pt\includegraphics[width=1em]{connector.pdf}\includegraphics[width=1em]{element-mustset.pdf}~positionAndVelocity & \hfuzz=500pt sequence & \hfuzz=500pt \\
\hfuzz=500pt\quad\includegraphics[width=1em]{connector.pdf}\includegraphics[width=1em]{element-mustset.pdf}~position0x & \hfuzz=500pt double & \hfuzz=500pt [m] in CRF\\
\hfuzz=500pt\quad\includegraphics[width=1em]{connector.pdf}\includegraphics[width=1em]{element-mustset.pdf}~position0y & \hfuzz=500pt double & \hfuzz=500pt [m] in CRF\\
\hfuzz=500pt\quad\includegraphics[width=1em]{connector.pdf}\includegraphics[width=1em]{element-mustset.pdf}~position0z & \hfuzz=500pt double & \hfuzz=500pt [m] in CRF\\
\hfuzz=500pt\quad\includegraphics[width=1em]{connector.pdf}\includegraphics[width=1em]{element-mustset.pdf}~velocity0x & \hfuzz=500pt double & \hfuzz=500pt [m/s]\\
\hfuzz=500pt\quad\includegraphics[width=1em]{connector.pdf}\includegraphics[width=1em]{element-mustset.pdf}~velocity0y & \hfuzz=500pt double & \hfuzz=500pt [m/s]\\
\hfuzz=500pt\quad\includegraphics[width=1em]{connector.pdf}\includegraphics[width=1em]{element-mustset.pdf}~velocity0z & \hfuzz=500pt double & \hfuzz=500pt [m/s]\\
\hfuzz=500pt\quad\includegraphics[width=1em]{connector.pdf}\includegraphics[width=1em]{element-mustset.pdf}~time & \hfuzz=500pt time & \hfuzz=500pt integration constants are valid at this epoch\\
\hline
\end{tabularx}

\clearpage
%==================================
\subsection{SimulateOrbit}\label{SimulateOrbit}
This program integrates an \file{orbit}{instrument} from a given force function (dynamic orbit).
The force functions are given by \configClass{forces}{forcesType}.
For computation of non-conservative forces a \file{satelliteModel}{satelliteModel} is needed.
The integration method must be selected with \configClass{propagator}{orbitPropagatorType}.
Because the orbit data are calculated in the celestial reference frame (CRF) you need
\configClass{earthRotation}{earthRotationType} to transform the force function
from the terrestrial reference frame (TRF).
The integration start and end time, as well as the sampling, are derived from
the \config{timeSeries} option. It is possible to integrate the arc in \config{reverse},
where the initial conditions are assumed to be met at the end time of the \config{timeSeries}.


\keepXColumns
\begin{tabularx}{\textwidth}{N T A}
\hline
Name & Type & Annotation\\
\hline
\hfuzz=500pt\includegraphics[width=1em]{element-mustset.pdf}~outputfileOrbit & \hfuzz=500pt filename & \hfuzz=500pt orbit file to be written.\\
\hfuzz=500pt\includegraphics[width=1em]{element.pdf}~inputfileSatelliteModel & \hfuzz=500pt filename & \hfuzz=500pt satellite macro model\\
\hfuzz=500pt\includegraphics[width=1em]{element-mustset-unbounded.pdf}~timeSeries & \hfuzz=500pt \hyperref[timeSeriesType]{timeSeries} & \hfuzz=500pt time points for simulated orbit epochs.\\
\hfuzz=500pt\includegraphics[width=1em]{element-mustset.pdf}~integrationConstants & \hfuzz=500pt choice & \hfuzz=500pt \\
\hfuzz=500pt\includegraphics[width=1em]{connector.pdf}\includegraphics[width=1em]{element-mustset.pdf}~kepler & \hfuzz=500pt sequence & \hfuzz=500pt \\
\hfuzz=500pt\quad\includegraphics[width=1em]{connector.pdf}\includegraphics[width=1em]{element-mustset.pdf}~majorAxis & \hfuzz=500pt double & \hfuzz=500pt [m]\\
\hfuzz=500pt\quad\includegraphics[width=1em]{connector.pdf}\includegraphics[width=1em]{element-mustset.pdf}~eccentricity & \hfuzz=500pt double & \hfuzz=500pt [-]\\
\hfuzz=500pt\quad\includegraphics[width=1em]{connector.pdf}\includegraphics[width=1em]{element-mustset.pdf}~inclination & \hfuzz=500pt angle & \hfuzz=500pt [degree]\\
\hfuzz=500pt\quad\includegraphics[width=1em]{connector.pdf}\includegraphics[width=1em]{element-mustset.pdf}~ascendingNode & \hfuzz=500pt angle & \hfuzz=500pt [degree]\\
\hfuzz=500pt\quad\includegraphics[width=1em]{connector.pdf}\includegraphics[width=1em]{element-mustset.pdf}~argumentOfPerigee & \hfuzz=500pt angle & \hfuzz=500pt [degree]\\
\hfuzz=500pt\quad\includegraphics[width=1em]{connector.pdf}\includegraphics[width=1em]{element-mustset.pdf}~meanAnomaly & \hfuzz=500pt angle & \hfuzz=500pt [degree]\\
\hfuzz=500pt\quad\includegraphics[width=1em]{connector.pdf}\includegraphics[width=1em]{element.pdf}~GM & \hfuzz=500pt double & \hfuzz=500pt Geocentric gravitational constant\\
\hfuzz=500pt\includegraphics[width=1em]{connector.pdf}\includegraphics[width=1em]{element-mustset.pdf}~positionAndVelocity & \hfuzz=500pt sequence & \hfuzz=500pt \\
\hfuzz=500pt\quad\includegraphics[width=1em]{connector.pdf}\includegraphics[width=1em]{element-mustset.pdf}~position0x & \hfuzz=500pt double & \hfuzz=500pt [m] in CRF\\
\hfuzz=500pt\quad\includegraphics[width=1em]{connector.pdf}\includegraphics[width=1em]{element-mustset.pdf}~position0y & \hfuzz=500pt double & \hfuzz=500pt [m] in CRF\\
\hfuzz=500pt\quad\includegraphics[width=1em]{connector.pdf}\includegraphics[width=1em]{element-mustset.pdf}~position0z & \hfuzz=500pt double & \hfuzz=500pt [m] in CRF\\
\hfuzz=500pt\quad\includegraphics[width=1em]{connector.pdf}\includegraphics[width=1em]{element-mustset.pdf}~velocity0x & \hfuzz=500pt double & \hfuzz=500pt [m/s]\\
\hfuzz=500pt\quad\includegraphics[width=1em]{connector.pdf}\includegraphics[width=1em]{element-mustset.pdf}~velocity0y & \hfuzz=500pt double & \hfuzz=500pt [m/s]\\
\hfuzz=500pt\quad\includegraphics[width=1em]{connector.pdf}\includegraphics[width=1em]{element-mustset.pdf}~velocity0z & \hfuzz=500pt double & \hfuzz=500pt [m/s]\\
\hfuzz=500pt\includegraphics[width=1em]{connector.pdf}\includegraphics[width=1em]{element-mustset.pdf}~file & \hfuzz=500pt sequence & \hfuzz=500pt \\
\hfuzz=500pt\quad\includegraphics[width=1em]{connector.pdf}\includegraphics[width=1em]{element-mustset.pdf}~inputfileOrbit & \hfuzz=500pt filename & \hfuzz=500pt only epoch at timeStart is used\\
\hfuzz=500pt\quad\includegraphics[width=1em]{connector.pdf}\includegraphics[width=1em]{element.pdf}~margin & \hfuzz=500pt double & \hfuzz=500pt [seconds] used when finding initial epoch in orbitFile\\
\hfuzz=500pt\includegraphics[width=1em]{element-mustset.pdf}~propagator & \hfuzz=500pt \hyperref[orbitPropagatorType]{orbitPropagator} & \hfuzz=500pt orbit propagation method.\\
\hfuzz=500pt\includegraphics[width=1em]{element-mustset.pdf}~earthRotation & \hfuzz=500pt \hyperref[earthRotationType]{earthRotation} & \hfuzz=500pt \\
\hfuzz=500pt\includegraphics[width=1em]{element.pdf}~ephemerides & \hfuzz=500pt \hyperref[ephemeridesType]{ephemerides} & \hfuzz=500pt \\
\hfuzz=500pt\includegraphics[width=1em]{element-mustset.pdf}~forces & \hfuzz=500pt \hyperref[forcesType]{forces} & \hfuzz=500pt considered in orbit propagation.\\
\hfuzz=500pt\includegraphics[width=1em]{element.pdf}~reverse & \hfuzz=500pt boolean & \hfuzz=500pt start integration at last epoch in timeSeries, going backward in time.\\
\hline
\end{tabularx}

\clearpage
%==================================
\subsection{SimulateSatelliteTracking}\label{SimulateSatelliteTracking}
This program simulates \file{tracking data}{instrument} (range, range-rate, range-accelerations)
between 2 satellites. The range is given by
\begin{equation}
\rho(t) = \left\lVert{\M r_B(t) - \M r_A(t)}\right\rVert = \M e_{AB}(t)\cdot\M r_{AB}(t),
\end{equation}
with $\M r_{AB} = \M r_B - \M r_A$ and the unit vector in line of sight (LOS) direction
\begin{equation}\label{sst.los}
\M e_{AB} = \frac{\M r_{AB}}{\left\lVert{\M r_{AB}}\right\rVert}=\frac{\M r_{AB}}{\rho}.
\end{equation}
Range-rates~$\dot{\rho}$ and range accelrations~$\ddot{\rho}$ are obtained by differentation
\begin{equation}\label{obsRangeRate}
\dot{\rho}  = \M e_{AB}\cdot\dot{\M r}_{AB} + \dot{\M e}_{AB}\cdot\M r_{AB}
            = \M e_{AB}\cdot\dot{\M r}_{AB},
\end{equation}
\begin{equation}\label{obsRangeAccl}
\begin{split}
\ddot{\rho} &= \M e_{AB}\cdot\ddot{\M r}_{AB} +\dot{\M e}_{AB}\cdot\dot{\M r}_{AB}
            = \M e_{AB}\cdot\ddot{\M r}_{AB} +
   \frac{1}{\rho}\left(\dot{\M r}_{AB}^2-\dot{\rho}^2\right). \\
\end{split}
\end{equation}
with the derivative of the unit vector
\begin{equation}
\dot{\M e}_{AB}=\frac{d}{dt}\left(\frac{\M r_{AB}}{\rho}\right)
=\frac{\dot{\M r}_{AB}}{\rho}-\frac{\dot{\rho}\cdot\M r_{AB}}{\rho^2}
=\frac{1}{\rho}\left({\dot{\M r}_{AB}-\dot{\rho}\cdot\M e_{AB}}\right).
\end{equation}
The \configFile{inputfileOrbit}{instrument}s must contain positions, velocities, and acceleration
(see \program{OrbitAddVelocityAndAcceleration}).


\keepXColumns
\begin{tabularx}{\textwidth}{N T A}
\hline
Name & Type & Annotation\\
\hline
\hfuzz=500pt\includegraphics[width=1em]{element-mustset.pdf}~outputfileSatelliteTracking & \hfuzz=500pt filename & \hfuzz=500pt \\
\hfuzz=500pt\includegraphics[width=1em]{element-mustset.pdf}~inputfileOrbit1 & \hfuzz=500pt filename & \hfuzz=500pt \\
\hfuzz=500pt\includegraphics[width=1em]{element-mustset.pdf}~inputfileOrbit2 & \hfuzz=500pt filename & \hfuzz=500pt \\
\hline
\end{tabularx}

This program is \reference{parallelized}{general.parallelization}.
\clearpage
%==================================
\subsection{SimulateStarCamera}\label{SimulateStarCamera}
This program simulates \file{star camera}{instrument} measurements at each satellite's position.
The orientation is simulated to be x-axis in along track (along velocity),
y-axis is cross track (normal to position and velocity vector)
and z-axis forms a right hand system (not exact radial).
The resulting rotation matrices rotate from satellite frame to inertial frame.


\keepXColumns
\begin{tabularx}{\textwidth}{N T A}
\hline
Name & Type & Annotation\\
\hline
\hfuzz=500pt\includegraphics[width=1em]{element-mustset.pdf}~outputfileStarCamera & \hfuzz=500pt filename & \hfuzz=500pt \\
\hfuzz=500pt\includegraphics[width=1em]{element-mustset.pdf}~inputfileOrbit & \hfuzz=500pt filename & \hfuzz=500pt position and velocity defines the orientation of the satellite at each epoch\\
\hline
\end{tabularx}

This program is \reference{parallelized}{general.parallelization}.
\clearpage
%==================================
\subsection{SimulateStarCameraGnss}\label{SimulateStarCameraGnss}
This program simulates \file{star camera}{instrument} measurements at each satellite position
of \configFile{inputfileOrbit}{instrument}.
The resulting rotation matrices rotate from body frame to inertial frame. The body frame refers
to the IGS-specific (not the manufacturer-specific) body frame, as described by
\href{https://doi.org/10.1016/j.asr.2015.06.019}{Montenbruck et al. (2015)}.
The \configFile{inputfileOrbit}{instrument} must contain velocities
(use \program{OrbitAddVelocityAndAcceleration} if needed).

Information about the attitude mode(s) used by the GNSS satellite may be provided via
\configFile{inputfileAttitudeInfo}{instrument}. This file can be created with
\program{GnssAttitudeInfoCreate}. It contains one or more time-dependent entries,
each defining the default attitude mode, the attitude modes used around orbit noon and
midnight, and some parameters required by the various modes.
If no \configFile{inputfileAttitudeInfo}{instrument} is selected, the program defaults
to a nominal yaw-steering attitude model.
A sufficiently high \config{modelingResolution} ensures that the attitude behavior is modeled properly
at all times.

The attitude behavior is defined by the respective mode. Here is a list of the supported
modes with a brief explanation and references:
\begin{itemize}
\item \textbf{nominalYawSteering}:
      Yaw to keep solar panels aligned to Sun (e.g. most GNSS satellites outside eclipse) [1]
\item \textbf{orbitNormal}:
      Keep fixed yaw angle, for example point X-axis in flight direction (e.g. BDS-2G, BDS-3G, QZS-2G) [1]
\item \textbf{catchUpYawSteering}:
      Yaw at maximum yaw rate to catch up to nominal yaw angle (e.g. GPS-* (noon), GPS-IIR (midnight)) [2, 3]
\item \textbf{shadowMaxYawSteeringAndRecovery}:
      Yaw at maximum yaw rate from shadow start to end, recover after shadow (e.g. GPS-IIA (midnight)) [2]
\item \textbf{shadowMaxYawSteeringAndStop}:
      Yaw at maximum yaw rate from shadow start until nominal yaw angle at shadow end is reached,
      then stop (e.g. GLO-M (midnight)) [4]
\item \textbf{shadowConstantYawSteering}:
      Yaw at constant yaw rate from shadow start to end (e.g. GPS-IIF (midnight)) [3]
\item \textbf{centeredMaxYawSteering}:
      Yaw at maximum yaw rate centered around noon/midnight (e.g. QZS-2I, GLO-M (noon)) [4, 8]
\item \textbf{smoothedYawSteering1}:
      Yaw based on an auxiliary Sun vector for a smooth yaw maneuver (e.g. GAL-1) [5]
\item \textbf{smoothedYawSteering2}:
      Yaw based on a modified yaw-steering law for a smooth yaw maneuver (e.g. GAL-2, BDS-3M, BDS-3I) [5, 6]
\item \textbf{betaDependentOrbitNormal}:
      Switch to orbit normal mode if below beta angle threshold (e.g. BDS-2M, BDS-2I, QZS-1) [7, 8]
\end{itemize}

\fig{!hb}{0.9}{gnssAttitudeModes}{fig:gnssAttitudeModes1}{Overview of attitude modes used by GNSS satellites}

See \program{GnssAttitudeInfoCreate} for more details on which satellite uses which attitude modes
and the required parameters for each mode.

References for the attitude modes:
\begin{enumerate}
\item \href{https://doi.org/10.1016/j.asr.2015.06.019}{Montenbruck et al. (2015)}
\item \href{https://doi.org/10.1007/s10291-008-0092-1}{Kouba (2009)}
\item \href{https://doi.org/10.1007/s10291-016-0562-9}{Kuang et al. (2017)}
\item \href{https://doi.org/10.1016/j.asr.2010.09.007}{Dilssner et al. (2011)}
\item \url{https://www.gsc-europa.eu/support-to-developers/galileo-satellite-metadata#3}
\item \href{https://doi.org/10.1007/s10291-018-0783-1}{Wang et al. (2018)}
\item \href{https://doi.org/10.1017/S0373463318000103}{Li et al. (2018)}
\item \url{https://qzss.go.jp/en/technical/qzssinfo/index.html}
\end{enumerate}


\keepXColumns
\begin{tabularx}{\textwidth}{N T A}
\hline
Name & Type & Annotation\\
\hline
\hfuzz=500pt\includegraphics[width=1em]{element-mustset.pdf}~outputfileStarCamera & \hfuzz=500pt filename & \hfuzz=500pt rotation from body frame to CRF\\
\hfuzz=500pt\includegraphics[width=1em]{element-mustset.pdf}~inputfileOrbit & \hfuzz=500pt filename & \hfuzz=500pt attitude is modeled based on this orbit\\
\hfuzz=500pt\includegraphics[width=1em]{element.pdf}~inputfileAttitudeInfo & \hfuzz=500pt filename & \hfuzz=500pt attitude modes used by the satellite and respective parameters\\
\hfuzz=500pt\includegraphics[width=1em]{element.pdf}~interpolationDegree & \hfuzz=500pt uint & \hfuzz=500pt polynomial degree for orbit interpolation\\
\hfuzz=500pt\includegraphics[width=1em]{element.pdf}~modelingResolution & \hfuzz=500pt double & \hfuzz=500pt [s] resolution for attitude model evaluation\\
\hfuzz=500pt\includegraphics[width=1em]{element-mustset.pdf}~ephemerides & \hfuzz=500pt \hyperref[ephemeridesType]{ephemerides} & \hfuzz=500pt \\
\hfuzz=500pt\includegraphics[width=1em]{element-mustset.pdf}~eclipse & \hfuzz=500pt \hyperref[eclipseType]{eclipse} & \hfuzz=500pt model to determine if satellite is in Earth's shadow\\
\hline
\end{tabularx}

\clearpage
%==================================
\subsection{SimulateStarCameraGrace}\label{SimulateStarCameraGrace}
Simulates \file{star camera data}{instrument} of the two GRACE satellites.
\begin{itemize}
\item x: the antenna center pointing to the other satellite.
\item y: normal to line of sight and the radial direction.
\item z: forms a right handed system.
\end{itemize}


\keepXColumns
\begin{tabularx}{\textwidth}{N T A}
\hline
Name & Type & Annotation\\
\hline
\hfuzz=500pt\includegraphics[width=1em]{element-mustset.pdf}~outputfileStarCamera1 & \hfuzz=500pt filename & \hfuzz=500pt \\
\hfuzz=500pt\includegraphics[width=1em]{element-mustset.pdf}~outputfileStarCamera2 & \hfuzz=500pt filename & \hfuzz=500pt \\
\hfuzz=500pt\includegraphics[width=1em]{element-mustset.pdf}~inputfileOrbit1 & \hfuzz=500pt filename & \hfuzz=500pt position define the orientation of the satellite at each epoch\\
\hfuzz=500pt\includegraphics[width=1em]{element-mustset.pdf}~inputfileOrbit2 & \hfuzz=500pt filename & \hfuzz=500pt position define the orientation of the satellite at each epoch\\
\hfuzz=500pt\includegraphics[width=1em]{element.pdf}~antennaCenters & \hfuzz=500pt choice & \hfuzz=500pt KBR antenna phase center\\
\hfuzz=500pt\includegraphics[width=1em]{connector.pdf}\includegraphics[width=1em]{element-mustset.pdf}~value & \hfuzz=500pt sequence & \hfuzz=500pt \\
\hfuzz=500pt\quad\includegraphics[width=1em]{connector.pdf}\includegraphics[width=1em]{element.pdf}~center1X & \hfuzz=500pt double & \hfuzz=500pt x-coordinate of antenna position in SRF [m] for GRACEA\\
\hfuzz=500pt\quad\includegraphics[width=1em]{connector.pdf}\includegraphics[width=1em]{element.pdf}~center1Y & \hfuzz=500pt double & \hfuzz=500pt y-coordinate of antenna position in SRF [m] for GRACEA\\
\hfuzz=500pt\quad\includegraphics[width=1em]{connector.pdf}\includegraphics[width=1em]{element.pdf}~center1Z & \hfuzz=500pt double & \hfuzz=500pt z-coordinate of antenna position in SRF [m] for GRACEA\\
\hfuzz=500pt\quad\includegraphics[width=1em]{connector.pdf}\includegraphics[width=1em]{element.pdf}~center2X & \hfuzz=500pt double & \hfuzz=500pt x-coordinate of antenna position in SRF [m] for GRACEB\\
\hfuzz=500pt\quad\includegraphics[width=1em]{connector.pdf}\includegraphics[width=1em]{element.pdf}~center2Y & \hfuzz=500pt double & \hfuzz=500pt y-coordinate of antenna position in SRF [m] for GRACEB\\
\hfuzz=500pt\quad\includegraphics[width=1em]{connector.pdf}\includegraphics[width=1em]{element.pdf}~center2Z & \hfuzz=500pt double & \hfuzz=500pt z-coordinate of antenna position in SRF [m] for GRACEB\\
\hfuzz=500pt\includegraphics[width=1em]{connector.pdf}\includegraphics[width=1em]{element-mustset.pdf}~file & \hfuzz=500pt sequence & \hfuzz=500pt \\
\hfuzz=500pt\quad\includegraphics[width=1em]{connector.pdf}\includegraphics[width=1em]{element-mustset.pdf}~inputAntennaCenters & \hfuzz=500pt filename & \hfuzz=500pt \\
\hline
\end{tabularx}

\clearpage
%==================================
\subsection{SimulateStarCameraSentinel1}\label{SimulateStarCameraSentinel1}
This program simulates \file{star camera}{instrument} measurements at each satellite's position for the Sentinel 1A satellite.
The \configFile{inputfileOrbit}{instrument} must contain positions and velocities (see \program{OrbitAddVelocityAndAcceleration}).
The resulting rotation matrices rotate from satellite frame to inertial frame.


\keepXColumns
\begin{tabularx}{\textwidth}{N T A}
\hline
Name & Type & Annotation\\
\hline
\hfuzz=500pt\includegraphics[width=1em]{element-mustset.pdf}~outputfileStarCamera & \hfuzz=500pt filename & \hfuzz=500pt \\
\hfuzz=500pt\includegraphics[width=1em]{element-mustset.pdf}~inputfileOrbit & \hfuzz=500pt filename & \hfuzz=500pt position and velocity defines the orientation of the satellite at each epoch\\
\hline
\end{tabularx}

This program is \reference{parallelized}{general.parallelization}.
\clearpage
%==================================
\section{Programs: System}
\subsection{FileConvert}\label{FileConvert}
Converts GROOPS file between different file formats (ASCII, XML, binary),
see \reference{file formats}{general.fileFormat} for details.
It prints also some information about the content.
Therefore it can be used to get an idea about the content of binary files.


\keepXColumns
\begin{tabularx}{\textwidth}{N T A}
\hline
Name & Type & Annotation\\
\hline
\hfuzz=500pt\includegraphics[width=1em]{element.pdf}~outputfile & \hfuzz=500pt filename & \hfuzz=500pt GROOPS formats: .xml, .txt, .dat\\
\hfuzz=500pt\includegraphics[width=1em]{element-mustset.pdf}~inputfile & \hfuzz=500pt filename & \hfuzz=500pt GROOPS formats: .xml, .txt, .dat\\
\hline
\end{tabularx}

\clearpage
%==================================
\subsection{FileCreateDirectories}\label{FileCreateDirectories}
Creates the directory and parent directories as needed.


\keepXColumns
\begin{tabularx}{\textwidth}{N T A}
\hline
Name & Type & Annotation\\
\hline
\hfuzz=500pt\includegraphics[width=1em]{element-mustset-unbounded.pdf}~directory & \hfuzz=500pt filename & \hfuzz=500pt \\
\hline
\end{tabularx}

\clearpage
%==================================
\subsection{FileRemove}\label{FileRemove}
Remove files or directories.
Deletes also the content recursivley if one of \config{files} is a directory.


\keepXColumns
\begin{tabularx}{\textwidth}{N T A}
\hline
Name & Type & Annotation\\
\hline
\hfuzz=500pt\includegraphics[width=1em]{element-mustset-unbounded.pdf}~files & \hfuzz=500pt filename & \hfuzz=500pt \\
\hline
\end{tabularx}

\clearpage
%==================================
\subsection{GroupPrograms}\label{GroupPrograms}
Runs \config{program}s in a group, which can be used to structure a config file.
If \config{catchErrors} is enabled and an error occurs, the remaining \config{program}s
are skipped and execution continues with \config{errorProgram}s, in case any are defined.
Otherwise an exception is thrown.

The \config{silently} option disables the screen ouput of the \config{program}s.
With \config{outputfileLog} a log file is written for this group additional to a global log file.
This might be helpful within \program{LoopPrograms} with parallel iterations.


\keepXColumns
\begin{tabularx}{\textwidth}{N T A}
\hline
Name & Type & Annotation\\
\hline
\hfuzz=500pt\includegraphics[width=1em]{element.pdf}~outputfileLog & \hfuzz=500pt filename & \hfuzz=500pt additional log file\\
\hfuzz=500pt\includegraphics[width=1em]{element.pdf}~silently & \hfuzz=500pt boolean & \hfuzz=500pt without showing the output.\\
\hfuzz=500pt\includegraphics[width=1em]{element-unbounded.pdf}~program & \hfuzz=500pt programType & \hfuzz=500pt \\
\hfuzz=500pt\includegraphics[width=1em]{element.pdf}~catchErrors & \hfuzz=500pt sequence & \hfuzz=500pt \\
\hfuzz=500pt\includegraphics[width=1em]{connector.pdf}\includegraphics[width=1em]{element-unbounded.pdf}~errorProgram & \hfuzz=500pt programType & \hfuzz=500pt executed if an error occured\\
\hline
\end{tabularx}

This program is \reference{parallelized}{general.parallelization}.
\clearpage
%==================================
\subsection{IfPrograms}\label{IfPrograms}
Runs a list of \config{program}s if a \configClass{condition}{conditionType} is met.
Otherwise \config{elseProgram}s are executed.


\keepXColumns
\begin{tabularx}{\textwidth}{N T A}
\hline
Name & Type & Annotation\\
\hline
\hfuzz=500pt\includegraphics[width=1em]{element-mustset.pdf}~condition & \hfuzz=500pt \hyperref[conditionType]{condition} & \hfuzz=500pt \\
\hfuzz=500pt\includegraphics[width=1em]{element-unbounded.pdf}~program & \hfuzz=500pt programType & \hfuzz=500pt executed if condition evaluates to true\\
\hfuzz=500pt\includegraphics[width=1em]{element-unbounded.pdf}~elseProgram & \hfuzz=500pt programType & \hfuzz=500pt executed if condition evaluates to false\\
\hline
\end{tabularx}

This program is \reference{parallelized}{general.parallelization}.
\clearpage
%==================================
\subsection{LoopPrograms}\label{LoopPrograms}
This program runs a list of programs in a \configClass{loop}{loopType}.

If \config{continueAfterError}=\verb|yes| and an error occurs, the remaining programs in the current iteration
are skipped and the loop continues with the next iteration. Otherwise an exception is thrown.

If this program is executed on multpile processing nodes, the iterations can be computed in parallel,
see \reference{parallelization}{general.parallelization}. The first process serves as load balancer
and the other processes are assigned to iterations according to \config{processCountPerIteration}.
For example, running a loop containing three iterations on 13 processes with \config{processCountPerIteration}=\verb|4|,
runs the three iterations in parallel, with each iteration being assigned four processes.
With \config{parallelLog}=\verb|yes| all processes write output to screen and the log file.
As the ouput can be quite confusing in this case, running \program{GroupPrograms} with an extra \config{outputfileLog}
for each iteration (use the loop variables for the name of the log files) might be helpful.


\keepXColumns
\begin{tabularx}{\textwidth}{N T A}
\hline
Name & Type & Annotation\\
\hline
\hfuzz=500pt\includegraphics[width=1em]{element-mustset.pdf}~loop & \hfuzz=500pt \hyperref[loopType]{loop} & \hfuzz=500pt subprograms are called for every iteration\\
\hfuzz=500pt\includegraphics[width=1em]{element.pdf}~continueAfterError & \hfuzz=500pt boolean & \hfuzz=500pt continue with next iteration after error, otherwise throw exception\\
\hfuzz=500pt\includegraphics[width=1em]{element.pdf}~processCountPerIteration & \hfuzz=500pt uint & \hfuzz=500pt 0: use all processes for each iteration\\
\hfuzz=500pt\includegraphics[width=1em]{element.pdf}~parallelLog & \hfuzz=500pt boolean & \hfuzz=500pt write to screen/log file from all processing nodes in parallelized loops\\
\hfuzz=500pt\includegraphics[width=1em]{element-unbounded.pdf}~program & \hfuzz=500pt programType & \hfuzz=500pt \\
\hline
\end{tabularx}

This program is \reference{parallelized}{general.parallelization}.
\clearpage
%==================================
\subsection{RunCommand}\label{RunCommand}
Execute system \config{command}s. If \config{executeParallel} is set and
multiple \config{command}s are given they are executed in parallel at
distributed nodes, otherwise they are executed consecutively at master node only.


\keepXColumns
\begin{tabularx}{\textwidth}{N T A}
\hline
Name & Type & Annotation\\
\hline
\hfuzz=500pt\includegraphics[width=1em]{element-mustset-unbounded.pdf}~command & \hfuzz=500pt filename & \hfuzz=500pt \\
\hfuzz=500pt\includegraphics[width=1em]{element.pdf}~silently & \hfuzz=500pt boolean & \hfuzz=500pt without showing the output.\\
\hfuzz=500pt\includegraphics[width=1em]{element.pdf}~continueAfterError & \hfuzz=500pt boolean & \hfuzz=500pt continue with next command after error, otherwise throw exception\\
\hfuzz=500pt\includegraphics[width=1em]{element.pdf}~executeParallel & \hfuzz=500pt boolean & \hfuzz=500pt execute several commands in parallel\\
\hline
\end{tabularx}

This program is \reference{parallelized}{general.parallelization}.
\clearpage
%==================================
\section{Programs: Conversion}
\subsection{BerneseKinematic2Orbit}\label{BerneseKinematic2Orbit}
Read kinematic orbits in Bernese format.


\keepXColumns
\begin{tabularx}{\textwidth}{N T A}
\hline
Name & Type & Annotation\\
\hline
\hfuzz=500pt\includegraphics[width=1em]{element-mustset.pdf}~outputfileOrbit & \hfuzz=500pt filename & \hfuzz=500pt \\
\hfuzz=500pt\includegraphics[width=1em]{element-mustset.pdf}~outputfileCovariance & \hfuzz=500pt filename & \hfuzz=500pt \\
\hfuzz=500pt\includegraphics[width=1em]{element.pdf}~earthRotation & \hfuzz=500pt \hyperref[earthRotationType]{earthRotation} & \hfuzz=500pt from TRF to CRF\\
\hfuzz=500pt\includegraphics[width=1em]{element-mustset-unbounded.pdf}~inputfile & \hfuzz=500pt filename & \hfuzz=500pt \\
\hline
\end{tabularx}

\clearpage
%==================================
\subsection{Champ2AccStar}\label{Champ2AccStar}
This program reads in CHAMP accelerometer and star camera data given in the special CHAMP format.
In case of CHAMP accelerometer and star camera data is both stored in one file.
A description of the format can be found under: \url{http://op.gfz-potsdam.de/champ/docs_CHAMP/CH-GFZ-FD-001.pdf}.


\keepXColumns
\begin{tabularx}{\textwidth}{N T A}
\hline
Name & Type & Annotation\\
\hline
\hfuzz=500pt\includegraphics[width=1em]{element.pdf}~outputfileAccelerometer & \hfuzz=500pt filename & \hfuzz=500pt \\
\hfuzz=500pt\includegraphics[width=1em]{element.pdf}~outputfileAngularAcceleration & \hfuzz=500pt filename & \hfuzz=500pt \\
\hfuzz=500pt\includegraphics[width=1em]{element.pdf}~outputfileStarCamera & \hfuzz=500pt filename & \hfuzz=500pt \\
\hfuzz=500pt\includegraphics[width=1em]{element-mustset-unbounded.pdf}~inputfile & \hfuzz=500pt filename & \hfuzz=500pt \\
\hline
\end{tabularx}

\clearpage
%==================================
\subsection{Champ2Orbit}\label{Champ2Orbit}
This program reads in CHAMP precise science orbits in the special CHORB format.
A description of the format can be found under: \url{http://op.gfz-potsdam.de/champ/docs_CHAMP/CH-GFZ-FD-002.pdf}


\keepXColumns
\begin{tabularx}{\textwidth}{N T A}
\hline
Name & Type & Annotation\\
\hline
\hfuzz=500pt\includegraphics[width=1em]{element-mustset.pdf}~outputfileOrbit & \hfuzz=500pt filename & \hfuzz=500pt \\
\hfuzz=500pt\includegraphics[width=1em]{element-mustset.pdf}~earthRotation & \hfuzz=500pt \hyperref[earthRotationType]{earthRotation} & \hfuzz=500pt \\
\hfuzz=500pt\includegraphics[width=1em]{element-unbounded.pdf}~timeSeries & \hfuzz=500pt \hyperref[timeSeriesType]{timeSeries} & \hfuzz=500pt \\
\hfuzz=500pt\includegraphics[width=1em]{element-mustset.pdf}~inputOrbit & \hfuzz=500pt sequence & \hfuzz=500pt \\
\hfuzz=500pt\includegraphics[width=1em]{connector.pdf}\includegraphics[width=1em]{element-mustset-unbounded.pdf}~inputfile & \hfuzz=500pt filename & \hfuzz=500pt orbits in SP3 format\\
\hline
\end{tabularx}

\clearpage
%==================================
\subsection{Cosmic2OrbitStar}\label{Cosmic2OrbitStar}
This program reads in cosmic orbit and star camera data given in the CHAMP format.
In case of cosmic orbit and star camera data is stored in one file.
A description of the format can be found under: \url{http://op.gfz-potsdam.de/champ/docs_CHAMP/CH-GFZ-FD-001.pdf}


\keepXColumns
\begin{tabularx}{\textwidth}{N T A}
\hline
Name & Type & Annotation\\
\hline
\hfuzz=500pt\includegraphics[width=1em]{element-mustset.pdf}~outputfileOrbit & \hfuzz=500pt filename & \hfuzz=500pt \\
\hfuzz=500pt\includegraphics[width=1em]{element-mustset.pdf}~outputfileStarCamera & \hfuzz=500pt filename & \hfuzz=500pt \\
\hfuzz=500pt\includegraphics[width=1em]{element-mustset-unbounded.pdf}~inputfile & \hfuzz=500pt filename & \hfuzz=500pt \\
\hline
\end{tabularx}

\clearpage
%==================================
\subsection{DoodsonHarmonics2IersPotential}\label{DoodsonHarmonics2IersPotential}
Convert doodson harmonics to IERS conventions according to FES2004.
cf. \url{ftp://tai.bipm.org/iers/conv2010/chapter6/tidemodels/fes2004.dat}.


\keepXColumns
\begin{tabularx}{\textwidth}{N T A}
\hline
Name & Type & Annotation\\
\hline
\hfuzz=500pt\includegraphics[width=1em]{element-mustset.pdf}~outputfile & \hfuzz=500pt filename & \hfuzz=500pt according to IERS2010, chapter 6.3.2, footnote 7\\
\hfuzz=500pt\includegraphics[width=1em]{element-mustset.pdf}~inputfileDoodsonHarmoncis & \hfuzz=500pt filename & \hfuzz=500pt \\
\hfuzz=500pt\includegraphics[width=1em]{element-mustset-unbounded.pdf}~header & \hfuzz=500pt string & \hfuzz=500pt info for output header\\
\hfuzz=500pt\includegraphics[width=1em]{element.pdf}~factor & \hfuzz=500pt double & \hfuzz=500pt \\
\hfuzz=500pt\includegraphics[width=1em]{element.pdf}~minDegree & \hfuzz=500pt uint & \hfuzz=500pt \\
\hfuzz=500pt\includegraphics[width=1em]{element.pdf}~maxDegree & \hfuzz=500pt uint & \hfuzz=500pt \\
\hline
\end{tabularx}

\clearpage
%==================================
\subsection{DoodsonHarmonics2IersWaterHeight}\label{DoodsonHarmonics2IersWaterHeight}
Convert doodson harmonics to IERS conventions according to FES2004.
cf. \url{ftp://tai.bipm.org/iers/conv2010/chapter6/tidemodels/fes2004.dat}.


\keepXColumns
\begin{tabularx}{\textwidth}{N T A}
\hline
Name & Type & Annotation\\
\hline
\hfuzz=500pt\includegraphics[width=1em]{element-mustset.pdf}~outputfile & \hfuzz=500pt filename & \hfuzz=500pt according to IERS2010, chapter 6.3.2, footnote 7\\
\hfuzz=500pt\includegraphics[width=1em]{element-mustset.pdf}~inputfileDoodsonHarmoncis & \hfuzz=500pt filename & \hfuzz=500pt \\
\hfuzz=500pt\includegraphics[width=1em]{element-mustset.pdf}~inputfileTideGeneratingPotential & \hfuzz=500pt filename & \hfuzz=500pt to compute Xi phase correction\\
\hfuzz=500pt\includegraphics[width=1em]{element-mustset-unbounded.pdf}~header & \hfuzz=500pt string & \hfuzz=500pt info for output header\\
\hfuzz=500pt\includegraphics[width=1em]{element-mustset.pdf}~kernel & \hfuzz=500pt \hyperref[kernelType]{kernel} & \hfuzz=500pt data type of output values\\
\hfuzz=500pt\includegraphics[width=1em]{element.pdf}~factor & \hfuzz=500pt double & \hfuzz=500pt e.g. from [m] to [cm]\\
\hfuzz=500pt\includegraphics[width=1em]{element.pdf}~minDegree & \hfuzz=500pt uint & \hfuzz=500pt \\
\hfuzz=500pt\includegraphics[width=1em]{element.pdf}~maxDegree & \hfuzz=500pt uint & \hfuzz=500pt \\
\hline
\end{tabularx}

\clearpage
%==================================
\subsection{GnssAntex2AntennaDefinition}\label{GnssAntex2AntennaDefinition}
Converts metadata and antenna definitions from the \href{https://files.igs.org/pub/data/format/antex14.txt}{IGS ANTEX format}.
to \configFile{transmitterInfo}{gnssStationInfo}, \configFile{antennaDefinition}{gnssAntennaDefinition}, \configFile{svnBlockTable}{stringTable},
and \configFile{transmitterList}{stringList} files for the respective GNSS and for the list of ground station antennas.


\keepXColumns
\begin{tabularx}{\textwidth}{N T A}
\hline
Name & Type & Annotation\\
\hline
\hfuzz=500pt\includegraphics[width=1em]{element.pdf}~outputfileAntennaDefinitionStation & \hfuzz=500pt filename & \hfuzz=500pt antenna center variations\\
\hfuzz=500pt\includegraphics[width=1em]{element.pdf}~outputfileAntennaDefinitionTransmitter & \hfuzz=500pt filename & \hfuzz=500pt antenna center variations\\
\hfuzz=500pt\includegraphics[width=1em]{element.pdf}~outputfileTransmitterInfo & \hfuzz=500pt filename & \hfuzz=500pt PRN is appended to file name\\
\hfuzz=500pt\includegraphics[width=1em]{element.pdf}~outputfileSvnBlockTableGps & \hfuzz=500pt filename & \hfuzz=500pt SVN to satellite block mapping\\
\hfuzz=500pt\includegraphics[width=1em]{element.pdf}~outputfileSvnBlockTableGlonass & \hfuzz=500pt filename & \hfuzz=500pt SVN to satellite block mapping\\
\hfuzz=500pt\includegraphics[width=1em]{element.pdf}~outputfileSvnBlockTableGalileo & \hfuzz=500pt filename & \hfuzz=500pt SVN to satellite block mapping\\
\hfuzz=500pt\includegraphics[width=1em]{element.pdf}~outputfileSvnBlockTableBeiDou & \hfuzz=500pt filename & \hfuzz=500pt SVN to satellite block mapping\\
\hfuzz=500pt\includegraphics[width=1em]{element.pdf}~outputfileSvnBlockTableQzss & \hfuzz=500pt filename & \hfuzz=500pt SVN to satellite block mapping\\
\hfuzz=500pt\includegraphics[width=1em]{element.pdf}~outputfileTransmitterListGps & \hfuzz=500pt filename & \hfuzz=500pt list of PRNs\\
\hfuzz=500pt\includegraphics[width=1em]{element.pdf}~outputfileTransmitterListGlonass & \hfuzz=500pt filename & \hfuzz=500pt list of PRNs\\
\hfuzz=500pt\includegraphics[width=1em]{element.pdf}~outputfileTransmitterListGalileo & \hfuzz=500pt filename & \hfuzz=500pt list of PRNs\\
\hfuzz=500pt\includegraphics[width=1em]{element.pdf}~outputfileTransmitterListBeiDou & \hfuzz=500pt filename & \hfuzz=500pt list of PRNs\\
\hfuzz=500pt\includegraphics[width=1em]{element.pdf}~outputfileTransmitterListQzss & \hfuzz=500pt filename & \hfuzz=500pt list of PRNs\\
\hfuzz=500pt\includegraphics[width=1em]{element-mustset.pdf}~inputfileAntex & \hfuzz=500pt filename & \hfuzz=500pt \\
\hfuzz=500pt\includegraphics[width=1em]{element.pdf}~timeStart & \hfuzz=500pt time & \hfuzz=500pt ignore older antenna definitions\\
\hfuzz=500pt\includegraphics[width=1em]{element.pdf}~createZeroModel & \hfuzz=500pt boolean & \hfuzz=500pt create empty antenna patterns\\
\hline
\end{tabularx}

\clearpage
%==================================
\subsection{GnssAttitude2Orbex}\label{GnssAttitude2Orbex}
Convert attitude of GNSS satellites to \href{http://acc.igs.org/misc/proposal_orbex_april2019.pdf}{ORBEX file format} (quaternions).

If \configClass{earthRotation}{earthRotationType} is provided, the output file contains quaternions for rotation from TRF to satellite
body frame (IGS/ORBEX convention), otherwise the rotation is from CRF to satellite body frame.

See also \program{GnssOrbex2StarCamera}, \program{SimulateStarCameraGnss}.


\keepXColumns
\begin{tabularx}{\textwidth}{N T A}
\hline
Name & Type & Annotation\\
\hline
\hfuzz=500pt\includegraphics[width=1em]{element-mustset.pdf}~outputfileOrbex & \hfuzz=500pt filename & \hfuzz=500pt ORBEX file\\
\hfuzz=500pt\includegraphics[width=1em]{element-mustset-unbounded.pdf}~inputfileTransmitterList & \hfuzz=500pt filename & \hfuzz=500pt ASCII list with transmitter PRNs\\
\hfuzz=500pt\includegraphics[width=1em]{element-mustset.pdf}~inputfileAttitude & \hfuzz=500pt filename & \hfuzz=500pt instrument file containing attitude\\
\hfuzz=500pt\includegraphics[width=1em]{element.pdf}~variablePrn & \hfuzz=500pt string & \hfuzz=500pt loop variable for PRNs from transmitter list\\
\hfuzz=500pt\includegraphics[width=1em]{element-unbounded.pdf}~timeSeries & \hfuzz=500pt \hyperref[timeSeriesType]{timeSeries} & \hfuzz=500pt resample to these epochs (otherwise input file epochs are used)\\
\hfuzz=500pt\includegraphics[width=1em]{element.pdf}~earthRotation & \hfuzz=500pt \hyperref[earthRotationType]{earthRotation} & \hfuzz=500pt rotate data into Earth-fixed frame\\
\hfuzz=500pt\includegraphics[width=1em]{element-mustset.pdf}~interpolationDegree & \hfuzz=500pt uint & \hfuzz=500pt for attitude and Earth rotation interpolation\\
\hfuzz=500pt\includegraphics[width=1em]{element-mustset.pdf}~description & \hfuzz=500pt string & \hfuzz=500pt description of file contents\\
\hfuzz=500pt\includegraphics[width=1em]{element-mustset.pdf}~createdBy & \hfuzz=500pt string & \hfuzz=500pt name of agency\\
\hfuzz=500pt\includegraphics[width=1em]{element-mustset.pdf}~inputData & \hfuzz=500pt string & \hfuzz=500pt description of input data (see ORBEX description)\\
\hfuzz=500pt\includegraphics[width=1em]{element-mustset.pdf}~contact & \hfuzz=500pt string & \hfuzz=500pt email address\\
\hfuzz=500pt\includegraphics[width=1em]{element-mustset.pdf}~referenceFrame & \hfuzz=500pt string & \hfuzz=500pt reference frame used in file\\
\hfuzz=500pt\includegraphics[width=1em]{element-unbounded.pdf}~comment & \hfuzz=500pt string & \hfuzz=500pt \\
\hline
\end{tabularx}

\clearpage
%==================================
\subsection{GnssClock2ClockRinex}\label{GnssClock2ClockRinex}
Converts GNSS clocks from GROOPS format to \href{https://files.igs.org/pub/data/format/rinex_clock304.txt}{IGS clock RINEX format}.
Clocks can be provided via \config{satelliteData} and/or \config{stationData}.
Observed signal types are inferred from \configFile{inputfileSignalBias}{gnssSignalBias}.
Satellites/stations used as clock references can be provided via \config{referenceClock}.

See IGS clock RINEX format description for further details on header information.


\keepXColumns
\begin{tabularx}{\textwidth}{N T A}
\hline
Name & Type & Annotation\\
\hline
\hfuzz=500pt\includegraphics[width=1em]{element-mustset.pdf}~outputfileClockRinex & \hfuzz=500pt filename & \hfuzz=500pt \\
\hfuzz=500pt\includegraphics[width=1em]{element-unbounded.pdf}~satelliteData & \hfuzz=500pt sequence & \hfuzz=500pt one element per satellite\\
\hfuzz=500pt\includegraphics[width=1em]{connector.pdf}\includegraphics[width=1em]{element-mustset.pdf}~inputfileClock & \hfuzz=500pt filename & \hfuzz=500pt clock instrument file\\
\hfuzz=500pt\includegraphics[width=1em]{connector.pdf}\includegraphics[width=1em]{element-mustset.pdf}~inputfileSignalBias & \hfuzz=500pt filename & \hfuzz=500pt signal bias file\\
\hfuzz=500pt\includegraphics[width=1em]{connector.pdf}\includegraphics[width=1em]{element-mustset.pdf}~identifier & \hfuzz=500pt string & \hfuzz=500pt PRN (e.g. G23)\\
\hfuzz=500pt\includegraphics[width=1em]{element-unbounded.pdf}~stationData & \hfuzz=500pt sequence & \hfuzz=500pt one element per station\\
\hfuzz=500pt\includegraphics[width=1em]{connector.pdf}\includegraphics[width=1em]{element-mustset.pdf}~inputfileClock & \hfuzz=500pt filename & \hfuzz=500pt clock instrument file\\
\hfuzz=500pt\includegraphics[width=1em]{connector.pdf}\includegraphics[width=1em]{element-mustset.pdf}~inputfilePosition & \hfuzz=500pt filename & \hfuzz=500pt station position file\\
\hfuzz=500pt\includegraphics[width=1em]{connector.pdf}\includegraphics[width=1em]{element-mustset.pdf}~inputfileStationInfo & \hfuzz=500pt filename & \hfuzz=500pt station info file\\
\hfuzz=500pt\includegraphics[width=1em]{connector.pdf}\includegraphics[width=1em]{element-mustset.pdf}~identifier & \hfuzz=500pt string & \hfuzz=500pt station name (e.g. wtzz)\\
\hfuzz=500pt\includegraphics[width=1em]{element-unbounded.pdf}~comment & \hfuzz=500pt string & \hfuzz=500pt comment in header\\
\hfuzz=500pt\includegraphics[width=1em]{element-mustset.pdf}~program & \hfuzz=500pt string & \hfuzz=500pt name of program (for first line)\\
\hfuzz=500pt\includegraphics[width=1em]{element-mustset.pdf}~institution & \hfuzz=500pt string & \hfuzz=500pt name of agency (for first line)\\
\hfuzz=500pt\includegraphics[width=1em]{element-mustset.pdf}~analysisCenter & \hfuzz=500pt string & \hfuzz=500pt name of analysis center\\
\hfuzz=500pt\includegraphics[width=1em]{element.pdf}~differentialCodeBias & \hfuzz=500pt string & \hfuzz=500pt program and source for applied differential code bias\\
\hfuzz=500pt\includegraphics[width=1em]{element-mustset.pdf}~phaseCenterVariations & \hfuzz=500pt string & \hfuzz=500pt program and source for applied phase center variations\\
\hfuzz=500pt\includegraphics[width=1em]{element-mustset-unbounded.pdf}~referenceClock & \hfuzz=500pt string & \hfuzz=500pt identifier of reference satellite/station\\
\hfuzz=500pt\includegraphics[width=1em]{element-mustset.pdf}~referenceFrame & \hfuzz=500pt string & \hfuzz=500pt terrestrial reference frame for the stations\\
\hline
\end{tabularx}

\clearpage
%==================================
\subsection{GnssClockRinex2InstrumentClock}\label{GnssClockRinex2InstrumentClock}
This program converts clocks from the \href{https://files.igs.org/pub/data/format/rinex_clock304.txt}{IGS clock RINEX format},
which contains the clocks of all satellites and stations in a single file,
into an \file{instrument file (MISCVALUE)}{instrument} for each \config{identifier}
(satellite and/or station).


\keepXColumns
\begin{tabularx}{\textwidth}{N T A}
\hline
Name & Type & Annotation\\
\hline
\hfuzz=500pt\includegraphics[width=1em]{element-mustset.pdf}~outputfileInstrument & \hfuzz=500pt filename & \hfuzz=500pt identifier is appended to each file\\
\hfuzz=500pt\includegraphics[width=1em]{element-mustset-unbounded.pdf}~inputfileClockRinex & \hfuzz=500pt filename & \hfuzz=500pt \\
\hfuzz=500pt\includegraphics[width=1em]{element-mustset-unbounded.pdf}~identifier & \hfuzz=500pt string & \hfuzz=500pt satellite or station identifier, e.g. G23 or alic\\
\hfuzz=500pt\includegraphics[width=1em]{element-mustset-unbounded.pdf}~intervals & \hfuzz=500pt \hyperref[timeSeriesType]{timeSeries} & \hfuzz=500pt \\
\hfuzz=500pt\includegraphics[width=1em]{element.pdf}~minEpochsPerInterval & \hfuzz=500pt uint & \hfuzz=500pt minimum number of epochs in an interval\\
\hline
\end{tabularx}

\clearpage
%==================================
\subsection{GnssEop2IgsErp}\label{GnssEop2IgsErp}
Write GNSS Earth orientation parameters to \href{https://files.igs.org/pub/data/format/erp.txt}{IGS ERP file format}.

Requires polar motion, polar motion rate, dUT1 and LOD parameters in the solution
vector \configFile{inputfileSolution}{matrix} and their sigmas in \configFile{inputfileSigmax}{matrix}.
Solution usually comes out of \program{GnssProcessing}.


\keepXColumns
\begin{tabularx}{\textwidth}{N T A}
\hline
Name & Type & Annotation\\
\hline
\hfuzz=500pt\includegraphics[width=1em]{element-mustset.pdf}~outputfileIgsErp & \hfuzz=500pt filename & \hfuzz=500pt IGS ERP file\\
\hfuzz=500pt\includegraphics[width=1em]{element-mustset-unbounded.pdf}~epoch & \hfuzz=500pt sequence & \hfuzz=500pt e.g. daily solution\\
\hfuzz=500pt\includegraphics[width=1em]{connector.pdf}\includegraphics[width=1em]{element-mustset.pdf}~inputfileSolution & \hfuzz=500pt filename & \hfuzz=500pt parameter vector\\
\hfuzz=500pt\includegraphics[width=1em]{connector.pdf}\includegraphics[width=1em]{element-mustset.pdf}~inputfileSigmax & \hfuzz=500pt filename & \hfuzz=500pt standard deviations of the parameters (sqrt of the diagonal of the inverse normal equation)\\
\hfuzz=500pt\includegraphics[width=1em]{connector.pdf}\includegraphics[width=1em]{element-mustset.pdf}~inputfileParameterNames & \hfuzz=500pt filename & \hfuzz=500pt parameter names\\
\hfuzz=500pt\includegraphics[width=1em]{connector.pdf}\includegraphics[width=1em]{element-unbounded.pdf}~inputfileTransmitterList & \hfuzz=500pt filename & \hfuzz=500pt transmitter PRNs used in solution (used for transmitter count)\\
\hfuzz=500pt\includegraphics[width=1em]{connector.pdf}\includegraphics[width=1em]{element.pdf}~inputfileStationList & \hfuzz=500pt filename & \hfuzz=500pt stations used in solution (used for station count)\\
\hfuzz=500pt\includegraphics[width=1em]{connector.pdf}\includegraphics[width=1em]{element-mustset.pdf}~time & \hfuzz=500pt time & \hfuzz=500pt reference time for epoch\\
\hfuzz=500pt\includegraphics[width=1em]{element-unbounded.pdf}~comment & \hfuzz=500pt string & \hfuzz=500pt \\
\hline
\end{tabularx}

\clearpage
%==================================
\subsection{GnssGriddedDataTimeSeries2Ionex}\label{GnssGriddedDataTimeSeries2Ionex}
Converts TEC maps from GROOPS \file{gridded data time series}{griddedDataTimeSeries} format
to IGS \href{https://files.igs.org/pub/data/format/ionex1.pdf}{IONEX file format}.

Currently only supports 2D TEC maps.

See also \program{GnssIonex2GriddedDataTimeSeries}, \configClass{IonosphereMap}{gnssParametrizationType:ionosphereMap}.


\keepXColumns
\begin{tabularx}{\textwidth}{N T A}
\hline
Name & Type & Annotation\\
\hline
\hfuzz=500pt\includegraphics[width=1em]{element-mustset.pdf}~outputfileIonex & \hfuzz=500pt filename & \hfuzz=500pt \\
\hfuzz=500pt\includegraphics[width=1em]{element-mustset.pdf}~inputfileGriddedDataTimeSeries & \hfuzz=500pt filename & \hfuzz=500pt must contain regular grid\\
\hfuzz=500pt\includegraphics[width=1em]{element-mustset.pdf}~value & \hfuzz=500pt expression & \hfuzz=500pt expression (e.g. data column)\\
\hfuzz=500pt\includegraphics[width=1em]{element-unbounded.pdf}~timeSeries & \hfuzz=500pt \hyperref[timeSeriesType]{timeSeries} & \hfuzz=500pt (empty = use input file time series)\\
\hfuzz=500pt\includegraphics[width=1em]{element-mustset.pdf}~program & \hfuzz=500pt string & \hfuzz=500pt name of program (for first line)\\
\hfuzz=500pt\includegraphics[width=1em]{element-mustset.pdf}~institution & \hfuzz=500pt string & \hfuzz=500pt name of agency (for first line)\\
\hfuzz=500pt\includegraphics[width=1em]{element-unbounded.pdf}~description & \hfuzz=500pt string & \hfuzz=500pt description in header\\
\hfuzz=500pt\includegraphics[width=1em]{element-unbounded.pdf}~comment & \hfuzz=500pt string & \hfuzz=500pt comment in header\\
\hfuzz=500pt\includegraphics[width=1em]{element.pdf}~mappingFunction & \hfuzz=500pt string & \hfuzz=500pt see IONEX documentation\\
\hfuzz=500pt\includegraphics[width=1em]{element.pdf}~elevationCutoff & \hfuzz=500pt double & \hfuzz=500pt see IONEX documentation (0 if unknown)\\
\hfuzz=500pt\includegraphics[width=1em]{element.pdf}~observablesUsed & \hfuzz=500pt string & \hfuzz=500pt see IONEX documentation\\
\hfuzz=500pt\includegraphics[width=1em]{element.pdf}~exponent & \hfuzz=500pt int & \hfuzz=500pt factor 10\textasciicircum{}exponent is applied to all values\\
\hline
\end{tabularx}

\clearpage
%==================================
\subsection{GnssIonex2GriddedDataTimeSeries}\label{GnssIonex2GriddedDataTimeSeries}
Converts TEC maps from IGS \href{https://files.igs.org/pub/data/format/ionex1.pdf}{IONEX file format}
to GROOPS \file{gridded data time series}{griddedDataTimeSeries} format.

Currently only supports 2D TEC maps.

See also \program{GnssGriddedDataTimeSeries2Ionex}, \configClass{IonosphereMap}{gnssParametrizationType:ionosphereMap}.


\keepXColumns
\begin{tabularx}{\textwidth}{N T A}
\hline
Name & Type & Annotation\\
\hline
\hfuzz=500pt\includegraphics[width=1em]{element-mustset.pdf}~outputfileGriddedDataTimeSeries & \hfuzz=500pt filename & \hfuzz=500pt \\
\hfuzz=500pt\includegraphics[width=1em]{element-mustset-unbounded.pdf}~inputfileIonex & \hfuzz=500pt filename & \hfuzz=500pt \\
\hline
\end{tabularx}

\clearpage
%==================================
\subsection{GnssNormals2Sinex}\label{GnssNormals2Sinex}
Write GNSS data/metadata and \file{normal equations}{normalEquation} to
\href{http://www.iers.org/IERS/EN/Organization/AnalysisCoordinator/SinexFormat/sinex.html}{SINEX format}.

Normal equations usually come from \program{GnssProcessing}
(e.g. from \reference{GNSS satellite orbit determination and station network analysis}{cookbook.gnssNetwork}).
Metadata input files include \configFile{stationInfo/transmitterInfo}{gnssStationInfo}, \configFile{antennaDefinition}{gnssAntennaDefinition},
and \configFile{stationList/transmitterList}{stringList}, see \program{GnssAntex2AntennaDefinition}.

See also \program{Sinex2Normals} and \program{NormalsSphericalHarmonics2Sinex}.


\keepXColumns
\begin{tabularx}{\textwidth}{N T A}
\hline
Name & Type & Annotation\\
\hline
\hfuzz=500pt\includegraphics[width=1em]{element.pdf}~outputfileSinexNormals & \hfuzz=500pt filename & \hfuzz=500pt full SINEX file including normal equations\\
\hfuzz=500pt\includegraphics[width=1em]{element.pdf}~outputfileSinexCoordinates & \hfuzz=500pt filename & \hfuzz=500pt SINEX file without normal equations (station coordinates file)\\
\hfuzz=500pt\includegraphics[width=1em]{element-mustset.pdf}~inputfileNormals & \hfuzz=500pt filename & \hfuzz=500pt normal equation matrix\\
\hfuzz=500pt\includegraphics[width=1em]{element.pdf}~inputfileSolution & \hfuzz=500pt filename & \hfuzz=500pt parameter vector\\
\hfuzz=500pt\includegraphics[width=1em]{element.pdf}~inputfileSigmax & \hfuzz=500pt filename & \hfuzz=500pt standard deviations of the parameters (sqrt of the diagonal of the inverse normal equation)\\
\hfuzz=500pt\includegraphics[width=1em]{element-mustset.pdf}~inputfileApriori & \hfuzz=500pt filename & \hfuzz=500pt apriori parameter vector\\
\hfuzz=500pt\includegraphics[width=1em]{element.pdf}~inputfileAprioriSigma & \hfuzz=500pt filename & \hfuzz=500pt constraint sigmas for apriori parameter vector\\
\hfuzz=500pt\includegraphics[width=1em]{element.pdf}~inputfileAprioriMatrix & \hfuzz=500pt filename & \hfuzz=500pt normal equation matrix of applied constraints\\
\hfuzz=500pt\includegraphics[width=1em]{element-mustset-unbounded.pdf}~transmitterConstellation & \hfuzz=500pt sequence & \hfuzz=500pt transmitter constellation metadata\\
\hfuzz=500pt\includegraphics[width=1em]{connector.pdf}\includegraphics[width=1em]{element-mustset.pdf}~inputfileTransmitterList & \hfuzz=500pt filename & \hfuzz=500pt transmitter PRNs used in solution\\
\hfuzz=500pt\includegraphics[width=1em]{connector.pdf}\includegraphics[width=1em]{element-mustset.pdf}~inputfileTransmitterInfo & \hfuzz=500pt filename & \hfuzz=500pt transmitter info file template\\
\hfuzz=500pt\includegraphics[width=1em]{connector.pdf}\includegraphics[width=1em]{element-mustset.pdf}~inputfileAntennaDefinition & \hfuzz=500pt filename & \hfuzz=500pt transmitter phase centers and variations (ANTEX)\\
\hfuzz=500pt\includegraphics[width=1em]{connector.pdf}\includegraphics[width=1em]{element.pdf}~variablePrn & \hfuzz=500pt string & \hfuzz=500pt loop variable for PRNs from transmitter list\\
\hfuzz=500pt\includegraphics[width=1em]{element-mustset.pdf}~stations & \hfuzz=500pt sequence & \hfuzz=500pt \\
\hfuzz=500pt\includegraphics[width=1em]{connector.pdf}\includegraphics[width=1em]{element-mustset.pdf}~inputfileStationList & \hfuzz=500pt filename & \hfuzz=500pt stations contained in normal equations\\
\hfuzz=500pt\includegraphics[width=1em]{connector.pdf}\includegraphics[width=1em]{element-mustset.pdf}~inputfileStationInfo & \hfuzz=500pt filename & \hfuzz=500pt station info file template\\
\hfuzz=500pt\includegraphics[width=1em]{connector.pdf}\includegraphics[width=1em]{element-mustset.pdf}~inputfileAntennaDefinition & \hfuzz=500pt filename & \hfuzz=500pt station phase centers and variations (ANTEX)\\
\hfuzz=500pt\includegraphics[width=1em]{connector.pdf}\includegraphics[width=1em]{element.pdf}~variableStationName & \hfuzz=500pt string & \hfuzz=500pt loop variable for station names from station list\\
\hfuzz=500pt\includegraphics[width=1em]{connector.pdf}\includegraphics[width=1em]{element-mustset.pdf}~observationTimeStart & \hfuzz=500pt time & \hfuzz=500pt start time for which solution has observations\\
\hfuzz=500pt\includegraphics[width=1em]{connector.pdf}\includegraphics[width=1em]{element-mustset.pdf}~observationTimeEnd & \hfuzz=500pt time & \hfuzz=500pt end time for which solution has observations\\
\hfuzz=500pt\includegraphics[width=1em]{element-mustset.pdf}~time & \hfuzz=500pt time & \hfuzz=500pt reference time for parameters\\
\hfuzz=500pt\includegraphics[width=1em]{element.pdf}~sampling & \hfuzz=500pt double & \hfuzz=500pt [seconds] observation sampling\\
\hfuzz=500pt\includegraphics[width=1em]{element-mustset.pdf}~antennaCalibrationModel & \hfuzz=500pt string & \hfuzz=500pt e.g. IGS14\_WWWW (WWWW = ANTEX release GPS week)\\
\hfuzz=500pt\includegraphics[width=1em]{element-mustset.pdf}~sinexHeader & \hfuzz=500pt sequence & \hfuzz=500pt \\
\hfuzz=500pt\includegraphics[width=1em]{connector.pdf}\includegraphics[width=1em]{element.pdf}~agencyCode & \hfuzz=500pt string & \hfuzz=500pt identify the agency providing the data\\
\hfuzz=500pt\includegraphics[width=1em]{connector.pdf}\includegraphics[width=1em]{element.pdf}~timeStart & \hfuzz=500pt time & \hfuzz=500pt start time of the data\\
\hfuzz=500pt\includegraphics[width=1em]{connector.pdf}\includegraphics[width=1em]{element.pdf}~timeEnd & \hfuzz=500pt time & \hfuzz=500pt end time of the data \\
\hfuzz=500pt\includegraphics[width=1em]{connector.pdf}\includegraphics[width=1em]{element.pdf}~observationCode & \hfuzz=500pt string & \hfuzz=500pt technique used to generate the SINEX solution\\
\hfuzz=500pt\includegraphics[width=1em]{connector.pdf}\includegraphics[width=1em]{element.pdf}~constraintCode & \hfuzz=500pt string & \hfuzz=500pt 0: tight constraint, 1: siginficant constraint, 2: unconstrained\\
\hfuzz=500pt\includegraphics[width=1em]{connector.pdf}\includegraphics[width=1em]{element.pdf}~solutionContent & \hfuzz=500pt string & \hfuzz=500pt solution types contained in the SINEX solution (S O E T C A)\\
\hfuzz=500pt\includegraphics[width=1em]{connector.pdf}\includegraphics[width=1em]{element.pdf}~description & \hfuzz=500pt string & \hfuzz=500pt organizitions gathering/alerting the file contents\\
\hfuzz=500pt\includegraphics[width=1em]{connector.pdf}\includegraphics[width=1em]{element.pdf}~contact & \hfuzz=500pt string & \hfuzz=500pt Address of the relevant contact. e-mail\\
\hfuzz=500pt\includegraphics[width=1em]{connector.pdf}\includegraphics[width=1em]{element.pdf}~output & \hfuzz=500pt string & \hfuzz=500pt Description of the file contents\\
\hfuzz=500pt\includegraphics[width=1em]{connector.pdf}\includegraphics[width=1em]{element.pdf}~input & \hfuzz=500pt string & \hfuzz=500pt Brief description of the input used to generate this solution\\
\hfuzz=500pt\includegraphics[width=1em]{connector.pdf}\includegraphics[width=1em]{element.pdf}~software & \hfuzz=500pt string & \hfuzz=500pt Software used to generate the file\\
\hfuzz=500pt\includegraphics[width=1em]{connector.pdf}\includegraphics[width=1em]{element.pdf}~hardware & \hfuzz=500pt string & \hfuzz=500pt Computer hardware on which above software was run\\
\hfuzz=500pt\includegraphics[width=1em]{connector.pdf}\includegraphics[width=1em]{element.pdf}~inputfileComment & \hfuzz=500pt filename & \hfuzz=500pt comments in the comment block from a file (truncated at 80 characters)\\
\hfuzz=500pt\includegraphics[width=1em]{connector.pdf}\includegraphics[width=1em]{element-unbounded.pdf}~comment & \hfuzz=500pt string & \hfuzz=500pt comments in the comment block\\
\hline
\end{tabularx}

\clearpage
%==================================
\subsection{GnssOrbex2StarCamera}\label{GnssOrbex2StarCamera}
Converts GNSS satellite attitude from \href{http://acc.igs.org/misc/proposal_orbex_april2019.pdf}{ORBEX file format}
(quaternions) to \file{instrument file (STARCAMERA)}{instrument}.
The resulting star camera files contain the rotation from satellite body frame to TRF, or to CRF in case
\configClass{earthRotation}{earthRotationType} is provided.

See also \program{GnssAttitude2Orbex}.


\keepXColumns
\begin{tabularx}{\textwidth}{N T A}
\hline
Name & Type & Annotation\\
\hline
\hfuzz=500pt\includegraphics[width=1em]{element-mustset.pdf}~outputfileStarCamera & \hfuzz=500pt filename & \hfuzz=500pt rotation from body frame to TRF/CRF, identifier is appended to each file\\
\hfuzz=500pt\includegraphics[width=1em]{element-mustset.pdf}~inputfileOrbex & \hfuzz=500pt filename & \hfuzz=500pt \\
\hfuzz=500pt\includegraphics[width=1em]{element-unbounded.pdf}~identifier & \hfuzz=500pt string & \hfuzz=500pt (empty = all) satellite identifier, e.g. G23 or E05\\
\hfuzz=500pt\includegraphics[width=1em]{element.pdf}~earthRotation & \hfuzz=500pt \hyperref[earthRotationType]{earthRotation} & \hfuzz=500pt rotation from TRF to CRF\\
\hline
\end{tabularx}

\clearpage
%==================================
\subsection{GnssRinexNavigation2OrbitClock}\label{GnssRinexNavigation2OrbitClock}
Evaluates orbit and clock parameters from \href{https://files.igs.org/pub/data/format/rinex_4.00.pdf}{RINEX} (version 2, 3, and 4)
navigation file \config{inputfileRinex} at epochs given by \configClass{timeSeries}{timeSeriesType} and writes them to
\configFile{outputfileOrbit}{instrument} and \configFile{outputfileClock}{instrument}, respectively.

Orbits are rotated from TRF (as broadcasted) to CRF via \configClass{earthRotation}{earthRotationType},
but system-specific TRFs (WGS84, PZ-90, etc.) are not aligned to a common TRF.

If \config{messageType} is set (e.g., to LNAV, CNAV, or other types defined in the RINEX 4 standard), only navigation records of this type are used.
Otherwise, if multiple records are defined for the same epoch, the first one is used.

See also \program{OrbitAddVelocityAndAcceleration}.


\keepXColumns
\begin{tabularx}{\textwidth}{N T A}
\hline
Name & Type & Annotation\\
\hline
\hfuzz=500pt\includegraphics[width=1em]{element.pdf}~outputfileOrbit & \hfuzz=500pt filename & \hfuzz=500pt PRN is appended to file name\\
\hfuzz=500pt\includegraphics[width=1em]{element.pdf}~outputfileClock & \hfuzz=500pt filename & \hfuzz=500pt PRN is appended to file name\\
\hfuzz=500pt\includegraphics[width=1em]{element-mustset.pdf}~inputfileRinex & \hfuzz=500pt filename & \hfuzz=500pt RINEX navigation file\\
\hfuzz=500pt\includegraphics[width=1em]{element-mustset-unbounded.pdf}~timeSeries & \hfuzz=500pt \hyperref[timeSeriesType]{timeSeries} & \hfuzz=500pt orbit and clock evaluation epochs\\
\hfuzz=500pt\includegraphics[width=1em]{element-mustset.pdf}~earthRotation & \hfuzz=500pt \hyperref[earthRotationType]{earthRotation} & \hfuzz=500pt for rotation from TRF to CRF\\
\hfuzz=500pt\includegraphics[width=1em]{element-unbounded.pdf}~usePrn & \hfuzz=500pt string & \hfuzz=500pt only export these PRNs instead of all\\
\hfuzz=500pt\includegraphics[width=1em]{element.pdf}~messageType & \hfuzz=500pt string & \hfuzz=500pt (RINEX4) only use this navigation message (LNAV, CNAV, ...)\\
\hline
\end{tabularx}

\clearpage
%==================================
\subsection{GnssSignalBias2SinexBias}\label{GnssSignalBias2SinexBias}
Convert \file{GNSS signal biases}{gnssSignalBias} from GROOPS format to \href{https://files.igs.org/pub/data/format/sinex_bias_100.pdf}{IGS SINEX Bias format}.
Biases can be provided via \config{transmitterBiases} and/or \config{receiverBiases}.
Phase biases without attribute (e.g. \verb|L1*|) are automatically expanded so each code
bias has a corresponding phase bias
(Example: \verb|C1C|, \verb|C1W|, \verb|L1*| are converted to \verb|C1C|, \verb|C1W|, \verb|L1C|, \verb|L1W|).

Time-variable biases (e.g. GPS L5 satellite phase bias) can be provided via \config{timeVariableBias}.
Their time span will be based on the provided epochs ($t \pm \Delta t / 2$).
The slope of the bias can be optionally provided in the second data column.

If GLONASS receiver biases depend on frequency number, those must be defined in \configFile{inputfileTransmitterInfo}{gnssStationInfo}
to get the correct PRN/SVN assignment to the biases.

See IGS SINEX Bias format description for further details on header information.

See also \program{GnssSinexBias2SignalBias} and \program{GnssBiasClockAlignment}.


\keepXColumns
\begin{tabularx}{\textwidth}{N T A}
\hline
Name & Type & Annotation\\
\hline
\hfuzz=500pt\includegraphics[width=1em]{element-mustset.pdf}~outputfileSinexBias & \hfuzz=500pt filename & \hfuzz=500pt \\
\hfuzz=500pt\includegraphics[width=1em]{element-mustset-unbounded.pdf}~inputfileTransmitterInfo & \hfuzz=500pt filename & \hfuzz=500pt one file per satellite\\
\hfuzz=500pt\includegraphics[width=1em]{element-unbounded.pdf}~transmitterBiases & \hfuzz=500pt sequence & \hfuzz=500pt one element per satellite\\
\hfuzz=500pt\includegraphics[width=1em]{connector.pdf}\includegraphics[width=1em]{element-mustset.pdf}~inputfileSignalBias & \hfuzz=500pt filename & \hfuzz=500pt signal bias file\\
\hfuzz=500pt\includegraphics[width=1em]{connector.pdf}\includegraphics[width=1em]{element-unbounded.pdf}~timeVariableBias & \hfuzz=500pt sequence & \hfuzz=500pt one entry per time variable bias type\\
\hfuzz=500pt\quad\includegraphics[width=1em]{connector.pdf}\includegraphics[width=1em]{element-mustset.pdf}~inputfileSignalBias & \hfuzz=500pt filename & \hfuzz=500pt columns: mjd, bias [m], (biasSlope [m/s])\\
\hfuzz=500pt\quad\includegraphics[width=1em]{connector.pdf}\includegraphics[width=1em]{element-mustset.pdf}~type & \hfuzz=500pt \hyperref[gnssType]{gnssType} & \hfuzz=500pt bias type\\
\hfuzz=500pt\includegraphics[width=1em]{connector.pdf}\includegraphics[width=1em]{element-mustset.pdf}~identifier & \hfuzz=500pt string & \hfuzz=500pt PRN or station name (e.g. G23 or wtzz)\\
\hfuzz=500pt\includegraphics[width=1em]{element-unbounded.pdf}~receiverBiases & \hfuzz=500pt sequence & \hfuzz=500pt one element per station\\
\hfuzz=500pt\includegraphics[width=1em]{connector.pdf}\includegraphics[width=1em]{element-mustset.pdf}~inputfileSignalBias & \hfuzz=500pt filename & \hfuzz=500pt signal bias file\\
\hfuzz=500pt\includegraphics[width=1em]{connector.pdf}\includegraphics[width=1em]{element-unbounded.pdf}~timeVariableBias & \hfuzz=500pt sequence & \hfuzz=500pt one entry per time variable bias type\\
\hfuzz=500pt\quad\includegraphics[width=1em]{connector.pdf}\includegraphics[width=1em]{element-mustset.pdf}~inputfileSignalBias & \hfuzz=500pt filename & \hfuzz=500pt columns: mjd, bias [m], (biasSlope [m/s])\\
\hfuzz=500pt\quad\includegraphics[width=1em]{connector.pdf}\includegraphics[width=1em]{element-mustset.pdf}~type & \hfuzz=500pt \hyperref[gnssType]{gnssType} & \hfuzz=500pt bias type\\
\hfuzz=500pt\includegraphics[width=1em]{connector.pdf}\includegraphics[width=1em]{element-mustset.pdf}~identifier & \hfuzz=500pt string & \hfuzz=500pt PRN or station name (e.g. G23 or wtzz)\\
\hfuzz=500pt\includegraphics[width=1em]{element-mustset.pdf}~agencyCode & \hfuzz=500pt string & \hfuzz=500pt identify the agency providing the data\\
\hfuzz=500pt\includegraphics[width=1em]{element-mustset.pdf}~fileAgencyCode & \hfuzz=500pt string & \hfuzz=500pt identify the agency creating the file\\
\hfuzz=500pt\includegraphics[width=1em]{element-mustset.pdf}~timeStart & \hfuzz=500pt time & \hfuzz=500pt start time of the data\\
\hfuzz=500pt\includegraphics[width=1em]{element-mustset.pdf}~timeEnd & \hfuzz=500pt time & \hfuzz=500pt end time of the data \\
\hfuzz=500pt\includegraphics[width=1em]{element-mustset.pdf}~biasMode & \hfuzz=500pt choice & \hfuzz=500pt absolute or relative bias estimates\\
\hfuzz=500pt\includegraphics[width=1em]{connector.pdf}\includegraphics[width=1em]{element-mustset.pdf}~absolute & \hfuzz=500pt  & \hfuzz=500pt \\
\hfuzz=500pt\includegraphics[width=1em]{connector.pdf}\includegraphics[width=1em]{element-mustset.pdf}~relative & \hfuzz=500pt  & \hfuzz=500pt \\
\hfuzz=500pt\includegraphics[width=1em]{element.pdf}~observationSampling & \hfuzz=500pt uint & \hfuzz=500pt [seconds]\\
\hfuzz=500pt\includegraphics[width=1em]{element.pdf}~intervalLength & \hfuzz=500pt uint & \hfuzz=500pt [seconds] interval for bias parameter representation\\
\hfuzz=500pt\includegraphics[width=1em]{element-mustset.pdf}~determinationMethod & \hfuzz=500pt string & \hfuzz=500pt determination method used to generate the bias results (see SINEX Bias format description)\\
\hfuzz=500pt\includegraphics[width=1em]{element.pdf}~receiverClockReferenceGnss & \hfuzz=500pt string & \hfuzz=500pt (G, R, E, C) reference GNSS used for receiver clock estimation\\
\hfuzz=500pt\includegraphics[width=1em]{element-unbounded.pdf}~satelliteClockReferenceObservables & \hfuzz=500pt string & \hfuzz=500pt one per system, reference code observable on first and second frequency (RINEX3 format)\\
\hfuzz=500pt\includegraphics[width=1em]{element-mustset.pdf}~description & \hfuzz=500pt string & \hfuzz=500pt organizition gathering/altering the file contents\\
\hfuzz=500pt\includegraphics[width=1em]{element-mustset.pdf}~contact & \hfuzz=500pt string & \hfuzz=500pt contact name and/or email address\\
\hfuzz=500pt\includegraphics[width=1em]{element-mustset.pdf}~input & \hfuzz=500pt string & \hfuzz=500pt brief description of the input used to generate this solution\\
\hfuzz=500pt\includegraphics[width=1em]{element-mustset.pdf}~output & \hfuzz=500pt string & \hfuzz=500pt description of the file contents\\
\hfuzz=500pt\includegraphics[width=1em]{element-mustset.pdf}~software & \hfuzz=500pt string & \hfuzz=500pt software used to generate the file\\
\hfuzz=500pt\includegraphics[width=1em]{element-mustset.pdf}~hardware & \hfuzz=500pt string & \hfuzz=500pt computer hardware on which above software was run\\
\hfuzz=500pt\includegraphics[width=1em]{element-unbounded.pdf}~comment & \hfuzz=500pt string & \hfuzz=500pt comments in the comment block\\
\hline
\end{tabularx}

\clearpage
%==================================
\subsection{GnssSinexBias2SignalBias}\label{GnssSinexBias2SignalBias}
Converts GNSS signal biases from \href{https://files.igs.org/pub/data/format/sinex_bias_100.pdf}{IGS SINEX Bias format}
to \file{GnssSignalBias format}{gnssSignalBias}.

Only satellite observable-specific signal biases (OSB) are supported at the moment.
If multiple entries exist for the same bias, the weighted average (based on time span) of all entries is used.
Time-variable biases are not supported at the moment.

See also \program{GnssSignalBias2SinexBias}.


\keepXColumns
\begin{tabularx}{\textwidth}{N T A}
\hline
Name & Type & Annotation\\
\hline
\hfuzz=500pt\includegraphics[width=1em]{element-mustset.pdf}~outputfileSignalBias & \hfuzz=500pt filename & \hfuzz=500pt identifier is appended to file name\\
\hfuzz=500pt\includegraphics[width=1em]{element-mustset.pdf}~inputfileSinexBias & \hfuzz=500pt filename & \hfuzz=500pt \\
\hfuzz=500pt\includegraphics[width=1em]{element.pdf}~inputfileGlonassSignalDefinition & \hfuzz=500pt filename & \hfuzz=500pt GLONASS frequency number mapping\\
\hfuzz=500pt\includegraphics[width=1em]{element-unbounded.pdf}~identifier & \hfuzz=500pt string & \hfuzz=500pt (empty = all) satellite PRN, e.g. G23 or E05\\
\hline
\end{tabularx}

\clearpage
%==================================
\subsection{GnssStationLog2StationInfo}\label{GnssStationLog2StationInfo}
Converts \href{https://files.igs.org/pub/station/general/blank.log}{IGS station log format} to \configFile{outputfileStationInfo}{gnssStationInfo}.

If \configFile{inputfileAntennaDefinition}{gnssAntennaDefinition} is provided, station log data is cross-checked with the given antenna definitions.
Cross-checking station log data with a \href{https://www.iers.org/IERS/EN/Organization/AnalysisCoordinator/SinexFormat/sinex.html}{SINEX file} is also
possible by providing \config{inputfileSinex}. Any failed checks result in warnings in the output log.


\keepXColumns
\begin{tabularx}{\textwidth}{N T A}
\hline
Name & Type & Annotation\\
\hline
\hfuzz=500pt\includegraphics[width=1em]{element-mustset.pdf}~outputfileStationInfo & \hfuzz=500pt filename & \hfuzz=500pt \\
\hfuzz=500pt\includegraphics[width=1em]{element-mustset.pdf}~inputfileStationLog & \hfuzz=500pt filename & \hfuzz=500pt \\
\hfuzz=500pt\includegraphics[width=1em]{element.pdf}~inputfileAntennaDefinition & \hfuzz=500pt filename & \hfuzz=500pt used to check antennas\\
\hfuzz=500pt\includegraphics[width=1em]{element.pdf}~inputfileSinex & \hfuzz=500pt filename & \hfuzz=500pt used to cross-check station log with SINEX file\\
\hline
\end{tabularx}

\clearpage
%==================================
\subsection{GnssTroposphere2TropoSinex}\label{GnssTroposphere2TropoSinex}
Convert GNSS troposphere data from GROOPS format to \href{https://files.igs.org/pub/data/format/sinex_tro_v2.00.pdf}{IGS SINEX TRO} format.

Specification of the station list is done via \config{inputfileStationList}.
\config{inputfileTroposphereData} needs the troposphere data provided from \program{GnssProcessing}.
Additional following station metadata are required: \config{inputfileStationInfo} using file type \file{Station info}{gnssStationInfo},
\config{inputfileAntennaDefinition} using file type \file{Antenna definition}{gnssAntennaDefinition} and \config{inputfileGridPos} which
uses the stations positions provided from \program{GnssProcessing}.
For considering the geoid height use \config{inputfileGeoidHeight}. The geoid height is provided by \program{Gravityfield2GriddedData}.


\keepXColumns
\begin{tabularx}{\textwidth}{N T A}
\hline
Name & Type & Annotation\\
\hline
\hfuzz=500pt\includegraphics[width=1em]{element-mustset.pdf}~outputfileTropoSinex & \hfuzz=500pt filename & \hfuzz=500pt \\
\hfuzz=500pt\includegraphics[width=1em]{element-mustset.pdf}~stations & \hfuzz=500pt sequence & \hfuzz=500pt \\
\hfuzz=500pt\includegraphics[width=1em]{connector.pdf}\includegraphics[width=1em]{element-mustset.pdf}~inputfileStationList & \hfuzz=500pt filename & \hfuzz=500pt ASCII file with station names\\
\hfuzz=500pt\includegraphics[width=1em]{connector.pdf}\includegraphics[width=1em]{element-mustset.pdf}~inputfileTroposphereData & \hfuzz=500pt filename & \hfuzz=500pt Troposphere data estimates template (columns: mjd, trodry, trowet, tgndry, tgnwet, tgedry, tgewet)\\
\hfuzz=500pt\includegraphics[width=1em]{connector.pdf}\includegraphics[width=1em]{element.pdf}~inputfileTroposphereSigmas & \hfuzz=500pt filename & \hfuzz=500pt Troposphere data sigmas template (columns: mjd, sigma\_trowet, sigma\_tgnwet, sigma\_tgewet)\\
\hfuzz=500pt\includegraphics[width=1em]{connector.pdf}\includegraphics[width=1em]{element-mustset.pdf}~inputfileStationInfo & \hfuzz=500pt filename & \hfuzz=500pt station info file template\\
\hfuzz=500pt\includegraphics[width=1em]{connector.pdf}\includegraphics[width=1em]{element.pdf}~inputfileGeoidHeight & \hfuzz=500pt filename & \hfuzz=500pt File including geoid height\\
\hfuzz=500pt\includegraphics[width=1em]{connector.pdf}\includegraphics[width=1em]{element-mustset.pdf}~inputfileGridPos & \hfuzz=500pt filename & \hfuzz=500pt File including stations positions\\
\hfuzz=500pt\includegraphics[width=1em]{connector.pdf}\includegraphics[width=1em]{element-mustset.pdf}~inputfileAntennaDefinition & \hfuzz=500pt filename & \hfuzz=500pt station phase centers and variations (ANTEX)\\
\hfuzz=500pt\includegraphics[width=1em]{connector.pdf}\includegraphics[width=1em]{element.pdf}~variableStationName & \hfuzz=500pt filename & \hfuzz=500pt Loop variable for station names from station list\\
\hfuzz=500pt\includegraphics[width=1em]{connector.pdf}\includegraphics[width=1em]{element-mustset.pdf}~observationTimeStart & \hfuzz=500pt time & \hfuzz=500pt Start time for which solution has observations\\
\hfuzz=500pt\includegraphics[width=1em]{connector.pdf}\includegraphics[width=1em]{element-mustset.pdf}~observationTimeEnd & \hfuzz=500pt time & \hfuzz=500pt End time for which solution has observations\\
\hfuzz=500pt\includegraphics[width=1em]{element-mustset.pdf}~dataSamplingInterval & \hfuzz=500pt double & \hfuzz=500pt [sec] GNSS data sampling rate\\
\hfuzz=500pt\includegraphics[width=1em]{element-mustset.pdf}~tropoSamplingInterval & \hfuzz=500pt double & \hfuzz=500pt [sec] Tropospheric parameter sampling interval\\
\hfuzz=500pt\includegraphics[width=1em]{element-mustset.pdf}~tropoModelingMethod & \hfuzz=500pt string & \hfuzz=500pt Tropospheric estimation method: Filter, Smoother, Least Squares, Piece-Wise Linear Interpolation\\
\hfuzz=500pt\includegraphics[width=1em]{element-mustset.pdf}~aPrioriTropoModel & \hfuzz=500pt string & \hfuzz=500pt A priori tropospheric model used\\
\hfuzz=500pt\includegraphics[width=1em]{element-mustset.pdf}~tropoMappingFunction & \hfuzz=500pt string & \hfuzz=500pt Name of mapping function used for hydrostatic and wet delay\\
\hfuzz=500pt\includegraphics[width=1em]{element-mustset.pdf}~gradientMappingFunction & \hfuzz=500pt string & \hfuzz=500pt Name of mapping function used for gradients\\
\hfuzz=500pt\includegraphics[width=1em]{element.pdf}~metDataSource & \hfuzz=500pt string & \hfuzz=500pt source of surface meteorological observations used (see format desc.)\\
\hfuzz=500pt\includegraphics[width=1em]{element-mustset.pdf}~observationWeighting & \hfuzz=500pt string & \hfuzz=500pt observation weighting model applied\\
\hfuzz=500pt\includegraphics[width=1em]{element-mustset.pdf}~elevationCutoff & \hfuzz=500pt double & \hfuzz=500pt [deg]\\
\hfuzz=500pt\includegraphics[width=1em]{element-mustset.pdf}~gnssSystems & \hfuzz=500pt string & \hfuzz=500pt G=GPS, R=GLONASS, E=Galileo, C=BeiDou\\
\hfuzz=500pt\includegraphics[width=1em]{element-mustset.pdf}~timeSystem & \hfuzz=500pt string & \hfuzz=500pt G (GPS) or UTC\\
\hfuzz=500pt\includegraphics[width=1em]{element.pdf}~oceanTideModel & \hfuzz=500pt string & \hfuzz=500pt Ocean tide loading model applied\\
\hfuzz=500pt\includegraphics[width=1em]{element.pdf}~atmosphericTideModel & \hfuzz=500pt string & \hfuzz=500pt Atmospheric tide loading model applied\\
\hfuzz=500pt\includegraphics[width=1em]{element.pdf}~geoidModel & \hfuzz=500pt string & \hfuzz=500pt Geoid model name for undulation values\\
\hfuzz=500pt\includegraphics[width=1em]{element-mustset.pdf}~systemCode & \hfuzz=500pt string & \hfuzz=500pt Terrestrial reference system code\\
\hfuzz=500pt\includegraphics[width=1em]{element-mustset.pdf}~remark & \hfuzz=500pt string & \hfuzz=500pt Remark used to identify the origin of the coordinates (AC acronym)\\
\hfuzz=500pt\includegraphics[width=1em]{element-mustset.pdf}~antennaCalibrationModel & \hfuzz=500pt string & \hfuzz=500pt e.g. IGS14\_WWWW (WWWW = ANTEX release GPS week)\\
\hfuzz=500pt\includegraphics[width=1em]{element-mustset.pdf}~sinexTroHeader & \hfuzz=500pt sequence & \hfuzz=500pt \\
\hfuzz=500pt\includegraphics[width=1em]{connector.pdf}\includegraphics[width=1em]{element-mustset.pdf}~agencyCode & \hfuzz=500pt string & \hfuzz=500pt Identify the agency providing the data\\
\hfuzz=500pt\includegraphics[width=1em]{connector.pdf}\includegraphics[width=1em]{element-mustset.pdf}~timeStart & \hfuzz=500pt time & \hfuzz=500pt Start time of the data\\
\hfuzz=500pt\includegraphics[width=1em]{connector.pdf}\includegraphics[width=1em]{element-mustset.pdf}~timeEnd & \hfuzz=500pt time & \hfuzz=500pt End time of the data \\
\hfuzz=500pt\includegraphics[width=1em]{connector.pdf}\includegraphics[width=1em]{element-mustset.pdf}~observationCode & \hfuzz=500pt string & \hfuzz=500pt Technique used to generate the SINEX solution\\
\hfuzz=500pt\includegraphics[width=1em]{connector.pdf}\includegraphics[width=1em]{element-mustset.pdf}~solutionContents & \hfuzz=500pt string & \hfuzz=500pt Marker name if single station, MIX if multiple stations\\
\hfuzz=500pt\includegraphics[width=1em]{connector.pdf}\includegraphics[width=1em]{element-mustset.pdf}~description & \hfuzz=500pt string & \hfuzz=500pt Organizitions gathering/alerting the file contents\\
\hfuzz=500pt\includegraphics[width=1em]{connector.pdf}\includegraphics[width=1em]{element-mustset.pdf}~output & \hfuzz=500pt string & \hfuzz=500pt Description of the file contents\\
\hfuzz=500pt\includegraphics[width=1em]{connector.pdf}\includegraphics[width=1em]{element-mustset.pdf}~contact & \hfuzz=500pt string & \hfuzz=500pt Address of the relevant contact e-mail\\
\hfuzz=500pt\includegraphics[width=1em]{connector.pdf}\includegraphics[width=1em]{element-mustset.pdf}~software & \hfuzz=500pt string & \hfuzz=500pt Software used to generate the file\\
\hfuzz=500pt\includegraphics[width=1em]{connector.pdf}\includegraphics[width=1em]{element-mustset.pdf}~hardware & \hfuzz=500pt string & \hfuzz=500pt Computer hardware on which above software was run\\
\hfuzz=500pt\includegraphics[width=1em]{connector.pdf}\includegraphics[width=1em]{element-mustset.pdf}~input & \hfuzz=500pt string & \hfuzz=500pt Brief description of the input used to generate this solution\\
\hfuzz=500pt\includegraphics[width=1em]{connector.pdf}\includegraphics[width=1em]{element-mustset.pdf}~versionNumber & \hfuzz=500pt string & \hfuzz=500pt Unique identifier of the product, same as in file name, e.g. 000\\
\hfuzz=500pt\includegraphics[width=1em]{connector.pdf}\includegraphics[width=1em]{element.pdf}~inputfileComment & \hfuzz=500pt filename & \hfuzz=500pt comments in the comment block from a file (truncated at 80 characters per line)\\
\hfuzz=500pt\includegraphics[width=1em]{connector.pdf}\includegraphics[width=1em]{element-unbounded.pdf}~comment & \hfuzz=500pt string & \hfuzz=500pt comments in the comment block\\
\hline
\end{tabularx}

\clearpage
%==================================
\subsection{GoceXml2Gradiometer}\label{GoceXml2Gradiometer}
Read ESA XML GOCE Data.
The \config{outputfileGradiometer} is written as \file{instrument file (GRADIOMETER)}{instrument}.


\keepXColumns
\begin{tabularx}{\textwidth}{N T A}
\hline
Name & Type & Annotation\\
\hline
\hfuzz=500pt\includegraphics[width=1em]{element-mustset.pdf}~outputfileGradiometer & \hfuzz=500pt filename & \hfuzz=500pt \\
\hfuzz=500pt\includegraphics[width=1em]{element-mustset-unbounded.pdf}~inputfile & \hfuzz=500pt filename & \hfuzz=500pt \\
\hline
\end{tabularx}

\clearpage
%==================================
\subsection{GoceXml2Orbit}\label{GoceXml2Orbit}
Read ESA XML GOCE Data.


\keepXColumns
\begin{tabularx}{\textwidth}{N T A}
\hline
Name & Type & Annotation\\
\hline
\hfuzz=500pt\includegraphics[width=1em]{element-mustset.pdf}~outputfileOrbit & \hfuzz=500pt filename & \hfuzz=500pt \\
\hfuzz=500pt\includegraphics[width=1em]{element.pdf}~earthRotation & \hfuzz=500pt \hyperref[earthRotationType]{earthRotation} & \hfuzz=500pt rotation from TRF to CRF\\
\hfuzz=500pt\includegraphics[width=1em]{element-mustset-unbounded.pdf}~inputfile & \hfuzz=500pt filename & \hfuzz=500pt \\
\hline
\end{tabularx}

\clearpage
%==================================
\subsection{GoceXml2StarCamera}\label{GoceXml2StarCamera}
Read ESA XML GOCE Data.


\keepXColumns
\begin{tabularx}{\textwidth}{N T A}
\hline
Name & Type & Annotation\\
\hline
\hfuzz=500pt\includegraphics[width=1em]{element-mustset.pdf}~outputfileStarCamera & \hfuzz=500pt filename & \hfuzz=500pt \\
\hfuzz=500pt\includegraphics[width=1em]{element-mustset-unbounded.pdf}~inputfile & \hfuzz=500pt filename & \hfuzz=500pt \\
\hline
\end{tabularx}

\clearpage
%==================================
\subsection{GoceXmlEggNom1b}\label{GoceXmlEggNom1b}
Read ESA XML GOCE Data.


\keepXColumns
\begin{tabularx}{\textwidth}{N T A}
\hline
Name & Type & Annotation\\
\hline
\hfuzz=500pt\includegraphics[width=1em]{element.pdf}~outputfileGradiometer & \hfuzz=500pt filename & \hfuzz=500pt \\
\hfuzz=500pt\includegraphics[width=1em]{element.pdf}~outputfileStarCamera & \hfuzz=500pt filename & \hfuzz=500pt \\
\hfuzz=500pt\includegraphics[width=1em]{element.pdf}~outputfileAngularRate & \hfuzz=500pt filename & \hfuzz=500pt \\
\hfuzz=500pt\includegraphics[width=1em]{element.pdf}~outputfileAngularAcc & \hfuzz=500pt filename & \hfuzz=500pt \\
\hfuzz=500pt\includegraphics[width=1em]{element-mustset-unbounded.pdf}~inputfile & \hfuzz=500pt filename & \hfuzz=500pt \\
\hline
\end{tabularx}

\clearpage
%==================================
\subsection{Grace2PotentialCoefficients}\label{Grace2PotentialCoefficients}
This program converts potential coefficients from the GRACE SDS format
into \file{potential coefficients file}{potentialCoefficients}.
The program supports file formats for RL04 to RL06.

Within the program, the variables \verb|epochStart|, \verb|epochEnd| and \verb|epochMid|
are populated with the corresponding time-stamps in the file.
These can be used in to \configFile{outputfilePotentialCoefficients}{potentialCoefficients}
to auto-generate the file name.


\keepXColumns
\begin{tabularx}{\textwidth}{N T A}
\hline
Name & Type & Annotation\\
\hline
\hfuzz=500pt\includegraphics[width=1em]{element-mustset.pdf}~outputfilePotentialCoefficients & \hfuzz=500pt filename & \hfuzz=500pt variables: epochStart, epochEnd, epochMid\\
\hfuzz=500pt\includegraphics[width=1em]{element-mustset.pdf}~inputfile & \hfuzz=500pt filename & \hfuzz=500pt \\
\hline
\end{tabularx}

\clearpage
%==================================
\subsection{GraceAccelerometer2L1bAscii}\label{GraceAccelerometer2L1bAscii}
Convert GROOPS accelerometer files to the GRACE SDS L1B ASCII format.


\keepXColumns
\begin{tabularx}{\textwidth}{N T A}
\hline
Name & Type & Annotation\\
\hline
\hfuzz=500pt\includegraphics[width=1em]{element-mustset.pdf}~outputfileAscii & \hfuzz=500pt filename & \hfuzz=500pt ASCII outputfile\\
\hfuzz=500pt\includegraphics[width=1em]{element-mustset.pdf}~inputfileAccelerometer & \hfuzz=500pt filename & \hfuzz=500pt GROOPS acceleromter file\\
\hfuzz=500pt\includegraphics[width=1em]{element-mustset.pdf}~satelliteIdentifier & \hfuzz=500pt string & \hfuzz=500pt satellite identifier (A or B for GRACE, C or D for GRACE-FO)\\
\hfuzz=500pt\includegraphics[width=1em]{element-unbounded.pdf}~globalAttributes & \hfuzz=500pt string & \hfuzz=500pt additional attributes as 'key: value' pairs\\
\hline
\end{tabularx}

\clearpage
%==================================
\subsection{GraceAod2DoodsonHarmonics}\label{GraceAod2DoodsonHarmonics}
This program converts the atmospheric and ocean tidal products (AOD1B)
from the GRACE SDS format into \configFile{outputfileDoodsonHarmonics}{doodsonHarmonic}.


\keepXColumns
\begin{tabularx}{\textwidth}{N T A}
\hline
Name & Type & Annotation\\
\hline
\hfuzz=500pt\includegraphics[width=1em]{element-mustset.pdf}~outputfileDoodsonHarmonics & \hfuzz=500pt filename & \hfuzz=500pt \\
\hfuzz=500pt\includegraphics[width=1em]{element-mustset.pdf}~inputfileTideGeneratingPotential & \hfuzz=500pt filename & \hfuzz=500pt to compute Xi phase correction\\
\hfuzz=500pt\includegraphics[width=1em]{element-mustset-unbounded.pdf}~inputfile & \hfuzz=500pt filename & \hfuzz=500pt \\
\hline
\end{tabularx}

\clearpage
%==================================
\subsection{GraceAod2TimeSplines}\label{GraceAod2TimeSplines}
This program converts the atmospheric and ocean de-aliasing product (AOD1B)
from the GRACE SDS format into \file{time spline files}{timeSplinesGravityField}.
Multiple \config{inputfile}s must be given in the correct time order.
A linear method is assumed for the interpolation between the given points in time.

The GRACE SDS format is described in "AOD1B Product Description Document"
given at \url{http://podaac.jpl.nasa.gov/grace/documentation.html}.


\keepXColumns
\begin{tabularx}{\textwidth}{N T A}
\hline
Name & Type & Annotation\\
\hline
\hfuzz=500pt\includegraphics[width=1em]{element.pdf}~outputfileDealiasing & \hfuzz=500pt filename & \hfuzz=500pt \\
\hfuzz=500pt\includegraphics[width=1em]{element.pdf}~outputfileAtmosphere & \hfuzz=500pt filename & \hfuzz=500pt \\
\hfuzz=500pt\includegraphics[width=1em]{element.pdf}~outputfileOcean & \hfuzz=500pt filename & \hfuzz=500pt \\
\hfuzz=500pt\includegraphics[width=1em]{element.pdf}~outputfileBottomPressure & \hfuzz=500pt filename & \hfuzz=500pt \\
\hfuzz=500pt\includegraphics[width=1em]{element.pdf}~outputfileMisc & \hfuzz=500pt filename & \hfuzz=500pt \\
\hfuzz=500pt\includegraphics[width=1em]{element-mustset-unbounded.pdf}~inputfile & \hfuzz=500pt filename & \hfuzz=500pt \\
\hline
\end{tabularx}

\clearpage
%==================================
\subsection{GraceCoefficients2BlockMeanTimeSplines}\label{GraceCoefficients2BlockMeanTimeSplines}
This program converts potential coefficients from the GRACE SDS RL06 format
into \configFile{outputfileTimeSplines}{timeSplinesGravityField}.

The \configFile{outputfileTimeSeries}{instrument} contains the mid points
of non-empty intervals and \configFile{outputfileTimeIntervals}{instrument}
contains the monthly interval boundaries from first to last solution.

The output will always be monthly block means. If the SDS solutions do vary or overlap,
the nearest solution in terms of reference epoch is used.


\keepXColumns
\begin{tabularx}{\textwidth}{N T A}
\hline
Name & Type & Annotation\\
\hline
\hfuzz=500pt\includegraphics[width=1em]{element-mustset.pdf}~outputfileTimeSplines & \hfuzz=500pt filename & \hfuzz=500pt \\
\hfuzz=500pt\includegraphics[width=1em]{element.pdf}~outputfileTimeSplinesCovariance & \hfuzz=500pt filename & \hfuzz=500pt only the variances are saved\\
\hfuzz=500pt\includegraphics[width=1em]{element.pdf}~outputfileTimeSeries & \hfuzz=500pt filename & \hfuzz=500pt mid points of non-empty intervals\\
\hfuzz=500pt\includegraphics[width=1em]{element.pdf}~outputfileTimeIntervals & \hfuzz=500pt filename & \hfuzz=500pt monthly interval boundaries from first to last solution\\
\hfuzz=500pt\includegraphics[width=1em]{element-mustset-unbounded.pdf}~inputfile & \hfuzz=500pt filename & \hfuzz=500pt \\
\hline
\end{tabularx}

\clearpage
%==================================
\subsection{GraceL1a2Accelerometer}\label{GraceL1a2Accelerometer}
This program converts Level-1A accelerometer data (ACC1A) to the GROOPS instrument file format.
The GRACE Level-1A format is described in \verb|GRACEiolib.h| given at
\url{http://podaac-tools.jpl.nasa.gov/drive/files/allData/grace/sw/GraceReadSW_L1_2010-03-31.tar.gz}.
The output is one arc of satellite data which can include data gaps.
To split the arc in multiple gap free arcs use \program{InstrumentSynchronize}.


\keepXColumns
\begin{tabularx}{\textwidth}{N T A}
\hline
Name & Type & Annotation\\
\hline
\hfuzz=500pt\includegraphics[width=1em]{element.pdf}~outputfileAccelerometer & \hfuzz=500pt filename & \hfuzz=500pt ACCELEROMETER in SRF\\
\hfuzz=500pt\includegraphics[width=1em]{element.pdf}~outputfileAngularAccelerometer & \hfuzz=500pt filename & \hfuzz=500pt ACCELEROMETER in SRF\\
\hfuzz=500pt\includegraphics[width=1em]{element-mustset-unbounded.pdf}~inputfile & \hfuzz=500pt filename & \hfuzz=500pt ACC1A\\
\hline
\end{tabularx}

\clearpage
%==================================
\subsection{GraceL1a2SatelliteTracking}\label{GraceL1a2SatelliteTracking}
This program converts Level-1A satellite tracking data (KBR1A) to the GROOPS instrument file format.
The GRACE Level-1A format is described in \verb|GRACEiolib.h| given at
\url{http://podaac-tools.jpl.nasa.gov/drive/files/allData/grace/sw/GraceReadSW_L1_2010-03-31.tar.gz}.
The output is one arc of satellite data which can include data gaps.
To split the arc in multiple gap free arcs use \program{InstrumentSynchronize}.


\keepXColumns
\begin{tabularx}{\textwidth}{N T A}
\hline
Name & Type & Annotation\\
\hline
\hfuzz=500pt\includegraphics[width=1em]{element.pdf}~outputfileSatelliteTracking & \hfuzz=500pt filename & \hfuzz=500pt MISCVALUES(ant\_id, K\_phase, Ka\_phase, K\_SNR, Ka\_SNR)\\
\hfuzz=500pt\includegraphics[width=1em]{element-mustset-unbounded.pdf}~inputfile & \hfuzz=500pt filename & \hfuzz=500pt KBR1A\\
\hline
\end{tabularx}

\clearpage
%==================================
\subsection{GraceL1a2StarCamera}\label{GraceL1a2StarCamera}
This program converts orientation data measured by the star cameras
from the GRACE Level-1A format (SCA1A) to the GROOPS instrument file format.
For further information see \program{GraceL1a2Accelerometer}.


\keepXColumns
\begin{tabularx}{\textwidth}{N T A}
\hline
Name & Type & Annotation\\
\hline
\hfuzz=500pt\includegraphics[width=1em]{element-mustset.pdf}~outputfileStarCamera1 & \hfuzz=500pt filename & \hfuzz=500pt STARCAMERA1A, head 1\\
\hfuzz=500pt\includegraphics[width=1em]{element-mustset.pdf}~outputfileStarCamera2 & \hfuzz=500pt filename & \hfuzz=500pt STARCAMERA1A, head 2\\
\hfuzz=500pt\includegraphics[width=1em]{element-mustset-unbounded.pdf}~inputfile & \hfuzz=500pt filename & \hfuzz=500pt SCA1A, !GRACE-FO is not working!\\
\hline
\end{tabularx}

\clearpage
%==================================
\subsection{GraceL1a2Temperature}\label{GraceL1a2Temperature}
This program converts Level-1A temperature measurments (HRT1B or HRT1A) to the GROOPS instrument file format.
The GRACE Level-1A format is described in GRACE given at \url{http://podaac-tools.jpl.nasa.gov/drive/files/allData/grace/sw/GraceReadSW_L1_2010-03-31.tar.gz}.
Multiple \config{inputfile}s must be given in the correct time order.
The output is one arc of satellite data which can include data gaps.
To split the arc in multiple gap free arcs use \program{InstrumentSynchronize}.


\keepXColumns
\begin{tabularx}{\textwidth}{N T A}
\hline
Name & Type & Annotation\\
\hline
\hfuzz=500pt\includegraphics[width=1em]{element-mustset.pdf}~outputfileTemperature & \hfuzz=500pt filename & \hfuzz=500pt MISCVALUES\\
\hfuzz=500pt\includegraphics[width=1em]{element-mustset-unbounded.pdf}~inputfile & \hfuzz=500pt filename & \hfuzz=500pt HRT1B or HRT1A\\
\hline
\end{tabularx}

\clearpage
%==================================
\subsection{GraceL1b2AccHousekeeping}\label{GraceL1b2AccHousekeeping}
This program converts ACC housekeeping data (AHK1B or AHK1A) from the GRACE SDS format into \file{instrument file (ACCHOUSEKEEPING)}{instrument}.
For further information see \program{GraceL1b2Accelerometer}.


\keepXColumns
\begin{tabularx}{\textwidth}{N T A}
\hline
Name & Type & Annotation\\
\hline
\hfuzz=500pt\includegraphics[width=1em]{element-mustset.pdf}~outputfileAccHousekeeping & \hfuzz=500pt filename & \hfuzz=500pt ACCHOUSEKEEPING\\
\hfuzz=500pt\includegraphics[width=1em]{element-mustset-unbounded.pdf}~inputfile & \hfuzz=500pt filename & \hfuzz=500pt AHK1B or AHK1A\\
\hline
\end{tabularx}

\clearpage
%==================================
\subsection{GraceL1b2Accelerometer}\label{GraceL1b2Accelerometer}
This program converts accelerometer data (ACC1B or ACT1B) from the GRACE SDS format into \file{instrument file (ACCELEROMETER)}{instrument}.

Multiple \config{inputfile}s must be given in the correct time order.
The output is one arc of satellite data which can include data gaps.
To split the arc in multiple gap free arcs use \program{InstrumentSynchronize}.

The GRACE SDS format is described in "GRACE Level 1B Data Product User Handbook JPL D-22027"
given at \url{https://podaac-tools.jpl.nasa.gov/drive/files/allData/grace/docs/Handbook_1B_v1.3.pdf}.


\keepXColumns
\begin{tabularx}{\textwidth}{N T A}
\hline
Name & Type & Annotation\\
\hline
\hfuzz=500pt\includegraphics[width=1em]{element.pdf}~outputfileAccelerometer & \hfuzz=500pt filename & \hfuzz=500pt ACCELEROMETER\\
\hfuzz=500pt\includegraphics[width=1em]{element.pdf}~outputfileAngularAccelerometer & \hfuzz=500pt filename & \hfuzz=500pt ACCELEROMETER\\
\hfuzz=500pt\includegraphics[width=1em]{element.pdf}~outputfileFlags & \hfuzz=500pt filename & \hfuzz=500pt MISCVALUES(qualflg, acl\_res.x, acl\_res.y, acl\_res.z)\\
\hfuzz=500pt\includegraphics[width=1em]{element-mustset-unbounded.pdf}~inputfile & \hfuzz=500pt filename & \hfuzz=500pt ACC1B or ACT1B\\
\hline
\end{tabularx}

\clearpage
%==================================
\subsection{GraceL1b2ClockOffset}\label{GraceL1b2ClockOffset}
This program converts clock data (CLK1B or LLK1B) from the GRACE SDS format into \file{instrument file (MISCVALUE)}{instrument}.
For further information see \program{GraceL1b2Accelerometer}.


\keepXColumns
\begin{tabularx}{\textwidth}{N T A}
\hline
Name & Type & Annotation\\
\hline
\hfuzz=500pt\includegraphics[width=1em]{element-mustset.pdf}~outputfileClock & \hfuzz=500pt filename & \hfuzz=500pt MISCVALUE\\
\hfuzz=500pt\includegraphics[width=1em]{element-mustset-unbounded.pdf}~inputfile & \hfuzz=500pt filename & \hfuzz=500pt CLK1B or LLK1B\\
\hline
\end{tabularx}

\clearpage
%==================================
\subsection{GraceL1b2GnssReceiver}\label{GraceL1b2GnssReceiver}
This program converts GPS receiver data (phase and pseudo range) data
from the GRACE SDS format (GPS1B or GPS1A) into \file{instrument file (GNSSRECEIVER)}{instrument}.
For further information see \program{GraceL1b2Accelerometer}.


\keepXColumns
\begin{tabularx}{\textwidth}{N T A}
\hline
Name & Type & Annotation\\
\hline
\hfuzz=500pt\includegraphics[width=1em]{element-mustset.pdf}~outputfileGnssReceiver & \hfuzz=500pt filename & \hfuzz=500pt GNSSRECEIVER\\
\hfuzz=500pt\includegraphics[width=1em]{element-mustset-unbounded.pdf}~inputfile & \hfuzz=500pt filename & \hfuzz=500pt GPS1B or GPS1A\\
\hline
\end{tabularx}

\clearpage
%==================================
\subsection{GraceL1b2Magnetometer}\label{GraceL1b2Magnetometer}
This program converts magnetometer data (MAG1B or MAG1A) from the GRACE SDS format into \file{instrument file (MAGNETOMETER)}{instrument}.
For further information see \program{GraceL1b2Accelerometer}.


\keepXColumns
\begin{tabularx}{\textwidth}{N T A}
\hline
Name & Type & Annotation\\
\hline
\hfuzz=500pt\includegraphics[width=1em]{element-mustset.pdf}~outputfileMagnetometer & \hfuzz=500pt filename & \hfuzz=500pt MAGNETOMETER\\
\hfuzz=500pt\includegraphics[width=1em]{element-mustset-unbounded.pdf}~inputfile & \hfuzz=500pt filename & \hfuzz=500pt MAG1B or MAG1A\\
\hline
\end{tabularx}

\clearpage
%==================================
\subsection{GraceL1b2Mass}\label{GraceL1b2Mass}
This program converts mass data (MAS1B or MAS1A) from the GRACE SDS format into \file{instrument file (MASS)}{instrument}.
For further information see \program{GraceL1b2Accelerometer}.


\keepXColumns
\begin{tabularx}{\textwidth}{N T A}
\hline
Name & Type & Annotation\\
\hline
\hfuzz=500pt\includegraphics[width=1em]{element-mustset.pdf}~outputfileMass & \hfuzz=500pt filename & \hfuzz=500pt MASS\\
\hfuzz=500pt\includegraphics[width=1em]{element-mustset-unbounded.pdf}~inputfile & \hfuzz=500pt filename & \hfuzz=500pt MAS1B or MAS1A\\
\hline
\end{tabularx}

\clearpage
%==================================
\subsection{GraceL1b2Orbit}\label{GraceL1b2Orbit}
This program converts the reduced dynamical orbit
from the GRACE/GRACE-FO SDS format (GNV1B, GNI1B) into \file{instrument file (ORBIT)}{instrument}.

When GNV1B is used, the orbit can be rotated from the terrestrial reference frame (TRF) transformed into the celestial reference frame (CRF) by
specifying \configClass{earthRotation}{earthRotationType}.

For further information see \program{GraceL1b2Accelerometer}.


\keepXColumns
\begin{tabularx}{\textwidth}{N T A}
\hline
Name & Type & Annotation\\
\hline
\hfuzz=500pt\includegraphics[width=1em]{element-mustset.pdf}~outputfileOrbit & \hfuzz=500pt filename & \hfuzz=500pt \\
\hfuzz=500pt\includegraphics[width=1em]{element.pdf}~earthRotation & \hfuzz=500pt \hyperref[earthRotationType]{earthRotation} & \hfuzz=500pt to rotate GNV1B into CRF\\
\hfuzz=500pt\includegraphics[width=1em]{element-mustset-unbounded.pdf}~inputfile & \hfuzz=500pt filename & \hfuzz=500pt GNV1B/GNI1B\\
\hline
\end{tabularx}

\clearpage
%==================================
\subsection{GraceL1b2SatelliteTracking}\label{GraceL1b2SatelliteTracking}
This program converts low-low satellite data measured by the K-band ranging system
from the GRACE SDS format (KBR1B or LRI1B) into \file{instrument file (SATELLITETRACKING)}{instrument}.
The \config{inputfile}s contain also corrections to antenna offsets
and the so called light time correction. The corrections can be stored in additional files
in the same format as the observations.
If a phase break is found an artificial gap is created.
For further information see \program{GraceL1b2Accelerometer}.


\keepXColumns
\begin{tabularx}{\textwidth}{N T A}
\hline
Name & Type & Annotation\\
\hline
\hfuzz=500pt\includegraphics[width=1em]{element.pdf}~outputfileSatelliteTracking & \hfuzz=500pt filename & \hfuzz=500pt SATELLITETRACKING\\
\hfuzz=500pt\includegraphics[width=1em]{element.pdf}~outputfileAntCentr & \hfuzz=500pt filename & \hfuzz=500pt SATELLITETRACKING\\
\hfuzz=500pt\includegraphics[width=1em]{element.pdf}~outputfileLighttime & \hfuzz=500pt filename & \hfuzz=500pt SATELLITETRACKING\\
\hfuzz=500pt\includegraphics[width=1em]{element.pdf}~outputfileSNR & \hfuzz=500pt filename & \hfuzz=500pt MISCVALUES(K\_A\_SNR, Ka\_A\_SNR, K\_B\_SNR, Ka\_B\_SNR, qualflg)\\
\hfuzz=500pt\includegraphics[width=1em]{element.pdf}~outputfileIonoCorr & \hfuzz=500pt filename & \hfuzz=500pt MISCVALUE\\
\hfuzz=500pt\includegraphics[width=1em]{element-mustset-unbounded.pdf}~inputfile & \hfuzz=500pt filename & \hfuzz=500pt KBR1B or LRI1B\\
\hline
\end{tabularx}

\clearpage
%==================================
\subsection{GraceL1b2StarCamera}\label{GraceL1b2StarCamera}
This program converts orientation data measured by a star camera (SRF to CRF)
from the GRACE SDS format (SCA1B) into \file{instrument file (STARCAMERA)}{instrument}.
For further information see \program{GraceL1b2Accelerometer}.


\keepXColumns
\begin{tabularx}{\textwidth}{N T A}
\hline
Name & Type & Annotation\\
\hline
\hfuzz=500pt\includegraphics[width=1em]{element.pdf}~outputfileStarCamera & \hfuzz=500pt filename & \hfuzz=500pt \\
\hfuzz=500pt\includegraphics[width=1em]{element.pdf}~outputfileStarCameraFlags & \hfuzz=500pt filename & \hfuzz=500pt MISCVALUES(sca\_id, qual\_rss, qualflg)\\
\hfuzz=500pt\includegraphics[width=1em]{element-mustset-unbounded.pdf}~inputfile & \hfuzz=500pt filename & \hfuzz=500pt SCA1B\\
\hline
\end{tabularx}

\clearpage
%==================================
\subsection{GraceL1b2StarCameraCovariance}\label{GraceL1b2StarCameraCovariance}
This program computes star camera covariance matrices (\file{instrument file, COVARIANE3D}{instrument})
for a GRACE satellite under consideration of the active camera heads and an a priori variance factor.


\keepXColumns
\begin{tabularx}{\textwidth}{N T A}
\hline
Name & Type & Annotation\\
\hline
\hfuzz=500pt\includegraphics[width=1em]{element-mustset.pdf}~outputfileStarCameraCovariance & \hfuzz=500pt filename & \hfuzz=500pt \\
\hfuzz=500pt\includegraphics[width=1em]{element-mustset.pdf}~inputfileStarCameraFlags & \hfuzz=500pt filename & \hfuzz=500pt \\
\hfuzz=500pt\includegraphics[width=1em]{element.pdf}~inputfileSequenceOfEventsQSA & \hfuzz=500pt filename & \hfuzz=500pt \\
\hfuzz=500pt\includegraphics[width=1em]{element.pdf}~sigma0 & \hfuzz=500pt double & \hfuzz=500pt [seconds of arc]\\
\hline
\end{tabularx}

This program is \reference{parallelized}{general.parallelization}.
\clearpage
%==================================
\subsection{GraceL1b2SteeringMirror}\label{GraceL1b2SteeringMirror}
This program converts GRACE-FO Steering Mirror output (LSM1B) to an \file{instrument file (STARCAMERA)}{instrument}.


\keepXColumns
\begin{tabularx}{\textwidth}{N T A}
\hline
Name & Type & Annotation\\
\hline
\hfuzz=500pt\includegraphics[width=1em]{element.pdf}~outputfileStarCamera & \hfuzz=500pt filename & \hfuzz=500pt \\
\hfuzz=500pt\includegraphics[width=1em]{element-mustset-unbounded.pdf}~inputfile & \hfuzz=500pt filename & \hfuzz=500pt LSM1B\\
\hline
\end{tabularx}

\clearpage
%==================================
\subsection{GraceL1b2Thruster}\label{GraceL1b2Thruster}
This program converts thruster data (THR1B or THR1A) from the GRACE SDS format into \file{instrument file (THRUSTER)}{instrument}.
For further information see \program{GraceL1b2Accelerometer}.


\keepXColumns
\begin{tabularx}{\textwidth}{N T A}
\hline
Name & Type & Annotation\\
\hline
\hfuzz=500pt\includegraphics[width=1em]{element-mustset.pdf}~outputfileThruster & \hfuzz=500pt filename & \hfuzz=500pt THRUSTER\\
\hfuzz=500pt\includegraphics[width=1em]{element-mustset-unbounded.pdf}~inputfile & \hfuzz=500pt filename & \hfuzz=500pt THR1B or THR1A\\
\hline
\end{tabularx}

\clearpage
%==================================
\subsection{GraceL1b2TimeOffset}\label{GraceL1b2TimeOffset}
This program converts time data (TIM1A or TIM1B) from the GRACE SDS format into \file{instrument file (MISCVALUE)}{instrument}.
For further information see \program{GraceL1b2Accelerometer}.


\keepXColumns
\begin{tabularx}{\textwidth}{N T A}
\hline
Name & Type & Annotation\\
\hline
\hfuzz=500pt\includegraphics[width=1em]{element-mustset.pdf}~outputfileTime & \hfuzz=500pt filename & \hfuzz=500pt MISCVALUE\\
\hfuzz=500pt\includegraphics[width=1em]{element-mustset.pdf}~fractionalScale & \hfuzz=500pt double & \hfuzz=500pt 1e-6 for GRACE, 1e-9 for GRACE-FO\\
\hfuzz=500pt\includegraphics[width=1em]{element-mustset-unbounded.pdf}~inputfile & \hfuzz=500pt filename & \hfuzz=500pt TIM1A or TIM1B\\
\hline
\end{tabularx}

\clearpage
%==================================
\subsection{GraceL1b2Uso}\label{GraceL1b2Uso}
This program converts clock data (USO1B) from the GRACE SDS format into \file{instrument file (MISCVALUES)}{instrument}.
For further information see \program{GraceL1b2Accelerometer}.


\keepXColumns
\begin{tabularx}{\textwidth}{N T A}
\hline
Name & Type & Annotation\\
\hline
\hfuzz=500pt\includegraphics[width=1em]{element-mustset.pdf}~outputfileUso & \hfuzz=500pt filename & \hfuzz=500pt MISCVALUES(uso\_freq, K\_freq, Ka\_freq)\\
\hfuzz=500pt\includegraphics[width=1em]{element-mustset-unbounded.pdf}~inputfile & \hfuzz=500pt filename & \hfuzz=500pt USO1B\\
\hline
\end{tabularx}

\clearpage
%==================================
\subsection{GraceL1b2Vector}\label{GraceL1b2Vector}
This program reads vector orientation data (positions of instruments in the satellite frame) from the GRACE SDS format
(VGB1B, VGN1B, VGO1B, VKB1B, or VCM1B).
The \configFile{outputfileVector}{matrix} is a $(3n\times1)$ matrix containing $(x,y,z)$ for each record.
The GRACE SDS format is described in "GRACE Level 1B Data Product User Handbook JPL D-22027"
given at \url{http://podaac.jpl.nasa.gov/grace/documentation.html}.


\keepXColumns
\begin{tabularx}{\textwidth}{N T A}
\hline
Name & Type & Annotation\\
\hline
\hfuzz=500pt\includegraphics[width=1em]{element-mustset.pdf}~outputfileVector & \hfuzz=500pt filename & \hfuzz=500pt \\
\hfuzz=500pt\includegraphics[width=1em]{element-mustset.pdf}~inputfile & \hfuzz=500pt filename & \hfuzz=500pt VGB1B, VGN1B, VGO1B, VKB1B, or VCM1B\\
\hline
\end{tabularx}

\clearpage
%==================================
\subsection{GraceSequenceOfEvents}\label{GraceSequenceOfEvents}
This program converts the GRACE SOE (sequence of events) file/format into \file{instrument file (MISCVALUES)}{instrument}.
The GRACE SOE format is described in "GRACE Level 1B Data Product User Handbook JPL D-22027" and "TN-03\_SOE\_format.txt"
given at \url{http://podaac.jpl.nasa.gov/grace/documentation.html}.
The output is one arc of satellite data which can include data gaps.


\keepXColumns
\begin{tabularx}{\textwidth}{N T A}
\hline
Name & Type & Annotation\\
\hline
\hfuzz=500pt\includegraphics[width=1em]{element.pdf}~outputfileGraceA & \hfuzz=500pt filename & \hfuzz=500pt \\
\hfuzz=500pt\includegraphics[width=1em]{element.pdf}~outputfileGraceB & \hfuzz=500pt filename & \hfuzz=500pt \\
\hfuzz=500pt\includegraphics[width=1em]{element-mustset.pdf}~inputfile & \hfuzz=500pt filename & \hfuzz=500pt SoE file\\
\hfuzz=500pt\includegraphics[width=1em]{element-mustset.pdf}~events & \hfuzz=500pt choice & \hfuzz=500pt \\
\hfuzz=500pt\includegraphics[width=1em]{connector.pdf}\includegraphics[width=1em]{element-mustset.pdf}~ACCT & \hfuzz=500pt sequence & \hfuzz=500pt DSHL HeaterDisconnect\\
\hfuzz=500pt\quad\includegraphics[width=1em]{connector.pdf}\includegraphics[width=1em]{element.pdf}~mode & \hfuzz=500pt choice & \hfuzz=500pt \\
\hfuzz=500pt\quad\quad\includegraphics[width=1em]{connector.pdf}\includegraphics[width=1em]{element-mustset.pdf}~Heater & \hfuzz=500pt  & \hfuzz=500pt DSHL HeaterDisconnect\\
\hfuzz=500pt\quad\quad\includegraphics[width=1em]{connector.pdf}\includegraphics[width=1em]{element-mustset.pdf}~SetPoint & \hfuzz=500pt  & \hfuzz=500pt temperature set point\\
\hfuzz=500pt\includegraphics[width=1em]{connector.pdf}\includegraphics[width=1em]{element-mustset.pdf}~AOCS & \hfuzz=500pt sequence & \hfuzz=500pt coarse pointing mode or attitude hold mode\\
\hfuzz=500pt\quad\includegraphics[width=1em]{connector.pdf}\includegraphics[width=1em]{element.pdf}~mode & \hfuzz=500pt choice & \hfuzz=500pt \\
\hfuzz=500pt\quad\quad\includegraphics[width=1em]{connector.pdf}\includegraphics[width=1em]{element-mustset.pdf}~CPM & \hfuzz=500pt  & \hfuzz=500pt coarse pointing mode\\
\hfuzz=500pt\quad\quad\includegraphics[width=1em]{connector.pdf}\includegraphics[width=1em]{element-mustset.pdf}~AHM & \hfuzz=500pt  & \hfuzz=500pt attitude hold mode\\
\hfuzz=500pt\quad\quad\includegraphics[width=1em]{connector.pdf}\includegraphics[width=1em]{element-mustset.pdf}~SM & \hfuzz=500pt  & \hfuzz=500pt science mode\\
\hfuzz=500pt\includegraphics[width=1em]{connector.pdf}\includegraphics[width=1em]{element-mustset.pdf}~ACCR & \hfuzz=500pt  & \hfuzz=500pt ACCR\\
\hfuzz=500pt\includegraphics[width=1em]{connector.pdf}\includegraphics[width=1em]{element-mustset.pdf}~CMCAL & \hfuzz=500pt sequence & \hfuzz=500pt CoM calibration maneuver\\
\hfuzz=500pt\quad\includegraphics[width=1em]{connector.pdf}\includegraphics[width=1em]{element.pdf}~sampling & \hfuzz=500pt double & \hfuzz=500pt [seconds] create events between start and end of maneuver\\
\hfuzz=500pt\includegraphics[width=1em]{connector.pdf}\includegraphics[width=1em]{element-mustset.pdf}~KBRCAL & \hfuzz=500pt sequence & \hfuzz=500pt KBR calibration maneuver\\
\hfuzz=500pt\quad\includegraphics[width=1em]{connector.pdf}\includegraphics[width=1em]{element.pdf}~sampling & \hfuzz=500pt double & \hfuzz=500pt [seconds] create events between start and end of maneuver\\
\hfuzz=500pt\includegraphics[width=1em]{connector.pdf}\includegraphics[width=1em]{element-mustset.pdf}~VCM & \hfuzz=500pt  & \hfuzz=500pt CoM coordinates in SRF (m)\\
\hfuzz=500pt\includegraphics[width=1em]{connector.pdf}\includegraphics[width=1em]{element-mustset.pdf}~VKB & \hfuzz=500pt  & \hfuzz=500pt KBR phase center coordinates in SRF (m)\\
\hfuzz=500pt\includegraphics[width=1em]{connector.pdf}\includegraphics[width=1em]{element-mustset.pdf}~ICUVP & \hfuzz=500pt  & \hfuzz=500pt ICUVP\\
\hfuzz=500pt\includegraphics[width=1em]{connector.pdf}\includegraphics[width=1em]{element-mustset.pdf}~IPU & \hfuzz=500pt  & \hfuzz=500pt IPU\\
\hfuzz=500pt\includegraphics[width=1em]{connector.pdf}\includegraphics[width=1em]{element-mustset.pdf}~IPUR & \hfuzz=500pt  & \hfuzz=500pt IPUR\\
\hfuzz=500pt\includegraphics[width=1em]{connector.pdf}\includegraphics[width=1em]{element-mustset.pdf}~KAMI & \hfuzz=500pt  & \hfuzz=500pt KAMI: time tag offset to Ka-phase meas.\\
\hfuzz=500pt\includegraphics[width=1em]{connector.pdf}\includegraphics[width=1em]{element-mustset.pdf}~KMI & \hfuzz=500pt  & \hfuzz=500pt K\_MI: time tag offset to K-phase meas.\\
\hfuzz=500pt\includegraphics[width=1em]{connector.pdf}\includegraphics[width=1em]{element-mustset.pdf}~KTOFF & \hfuzz=500pt  & \hfuzz=500pt KTOFF: time tag offset to KBR meas.\\
\hfuzz=500pt\includegraphics[width=1em]{connector.pdf}\includegraphics[width=1em]{element-mustset.pdf}~MANV & \hfuzz=500pt  & \hfuzz=500pt MANV\\
\hfuzz=500pt\includegraphics[width=1em]{connector.pdf}\includegraphics[width=1em]{element-mustset.pdf}~MTE1 & \hfuzz=500pt  & \hfuzz=500pt MTE1\\
\hfuzz=500pt\includegraphics[width=1em]{connector.pdf}\includegraphics[width=1em]{element-mustset.pdf}~MTE2 & \hfuzz=500pt  & \hfuzz=500pt MTE2\\
\hfuzz=500pt\includegraphics[width=1em]{connector.pdf}\includegraphics[width=1em]{element-mustset.pdf}~OCC & \hfuzz=500pt  & \hfuzz=500pt OCC\\
\hfuzz=500pt\includegraphics[width=1em]{connector.pdf}\includegraphics[width=1em]{element-mustset.pdf}~QSA & \hfuzz=500pt  & \hfuzz=500pt SCA to SRF frame rotation\\
\hfuzz=500pt\includegraphics[width=1em]{connector.pdf}\includegraphics[width=1em]{element-mustset.pdf}~QKS & \hfuzz=500pt  & \hfuzz=500pt SCA to KBR frame rotation\\
\hline
\end{tabularx}

\clearpage
%==================================
\subsection{GrailCdr2Orbit}\label{GrailCdr2Orbit}
This program converts the orbit from the GRAIL SDS format into  \file{instrument file (ORBIT)}{instrument}.


\keepXColumns
\begin{tabularx}{\textwidth}{N T A}
\hline
Name & Type & Annotation\\
\hline
\hfuzz=500pt\includegraphics[width=1em]{element-mustset.pdf}~outputfileOrbit & \hfuzz=500pt filename & \hfuzz=500pt \\
\hfuzz=500pt\includegraphics[width=1em]{element-mustset-unbounded.pdf}~inputfile & \hfuzz=500pt filename & \hfuzz=500pt \\
\hline
\end{tabularx}

\clearpage
%==================================
\subsection{GrailCdr2SatelliteTracking}\label{GrailCdr2SatelliteTracking}
This program converts low-low satellite data measured by the K-band ranging system
from the GRAIL format into \file{instrument file (SATELLITETRACKING)}{instrument}.
The \config{inputfile}s contain also corrections for antenna offsets
and the so called light time correction.
The corrections can be stored in additional files in the same format as the observations.
If a phase break is found an artificial gap is created.


\keepXColumns
\begin{tabularx}{\textwidth}{N T A}
\hline
Name & Type & Annotation\\
\hline
\hfuzz=500pt\includegraphics[width=1em]{element.pdf}~outputfileSatelliteTracking & \hfuzz=500pt filename & \hfuzz=500pt \\
\hfuzz=500pt\includegraphics[width=1em]{element.pdf}~outputfileAntCentr & \hfuzz=500pt filename & \hfuzz=500pt \\
\hfuzz=500pt\includegraphics[width=1em]{element.pdf}~outputfileLighttime & \hfuzz=500pt filename & \hfuzz=500pt \\
\hfuzz=500pt\includegraphics[width=1em]{element.pdf}~outputfileTemperature & \hfuzz=500pt filename & \hfuzz=500pt \\
\hfuzz=500pt\includegraphics[width=1em]{element.pdf}~approximateTimeBias & \hfuzz=500pt double & \hfuzz=500pt [seconds]\\
\hfuzz=500pt\includegraphics[width=1em]{element-mustset-unbounded.pdf}~inputfile & \hfuzz=500pt filename & \hfuzz=500pt \\
\hline
\end{tabularx}

\clearpage
%==================================
\subsection{GrailCdr2StarCamera}\label{GrailCdr2StarCamera}
This program converts orientation data measured by a star camera (SRF to CRF)
from the GRAIL SDS format into \file{instrument file (STARCAMERA)}{instrument}.
For further information see \program{GraceL1b2Accelerometer}.


\keepXColumns
\begin{tabularx}{\textwidth}{N T A}
\hline
Name & Type & Annotation\\
\hline
\hfuzz=500pt\includegraphics[width=1em]{element.pdf}~outputfileStarCamera & \hfuzz=500pt filename & \hfuzz=500pt \\
\hfuzz=500pt\includegraphics[width=1em]{element-mustset-unbounded.pdf}~inputfile & \hfuzz=500pt filename & \hfuzz=500pt \\
\hline
\end{tabularx}

\clearpage
%==================================
\subsection{GridRectangular2NetCdf}\label{GridRectangular2NetCdf}
This program converts a sequence of \configFile{inputfileGridRectangular}{griddedData}
to a COARDS compliant NetCDF file.

See also \program{NetCdfInfo}, \program{NetCdf2GridRectangular}.


\keepXColumns
\begin{tabularx}{\textwidth}{N T A}
\hline
Name & Type & Annotation\\
\hline
\hfuzz=500pt\includegraphics[width=1em]{element-mustset.pdf}~outputfileNetCdf & \hfuzz=500pt filename & \hfuzz=500pt file name of NetCDF output\\
\hfuzz=500pt\includegraphics[width=1em]{element-mustset-unbounded.pdf}~inputfileGridRectangular & \hfuzz=500pt filename & \hfuzz=500pt input grid sequence\\
\hfuzz=500pt\includegraphics[width=1em]{element-unbounded.pdf}~times & \hfuzz=500pt \hyperref[timeSeriesType]{timeSeries} & \hfuzz=500pt values for time axis (COARDS specification)\\
\hfuzz=500pt\includegraphics[width=1em]{element-mustset-unbounded.pdf}~dataVariable & \hfuzz=500pt sequence & \hfuzz=500pt metadata for data variables\\
\hfuzz=500pt\includegraphics[width=1em]{connector.pdf}\includegraphics[width=1em]{element.pdf}~selectDataField & \hfuzz=500pt uint & \hfuzz=500pt input data column\\
\hfuzz=500pt\includegraphics[width=1em]{connector.pdf}\includegraphics[width=1em]{element-mustset.pdf}~name & \hfuzz=500pt string & \hfuzz=500pt netCDF variable name\\
\hfuzz=500pt\includegraphics[width=1em]{connector.pdf}\includegraphics[width=1em]{element-mustset.pdf}~dataType & \hfuzz=500pt choice & \hfuzz=500pt \\
\hfuzz=500pt\quad\includegraphics[width=1em]{connector.pdf}\includegraphics[width=1em]{element-mustset.pdf}~double & \hfuzz=500pt  & \hfuzz=500pt \\
\hfuzz=500pt\quad\includegraphics[width=1em]{connector.pdf}\includegraphics[width=1em]{element-mustset.pdf}~float & \hfuzz=500pt  & \hfuzz=500pt \\
\hfuzz=500pt\quad\includegraphics[width=1em]{connector.pdf}\includegraphics[width=1em]{element-mustset.pdf}~int & \hfuzz=500pt  & \hfuzz=500pt \\
\hfuzz=500pt\includegraphics[width=1em]{connector.pdf}\includegraphics[width=1em]{element-unbounded.pdf}~attribute & \hfuzz=500pt choice & \hfuzz=500pt netCDF attributes\\
\hfuzz=500pt\quad\includegraphics[width=1em]{connector.pdf}\includegraphics[width=1em]{element-mustset.pdf}~text & \hfuzz=500pt sequence & \hfuzz=500pt \\
\hfuzz=500pt\quad\quad\includegraphics[width=1em]{connector.pdf}\includegraphics[width=1em]{element-mustset.pdf}~name & \hfuzz=500pt string & \hfuzz=500pt \\
\hfuzz=500pt\quad\quad\includegraphics[width=1em]{connector.pdf}\includegraphics[width=1em]{element-mustset.pdf}~value & \hfuzz=500pt string & \hfuzz=500pt \\
\hfuzz=500pt\quad\includegraphics[width=1em]{connector.pdf}\includegraphics[width=1em]{element-mustset.pdf}~value & \hfuzz=500pt sequence & \hfuzz=500pt \\
\hfuzz=500pt\quad\quad\includegraphics[width=1em]{connector.pdf}\includegraphics[width=1em]{element-mustset.pdf}~name & \hfuzz=500pt string & \hfuzz=500pt \\
\hfuzz=500pt\quad\quad\includegraphics[width=1em]{connector.pdf}\includegraphics[width=1em]{element-mustset-unbounded.pdf}~value & \hfuzz=500pt double & \hfuzz=500pt \\
\hfuzz=500pt\quad\quad\includegraphics[width=1em]{connector.pdf}\includegraphics[width=1em]{element-mustset.pdf}~dataType & \hfuzz=500pt choice & \hfuzz=500pt \\
\hfuzz=500pt\quad\quad\quad\includegraphics[width=1em]{connector.pdf}\includegraphics[width=1em]{element-mustset.pdf}~double & \hfuzz=500pt  & \hfuzz=500pt \\
\hfuzz=500pt\quad\quad\quad\includegraphics[width=1em]{connector.pdf}\includegraphics[width=1em]{element-mustset.pdf}~float & \hfuzz=500pt  & \hfuzz=500pt \\
\hfuzz=500pt\quad\quad\quad\includegraphics[width=1em]{connector.pdf}\includegraphics[width=1em]{element-mustset.pdf}~int & \hfuzz=500pt  & \hfuzz=500pt \\
\hfuzz=500pt\includegraphics[width=1em]{element-unbounded.pdf}~globalAttribute & \hfuzz=500pt choice & \hfuzz=500pt additional meta data\\
\hfuzz=500pt\includegraphics[width=1em]{connector.pdf}\includegraphics[width=1em]{element-mustset.pdf}~text & \hfuzz=500pt sequence & \hfuzz=500pt \\
\hfuzz=500pt\quad\includegraphics[width=1em]{connector.pdf}\includegraphics[width=1em]{element-mustset.pdf}~name & \hfuzz=500pt string & \hfuzz=500pt \\
\hfuzz=500pt\quad\includegraphics[width=1em]{connector.pdf}\includegraphics[width=1em]{element-mustset.pdf}~value & \hfuzz=500pt string & \hfuzz=500pt \\
\hfuzz=500pt\includegraphics[width=1em]{connector.pdf}\includegraphics[width=1em]{element-mustset.pdf}~value & \hfuzz=500pt sequence & \hfuzz=500pt \\
\hfuzz=500pt\quad\includegraphics[width=1em]{connector.pdf}\includegraphics[width=1em]{element-mustset.pdf}~name & \hfuzz=500pt string & \hfuzz=500pt \\
\hfuzz=500pt\quad\includegraphics[width=1em]{connector.pdf}\includegraphics[width=1em]{element-mustset-unbounded.pdf}~value & \hfuzz=500pt double & \hfuzz=500pt \\
\hfuzz=500pt\quad\includegraphics[width=1em]{connector.pdf}\includegraphics[width=1em]{element-mustset.pdf}~dataType & \hfuzz=500pt choice & \hfuzz=500pt \\
\hfuzz=500pt\quad\quad\includegraphics[width=1em]{connector.pdf}\includegraphics[width=1em]{element-mustset.pdf}~double & \hfuzz=500pt  & \hfuzz=500pt \\
\hfuzz=500pt\quad\quad\includegraphics[width=1em]{connector.pdf}\includegraphics[width=1em]{element-mustset.pdf}~float & \hfuzz=500pt  & \hfuzz=500pt \\
\hfuzz=500pt\quad\quad\includegraphics[width=1em]{connector.pdf}\includegraphics[width=1em]{element-mustset.pdf}~int & \hfuzz=500pt  & \hfuzz=500pt \\
\hline
\end{tabularx}

\clearpage
%==================================
\subsection{GroopsAscii2Orbit}\label{GroopsAscii2Orbit}
Read Orbits given in groops kinematic orbit ASCII format with covariance information.

See also \program{Orbit2GroopsAscii}.


\keepXColumns
\begin{tabularx}{\textwidth}{N T A}
\hline
Name & Type & Annotation\\
\hline
\hfuzz=500pt\includegraphics[width=1em]{element.pdf}~outputfileOrbit & \hfuzz=500pt filename & \hfuzz=500pt \\
\hfuzz=500pt\includegraphics[width=1em]{element.pdf}~outputfileCovariance & \hfuzz=500pt filename & \hfuzz=500pt \\
\hfuzz=500pt\includegraphics[width=1em]{element-mustset.pdf}~earthRotation & \hfuzz=500pt \hyperref[earthRotationType]{earthRotation} & \hfuzz=500pt \\
\hfuzz=500pt\includegraphics[width=1em]{element-mustset-unbounded.pdf}~inputfile & \hfuzz=500pt filename & \hfuzz=500pt \\
\hline
\end{tabularx}

\clearpage
%==================================
\subsection{Hw2TideGeneratingPotential}\label{Hw2TideGeneratingPotential}
Write \file{tide generating potential}{tideGeneratingPotential}
from Hartmann and Wenzel 1995 file, \url{https://doi.org/10.1029/95GL03324}.


\keepXColumns
\begin{tabularx}{\textwidth}{N T A}
\hline
Name & Type & Annotation\\
\hline
\hfuzz=500pt\includegraphics[width=1em]{element-mustset.pdf}~outputfileTideGeneratingPotential & \hfuzz=500pt filename & \hfuzz=500pt \\
\hfuzz=500pt\includegraphics[width=1em]{element-mustset.pdf}~inputfile & \hfuzz=500pt filename & \hfuzz=500pt \\
\hfuzz=500pt\includegraphics[width=1em]{element.pdf}~headerLines & \hfuzz=500pt uint & \hfuzz=500pt skip number of header lines\\
\hfuzz=500pt\includegraphics[width=1em]{element.pdf}~referenceTime & \hfuzz=500pt time & \hfuzz=500pt reference time\\
\hline
\end{tabularx}

\clearpage
%==================================
\subsection{Icgem2PotentialCoefficients}\label{Icgem2PotentialCoefficients}
Read spherical harmonics in ICGEM format (\url{http://icgem.gfz-potsdam.de/}).


\keepXColumns
\begin{tabularx}{\textwidth}{N T A}
\hline
Name & Type & Annotation\\
\hline
\hfuzz=500pt\includegraphics[width=1em]{element-mustset.pdf}~outputfileStaticCoefficients & \hfuzz=500pt filename & \hfuzz=500pt static potential coefficients in GROOPS gfc format. Available variables (icgem2.0): epochStart, epochEnd, epochMid; (icgem1.0) epochReference\\
\hfuzz=500pt\includegraphics[width=1em]{element.pdf}~outputfileTrendCoefficients & \hfuzz=500pt filename & \hfuzz=500pt trend potential coefficients in GROOPS gfc format.  Available variables (icgem2.0): epochStart, epochEnd, epochMid; (icgem1.0) epochReference\\
\hfuzz=500pt\includegraphics[width=1em]{element.pdf}~outputfileOscillationCosine & \hfuzz=500pt filename & \hfuzz=500pt oscillation cosine coefficients in GROOPS gfc format. Available variables (icgem2.0): epochStart, epochEnd, epochMid, oscillationPeriod; (icgem1.0) epochReference, oscillationPeriod\\
\hfuzz=500pt\includegraphics[width=1em]{element.pdf}~outputfileOscillationSine & \hfuzz=500pt filename & \hfuzz=500pt oscillation sine coefficients in GROOPS gfc format. Available variables (icgem2.0): epochStart, epochEnd, epochMid, oscillationPeriod; (icgem1.0) epochReference, oscillationPeriod\\
\hfuzz=500pt\includegraphics[width=1em]{element.pdf}~outputfileIntervals & \hfuzz=500pt filename & \hfuzz=500pt two column ASCII file with all intervals found (only sensible for icgem2.0). The base name will be extended with .static, .trend, .annualCos, and .annualSin.\\
\hfuzz=500pt\includegraphics[width=1em]{element-mustset.pdf}~inputfileIcgem & \hfuzz=500pt filename & \hfuzz=500pt ICGEM GFC file\\
\hfuzz=500pt\includegraphics[width=1em]{element.pdf}~useFormalErrors & \hfuzz=500pt boolean & \hfuzz=500pt use formal errors if both formal and calibrated errors are given\\
\hline
\end{tabularx}

\clearpage
%==================================
\subsection{Iers2OceanPoleTide}\label{Iers2OceanPoleTide}
Read ocean pole tide model according to IERS conventions
and convert into \file{oceanPoleTide file}{oceanPoleTide}.


\keepXColumns
\begin{tabularx}{\textwidth}{N T A}
\hline
Name & Type & Annotation\\
\hline
\hfuzz=500pt\includegraphics[width=1em]{element-mustset.pdf}~outputfileOceanPole & \hfuzz=500pt filename & \hfuzz=500pt \\
\hfuzz=500pt\includegraphics[width=1em]{element-mustset.pdf}~inputfile & \hfuzz=500pt filename & \hfuzz=500pt \\
\hfuzz=500pt\includegraphics[width=1em]{element-mustset.pdf}~inputfileLoadingLoveNumber & \hfuzz=500pt filename & \hfuzz=500pt \\
\hfuzz=500pt\includegraphics[width=1em]{element-mustset.pdf}~maxDegree & \hfuzz=500pt uint & \hfuzz=500pt \\
\hfuzz=500pt\includegraphics[width=1em]{element.pdf}~GM & \hfuzz=500pt double & \hfuzz=500pt Geocentric gravitational constant\\
\hfuzz=500pt\includegraphics[width=1em]{element.pdf}~R & \hfuzz=500pt double & \hfuzz=500pt Reference radius\\
\hfuzz=500pt\includegraphics[width=1em]{element.pdf}~Omega & \hfuzz=500pt double & \hfuzz=500pt [rad/s] earth rotation\\
\hfuzz=500pt\includegraphics[width=1em]{element.pdf}~rho & \hfuzz=500pt double & \hfuzz=500pt [kg/m**3] density of sea water\\
\hfuzz=500pt\includegraphics[width=1em]{element.pdf}~G & \hfuzz=500pt double & \hfuzz=500pt [m**3/(kg*s**2)] gravitational constant\\
\hfuzz=500pt\includegraphics[width=1em]{element.pdf}~g & \hfuzz=500pt double & \hfuzz=500pt [m/s**2] gravity\\
\hline
\end{tabularx}

\clearpage
%==================================
\subsection{IersC04IAU2000EarthOrientationParameter}\label{IersC04IAU2000EarthOrientationParameter}
Read a IERS Earth orientation data C04 (IAU2000A) file
and write it as \configFile{outputfileEOP}{earthOrientationParameter}.


\keepXColumns
\begin{tabularx}{\textwidth}{N T A}
\hline
Name & Type & Annotation\\
\hline
\hfuzz=500pt\includegraphics[width=1em]{element-mustset.pdf}~outputfileEOP & \hfuzz=500pt filename & \hfuzz=500pt \\
\hfuzz=500pt\includegraphics[width=1em]{element-mustset.pdf}~inputfile & \hfuzz=500pt filename & \hfuzz=500pt \\
\hfuzz=500pt\includegraphics[width=1em]{element.pdf}~timeStart & \hfuzz=500pt time & \hfuzz=500pt \\
\hfuzz=500pt\includegraphics[width=1em]{element.pdf}~timeEnd & \hfuzz=500pt time & \hfuzz=500pt \\
\hline
\end{tabularx}

\clearpage
%==================================
\subsection{IersHighFrequentEop2DoodsonEop}\label{IersHighFrequentEop2DoodsonEop}
Read Diurnal and Subdiurnal Earth Orientation variations according to updated IERS 2010 conventions
and write them as \configFile{outputfileDoodsonEOP}{doodsonEarthOrientationParameter}.


\keepXColumns
\begin{tabularx}{\textwidth}{N T A}
\hline
Name & Type & Annotation\\
\hline
\hfuzz=500pt\includegraphics[width=1em]{element-mustset.pdf}~outputfileDoodsonEOP & \hfuzz=500pt filename & \hfuzz=500pt \\
\hfuzz=500pt\includegraphics[width=1em]{element-mustset.pdf}~inputfile & \hfuzz=500pt filename & \hfuzz=500pt \\
\hline
\end{tabularx}

\clearpage
%==================================
\subsection{IersPotential2DoodsonHarmonics}\label{IersPotential2DoodsonHarmonics}
Read ocean tide file in IERS format.


\keepXColumns
\begin{tabularx}{\textwidth}{N T A}
\hline
Name & Type & Annotation\\
\hline
\hfuzz=500pt\includegraphics[width=1em]{element-mustset.pdf}~outputfileDoodsonHarmoncis & \hfuzz=500pt filename & \hfuzz=500pt \\
\hfuzz=500pt\includegraphics[width=1em]{element-mustset.pdf}~inputfile & \hfuzz=500pt filename & \hfuzz=500pt \\
\hfuzz=500pt\includegraphics[width=1em]{element-mustset.pdf}~headerLines & \hfuzz=500pt uint & \hfuzz=500pt skip number of header lines\\
\hfuzz=500pt\includegraphics[width=1em]{element.pdf}~minDegree & \hfuzz=500pt uint & \hfuzz=500pt \\
\hfuzz=500pt\includegraphics[width=1em]{element-mustset.pdf}~maxDegree & \hfuzz=500pt uint & \hfuzz=500pt \\
\hfuzz=500pt\includegraphics[width=1em]{element.pdf}~GM & \hfuzz=500pt double & \hfuzz=500pt Geocentric gravitational constant\\
\hfuzz=500pt\includegraphics[width=1em]{element.pdf}~R & \hfuzz=500pt double & \hfuzz=500pt reference radius\\
\hline
\end{tabularx}

\clearpage
%==================================
\subsection{IersRapidIAU2000EarthOrientationParameter}\label{IersRapidIAU2000EarthOrientationParameter}
Read a IERS Earth orientation rapid data and prediction file (IAU2000)
and write it as \configFile{outputfileEOP}{earthOrientationParameter}.


\keepXColumns
\begin{tabularx}{\textwidth}{N T A}
\hline
Name & Type & Annotation\\
\hline
\hfuzz=500pt\includegraphics[width=1em]{element-mustset.pdf}~outputfileEOP & \hfuzz=500pt filename & \hfuzz=500pt \\
\hfuzz=500pt\includegraphics[width=1em]{element-mustset.pdf}~inputfile & \hfuzz=500pt filename & \hfuzz=500pt \\
\hfuzz=500pt\includegraphics[width=1em]{element.pdf}~timeStart & \hfuzz=500pt time & \hfuzz=500pt \\
\hfuzz=500pt\includegraphics[width=1em]{element.pdf}~timeEnd & \hfuzz=500pt time & \hfuzz=500pt \\
\hline
\end{tabularx}

\clearpage
%==================================
\subsection{IersWaterHeight2DoodsonHarmonics}\label{IersWaterHeight2DoodsonHarmonics}
Read ocean tide file in IERS format.


\keepXColumns
\begin{tabularx}{\textwidth}{N T A}
\hline
Name & Type & Annotation\\
\hline
\hfuzz=500pt\includegraphics[width=1em]{element-mustset.pdf}~outputfileDoodsonHarmoncis & \hfuzz=500pt filename & \hfuzz=500pt \\
\hfuzz=500pt\includegraphics[width=1em]{element-mustset.pdf}~inputfile & \hfuzz=500pt filename & \hfuzz=500pt \\
\hfuzz=500pt\includegraphics[width=1em]{element-mustset.pdf}~headerLines & \hfuzz=500pt uint & \hfuzz=500pt skip number of header lines\\
\hfuzz=500pt\includegraphics[width=1em]{element-mustset.pdf}~inputfileTideGeneratingPotential & \hfuzz=500pt filename & \hfuzz=500pt to compute Xi phase correction\\
\hfuzz=500pt\includegraphics[width=1em]{element-mustset.pdf}~kernel & \hfuzz=500pt \hyperref[kernelType]{kernel} & \hfuzz=500pt data type of input values\\
\hfuzz=500pt\includegraphics[width=1em]{element.pdf}~factor & \hfuzz=500pt double & \hfuzz=500pt to convert in SI units\\
\hfuzz=500pt\includegraphics[width=1em]{element.pdf}~minDegree & \hfuzz=500pt uint & \hfuzz=500pt \\
\hfuzz=500pt\includegraphics[width=1em]{element-mustset.pdf}~maxDegree & \hfuzz=500pt uint & \hfuzz=500pt \\
\hfuzz=500pt\includegraphics[width=1em]{element.pdf}~GM & \hfuzz=500pt double & \hfuzz=500pt Geocentric gravitational constant\\
\hfuzz=500pt\includegraphics[width=1em]{element.pdf}~R & \hfuzz=500pt double & \hfuzz=500pt reference radius\\
\hline
\end{tabularx}

\clearpage
%==================================
\subsection{Igs2EarthOrientationParameter}\label{Igs2EarthOrientationParameter}
Read Rapid Earth Orientation Parameter from IGS daily file
and write it as \configFile{outputfileEOP}{earthOrientationParameter}.


\keepXColumns
\begin{tabularx}{\textwidth}{N T A}
\hline
Name & Type & Annotation\\
\hline
\hfuzz=500pt\includegraphics[width=1em]{element-mustset.pdf}~outputfileEOP & \hfuzz=500pt filename & \hfuzz=500pt \\
\hfuzz=500pt\includegraphics[width=1em]{element-mustset-unbounded.pdf}~inputfile & \hfuzz=500pt filename & \hfuzz=500pt \\
\hfuzz=500pt\includegraphics[width=1em]{element.pdf}~timeStart & \hfuzz=500pt time & \hfuzz=500pt \\
\hfuzz=500pt\includegraphics[width=1em]{element.pdf}~timeEnd & \hfuzz=500pt time & \hfuzz=500pt \\
\hline
\end{tabularx}

\clearpage
%==================================
\subsection{Jason2Starcamera}\label{Jason2Starcamera}
This program reads in Jason star camera data given in a special  format.
Files available at: \url{cddis.gsfc.nasa.gov/pub/doris/ancillary/quaternions/ja2/}.
A description of the format can be found under:
\url{ftp://ftp.ids-doris.org/pub/ids/ancillary/quaternions/jason1_2_quaternion_solar_panel.pdf}


\keepXColumns
\begin{tabularx}{\textwidth}{N T A}
\hline
Name & Type & Annotation\\
\hline
\hfuzz=500pt\includegraphics[width=1em]{element-mustset.pdf}~outputfileStarCamera & \hfuzz=500pt filename & \hfuzz=500pt \\
\hfuzz=500pt\includegraphics[width=1em]{element-mustset.pdf}~jasonNumber & \hfuzz=500pt uint & \hfuzz=500pt Jason number (different file format), 1 for Sentinel\\
\hfuzz=500pt\includegraphics[width=1em]{element-mustset-unbounded.pdf}~inputfile & \hfuzz=500pt filename & \hfuzz=500pt \\
\hline
\end{tabularx}

\clearpage
%==================================
\subsection{JplAscii2Ephemerides}\label{JplAscii2Ephemerides}
Read JPL DExxx (ASCII) ephemerides.


\keepXColumns
\begin{tabularx}{\textwidth}{N T A}
\hline
Name & Type & Annotation\\
\hline
\hfuzz=500pt\includegraphics[width=1em]{element-mustset.pdf}~outputfileEphemerides & \hfuzz=500pt filename & \hfuzz=500pt \\
\hfuzz=500pt\includegraphics[width=1em]{element-mustset.pdf}~inputfileHeader & \hfuzz=500pt filename & \hfuzz=500pt \\
\hfuzz=500pt\includegraphics[width=1em]{element-mustset-unbounded.pdf}~inputfileData & \hfuzz=500pt filename & \hfuzz=500pt \\
\hline
\end{tabularx}

\clearpage
%==================================
\subsection{Metop2Starcamera}\label{Metop2Starcamera}
This program reads in star camera data from MetOp satellites given in the special CHAMP format.
A description of the format can be found under: \url{http://op.gfz-potsdam.de/champ/docs_CHAMP/CH-GFZ-FD-001.pdf}


\keepXColumns
\begin{tabularx}{\textwidth}{N T A}
\hline
Name & Type & Annotation\\
\hline
\hfuzz=500pt\includegraphics[width=1em]{element-mustset.pdf}~outputfileStarCamera & \hfuzz=500pt filename & \hfuzz=500pt \\
\hfuzz=500pt\includegraphics[width=1em]{element-mustset-unbounded.pdf}~inputfile & \hfuzz=500pt filename & \hfuzz=500pt \\
\hline
\end{tabularx}

\clearpage
%==================================
\subsection{NetCdf2GridRectangular}\label{NetCdf2GridRectangular}
This program converts a COARDS compliant NetCDF file into a sequence of
\configFile{outputfileGridRectangular}{griddedData}.

See also \program{NetCdfInfo}, \program{GridRectangular2NetCdf}.


\keepXColumns
\begin{tabularx}{\textwidth}{N T A}
\hline
Name & Type & Annotation\\
\hline
\hfuzz=500pt\includegraphics[width=1em]{element-mustset.pdf}~outputfileGridRectangular & \hfuzz=500pt filename & \hfuzz=500pt One grid for each epoch in the NetCDF file is written. Use loopTimeVariable as template.\\
\hfuzz=500pt\includegraphics[width=1em]{element-mustset.pdf}~loopTimeVariable & \hfuzz=500pt string & \hfuzz=500pt \\
\hfuzz=500pt\includegraphics[width=1em]{element-mustset.pdf}~inputfileNetCdf & \hfuzz=500pt filename & \hfuzz=500pt \\
\hfuzz=500pt\includegraphics[width=1em]{element-mustset.pdf}~variableNameLongitude & \hfuzz=500pt string & \hfuzz=500pt name of NetCDF variable\\
\hfuzz=500pt\includegraphics[width=1em]{element-mustset.pdf}~variableNameLatitude & \hfuzz=500pt string & \hfuzz=500pt name of NetCDF variable\\
\hfuzz=500pt\includegraphics[width=1em]{element.pdf}~variableNameTime & \hfuzz=500pt string & \hfuzz=500pt name of NetCDF variable (leave blank for static grids)\\
\hfuzz=500pt\includegraphics[width=1em]{element-unbounded.pdf}~variableNameData & \hfuzz=500pt string & \hfuzz=500pt name of NetCDF variable\\
\hfuzz=500pt\includegraphics[width=1em]{element.pdf}~R & \hfuzz=500pt double & \hfuzz=500pt reference radius for ellipsoidal coordinates\\
\hfuzz=500pt\includegraphics[width=1em]{element.pdf}~inverseFlattening & \hfuzz=500pt double & \hfuzz=500pt reference flattening for ellipsoidal coordinates\\
\hline
\end{tabularx}

\clearpage
%==================================
\subsection{NetCdf2PotentialCoefficients}\label{NetCdf2PotentialCoefficients}
This program converts a COARDS compliant NetCDF file into potential coefficients by least squares.
If multiple \config{variableNameData} are given the grids values are accumulated before the adjustment.

See also \program{NetCdfInfo}, \program{NetCdf2GridRectangular}.


\keepXColumns
\begin{tabularx}{\textwidth}{N T A}
\hline
Name & Type & Annotation\\
\hline
\hfuzz=500pt\includegraphics[width=1em]{element-mustset.pdf}~outputfilePotentialCoefficients & \hfuzz=500pt filename & \hfuzz=500pt One file for each epoch in the NetCDF file. Use loopTimeVariable as template.\\
\hfuzz=500pt\includegraphics[width=1em]{element-mustset.pdf}~loopTimeVariable & \hfuzz=500pt string & \hfuzz=500pt \\
\hfuzz=500pt\includegraphics[width=1em]{element-mustset.pdf}~inputfileNetCdf & \hfuzz=500pt filename & \hfuzz=500pt \\
\hfuzz=500pt\includegraphics[width=1em]{element-mustset.pdf}~variableNameLongitude & \hfuzz=500pt string & \hfuzz=500pt name of NetCDF variable\\
\hfuzz=500pt\includegraphics[width=1em]{element-mustset.pdf}~variableNameLatitude & \hfuzz=500pt string & \hfuzz=500pt name of NetCDF variable\\
\hfuzz=500pt\includegraphics[width=1em]{element.pdf}~variableNameTime & \hfuzz=500pt string & \hfuzz=500pt name of NetCDF variable (leave blank for static grids)\\
\hfuzz=500pt\includegraphics[width=1em]{element-unbounded.pdf}~variableNameData & \hfuzz=500pt string & \hfuzz=500pt name of NetCDF variable\\
\hfuzz=500pt\includegraphics[width=1em]{element.pdf}~noDataValue & \hfuzz=500pt double & \hfuzz=500pt no data value for data variables\\
\hfuzz=500pt\includegraphics[width=1em]{element.pdf}~semimajorAxis & \hfuzz=500pt double & \hfuzz=500pt reference semimajor axis for ellipsoidal coordinates\\
\hfuzz=500pt\includegraphics[width=1em]{element.pdf}~inverseFlattening & \hfuzz=500pt double & \hfuzz=500pt reference flattening for ellipsoidal coordinates\\
\hfuzz=500pt\includegraphics[width=1em]{element-mustset.pdf}~kernel & \hfuzz=500pt \hyperref[kernelType]{kernel} & \hfuzz=500pt kernel in which the grid values are given\\
\hfuzz=500pt\includegraphics[width=1em]{element.pdf}~minDegree & \hfuzz=500pt uint & \hfuzz=500pt \\
\hfuzz=500pt\includegraphics[width=1em]{element-mustset.pdf}~maxDegree & \hfuzz=500pt uint & \hfuzz=500pt \\
\hfuzz=500pt\includegraphics[width=1em]{element.pdf}~GM & \hfuzz=500pt double & \hfuzz=500pt Geocentric gravitational constant\\
\hfuzz=500pt\includegraphics[width=1em]{element.pdf}~R & \hfuzz=500pt double & \hfuzz=500pt reference radius for potential coefficients\\
\hline
\end{tabularx}

This program is \reference{parallelized}{general.parallelization}.
\clearpage
%==================================
\subsection{NetCdfInfo}\label{NetCdfInfo}
Print content information of a NetCDF file like
dimensions, variables and attributes.

See also \program{NetCdf2GridRectangular}, \program{GridRectangular2NetCdf}.


\keepXColumns
\begin{tabularx}{\textwidth}{N T A}
\hline
Name & Type & Annotation\\
\hline
\hfuzz=500pt\includegraphics[width=1em]{element-mustset.pdf}~inputfileNetCdf & \hfuzz=500pt filename & \hfuzz=500pt \\
\hline
\end{tabularx}

\clearpage
%==================================
\subsection{NormalsSphericalHarmonics2Sinex}\label{NormalsSphericalHarmonics2Sinex}
Write potential coefficients and \file{normal equations}{normalEquation} to
\href{http://www.iers.org/IERS/EN/Organization/AnalysisCoordinator/SinexFormat/sinex.html}{SINEX format}.

See also \program{Sinex2Normals} and \program{GnssNormals2Sinex}.


\keepXColumns
\begin{tabularx}{\textwidth}{N T A}
\hline
Name & Type & Annotation\\
\hline
\hfuzz=500pt\includegraphics[width=1em]{element-mustset.pdf}~outputfileSinex & \hfuzz=500pt filename & \hfuzz=500pt solutions in SINEX format\\
\hfuzz=500pt\includegraphics[width=1em]{element-mustset.pdf}~inputfileNormals & \hfuzz=500pt filename & \hfuzz=500pt normal equation matrix\\
\hfuzz=500pt\includegraphics[width=1em]{element.pdf}~inputfileSolution & \hfuzz=500pt filename & \hfuzz=500pt parameter vector\\
\hfuzz=500pt\includegraphics[width=1em]{element.pdf}~inputfileSigmax & \hfuzz=500pt filename & \hfuzz=500pt standard deviations of the parameters (sqrt of the diagonal of the inverse normal equation)\\
\hfuzz=500pt\includegraphics[width=1em]{element-mustset.pdf}~inputfileApriori & \hfuzz=500pt filename & \hfuzz=500pt apriori parameter vector\\
\hfuzz=500pt\includegraphics[width=1em]{element.pdf}~inputfileAprioriMatrix & \hfuzz=500pt filename & \hfuzz=500pt normal equation matrix of applied constraints\\
\hfuzz=500pt\includegraphics[width=1em]{element-mustset.pdf}~time & \hfuzz=500pt time & \hfuzz=500pt reference time for parameters\\
\hfuzz=500pt\includegraphics[width=1em]{element-mustset.pdf}~sinexHeader & \hfuzz=500pt sequence & \hfuzz=500pt \\
\hfuzz=500pt\includegraphics[width=1em]{connector.pdf}\includegraphics[width=1em]{element.pdf}~agencyCode & \hfuzz=500pt string & \hfuzz=500pt identify the agency providing the data\\
\hfuzz=500pt\includegraphics[width=1em]{connector.pdf}\includegraphics[width=1em]{element.pdf}~timeStart & \hfuzz=500pt time & \hfuzz=500pt start time of the data\\
\hfuzz=500pt\includegraphics[width=1em]{connector.pdf}\includegraphics[width=1em]{element.pdf}~timeEnd & \hfuzz=500pt time & \hfuzz=500pt end time of the data \\
\hfuzz=500pt\includegraphics[width=1em]{connector.pdf}\includegraphics[width=1em]{element.pdf}~observationCode & \hfuzz=500pt string & \hfuzz=500pt technique used to generate the SINEX solution\\
\hfuzz=500pt\includegraphics[width=1em]{connector.pdf}\includegraphics[width=1em]{element.pdf}~constraintCode & \hfuzz=500pt string & \hfuzz=500pt 0: tight constraint, 1: siginficant constraint, 2: unconstrained\\
\hfuzz=500pt\includegraphics[width=1em]{connector.pdf}\includegraphics[width=1em]{element.pdf}~solutionContent & \hfuzz=500pt string & \hfuzz=500pt solution types contained in the SINEX solution (S O E T C A)\\
\hfuzz=500pt\includegraphics[width=1em]{connector.pdf}\includegraphics[width=1em]{element.pdf}~description & \hfuzz=500pt string & \hfuzz=500pt organizitions gathering/alerting the file contents\\
\hfuzz=500pt\includegraphics[width=1em]{connector.pdf}\includegraphics[width=1em]{element.pdf}~contact & \hfuzz=500pt string & \hfuzz=500pt Address of the relevant contact. e-mail\\
\hfuzz=500pt\includegraphics[width=1em]{connector.pdf}\includegraphics[width=1em]{element.pdf}~output & \hfuzz=500pt string & \hfuzz=500pt Description of the file contents\\
\hfuzz=500pt\includegraphics[width=1em]{connector.pdf}\includegraphics[width=1em]{element.pdf}~input & \hfuzz=500pt string & \hfuzz=500pt Brief description of the input used to generate this solution\\
\hfuzz=500pt\includegraphics[width=1em]{connector.pdf}\includegraphics[width=1em]{element.pdf}~software & \hfuzz=500pt string & \hfuzz=500pt Software used to generate the file\\
\hfuzz=500pt\includegraphics[width=1em]{connector.pdf}\includegraphics[width=1em]{element.pdf}~hardware & \hfuzz=500pt string & \hfuzz=500pt Computer hardware on which above software was run\\
\hfuzz=500pt\includegraphics[width=1em]{connector.pdf}\includegraphics[width=1em]{element.pdf}~inputfileComment & \hfuzz=500pt filename & \hfuzz=500pt comments in the comment block from a file (truncated at 80 characters)\\
\hfuzz=500pt\includegraphics[width=1em]{connector.pdf}\includegraphics[width=1em]{element-unbounded.pdf}~comment & \hfuzz=500pt string & \hfuzz=500pt comments in the comment block\\
\hline
\end{tabularx}

\clearpage
%==================================
\subsection{Orbit2GroopsAscii}\label{Orbit2GroopsAscii}
Convert groops orbits and corresponding covariance information to ASCII format.
The format is used to publish TUG orbits. It contains a two line header
with a short description of the orbit defined in \config{firstLine}.
The orbit is rotated to the Earth fixed frame (TRF) with \configClass{earthRotation}{earthRotationType} and given as one line per epoch.
The epoch lines contained time [MJD GPS time], position x, y and z [m], and the epoch covariance xx, yy, zz, xy, xz and yz [$m^2$].

See also \program{GroopsAscii2Orbit}.


\keepXColumns
\begin{tabularx}{\textwidth}{N T A}
\hline
Name & Type & Annotation\\
\hline
\hfuzz=500pt\includegraphics[width=1em]{element-mustset.pdf}~outputfile & \hfuzz=500pt filename & \hfuzz=500pt \\
\hfuzz=500pt\includegraphics[width=1em]{element-mustset.pdf}~inputfileOrbit & \hfuzz=500pt filename & \hfuzz=500pt \\
\hfuzz=500pt\includegraphics[width=1em]{element-mustset.pdf}~inputfileCovariance & \hfuzz=500pt filename & \hfuzz=500pt \\
\hfuzz=500pt\includegraphics[width=1em]{element-mustset.pdf}~earthRotation & \hfuzz=500pt \hyperref[earthRotationType]{earthRotation} & \hfuzz=500pt \\
\hfuzz=500pt\includegraphics[width=1em]{element.pdf}~firstLine & \hfuzz=500pt string & \hfuzz=500pt Text for first line\\
\hline
\end{tabularx}

\clearpage
%==================================
\subsection{Orbit2Sp3Format}\label{Orbit2Sp3Format}
Writes orbits to \href{https://files.igs.org/pub/data/format/sp3d.pdf}{SP3 format}.

SP3 orbits are usually given in the terrestrial reference frame (TRF), so providing \configClass{earthRotation}{earthRotationType}
automatically rotates the orbits from the celestial reference frame (CRF) to the TRF.
Since SP3 orbits often use the center of Earth as a reference, a correction from center of mass to center
of Earth can be applied to the orbits by providing \configClass{gravityfield}{gravityfieldType} (e.g. ocean tides).

See also \program{Sp3Format2Orbit}.


\keepXColumns
\begin{tabularx}{\textwidth}{N T A}
\hline
Name & Type & Annotation\\
\hline
\hfuzz=500pt\includegraphics[width=1em]{element-mustset.pdf}~outputfile & \hfuzz=500pt filename & \hfuzz=500pt \\
\hfuzz=500pt\includegraphics[width=1em]{element-mustset-unbounded.pdf}~satellite & \hfuzz=500pt sequence & \hfuzz=500pt \\
\hfuzz=500pt\includegraphics[width=1em]{connector.pdf}\includegraphics[width=1em]{element-mustset.pdf}~inputfileOrbit & \hfuzz=500pt filename & \hfuzz=500pt \\
\hfuzz=500pt\includegraphics[width=1em]{connector.pdf}\includegraphics[width=1em]{element.pdf}~inputfileClock & \hfuzz=500pt filename & \hfuzz=500pt \\
\hfuzz=500pt\includegraphics[width=1em]{connector.pdf}\includegraphics[width=1em]{element.pdf}~inputfileCovariance & \hfuzz=500pt filename & \hfuzz=500pt \\
\hfuzz=500pt\includegraphics[width=1em]{connector.pdf}\includegraphics[width=1em]{element-mustset.pdf}~identifier & \hfuzz=500pt string & \hfuzz=500pt 3 characters (e.g. GNSS PRN: G01)\\
\hfuzz=500pt\includegraphics[width=1em]{connector.pdf}\includegraphics[width=1em]{element.pdf}~orbitAccuracy & \hfuzz=500pt double & \hfuzz=500pt [m] used for accuracy codes in header (0 = unknown)\\
\hfuzz=500pt\includegraphics[width=1em]{element.pdf}~earthRotation & \hfuzz=500pt \hyperref[earthRotationType]{earthRotation} & \hfuzz=500pt rotate data into Earth-fixed frame\\
\hfuzz=500pt\includegraphics[width=1em]{element-unbounded.pdf}~gravityfield & \hfuzz=500pt \hyperref[gravityfieldType]{gravityfield} & \hfuzz=500pt degree 1 fluid mantle for CM2CE correction (SP3 orbits should be in center of Earth)\\
\hfuzz=500pt\includegraphics[width=1em]{element-unbounded.pdf}~comment & \hfuzz=500pt string & \hfuzz=500pt comment lines (77 char max)\\
\hfuzz=500pt\includegraphics[width=1em]{element.pdf}~firstLine & \hfuzz=500pt string & \hfuzz=500pt Text for first line e.g:  u+U  IGb14 KIN ITSG\\
\hfuzz=500pt\includegraphics[width=1em]{element.pdf}~writeVelocity & \hfuzz=500pt boolean & \hfuzz=500pt write velocity in addition to position\\
\hfuzz=500pt\includegraphics[width=1em]{element.pdf}~useSp3kFormat & \hfuzz=500pt boolean & \hfuzz=500pt use the extended sp3k format\\
\hline
\end{tabularx}

\clearpage
%==================================
\subsection{PotentialCoefficients2Icgem}\label{PotentialCoefficients2Icgem}
Write spherical harmonics in ICGEM format.
GROOPS uses this format as default but this program enables
the possibility to include comments and set the modelname.


\keepXColumns
\begin{tabularx}{\textwidth}{N T A}
\hline
Name & Type & Annotation\\
\hline
\hfuzz=500pt\includegraphics[width=1em]{element-mustset.pdf}~outputfile & \hfuzz=500pt filename & \hfuzz=500pt \\
\hfuzz=500pt\includegraphics[width=1em]{element-mustset.pdf}~inputfilePotentialCoefficients & \hfuzz=500pt filename & \hfuzz=500pt \\
\hfuzz=500pt\includegraphics[width=1em]{element.pdf}~inputfileTrend & \hfuzz=500pt filename & \hfuzz=500pt \\
\hfuzz=500pt\includegraphics[width=1em]{element-unbounded.pdf}~oscillation & \hfuzz=500pt sequence & \hfuzz=500pt \\
\hfuzz=500pt\includegraphics[width=1em]{connector.pdf}\includegraphics[width=1em]{element-mustset.pdf}~inputfileCosPotentialCoefficients & \hfuzz=500pt filename & \hfuzz=500pt \\
\hfuzz=500pt\includegraphics[width=1em]{connector.pdf}\includegraphics[width=1em]{element-mustset.pdf}~inputfileSinPotentialCoefficients & \hfuzz=500pt filename & \hfuzz=500pt \\
\hfuzz=500pt\includegraphics[width=1em]{connector.pdf}\includegraphics[width=1em]{element-mustset.pdf}~period & \hfuzz=500pt string & \hfuzz=500pt period of oscillation [year]\\
\hfuzz=500pt\includegraphics[width=1em]{element.pdf}~inputfileComment & \hfuzz=500pt filename & \hfuzz=500pt file containing comments for header\\
\hfuzz=500pt\includegraphics[width=1em]{element-unbounded.pdf}~comment & \hfuzz=500pt string & \hfuzz=500pt comment in header\\
\hfuzz=500pt\includegraphics[width=1em]{element-mustset.pdf}~modelname & \hfuzz=500pt string & \hfuzz=500pt name of the model\\
\hfuzz=500pt\includegraphics[width=1em]{element.pdf}~tideSystem & \hfuzz=500pt choice & \hfuzz=500pt tide system of model\\
\hfuzz=500pt\includegraphics[width=1em]{connector.pdf}\includegraphics[width=1em]{element-mustset.pdf}~zero\_tide & \hfuzz=500pt  & \hfuzz=500pt \\
\hfuzz=500pt\includegraphics[width=1em]{connector.pdf}\includegraphics[width=1em]{element-mustset.pdf}~tide\_free & \hfuzz=500pt  & \hfuzz=500pt \\
\hfuzz=500pt\includegraphics[width=1em]{element.pdf}~minDegree & \hfuzz=500pt uint & \hfuzz=500pt \\
\hfuzz=500pt\includegraphics[width=1em]{element.pdf}~maxDegree & \hfuzz=500pt uint & \hfuzz=500pt \\
\hfuzz=500pt\includegraphics[width=1em]{element.pdf}~GM & \hfuzz=500pt double & \hfuzz=500pt Geocentric gravitational constant\\
\hfuzz=500pt\includegraphics[width=1em]{element.pdf}~R & \hfuzz=500pt double & \hfuzz=500pt reference radius\\
\hfuzz=500pt\includegraphics[width=1em]{element.pdf}~time & \hfuzz=500pt time & \hfuzz=500pt reference time\\
\hline
\end{tabularx}

\clearpage
%==================================
\subsection{PsmslOceanBottomPressure2TimeSeries}\label{PsmslOceanBottomPressure2TimeSeries}
This programs reads ocean bottom pressure time series from the Permanent Service for Mean Sea Level (PSMSL).

In addition to the OBP measurements, the recorder position can be written to a \file{grid file}{griddedData}.


\keepXColumns
\begin{tabularx}{\textwidth}{N T A}
\hline
Name & Type & Annotation\\
\hline
\hfuzz=500pt\includegraphics[width=1em]{element-mustset.pdf}~outputfileTimeSeries & \hfuzz=500pt filename & \hfuzz=500pt \\
\hfuzz=500pt\includegraphics[width=1em]{element.pdf}~outputfilePosition & \hfuzz=500pt filename & \hfuzz=500pt recorder position as gridded data\\
\hfuzz=500pt\includegraphics[width=1em]{element-mustset.pdf}~inputfile & \hfuzz=500pt filename & \hfuzz=500pt \\
\hfuzz=500pt\includegraphics[width=1em]{element-mustset.pdf}~isDaily & \hfuzz=500pt boolean & \hfuzz=500pt \\
\hfuzz=500pt\includegraphics[width=1em]{element-mustset.pdf}~ignoreBadData & \hfuzz=500pt boolean & \hfuzz=500pt \\
\hfuzz=500pt\includegraphics[width=1em]{element.pdf}~R & \hfuzz=500pt double & \hfuzz=500pt \\
\hfuzz=500pt\includegraphics[width=1em]{element.pdf}~inverseFlattening & \hfuzz=500pt double & \hfuzz=500pt \\
\hfuzz=500pt\includegraphics[width=1em]{element-mustset-unbounded.pdf}~timeSeries & \hfuzz=500pt \hyperref[timeSeriesType]{timeSeries} & \hfuzz=500pt \\
\hline
\end{tabularx}

\clearpage
%==================================
\subsection{RinexObservation2GnssReceiver}\label{RinexObservation2GnssReceiver}
Converts \href{https://files.igs.org/pub/data/format/rinex_4.00.pdf}{RINEX} (version 2, 3, and 4) and
\href{https://terras.gsi.go.jp/ja/crx2rnx.html}{Compact RINEX} observation files to
\file{GnssReceiver Instrument file}{instrument}.

In case of \href{https://files.igs.org/pub/data/format/rinex211.txt}{RINEX v2.x} observation files
containing GLONASS satellites, a mapping from PRN
to frequency number must be provided via \config{inputfileMatrixPrn2FrequencyNumber}
in the form of a \file{matrix file}{matrix} with columns: GLONASS PRN, mjdStart, mjdEnd, frequencyNumber.
Source for mapping: \url{http://semisys.gfz-potsdam.de/semisys/api/?symname=2002&format=json&satellite=GLO}.
RINEX v3+ observation files already contain this information.

\configClass{useType}{gnssType} and \configClass{ignoreType}{gnssType} can be used to filter
the observation types that will be exported.

If \configFile{inputfileStationInfo}{gnssStationInfo} is set, RINEX antenna and receiver info
will be cross-checked with the provided file and warnings are raised in case of differences.

A list of semi-codeless GPS receivers (observing C2D instead of C2W) can be provided via
\configFile{inputfileSemiCodelessReceivers}{stringList} in ASCII format with one receiver name per line.
Observation types will be automatically corrected for these receivers.

Some LEO satellites use special RINEX observation types, either from the unofficial RINEX v2.20
or custom ones. These can be provided via \configFile{inputfileSpecialObservationTypes}{stringTable}
in ASCII format. The file must  must contain a table with two columns, the first being the special type,
and the second being the equivalent RINEX v3 type.

%Example for RINEX v2.20:

%\begin{tabular}{ll}
%LA & L1C \\
%L1 & L1W \\
%L2 & L2W \\
%SA & S1C \\
%S1 & S1W \\
%S2 & S2W \\
%\end{tabular}


\keepXColumns
\begin{tabularx}{\textwidth}{N T A}
\hline
Name & Type & Annotation\\
\hline
\hfuzz=500pt\includegraphics[width=1em]{element-mustset.pdf}~outputfileGnssReceiver & \hfuzz=500pt filename & \hfuzz=500pt \\
\hfuzz=500pt\includegraphics[width=1em]{element-mustset-unbounded.pdf}~inputfileRinexObservation & \hfuzz=500pt filename & \hfuzz=500pt RINEX or Compact RINEX observation files\\
\hfuzz=500pt\includegraphics[width=1em]{element.pdf}~inputfileMatrixPrn2FrequencyNumber & \hfuzz=500pt filename & \hfuzz=500pt (required for RINEX v2 files containing GLONASS observations) GROOPS matrix with columns: GLONASS PRN, SVN, mjdStart, mjdEnd, frequencyNumber\\
\hfuzz=500pt\includegraphics[width=1em]{element.pdf}~inputfileStationInfo & \hfuzz=500pt filename & \hfuzz=500pt used to determine semi-codeless receivers and to cross-check antenna and receiver info\\
\hfuzz=500pt\includegraphics[width=1em]{element.pdf}~inputfileSemiCodelessReceivers & \hfuzz=500pt filename & \hfuzz=500pt ASCII list with one receiver name per line\\
\hfuzz=500pt\includegraphics[width=1em]{element.pdf}~inputfileSpecialObservationTypes & \hfuzz=500pt filename & \hfuzz=500pt ASCII table mapping special observation types to RINEX 3 types, e.g.: LA L1C\\
\hfuzz=500pt\includegraphics[width=1em]{element-unbounded.pdf}~useType & \hfuzz=500pt \hyperref[gnssType]{gnssType} & \hfuzz=500pt only use observations that match any of these patterns\\
\hfuzz=500pt\includegraphics[width=1em]{element-unbounded.pdf}~ignoreType & \hfuzz=500pt \hyperref[gnssType]{gnssType} & \hfuzz=500pt ignore observations that match any of these patterns\\
\hline
\end{tabularx}

\clearpage
%==================================
\subsection{Sacc2Orbit}\label{Sacc2Orbit}
This program reads in SACC orbit data.


\keepXColumns
\begin{tabularx}{\textwidth}{N T A}
\hline
Name & Type & Annotation\\
\hline
\hfuzz=500pt\includegraphics[width=1em]{element-mustset.pdf}~outputfileOrbit & \hfuzz=500pt filename & \hfuzz=500pt \\
\hfuzz=500pt\includegraphics[width=1em]{element-mustset-unbounded.pdf}~inputfile & \hfuzz=500pt filename & \hfuzz=500pt \\
\hline
\end{tabularx}

\clearpage
%==================================
\subsection{Sentinel2StarCamera}\label{Sentinel2StarCamera}
This program reads in Sentinel-1/2/3 star camera data given in the special format.


\keepXColumns
\begin{tabularx}{\textwidth}{N T A}
\hline
Name & Type & Annotation\\
\hline
\hfuzz=500pt\includegraphics[width=1em]{element-mustset.pdf}~outputfileStarCamera & \hfuzz=500pt filename & \hfuzz=500pt \\
\hfuzz=500pt\includegraphics[width=1em]{element-mustset-unbounded.pdf}~inputfile & \hfuzz=500pt filename & \hfuzz=500pt \\
\hline
\end{tabularx}

\clearpage
%==================================
\subsection{SentinelXml2Orbit}\label{SentinelXml2Orbit}
Read Sentinel orbits from XML format.


\keepXColumns
\begin{tabularx}{\textwidth}{N T A}
\hline
Name & Type & Annotation\\
\hline
\hfuzz=500pt\includegraphics[width=1em]{element-mustset.pdf}~outputfileOrbit & \hfuzz=500pt filename & \hfuzz=500pt \\
\hfuzz=500pt\includegraphics[width=1em]{element-mustset.pdf}~earthRotation & \hfuzz=500pt \hyperref[earthRotationType]{earthRotation} & \hfuzz=500pt \\
\hfuzz=500pt\includegraphics[width=1em]{element-mustset-unbounded.pdf}~inputfile & \hfuzz=500pt filename & \hfuzz=500pt \\
\hline
\end{tabularx}

\clearpage
%==================================
\subsection{Sinex2Normals}\label{Sinex2Normals}
Convert normal equations from \href{http://www.iers.org/IERS/EN/Organization/AnalysisCoordinator/SinexFormat/sinex.html}{SINEX format}
to \file{normal equations}{normalEquation}.

See also \program{GnssNormals2Sinex} and \program{NormalsSphericalHarmonics2Sinex}.


\keepXColumns
\begin{tabularx}{\textwidth}{N T A}
\hline
Name & Type & Annotation\\
\hline
\hfuzz=500pt\includegraphics[width=1em]{element.pdf}~outputfileNormals & \hfuzz=500pt filename & \hfuzz=500pt N, n: unconstrained normal equations\\
\hfuzz=500pt\includegraphics[width=1em]{element.pdf}~outputfileNormalsConstraint & \hfuzz=500pt filename & \hfuzz=500pt N0, n0: normal equations of applied constraints\\
\hfuzz=500pt\includegraphics[width=1em]{element.pdf}~outputfileSolution & \hfuzz=500pt filename & \hfuzz=500pt x: parameter vector\\
\hfuzz=500pt\includegraphics[width=1em]{element.pdf}~outputfileSolutionApriori & \hfuzz=500pt filename & \hfuzz=500pt x0: a priori parameter vector\\
\hfuzz=500pt\includegraphics[width=1em]{element-mustset.pdf}~inputFileSinex & \hfuzz=500pt filename & \hfuzz=500pt \\
\hline
\end{tabularx}

\clearpage
%==================================
\subsection{Sinex2StationDiscontinuities}\label{Sinex2StationDiscontinuities}
Convert station discontinuities from
\href{http://www.iers.org/IERS/EN/Organization/AnalysisCoordinator/SinexFormat/sinex.html}{SINEX format}
(e.g. ITRF14) to \configFile{outputfileInstrument}{instrument} (MISCVALUE).
A value of 1 means position discontinuity, a value of 2 means velocity discontinuity.
Start and end epochs with value 0 are added in addition to the discontinuities from
SINEX to define continuity interval borders.

See also \program{Sinex2StationPosition} and \program{Sinex2StationPostSeismicDeformation}.


\keepXColumns
\begin{tabularx}{\textwidth}{N T A}
\hline
Name & Type & Annotation\\
\hline
\hfuzz=500pt\includegraphics[width=1em]{element-mustset.pdf}~outputfileInstrument & \hfuzz=500pt filename & \hfuzz=500pt loop variable is replaced with station name (e.g. wtzz)\\
\hfuzz=500pt\includegraphics[width=1em]{element-mustset.pdf}~inputfileDiscontinuities & \hfuzz=500pt filename & \hfuzz=500pt SINEX (e.g. ITRF14) station discontinuities\\
\hfuzz=500pt\includegraphics[width=1em]{element.pdf}~variableLoopStation & \hfuzz=500pt string & \hfuzz=500pt variable name for station loop\\
\hfuzz=500pt\includegraphics[width=1em]{element-unbounded.pdf}~stationName & \hfuzz=500pt string & \hfuzz=500pt only export these stations\\
\hline
\end{tabularx}

\clearpage
%==================================
\subsection{Sinex2StationPosition}\label{Sinex2StationPosition}
Extracts station positions from \config{inputfileSinex}
(\href{http://www.iers.org/IERS/EN/Organization/AnalysisCoordinator/SinexFormat/sinex.html}{SINEX format description})
and writes an Instrument file \configFile{outputfileInstrument}{instrument} of type VECTOR3D
for each station. Stations can be limited via \config{stationName}, otherwise all stations in \config{inputfileSinex} will be used.
If \config{timeSeries} is provided, positions will be computed at these epochs based on velocity, otherwise reference epoch is used.
If \config{extrapolateBackward} or \config{extrapolateForward} are provided, positions will also be computed for epochs
before the first interval/after the last interval, based on the position and velocity of the first/last interval.
If \config{inputfileDiscontinuities} is provided (one file per station, create with \program{Sinex2StationDiscontinuities}),
position extrapolation will stop at the first discontinuity before the first interval/after the last interval.

See also \program{Sinex2StationPostSeismicDeformation}.


\keepXColumns
\begin{tabularx}{\textwidth}{N T A}
\hline
Name & Type & Annotation\\
\hline
\hfuzz=500pt\includegraphics[width=1em]{element-mustset.pdf}~outputfileInstrument & \hfuzz=500pt filename & \hfuzz=500pt loop variable is replaced with station name (e.g. wtzz)\\
\hfuzz=500pt\includegraphics[width=1em]{element-mustset.pdf}~inputfileSinex & \hfuzz=500pt filename & \hfuzz=500pt SINEX file (.snx or .ssc)\\
\hfuzz=500pt\includegraphics[width=1em]{element.pdf}~inputfileDiscontinuities & \hfuzz=500pt filename & \hfuzz=500pt discontinuities file per station; loop variable is replaced with station name (e.g. wtzz)\\
\hfuzz=500pt\includegraphics[width=1em]{element.pdf}~variableLoopStation & \hfuzz=500pt string & \hfuzz=500pt variable name for station loop\\
\hfuzz=500pt\includegraphics[width=1em]{element-unbounded.pdf}~stationName & \hfuzz=500pt string & \hfuzz=500pt convert only these stations\\
\hfuzz=500pt\includegraphics[width=1em]{element-unbounded.pdf}~timeSeries & \hfuzz=500pt \hyperref[timeSeriesType]{timeSeries} & \hfuzz=500pt compute positions for these epochs based on velocity\\
\hfuzz=500pt\includegraphics[width=1em]{element.pdf}~extrapolateForward & \hfuzz=500pt boolean & \hfuzz=500pt also compute positions for epochs after last interval defined in SINEX file\\
\hfuzz=500pt\includegraphics[width=1em]{element.pdf}~extrapolateBackward & \hfuzz=500pt boolean & \hfuzz=500pt also compute positions for epochs before first interval defined in SINEX file\\
\hline
\end{tabularx}

\clearpage
%==================================
\subsection{Sinex2StationPostSeismicDeformation}\label{Sinex2StationPostSeismicDeformation}
Convert ITRF post-seismic deformation
\href{http://www.iers.org/IERS/EN/Organization/AnalysisCoordinator/SinexFormat/sinex.html}{SINEX file}
to \configFile{outputfileInstrument}{instrument} (VECTOR3D).

See also \program{Sinex2StationPosition} and \program{Sinex2StationDiscontinuities}.


\keepXColumns
\begin{tabularx}{\textwidth}{N T A}
\hline
Name & Type & Annotation\\
\hline
\hfuzz=500pt\includegraphics[width=1em]{element-mustset.pdf}~outputfileInstrument & \hfuzz=500pt filename & \hfuzz=500pt deformation time series\\
\hfuzz=500pt\includegraphics[width=1em]{element-mustset.pdf}~inputfileSinex & \hfuzz=500pt filename & \hfuzz=500pt ITRF post-seismic deformation SINEX file\\
\hfuzz=500pt\includegraphics[width=1em]{element-mustset-unbounded.pdf}~timeSeries & \hfuzz=500pt \hyperref[timeSeriesType]{timeSeries} & \hfuzz=500pt compute deformation for these epochs\\
\hfuzz=500pt\includegraphics[width=1em]{element-mustset.pdf}~stationName & \hfuzz=500pt string & \hfuzz=500pt \\
\hfuzz=500pt\includegraphics[width=1em]{element.pdf}~localLevelFrame & \hfuzz=500pt boolean & \hfuzz=500pt output in North, East, Up local-level frame\\
\hline
\end{tabularx}

\clearpage
%==================================
\subsection{SinexMetadata2SatelliteModel}\label{SinexMetadata2SatelliteModel}
Create \configFile{outputfileSatelliteModel}{satelliteModel} from \href{https://www.igs.org/mgex/metadata/#metadata}{IGS SINEX metadata format}.

If \configFile{inputfileSatelliteModel}{satelliteModel} is provided it is used as a basis and values are updated from the metadata file.

See also \program{SatelliteModelCreate}.


\keepXColumns
\begin{tabularx}{\textwidth}{N T A}
\hline
Name & Type & Annotation\\
\hline
\hfuzz=500pt\includegraphics[width=1em]{element-mustset.pdf}~outputfileSatelliteModel & \hfuzz=500pt filename & \hfuzz=500pt \\
\hfuzz=500pt\includegraphics[width=1em]{element-mustset.pdf}~inputfileSinexMetadata & \hfuzz=500pt filename & \hfuzz=500pt IGS SINEX metadata file\\
\hfuzz=500pt\includegraphics[width=1em]{element.pdf}~inputfileSatelliteModel & \hfuzz=500pt filename & \hfuzz=500pt base satellite model\\
\hfuzz=500pt\includegraphics[width=1em]{element-mustset.pdf}~svn & \hfuzz=500pt string & \hfuzz=500pt e.g. G040, R736, E204, C211\\
\hline
\end{tabularx}

\clearpage
%==================================
\subsection{Sp3Format2Orbit}\label{Sp3Format2Orbit}
Read IGS orbits from \href{https://files.igs.org/pub/data/format/sp3d.pdf}{SP3 format}
and write an \file{instrument file (ORBIT)}{instrument}.
The additional \config{outputfileClock} is an \file{instrument file (MISCVALUE)}{instrument}
and \config{outputfileCovariance} is an \file{instrument file (COVARIANCE3D)}{instrument}.

If \configClass{earthRotation}{earthRotationType} is provided the data are transformed
from terrestrial (TRF) to celestial reference frame (CRF).

See also \program{Orbit2Sp3Format}.


\keepXColumns
\begin{tabularx}{\textwidth}{N T A}
\hline
Name & Type & Annotation\\
\hline
\hfuzz=500pt\includegraphics[width=1em]{element-mustset.pdf}~outputfileOrbit & \hfuzz=500pt filename & \hfuzz=500pt \\
\hfuzz=500pt\includegraphics[width=1em]{element.pdf}~outputfileClock & \hfuzz=500pt filename & \hfuzz=500pt \\
\hfuzz=500pt\includegraphics[width=1em]{element.pdf}~outputfileCovariance & \hfuzz=500pt filename & \hfuzz=500pt 3x3 epoch covariance\\
\hfuzz=500pt\includegraphics[width=1em]{element.pdf}~satelliteIdentifier & \hfuzz=500pt string & \hfuzz=500pt e.g. L09 for GRACE A, empty: take first satellite\\
\hfuzz=500pt\includegraphics[width=1em]{element.pdf}~earthRotation & \hfuzz=500pt \hyperref[earthRotationType]{earthRotation} & \hfuzz=500pt rotation from TRF to CRF\\
\hfuzz=500pt\includegraphics[width=1em]{element-mustset-unbounded.pdf}~inputfile & \hfuzz=500pt filename & \hfuzz=500pt orbits in SP3 format\\
\hline
\end{tabularx}

\clearpage
%==================================
\subsection{Swarm2Starcamera}\label{Swarm2Starcamera}
This program reads SWARM star camera data given in the cdf format
and before converted to an ascii file using the program \verb|cdfexport|
provided by the Goddard Space Flight Center (\url{http://cdf.gsfc.nasa.gov/}).


\keepXColumns
\begin{tabularx}{\textwidth}{N T A}
\hline
Name & Type & Annotation\\
\hline
\hfuzz=500pt\includegraphics[width=1em]{element-mustset.pdf}~outputfileStarCamera & \hfuzz=500pt filename & \hfuzz=500pt \\
\hfuzz=500pt\includegraphics[width=1em]{element-mustset.pdf}~earthRotation & \hfuzz=500pt \hyperref[earthRotationType]{earthRotation} & \hfuzz=500pt \\
\hfuzz=500pt\includegraphics[width=1em]{element-mustset-unbounded.pdf}~inputfile & \hfuzz=500pt filename & \hfuzz=500pt \\
\hline
\end{tabularx}

\clearpage
%==================================
\subsection{TerraSarTandem2Orbit}\label{TerraSarTandem2Orbit}
This program reads in TerraSar-X or Tandem-X orbits in the special CHORB format and takes the appropriate
time frame as stated in the document header.
A description of the format can be found under: \url{http://op.gfz-potsdam.de/champ/docs_CHAMP/CH-GFZ-FD-002.pdf}


\keepXColumns
\begin{tabularx}{\textwidth}{N T A}
\hline
Name & Type & Annotation\\
\hline
\hfuzz=500pt\includegraphics[width=1em]{element-mustset.pdf}~outputfileOrbit & \hfuzz=500pt filename & \hfuzz=500pt \\
\hfuzz=500pt\includegraphics[width=1em]{element-mustset.pdf}~earthRotation & \hfuzz=500pt \hyperref[earthRotationType]{earthRotation} & \hfuzz=500pt \\
\hfuzz=500pt\includegraphics[width=1em]{element-mustset-unbounded.pdf}~inputfile & \hfuzz=500pt filename & \hfuzz=500pt orbits in CHORB format\\
\hline
\end{tabularx}

\clearpage
%==================================
\subsection{TerraSarTandem2StarCamera}\label{TerraSarTandem2StarCamera}
This program reads in TerraSar-X or Tandem-X star camera data given in the special format.


\keepXColumns
\begin{tabularx}{\textwidth}{N T A}
\hline
Name & Type & Annotation\\
\hline
\hfuzz=500pt\includegraphics[width=1em]{element-mustset.pdf}~outputfileStarCamera & \hfuzz=500pt filename & \hfuzz=500pt \\
\hfuzz=500pt\includegraphics[width=1em]{element-mustset-unbounded.pdf}~inputfile & \hfuzz=500pt filename & \hfuzz=500pt \\
\hline
\end{tabularx}

\clearpage
%==================================
\subsection{ViennaMappingFunctionGrid2File}\label{ViennaMappingFunctionGrid2File}
This program converts the gridded time series of the Vienna Mapping Functions (VMF) into
the \file{GROOPS file format}{griddedDataTimeSeries}.

Gridded VMF data is available at: \url{https://vmf.geo.tuwien.ac.at/trop_products/GRID/}


\keepXColumns
\begin{tabularx}{\textwidth}{N T A}
\hline
Name & Type & Annotation\\
\hline
\hfuzz=500pt\includegraphics[width=1em]{element.pdf}~outputfileVmfCoefficients & \hfuzz=500pt filename & \hfuzz=500pt \\
\hfuzz=500pt\includegraphics[width=1em]{element-mustset-unbounded.pdf}~inputfile & \hfuzz=500pt filename & \hfuzz=500pt files must be given for each point in time\\
\hfuzz=500pt\includegraphics[width=1em]{element-mustset-unbounded.pdf}~timeSeries & \hfuzz=500pt \hyperref[timeSeriesType]{timeSeries} & \hfuzz=500pt times of input files\\
\hfuzz=500pt\includegraphics[width=1em]{element.pdf}~deltaLambda & \hfuzz=500pt angle & \hfuzz=500pt [deg] sampling in longitude\\
\hfuzz=500pt\includegraphics[width=1em]{element.pdf}~deltaPhi & \hfuzz=500pt angle & \hfuzz=500pt [deg] sampling in latitude\\
\hfuzz=500pt\includegraphics[width=1em]{element.pdf}~isCellRegistered & \hfuzz=500pt boolean & \hfuzz=500pt grid points represent cells (VMF3), not grid corners (VMF1)\\
\hline
\end{tabularx}

\clearpage
%==================================
\subsection{ViennaMappingFunctionStation2File}\label{ViennaMappingFunctionStation2File}
Converts Vienna Mapping Functions (VMF) station time series into \file{GROOPS file format}{griddedDataTimeSeries}.

Station-wise VMF data for GNSS is available at: \url{https://vmf.geo.tuwien.ac.at/trop_products/GNSS/}


\keepXColumns
\begin{tabularx}{\textwidth}{N T A}
\hline
Name & Type & Annotation\\
\hline
\hfuzz=500pt\includegraphics[width=1em]{element.pdf}~outputfileVmfCoefficients & \hfuzz=500pt filename & \hfuzz=500pt \\
\hfuzz=500pt\includegraphics[width=1em]{element-mustset.pdf}~inputfileStationInfo & \hfuzz=500pt filename & \hfuzz=500pt \\
\hfuzz=500pt\includegraphics[width=1em]{element-mustset.pdf}~inputfileStation & \hfuzz=500pt filename & \hfuzz=500pt \\
\hfuzz=500pt\includegraphics[width=1em]{element-mustset.pdf}~inputfileVmf & \hfuzz=500pt filename & \hfuzz=500pt \\
\hline
\end{tabularx}

\clearpage
%==================================
