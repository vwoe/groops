% auto generated by GROOPS

\section{AutoregressiveModelSequence}\label{autoregressiveModelSequenceType}
Represents a sequence of multivariate autoregressive (AR) models with increasing order $p$.
The AR models should be stored as \file{matrix file}{matrix} in the \reference{GROOPS definition of
AR models}{fundamentals.autoregressiveModel}.
The required AR models can be computed with \program{CovarianceMatrix2AutoregressiveModel},
and passed to this class through
\config{inputfileAutoregressiveModel} in increasing order.

The main purpose of AutoregressiveModelSequence is to use AR models of the form
\begin{equation}
  \label{eq:ar-model}
  \mathbf{y}_e(t_i) = \sum_{k=1}^p \mathbf{\Phi}^{(p)}_k\mathbf{y}_e(t_{i-k}) + \mathbf{w}(t_i),
  \hspace{5pt} \mathbf{w}(t_i) \sim \mathcal{N}(0, \mathbf{\Sigma}^{(p)}_\mathbf{w}),
\end{equation}
to create pseudo-observation equations
\begin{equation}
  \label{eq:pseudo-observations-transformed}
  0 = \bar{\mathbf{\Phi}} \Delta\mathbf{y} + \bar{\mathbf{w}}, \hspace{5pt} \bar{\mathbf{w}} \sim
  \mathcal{N}(0, \bar{\mathbf{\Sigma}}_{\bar{\mathbf{w}}}),
\end{equation}
with
\begin{equation}
  \label{eq:pseudo-observations-ar}
  \bar{\mathbf{\Phi}} =
  \begin{bmatrix}
    \mathbf{I} & & & & & \\
    -\mathbf{\Phi}^{(1)}_1 & \mathbf{I} & & & &  \\
    -\mathbf{\Phi}^{(2)}_2 & -\mathbf{\Phi}^{(2)}_1 & \mathbf{I} & & & \\
    -\mathbf{\Phi}^{(3)}_3 & -\mathbf{\Phi}^ {(3)}_2 & -\mathbf{\Phi}^ {(3)}_1 & \mathbf{I} & &  \\
    & -\mathbf{\Phi}^{(3)}_3 & -\mathbf{\Phi}^ {(3)}_2 & -\mathbf{\Phi}^ {(3)}_1 & \mathbf{I} &  \\
    & & \ddots & \ddots & \ddots & \ddots  \\
  \end{bmatrix},
  \hspace{15pt}
  \bar{\mathbf{\Sigma}}_{\bar{\mathbf{w}}} =
  \bar{\mathbf{\Sigma}}_{\bar{\mathbf{w}}} =
  \begin{bmatrix}
    \mathbf{\Sigma}^{(0)}_{\mathbf{w}} & & & & & \\
    & \mathbf{\Sigma}^{(1)}_{\mathbf{w}} & & & & \\
    & & \mathbf{\Sigma}^{(2)}_{\mathbf{w}} & & & \\
    & & & \mathbf{\Sigma}^{(3)}_{\mathbf{w}} & & \\
    & & & & \mathbf{\Sigma}^{(3)}_{\mathbf{w}} &  \\
    & & & & & \ddots \\
  \end{bmatrix}.
\end{equation}
used to constrain high-frequency temporal gravity field variations (see
\program{KalmanSmootherLeastSquares}, \program{NormalsBuildShortTimeStaticLongTime},
\program{PreprocessingSst}).

The corresponding normal equation coefficient matrix is given by
\begin{equation}
  \label{eq:ar-normals}
  \bar{\mathbf{\Phi}}^T\bar{\mathbf{\Sigma}}^{-1}_{\bar{\mathbf{w}}}\bar{\mathbf{\Phi}}
\end{equation}
and if all AR models are estimated from the same sample its inverse is a block-Toeplitz covariance matrix
\begin{equation}
  (\mathbf{\Sigma}_{\mathbf{y}_m})_{ij} =
  \begin{cases}
 \mathbf{\Sigma}(|j-i|) & \text{for } i \leq j \\
 \mathbf{\Sigma}(|j-i|))^T & \text{otherwise}
 \end{cases},
\end{equation}
which can be computed using \program{AutoregressiveModel2CovarianceMatrix}.

A detailed description with applications can be found in:
Kvas, A., Mayer-Gürr, T. GRACE gravity field recovery with background model uncertainties.
J Geod 93, 2543–2552 (2019). \url{https://doi.org/10.1007/s00190-019-01314-1}


\keepXColumns
\begin{tabularx}{\textwidth}{N T A}
\hline
Name & Type & Annotation\\
\hline
\hfuzz=500pt\includegraphics[width=1em]{element-mustset.pdf}~autoregressiveModelSequenceType & \hfuzz=500pt sequence & \hfuzz=500pt \\
\hfuzz=500pt\includegraphics[width=1em]{connector.pdf}\includegraphics[width=1em]{element-mustset-unbounded.pdf}~inputfileAutoregressiveModel & \hfuzz=500pt filename & \hfuzz=500pt matrix file containing an AR model\\
\hfuzz=500pt\includegraphics[width=1em]{connector.pdf}\includegraphics[width=1em]{element.pdf}~sigma0 & \hfuzz=500pt double & \hfuzz=500pt a-priori sigma for white noise covariance\\
\hline
\end{tabularx}

\clearpage
%==================================

\section{Border}\label{borderType}
With this class you can select one or more region on the surface of the Earth.
In every instance of Border you can choose whether the specific region is excluded
from the overall result with the switch \config{exclude}.
To determine whether a specific point will be used furthermore the following algorithm will be applied:
In a first step all points are selected if first border excludes points otherwise all points excluded.
When every point will be tested for each instance of border from top to bottom.
If the point is not in the selected region nothing happens.
Otherwise it will included or excluded depending on the switch \config{exclude}.

First Example: The border excludes all continental areas.
The result are points on the oceans only.

Second Example: First border describes the continent north america. The next borders
excludes the great lakes and the last border describes Washington island.
In this configuration points are selected if they are inside north america
but not in the area of the great lakes. But if the point is on Washington island
it will be included again.


\subsection{Rectangle}
The region is restricted along lines of geographical coordinates.
\config{minPhi} and \config{maxPhi} describe the lower and the upper bound of the region.
\config{minLambda} and \config{maxLambda} define the left and right bound.


\keepXColumns
\begin{tabularx}{\textwidth}{N T A}
\hline
Name & Type & Annotation\\
\hline
\hfuzz=500pt\includegraphics[width=1em]{element-mustset.pdf}~minLambda & \hfuzz=500pt angle & \hfuzz=500pt \\
\hfuzz=500pt\includegraphics[width=1em]{element-mustset.pdf}~maxLambda & \hfuzz=500pt angle & \hfuzz=500pt \\
\hfuzz=500pt\includegraphics[width=1em]{element-mustset.pdf}~minPhi & \hfuzz=500pt angle & \hfuzz=500pt \\
\hfuzz=500pt\includegraphics[width=1em]{element-mustset.pdf}~maxPhi & \hfuzz=500pt angle & \hfuzz=500pt \\
\hfuzz=500pt\includegraphics[width=1em]{element.pdf}~exclude & \hfuzz=500pt boolean & \hfuzz=500pt dismiss points inside\\
\hline
\end{tabularx}


\subsection{Cap}
The region is defined by a spherical cap with the center given in geographical coordinates
longitude (\config{lambdaCenter}) and latitude (\config{phiCenter}).
The radius of the cap is given as aperture angle \config{psi}.

\fig{!hb}{0.4}{borderCap}{fig:borderCap}{spherical cap}


\keepXColumns
\begin{tabularx}{\textwidth}{N T A}
\hline
Name & Type & Annotation\\
\hline
\hfuzz=500pt\includegraphics[width=1em]{element-mustset.pdf}~lambdaCenter & \hfuzz=500pt angle & \hfuzz=500pt longitude of the center of the cap\\
\hfuzz=500pt\includegraphics[width=1em]{element-mustset.pdf}~phiCenter & \hfuzz=500pt angle & \hfuzz=500pt latitude of the center of the cap\\
\hfuzz=500pt\includegraphics[width=1em]{element-mustset.pdf}~psi & \hfuzz=500pt angle & \hfuzz=500pt aperture angle (radius)\\
\hfuzz=500pt\includegraphics[width=1em]{element.pdf}~exclude & \hfuzz=500pt boolean & \hfuzz=500pt dismiss points inside\\
\hline
\end{tabularx}


\subsection{Polygon}\label{borderType:polygon}
The region is defined by \configFile{inputfilePolygon}{polygon}
containing one or more polygons given in longitude and latitude.
An additional \config{buffer} around the polygon can be defined.
Use a negative value to shrink the polygon area.


\keepXColumns
\begin{tabularx}{\textwidth}{N T A}
\hline
Name & Type & Annotation\\
\hline
\hfuzz=500pt\includegraphics[width=1em]{element-mustset.pdf}~inputfilePolygon & \hfuzz=500pt filename & \hfuzz=500pt \\
\hfuzz=500pt\includegraphics[width=1em]{element.pdf}~buffer & \hfuzz=500pt double & \hfuzz=500pt buffer around polygon [km], \$<\$0: inside\\
\hfuzz=500pt\includegraphics[width=1em]{element.pdf}~exclude & \hfuzz=500pt boolean & \hfuzz=500pt dismiss points inside\\
\hline
\end{tabularx}

\clearpage
%==================================

\section{Condition}\label{conditionType}
Test for conditions. See \reference{Loop and conditions}{general.loopsAndConditions} for usage.


\subsection{FileExist}
Check for a file or directory existing.


\keepXColumns
\begin{tabularx}{\textwidth}{N T A}
\hline
Name & Type & Annotation\\
\hline
\hfuzz=500pt\includegraphics[width=1em]{element-mustset.pdf}~file & \hfuzz=500pt filename & \hfuzz=500pt \\
\hline
\end{tabularx}


\subsection{Command}
Execute command and check success.


\keepXColumns
\begin{tabularx}{\textwidth}{N T A}
\hline
Name & Type & Annotation\\
\hline
\hfuzz=500pt\includegraphics[width=1em]{element-mustset.pdf}~command & \hfuzz=500pt filename & \hfuzz=500pt \\
\hfuzz=500pt\includegraphics[width=1em]{element.pdf}~silently & \hfuzz=500pt boolean & \hfuzz=500pt without showing the output.\\
\hline
\end{tabularx}


\subsection{Expression}
Evaluate expression.


\keepXColumns
\begin{tabularx}{\textwidth}{N T A}
\hline
Name & Type & Annotation\\
\hline
\hfuzz=500pt\includegraphics[width=1em]{element-mustset.pdf}~expression & \hfuzz=500pt expression & \hfuzz=500pt \\
\hline
\end{tabularx}


\subsection{Matrix}
Evaluate elements of a \configClass{matrix}{matrixGeneratorType} based on an expression.
If \config{all}=\verb|yes|, all elements of the matrix must evaluate to true
for the condition to be fulfilled, otherwise any element evaluating to true is sufficient.


\keepXColumns
\begin{tabularx}{\textwidth}{N T A}
\hline
Name & Type & Annotation\\
\hline
\hfuzz=500pt\includegraphics[width=1em]{element-mustset-unbounded.pdf}~matrix & \hfuzz=500pt \hyperref[matrixGeneratorType]{matrixGenerator} & \hfuzz=500pt expression is evaluated for each element of resulting matrix\\
\hfuzz=500pt\includegraphics[width=1em]{element-mustset.pdf}~expression & \hfuzz=500pt expression & \hfuzz=500pt (variable: data) evaluated for each element\\
\hfuzz=500pt\includegraphics[width=1em]{element.pdf}~all & \hfuzz=500pt boolean & \hfuzz=500pt all (=yes)/any (=no) elements must evaluate to true\\
\hline
\end{tabularx}


\subsection{MatrixEmpty}
Evaluate if \file{matrix}{matrix} (or \file{instrument}{instrument}) file is empty/has zero size.


\keepXColumns
\begin{tabularx}{\textwidth}{N T A}
\hline
Name & Type & Annotation\\
\hline
\hfuzz=500pt\includegraphics[width=1em]{element-mustset.pdf}~inputfileMatrix & \hfuzz=500pt filename & \hfuzz=500pt \\
\hline
\end{tabularx}


\subsection{StringContainsPattern}
Determines if there is a match between a pattern or a regular expression and some subsequence in a string.


\keepXColumns
\begin{tabularx}{\textwidth}{N T A}
\hline
Name & Type & Annotation\\
\hline
\hfuzz=500pt\includegraphics[width=1em]{element.pdf}~string & \hfuzz=500pt string & \hfuzz=500pt should contain a \{variable\}\\
\hfuzz=500pt\includegraphics[width=1em]{element.pdf}~pattern & \hfuzz=500pt string & \hfuzz=500pt \\
\hfuzz=500pt\includegraphics[width=1em]{element.pdf}~isRegularExpression & \hfuzz=500pt boolean & \hfuzz=500pt pattern is  a regular expression\\
\hfuzz=500pt\includegraphics[width=1em]{element.pdf}~caseSensitive & \hfuzz=500pt boolean & \hfuzz=500pt treat lower and upper case as distinct\\
\hline
\end{tabularx}


\subsection{StringMatchPattern}
Determines if a pattern or a regular expression matches the entire string.


\keepXColumns
\begin{tabularx}{\textwidth}{N T A}
\hline
Name & Type & Annotation\\
\hline
\hfuzz=500pt\includegraphics[width=1em]{element.pdf}~string & \hfuzz=500pt string & \hfuzz=500pt should contain a \{variable\}\\
\hfuzz=500pt\includegraphics[width=1em]{element.pdf}~pattern & \hfuzz=500pt string & \hfuzz=500pt \\
\hfuzz=500pt\includegraphics[width=1em]{element.pdf}~isRegularExpression & \hfuzz=500pt boolean & \hfuzz=500pt pattern is  a regular expression\\
\hfuzz=500pt\includegraphics[width=1em]{element.pdf}~caseSensitive & \hfuzz=500pt boolean & \hfuzz=500pt treat lower and upper case as distinct\\
\hline
\end{tabularx}


\subsection{And}
All conditions must be met (with short-circuit evaluation).


\keepXColumns
\begin{tabularx}{\textwidth}{N T A}
\hline
Name & Type & Annotation\\
\hline
\hfuzz=500pt\includegraphics[width=1em]{element-mustset-unbounded.pdf}~condition & \hfuzz=500pt \hyperref[conditionType]{condition} & \hfuzz=500pt \\
\hline
\end{tabularx}


\subsection{Or}
One of the conditions must be met (with short-circuit evaluation).


\keepXColumns
\begin{tabularx}{\textwidth}{N T A}
\hline
Name & Type & Annotation\\
\hline
\hfuzz=500pt\includegraphics[width=1em]{element-mustset-unbounded.pdf}~condition & \hfuzz=500pt \hyperref[conditionType]{condition} & \hfuzz=500pt \\
\hline
\end{tabularx}


\subsection{Not}
The result of the condition is inverted.


\keepXColumns
\begin{tabularx}{\textwidth}{N T A}
\hline
Name & Type & Annotation\\
\hline
\hfuzz=500pt\includegraphics[width=1em]{element-mustset.pdf}~condition & \hfuzz=500pt \hyperref[conditionType]{condition} & \hfuzz=500pt \\
\hline
\end{tabularx}

\clearpage
%==================================

\section{CovariancePod}\label{covariancePodType}
Provides arc wise covariance matrices for precise orbit data.
Temporal correlations are modeled in the orbit system (along, cross, radial).
The \configFile{inputfileCovarianceFunction}{matrix} provides temporal covariance functions for each axis.
From the diagonal matrix for each time step
\begin{equation}
  Cov_{3\times3}(t) = \text{diag}(cov_x(t), cov_y(t), cov_z(t))
\end{equation}
the Toeplitz covariance matrix for an arc is constructed
\begin{equation}
  \M C = \begin{pmatrix}
    Cov(t_0) & Cov(t_1) & \cdots   &          &        &        \\
    Cov(t_1) & Cov(t_0) & Cov(t_1) & \cdots   &        &        \\
    \cdots   & Cov(t_1) & Cov(t_0) & Cov(t_1) & \cdots &        \\
             & \cdots   & \ddots   & \ddots   & \ddots & \cdots \\
  \end{pmatrix}
\end{equation}

The epoch wise $3\times3$ covariance matrices given by \configFile{inputfileCovariancePodEpoch}{instrument}
are eigen value decomposed
\begin{equation}
  \M C_{3\times3}(t_i) = \M Q \M\Lambda \M Q^T,
\end{equation}
where $\M Q$ is an orthgonal matrix and $\M\Lambda$ diagonal.
This used to split the covariances matrices
\begin{equation}
  \M C_{3\times3}(t_i) = \M D(t_i) \M D(t_i)^T = (\M Q \M\Lambda^{1/2} \M Q^T)(\M Q \M\Lambda^{1/2} \M Q^T)^T,
\end{equation}
and to compose a block diagonal matrix for an arc
\begin{equation}
  \M D = \text{diag}(\M D(t_1), \M D(t_2), \ldots, \M D(t_2)).
\end{equation}

The complete covariance matrix of an arc is given by
\begin{equation}
  \M C_{arc} = \sigma_0^2 \sigma_{arc}^2 \M D \M C \M D^T +
  \text{diag}(\sigma_1^2\M I_{3\times3}, \sigma_2^2\M I_{3\times3}, \ldots, \sigma_n^2\M I_{3\times3})
\end{equation}
where \config{sigma}~$\sigma_0$ is an overall factor
and the arc specific factors $\sigma_{arc}$ can be provided with \configFile{inputfileSigmasPerArc}{matrix}.
The last matrix can be used to downweight outliers in single epochs and will be added if
\configFile{inputfileSigmasPerEpoch}{instrument} is provided.


\keepXColumns
\begin{tabularx}{\textwidth}{N T A}
\hline
Name & Type & Annotation\\
\hline
\hfuzz=500pt\includegraphics[width=1em]{element-mustset.pdf}~covariancePodType & \hfuzz=500pt sequence & \hfuzz=500pt \\
\hfuzz=500pt\includegraphics[width=1em]{connector.pdf}\includegraphics[width=1em]{element.pdf}~sigma & \hfuzz=500pt double & \hfuzz=500pt general variance factor\\
\hfuzz=500pt\includegraphics[width=1em]{connector.pdf}\includegraphics[width=1em]{element.pdf}~inputfileSigmasPerArc & \hfuzz=500pt filename & \hfuzz=500pt different accuaries for each arc (multplicated with sigma)\\
\hfuzz=500pt\includegraphics[width=1em]{connector.pdf}\includegraphics[width=1em]{element.pdf}~inputfileSigmasPerEpoch & \hfuzz=500pt filename & \hfuzz=500pt different accuaries for each epoch (added)\\
\hfuzz=500pt\includegraphics[width=1em]{connector.pdf}\includegraphics[width=1em]{element.pdf}~inputfileCovarianceFunction & \hfuzz=500pt filename & \hfuzz=500pt covariances in time for along, cross, and radial direction\\
\hfuzz=500pt\includegraphics[width=1em]{connector.pdf}\includegraphics[width=1em]{element.pdf}~inputfileCovariancePodEpoch & \hfuzz=500pt filename & \hfuzz=500pt 3x3 epoch wise covariances\\
\hline
\end{tabularx}

\clearpage
%==================================

\section{CovarianceSst}\label{covarianceSstType}
Provides arc wise covariance matrices for satellite-to-satellite observations SST).
The \configFile{inputfileCovarianceFunction}{matrix} provides a temporal covariance function.
From it the Toeplitz covariance matrix is constructed
\begin{equation}
  \M C = \begin{pmatrix}
    cov(t_0) & cov(t_1) & \cdots   &          &        &        \\
    cov(t_1) & cov(t_0) & cov(t_1) & \cdots   &        &        \\
    \cdots   & cov(t_1) & cov(t_0) & cov(t_1) & \cdots &        \\
             & \cdots   & \ddots   & \ddots   & \ddots & \cdots \\
  \end{pmatrix} \\
\end{equation}

The complete covariance matrix of an arc is given by
\begin{equation}
  \M C_{arc} = \sigma_0^2 \sigma_{arc}^2 \M C + \sigma_{S,arc}^2 \M S_{arc}+ \text{diag}(\sigma_1^2, \sigma_2^2, \ldots, \sigma_n^2)
\end{equation}
where \config{sigma}~$\sigma_0$ is an overall factor and the arc specific factors $\sigma_{arc}$
can be provided with \configFile{inputfileSigmasPerArc}{matrix}.
The second term describes general covariance matrices for each arc
\configFile{inputfileCovarianceMatrixArc}{matrix} together with the factors $\sigma_{S,arc}$ from \config{sigmasCovarianceMatrixArc}.
The last matrix can be used to downweight outliers in single epochs and will be added if
\configFile{inputfileSigmasPerEpoch}{instrument} is provided.



\keepXColumns
\begin{tabularx}{\textwidth}{N T A}
\hline
Name & Type & Annotation\\
\hline
\hfuzz=500pt\includegraphics[width=1em]{element-mustset.pdf}~covarianceSstType & \hfuzz=500pt sequence & \hfuzz=500pt \\
\hfuzz=500pt\includegraphics[width=1em]{connector.pdf}\includegraphics[width=1em]{element.pdf}~sigma & \hfuzz=500pt double & \hfuzz=500pt general variance factor\\
\hfuzz=500pt\includegraphics[width=1em]{connector.pdf}\includegraphics[width=1em]{element.pdf}~inputfileSigmasPerArc & \hfuzz=500pt filename & \hfuzz=500pt different accuaries for each arc (multplicated with sigma)\\
\hfuzz=500pt\includegraphics[width=1em]{connector.pdf}\includegraphics[width=1em]{element.pdf}~inputfileSigmasPerEpoch & \hfuzz=500pt filename & \hfuzz=500pt different accuaries for each epoch (added)\\
\hfuzz=500pt\includegraphics[width=1em]{connector.pdf}\includegraphics[width=1em]{element.pdf}~inputfileCovarianceFunction & \hfuzz=500pt filename & \hfuzz=500pt covariance function in time\\
\hfuzz=500pt\includegraphics[width=1em]{connector.pdf}\includegraphics[width=1em]{element-unbounded.pdf}~inputfileCovarianceMatrixArc & \hfuzz=500pt filename & \hfuzz=500pt one matrix file per arc. Use \{arcNo\} as template\\
\hfuzz=500pt\includegraphics[width=1em]{connector.pdf}\includegraphics[width=1em]{element.pdf}~sigmasCovarianceMatrixArc & \hfuzz=500pt filename & \hfuzz=500pt vector with one sigma for each covarianceMatrixArc\\
\hline
\end{tabularx}

\clearpage
%==================================

\section{DigitalFilter}\label{digitalFilterType}
Digital filter implementation for the filtering of equally spaced time series. This class implements the filter equations as
\begin{equation}\label{digitalFilterType:arma}
  \sum_{l=0}^Q a_l y_{n-l} = \sum_{k=-p_0}^{P-p_0-1} b_k x_{n-k}, \hspace{25pt} a_0 = 1,
\end{equation}
where $Q$ is the autoregressive (AR) order and $P$ is the moving average (MA) order. Note that the MA part can also be non-causal.
The characteristics of a filter cascade can be computed by the programs \program{DigitalFilter2FrequencyResponse} and \program{DigitalFilter2ImpulseResponse}.
To apply a filter cascade to a time series (or an instrument file ) use \program{InstrumentFilter}.
Each filter can be applyed in forward and backward direction by setting \config{backwardDirection}.
If the same filter is applied in both directions, the combined filter has zero phase and the squared magnitude response.
Setting \config{inFrequencyDomain} to true applies the transfer function of the filter to the DFT of the input and synthesizes the result, i.e.:
\begin{equation}
  y_n = \mathcal{F}^{-1}\{H\cdot\mathcal{F}\{x_n\}\}.
\end{equation}
This is equivalent to setting \config{padType} to \config{periodic}.

To reduce warmup effects, the input time series can be padded by choosing a \config{padType}:
\begin{itemize}
\item \config{none}: no padding is applied
\item \config{zero}: zeros are appended at the beginning and end of the input time series
\item \config{constant}: the beginning of the input time series is padded with the first value, the end is padded with the last value
\item \config{periodic}: periodic continuation of the input time series (i.,e. the beginning is padded with the last epochs and the end is padded with the first epochs)
\item \config{symmetric}: beginning and end are reflected around the first and last epoch respectively
\end{itemize}


\subsection{MovingAverage}
Moving average (boxcar) filter. For odd lengths, this filter is symmetric and has therefore no phase shift. For even lengths, a phase shift of half a cycle is introduced.

\[
  y_n = \sum_{k=-\lfloor\frac{P}{2}\rfloor}^{\lfloor\frac{P}{2}\rfloor} \frac{1}{P}x_{n-k}
\]



\keepXColumns
\begin{tabularx}{\textwidth}{N T A}
\hline
Name & Type & Annotation\\
\hline
\hfuzz=500pt\includegraphics[width=1em]{element-mustset.pdf}~length & \hfuzz=500pt uint & \hfuzz=500pt number of epochs in averaging operator\\
\hfuzz=500pt\includegraphics[width=1em]{element.pdf}~inFrequencyDomain & \hfuzz=500pt boolean & \hfuzz=500pt apply filter in frequency domain\\
\hfuzz=500pt\includegraphics[width=1em]{element-mustset.pdf}~padType & \hfuzz=500pt choice & \hfuzz=500pt \\
\hfuzz=500pt\includegraphics[width=1em]{connector.pdf}\includegraphics[width=1em]{element-mustset.pdf}~none & \hfuzz=500pt  & \hfuzz=500pt no padding is applied\\
\hfuzz=500pt\includegraphics[width=1em]{connector.pdf}\includegraphics[width=1em]{element-mustset.pdf}~zero & \hfuzz=500pt  & \hfuzz=500pt zero padding\\
\hfuzz=500pt\includegraphics[width=1em]{connector.pdf}\includegraphics[width=1em]{element-mustset.pdf}~constant & \hfuzz=500pt  & \hfuzz=500pt pad using first and last value\\
\hfuzz=500pt\includegraphics[width=1em]{connector.pdf}\includegraphics[width=1em]{element-mustset.pdf}~periodic & \hfuzz=500pt  & \hfuzz=500pt periodic continuation of matrix\\
\hfuzz=500pt\includegraphics[width=1em]{connector.pdf}\includegraphics[width=1em]{element-mustset.pdf}~symmetric & \hfuzz=500pt  & \hfuzz=500pt symmetric continuation around the matrix edges\\
\hline
\end{tabularx}


\subsection{Median}
Moving median filter of length $n$. The filter output at epoch $k$ is the median of the set start at $k-n/2$ to $k+n/2$.
The filter length $n$ should be uneven to avoid a phase shift.


\keepXColumns
\begin{tabularx}{\textwidth}{N T A}
\hline
Name & Type & Annotation\\
\hline
\hfuzz=500pt\includegraphics[width=1em]{element-mustset.pdf}~length & \hfuzz=500pt uint & \hfuzz=500pt length of the moving window [epochs]\\
\hfuzz=500pt\includegraphics[width=1em]{element-mustset.pdf}~padType & \hfuzz=500pt choice & \hfuzz=500pt \\
\hfuzz=500pt\includegraphics[width=1em]{connector.pdf}\includegraphics[width=1em]{element-mustset.pdf}~none & \hfuzz=500pt  & \hfuzz=500pt no padding is applied\\
\hfuzz=500pt\includegraphics[width=1em]{connector.pdf}\includegraphics[width=1em]{element-mustset.pdf}~zero & \hfuzz=500pt  & \hfuzz=500pt zero padding\\
\hfuzz=500pt\includegraphics[width=1em]{connector.pdf}\includegraphics[width=1em]{element-mustset.pdf}~constant & \hfuzz=500pt  & \hfuzz=500pt pad using first and last value\\
\hfuzz=500pt\includegraphics[width=1em]{connector.pdf}\includegraphics[width=1em]{element-mustset.pdf}~periodic & \hfuzz=500pt  & \hfuzz=500pt periodic continuation of matrix\\
\hfuzz=500pt\includegraphics[width=1em]{connector.pdf}\includegraphics[width=1em]{element-mustset.pdf}~symmetric & \hfuzz=500pt  & \hfuzz=500pt symmetric continuation around the matrix edges\\
\hline
\end{tabularx}


\subsection{Derivative}
Symmetric MA filter for numerical differentiation using polynomial approximation. The input time series is approximated by a moving polynomial of degree \config{polynomialDegree}, by solving
\begin{equation}
  \begin{bmatrix} x(t_k+\tau_0) \\ \vdots \\ x(t_k+\tau_M) \end{bmatrix}
  =
  \begin{bmatrix}
  1      & \tau_0 & \tau_0^2 & \cdots & \tau_0^M \\
  \vdots & \vdots & \vdots   &        & \vdots   \\
  1      & \tau_M & \tau_M^2 & \cdots & \tau_M^M \\
  \end{bmatrix}%^{-1}
  \begin{bmatrix}
  a_0 \\ \vdots \\ a_M
  \end{bmatrix}
  \qquad\text{with}\quad
  \tau_j =  (j-M/2)\cdot \Delta t,
\end{equation}
for each time step $t_k$ ($\Delta t$ is the \config{sampling} of the time series).
The filter coefficients for the $k$-th derivative are obtained by taking the appropriate row of the inverse coefficient matrix $\mathbf{W}$:
\begin{equation}
  b_n = \prod_{i=0}^{k-1} (k-i) \mathbf{w}_{2,:}.
\end{equation}
The \config{polynomialDegree} should be even if no phase shift should be introduced.


\keepXColumns
\begin{tabularx}{\textwidth}{N T A}
\hline
Name & Type & Annotation\\
\hline
\hfuzz=500pt\includegraphics[width=1em]{element-mustset.pdf}~polynomialDegree & \hfuzz=500pt uint & \hfuzz=500pt degree of approximation polynomial\\
\hfuzz=500pt\includegraphics[width=1em]{element.pdf}~derivative & \hfuzz=500pt uint & \hfuzz=500pt take kth derivative\\
\hfuzz=500pt\includegraphics[width=1em]{element.pdf}~sampling & \hfuzz=500pt double & \hfuzz=500pt assumed time step between points\\
\hfuzz=500pt\includegraphics[width=1em]{element-mustset.pdf}~padType & \hfuzz=500pt choice & \hfuzz=500pt \\
\hfuzz=500pt\includegraphics[width=1em]{connector.pdf}\includegraphics[width=1em]{element-mustset.pdf}~none & \hfuzz=500pt  & \hfuzz=500pt no padding is applied\\
\hfuzz=500pt\includegraphics[width=1em]{connector.pdf}\includegraphics[width=1em]{element-mustset.pdf}~zero & \hfuzz=500pt  & \hfuzz=500pt zero padding\\
\hfuzz=500pt\includegraphics[width=1em]{connector.pdf}\includegraphics[width=1em]{element-mustset.pdf}~constant & \hfuzz=500pt  & \hfuzz=500pt pad using first and last value\\
\hfuzz=500pt\includegraphics[width=1em]{connector.pdf}\includegraphics[width=1em]{element-mustset.pdf}~periodic & \hfuzz=500pt  & \hfuzz=500pt periodic continuation of matrix\\
\hfuzz=500pt\includegraphics[width=1em]{connector.pdf}\includegraphics[width=1em]{element-mustset.pdf}~symmetric & \hfuzz=500pt  & \hfuzz=500pt symmetric continuation around the matrix edges\\
\hline
\end{tabularx}


\subsection{Integral}
Numerical integration using polynomial approximation.
The input time series is approximated by a moving polynomial of degree \config{polynomialDegree}
by solving
\begin{equation}
  \begin{bmatrix} x(t_k+\tau_0) \\ \vdots \\ x(t_k+\tau_M) \end{bmatrix}
  =
  \begin{bmatrix}
  1      & \tau_0 & \tau_0^2 & \cdots & \tau_0^M \\
  \vdots & \vdots & \vdots   &        & \vdots   \\
  1      & \tau_M & \tau_M^2 & \cdots & \tau_M^M \\
  \end{bmatrix}%^{-1}
  \begin{bmatrix}
  a_0 \\ \vdots \\ a_M
  \end{bmatrix}
  \qquad\text{with}\quad
  \tau_j =  (j-M/2)\cdot \Delta t,
\end{equation}
for each time step $t_k$ ($\Delta t$ is the \config{sampling} of the time series).
The numerical integral for each time step $t_k$ is approximated by the center interval of the estimated polynomial.

\fig{!hb}{0.7}{DigitalFilter_integral}{fig:DigitalFilterIntegral}{Numerical integration by polynomial approximation.}

\config{polynomialDegree} should be even to avoid a phase shift.


\keepXColumns
\begin{tabularx}{\textwidth}{N T A}
\hline
Name & Type & Annotation\\
\hline
\hfuzz=500pt\includegraphics[width=1em]{element-mustset.pdf}~polynomialDegree & \hfuzz=500pt uint & \hfuzz=500pt degree of approximation polynomial\\
\hfuzz=500pt\includegraphics[width=1em]{element.pdf}~sampling & \hfuzz=500pt double & \hfuzz=500pt assumed time step between points\\
\hfuzz=500pt\includegraphics[width=1em]{element-mustset.pdf}~padType & \hfuzz=500pt choice & \hfuzz=500pt \\
\hfuzz=500pt\includegraphics[width=1em]{connector.pdf}\includegraphics[width=1em]{element-mustset.pdf}~none & \hfuzz=500pt  & \hfuzz=500pt no padding is applied\\
\hfuzz=500pt\includegraphics[width=1em]{connector.pdf}\includegraphics[width=1em]{element-mustset.pdf}~zero & \hfuzz=500pt  & \hfuzz=500pt zero padding\\
\hfuzz=500pt\includegraphics[width=1em]{connector.pdf}\includegraphics[width=1em]{element-mustset.pdf}~constant & \hfuzz=500pt  & \hfuzz=500pt pad using first and last value\\
\hfuzz=500pt\includegraphics[width=1em]{connector.pdf}\includegraphics[width=1em]{element-mustset.pdf}~periodic & \hfuzz=500pt  & \hfuzz=500pt periodic continuation of matrix\\
\hfuzz=500pt\includegraphics[width=1em]{connector.pdf}\includegraphics[width=1em]{element-mustset.pdf}~symmetric & \hfuzz=500pt  & \hfuzz=500pt symmetric continuation around the matrix edges\\
\hline
\end{tabularx}


\subsection{Correlation}
Correlation ($\rho$) of \config{corr} is introduced into the time series:
\begin{equation}
  y_n = \rho\cdot y_{n-1} + \sqrt{1-\rho^2}x_n.
\end{equation}


\keepXColumns
\begin{tabularx}{\textwidth}{N T A}
\hline
Name & Type & Annotation\\
\hline
\hfuzz=500pt\includegraphics[width=1em]{element-mustset.pdf}~correlation & \hfuzz=500pt double & \hfuzz=500pt correlation\\
\hfuzz=500pt\includegraphics[width=1em]{element.pdf}~backwardDirection & \hfuzz=500pt boolean & \hfuzz=500pt apply filter in backward direction\\
\hfuzz=500pt\includegraphics[width=1em]{element.pdf}~inFrequencyDomain & \hfuzz=500pt boolean & \hfuzz=500pt apply filter in frequency domain\\
\hfuzz=500pt\includegraphics[width=1em]{element-mustset.pdf}~padType & \hfuzz=500pt choice & \hfuzz=500pt \\
\hfuzz=500pt\includegraphics[width=1em]{connector.pdf}\includegraphics[width=1em]{element-mustset.pdf}~none & \hfuzz=500pt  & \hfuzz=500pt no padding is applied\\
\hfuzz=500pt\includegraphics[width=1em]{connector.pdf}\includegraphics[width=1em]{element-mustset.pdf}~zero & \hfuzz=500pt  & \hfuzz=500pt zero padding\\
\hfuzz=500pt\includegraphics[width=1em]{connector.pdf}\includegraphics[width=1em]{element-mustset.pdf}~constant & \hfuzz=500pt  & \hfuzz=500pt pad using first and last value\\
\hfuzz=500pt\includegraphics[width=1em]{connector.pdf}\includegraphics[width=1em]{element-mustset.pdf}~periodic & \hfuzz=500pt  & \hfuzz=500pt periodic continuation of matrix\\
\hfuzz=500pt\includegraphics[width=1em]{connector.pdf}\includegraphics[width=1em]{element-mustset.pdf}~symmetric & \hfuzz=500pt  & \hfuzz=500pt symmetric continuation around the matrix edges\\
\hline
\end{tabularx}


\subsection{GraceLowpass}
Low pass and differentation filter as used for GRACE KBR and ACC data in the Level1A processing.

\fig{!hb}{0.8}{DigitalFilter_graceLowpass}{fig:DigitalFilterGraceLowpass}{Amplitude response of the low pass filter used in the L1A processing.}


\keepXColumns
\begin{tabularx}{\textwidth}{N T A}
\hline
Name & Type & Annotation\\
\hline
\hfuzz=500pt\includegraphics[width=1em]{element-mustset.pdf}~rawDataRate & \hfuzz=500pt double & \hfuzz=500pt sampling frequency in Hz (fs).\\
\hfuzz=500pt\includegraphics[width=1em]{element.pdf}~convolutionNumber & \hfuzz=500pt uint & \hfuzz=500pt number of self convolutions of the filter kernel\\
\hfuzz=500pt\includegraphics[width=1em]{element.pdf}~fitInterval & \hfuzz=500pt double & \hfuzz=500pt length of the filter kernel [seconds]\\
\hfuzz=500pt\includegraphics[width=1em]{element.pdf}~lowPassBandwith & \hfuzz=500pt double & \hfuzz=500pt target low pass bandwidth\\
\hfuzz=500pt\includegraphics[width=1em]{element.pdf}~normFrequency & \hfuzz=500pt double & \hfuzz=500pt norm filter at this frequency [Hz] (default: GRACE dominant (J2) signal frequency)\\
\hfuzz=500pt\includegraphics[width=1em]{element.pdf}~reduceQuadraticFit & \hfuzz=500pt boolean & \hfuzz=500pt remove-\$>\$filter-\$>\$restore quadratic fit\\
\hfuzz=500pt\includegraphics[width=1em]{element.pdf}~derivative & \hfuzz=500pt choice & \hfuzz=500pt \\
\hfuzz=500pt\includegraphics[width=1em]{connector.pdf}\includegraphics[width=1em]{element-mustset.pdf}~derivative1st & \hfuzz=500pt  & \hfuzz=500pt range rate\\
\hfuzz=500pt\includegraphics[width=1em]{connector.pdf}\includegraphics[width=1em]{element-mustset.pdf}~derivative2nd & \hfuzz=500pt  & \hfuzz=500pt range acceleration\\
\hfuzz=500pt\includegraphics[width=1em]{element.pdf}~inFrequencyDomain & \hfuzz=500pt boolean & \hfuzz=500pt apply filter in frequency domain\\
\hfuzz=500pt\includegraphics[width=1em]{element-mustset.pdf}~padType & \hfuzz=500pt choice & \hfuzz=500pt \\
\hfuzz=500pt\includegraphics[width=1em]{connector.pdf}\includegraphics[width=1em]{element-mustset.pdf}~none & \hfuzz=500pt  & \hfuzz=500pt no padding is applied\\
\hfuzz=500pt\includegraphics[width=1em]{connector.pdf}\includegraphics[width=1em]{element-mustset.pdf}~zero & \hfuzz=500pt  & \hfuzz=500pt zero padding\\
\hfuzz=500pt\includegraphics[width=1em]{connector.pdf}\includegraphics[width=1em]{element-mustset.pdf}~constant & \hfuzz=500pt  & \hfuzz=500pt pad using first and last value\\
\hfuzz=500pt\includegraphics[width=1em]{connector.pdf}\includegraphics[width=1em]{element-mustset.pdf}~periodic & \hfuzz=500pt  & \hfuzz=500pt periodic continuation of matrix\\
\hfuzz=500pt\includegraphics[width=1em]{connector.pdf}\includegraphics[width=1em]{element-mustset.pdf}~symmetric & \hfuzz=500pt  & \hfuzz=500pt symmetric continuation around the matrix edges\\
\hline
\end{tabularx}


\subsection{Butterworth}
Digital implementation of the Butterworth filter. The design of the filter is done by modifying the analog (continuous time) transfer function, which is
then transformed into the digital domain by using the bilinear transform. The filter coefficients are then determined by a least squares adjustment in time domain.

The \config{filterType} can be \config{lowpass}, \config{highpass}, where one cutoff frequency has to be specified, and \config{bandpass} and \config{bandstop} where to cutoff frequencies have to be specified.
Cutoff frequencies must be given as normalized frequency $w_n = f/f_{\text{nyq}}$. For a cutoff frequency of 30~mHz for a time series sampled with 5~seconds gives a normalized frequency of $0.03/0.1 = 0.3$.


\keepXColumns
\begin{tabularx}{\textwidth}{N T A}
\hline
Name & Type & Annotation\\
\hline
\hfuzz=500pt\includegraphics[width=1em]{element-mustset.pdf}~order & \hfuzz=500pt uint & \hfuzz=500pt filter order\\
\hfuzz=500pt\includegraphics[width=1em]{element-mustset.pdf}~type & \hfuzz=500pt choice & \hfuzz=500pt filter type\\
\hfuzz=500pt\includegraphics[width=1em]{connector.pdf}\includegraphics[width=1em]{element-mustset.pdf}~lowpass & \hfuzz=500pt sequence & \hfuzz=500pt \\
\hfuzz=500pt\quad\includegraphics[width=1em]{connector.pdf}\includegraphics[width=1em]{element-mustset.pdf}~Wn & \hfuzz=500pt double & \hfuzz=500pt normalized cutoff frequency (f\_c / f\_nyq)\\
\hfuzz=500pt\includegraphics[width=1em]{connector.pdf}\includegraphics[width=1em]{element-mustset.pdf}~highpass & \hfuzz=500pt sequence & \hfuzz=500pt \\
\hfuzz=500pt\quad\includegraphics[width=1em]{connector.pdf}\includegraphics[width=1em]{element-mustset.pdf}~Wn & \hfuzz=500pt double & \hfuzz=500pt normalized cutoff frequency (f\_c / f\_nyq)\\
\hfuzz=500pt\includegraphics[width=1em]{connector.pdf}\includegraphics[width=1em]{element-mustset.pdf}~bandpass & \hfuzz=500pt sequence & \hfuzz=500pt \\
\hfuzz=500pt\quad\includegraphics[width=1em]{connector.pdf}\includegraphics[width=1em]{element-mustset.pdf}~Wn1 & \hfuzz=500pt double & \hfuzz=500pt lower normalized cutoff frequency (f\_c / f\_nyq)\\
\hfuzz=500pt\quad\includegraphics[width=1em]{connector.pdf}\includegraphics[width=1em]{element-mustset.pdf}~Wn2 & \hfuzz=500pt double & \hfuzz=500pt upper normalized cutoff frequency (f\_c / f\_nyq)\\
\hfuzz=500pt\includegraphics[width=1em]{connector.pdf}\includegraphics[width=1em]{element-mustset.pdf}~bandstop & \hfuzz=500pt sequence & \hfuzz=500pt \\
\hfuzz=500pt\quad\includegraphics[width=1em]{connector.pdf}\includegraphics[width=1em]{element-mustset.pdf}~Wn1 & \hfuzz=500pt double & \hfuzz=500pt lower normalized cutoff frequency (f\_c / f\_nyq)\\
\hfuzz=500pt\quad\includegraphics[width=1em]{connector.pdf}\includegraphics[width=1em]{element-mustset.pdf}~Wn2 & \hfuzz=500pt double & \hfuzz=500pt upper normalized cutoff frequency (f\_c / f\_nyq)\\
\hfuzz=500pt\includegraphics[width=1em]{element.pdf}~backwardDirection & \hfuzz=500pt boolean & \hfuzz=500pt apply filter in backward direction\\
\hfuzz=500pt\includegraphics[width=1em]{element.pdf}~inFrequencyDomain & \hfuzz=500pt boolean & \hfuzz=500pt apply filter in frequency domain\\
\hfuzz=500pt\includegraphics[width=1em]{element-mustset.pdf}~padType & \hfuzz=500pt choice & \hfuzz=500pt \\
\hfuzz=500pt\includegraphics[width=1em]{connector.pdf}\includegraphics[width=1em]{element-mustset.pdf}~none & \hfuzz=500pt  & \hfuzz=500pt no padding is applied\\
\hfuzz=500pt\includegraphics[width=1em]{connector.pdf}\includegraphics[width=1em]{element-mustset.pdf}~zero & \hfuzz=500pt  & \hfuzz=500pt zero padding\\
\hfuzz=500pt\includegraphics[width=1em]{connector.pdf}\includegraphics[width=1em]{element-mustset.pdf}~constant & \hfuzz=500pt  & \hfuzz=500pt pad using first and last value\\
\hfuzz=500pt\includegraphics[width=1em]{connector.pdf}\includegraphics[width=1em]{element-mustset.pdf}~periodic & \hfuzz=500pt  & \hfuzz=500pt periodic continuation of matrix\\
\hfuzz=500pt\includegraphics[width=1em]{connector.pdf}\includegraphics[width=1em]{element-mustset.pdf}~symmetric & \hfuzz=500pt  & \hfuzz=500pt symmetric continuation around the matrix edges\\
\hline
\end{tabularx}


\subsection{File}
Read filter coefficients of \eqref{digitalFilterType:arma} from a coefficient file.
One column might define the index $n$
of the coefficients $a_n$ and $b_n$ in the other columns.


\keepXColumns
\begin{tabularx}{\textwidth}{N T A}
\hline
Name & Type & Annotation\\
\hline
\hfuzz=500pt\includegraphics[width=1em]{element-mustset.pdf}~inputfileMatrix & \hfuzz=500pt filename & \hfuzz=500pt matrix with filter coefficients\\
\hfuzz=500pt\includegraphics[width=1em]{element.pdf}~index & \hfuzz=500pt expression & \hfuzz=500pt index of coefficients (input columns are named data0, data1, ...)\\
\hfuzz=500pt\includegraphics[width=1em]{element.pdf}~bn & \hfuzz=500pt expression & \hfuzz=500pt MA coefficients (moving average) (input columns are named data0, data1, ...)\\
\hfuzz=500pt\includegraphics[width=1em]{element.pdf}~an & \hfuzz=500pt expression & \hfuzz=500pt AR coefficients (autoregressive) (input columns are named data0, data1, ...)\\
\hfuzz=500pt\includegraphics[width=1em]{element.pdf}~backwardDirection & \hfuzz=500pt boolean & \hfuzz=500pt apply filter in backward direction\\
\hfuzz=500pt\includegraphics[width=1em]{element.pdf}~inFrequencyDomain & \hfuzz=500pt boolean & \hfuzz=500pt apply filter in frequency domain\\
\hfuzz=500pt\includegraphics[width=1em]{element-mustset.pdf}~padType & \hfuzz=500pt choice & \hfuzz=500pt \\
\hfuzz=500pt\includegraphics[width=1em]{connector.pdf}\includegraphics[width=1em]{element-mustset.pdf}~none & \hfuzz=500pt  & \hfuzz=500pt no padding is applied\\
\hfuzz=500pt\includegraphics[width=1em]{connector.pdf}\includegraphics[width=1em]{element-mustset.pdf}~zero & \hfuzz=500pt  & \hfuzz=500pt zero padding\\
\hfuzz=500pt\includegraphics[width=1em]{connector.pdf}\includegraphics[width=1em]{element-mustset.pdf}~constant & \hfuzz=500pt  & \hfuzz=500pt pad using first and last value\\
\hfuzz=500pt\includegraphics[width=1em]{connector.pdf}\includegraphics[width=1em]{element-mustset.pdf}~periodic & \hfuzz=500pt  & \hfuzz=500pt periodic continuation of matrix\\
\hfuzz=500pt\includegraphics[width=1em]{connector.pdf}\includegraphics[width=1em]{element-mustset.pdf}~symmetric & \hfuzz=500pt  & \hfuzz=500pt symmetric continuation around the matrix edges\\
\hline
\end{tabularx}


\subsection{Wavelet}
Filter representation of a wavelet.


\keepXColumns
\begin{tabularx}{\textwidth}{N T A}
\hline
Name & Type & Annotation\\
\hline
\hfuzz=500pt\includegraphics[width=1em]{element-mustset.pdf}~inputfileWavelet & \hfuzz=500pt filename & \hfuzz=500pt wavelet coefficients\\
\hfuzz=500pt\includegraphics[width=1em]{element-mustset.pdf}~type & \hfuzz=500pt choice & \hfuzz=500pt filter type\\
\hfuzz=500pt\includegraphics[width=1em]{connector.pdf}\includegraphics[width=1em]{element-mustset.pdf}~lowpass & \hfuzz=500pt  & \hfuzz=500pt \\
\hfuzz=500pt\includegraphics[width=1em]{connector.pdf}\includegraphics[width=1em]{element-mustset.pdf}~highpass & \hfuzz=500pt  & \hfuzz=500pt \\
\hfuzz=500pt\includegraphics[width=1em]{element.pdf}~level & \hfuzz=500pt uint & \hfuzz=500pt compute filter for specific decomposition level\\
\hfuzz=500pt\includegraphics[width=1em]{element.pdf}~backwardDirection & \hfuzz=500pt boolean & \hfuzz=500pt apply filter in backward direction\\
\hfuzz=500pt\includegraphics[width=1em]{element.pdf}~inFrequencyDomain & \hfuzz=500pt boolean & \hfuzz=500pt apply filter in frequency domain\\
\hfuzz=500pt\includegraphics[width=1em]{element-mustset.pdf}~padType & \hfuzz=500pt choice & \hfuzz=500pt \\
\hfuzz=500pt\includegraphics[width=1em]{connector.pdf}\includegraphics[width=1em]{element-mustset.pdf}~none & \hfuzz=500pt  & \hfuzz=500pt no padding is applied\\
\hfuzz=500pt\includegraphics[width=1em]{connector.pdf}\includegraphics[width=1em]{element-mustset.pdf}~zero & \hfuzz=500pt  & \hfuzz=500pt zero padding\\
\hfuzz=500pt\includegraphics[width=1em]{connector.pdf}\includegraphics[width=1em]{element-mustset.pdf}~constant & \hfuzz=500pt  & \hfuzz=500pt pad using first and last value\\
\hfuzz=500pt\includegraphics[width=1em]{connector.pdf}\includegraphics[width=1em]{element-mustset.pdf}~periodic & \hfuzz=500pt  & \hfuzz=500pt periodic continuation of matrix\\
\hfuzz=500pt\includegraphics[width=1em]{connector.pdf}\includegraphics[width=1em]{element-mustset.pdf}~symmetric & \hfuzz=500pt  & \hfuzz=500pt symmetric continuation around the matrix edges\\
\hline
\end{tabularx}


\subsection{Notch}
Implemented after Christian Siemes' dissertation, page 106.

\fig{!hb}{0.6}{DigitalFilter_notch}{fig:DigitalFilterNotch}{Amplitude response of a notch filter of order three with default settings.}


\keepXColumns
\begin{tabularx}{\textwidth}{N T A}
\hline
Name & Type & Annotation\\
\hline
\hfuzz=500pt\includegraphics[width=1em]{element-mustset.pdf}~notchFrequency & \hfuzz=500pt double & \hfuzz=500pt normalized notch frequency w\_n = (f\_n/f\_nyq)\\
\hfuzz=500pt\includegraphics[width=1em]{element-mustset.pdf}~bandWidth & \hfuzz=500pt double & \hfuzz=500pt bandwidth at -3db. Quality factor of filter Q = w\_n/bw\\
\hfuzz=500pt\includegraphics[width=1em]{element.pdf}~backwardDirection & \hfuzz=500pt boolean & \hfuzz=500pt apply filter in backward direction\\
\hfuzz=500pt\includegraphics[width=1em]{element.pdf}~inFrequencyDomain & \hfuzz=500pt boolean & \hfuzz=500pt apply filter in frequency domain\\
\hfuzz=500pt\includegraphics[width=1em]{element-mustset.pdf}~padType & \hfuzz=500pt choice & \hfuzz=500pt \\
\hfuzz=500pt\includegraphics[width=1em]{connector.pdf}\includegraphics[width=1em]{element-mustset.pdf}~none & \hfuzz=500pt  & \hfuzz=500pt no padding is applied\\
\hfuzz=500pt\includegraphics[width=1em]{connector.pdf}\includegraphics[width=1em]{element-mustset.pdf}~zero & \hfuzz=500pt  & \hfuzz=500pt zero padding\\
\hfuzz=500pt\includegraphics[width=1em]{connector.pdf}\includegraphics[width=1em]{element-mustset.pdf}~constant & \hfuzz=500pt  & \hfuzz=500pt pad using first and last value\\
\hfuzz=500pt\includegraphics[width=1em]{connector.pdf}\includegraphics[width=1em]{element-mustset.pdf}~periodic & \hfuzz=500pt  & \hfuzz=500pt periodic continuation of matrix\\
\hfuzz=500pt\includegraphics[width=1em]{connector.pdf}\includegraphics[width=1em]{element-mustset.pdf}~symmetric & \hfuzz=500pt  & \hfuzz=500pt symmetric continuation around the matrix edges\\
\hline
\end{tabularx}


\subsection{Decorrelation}
Moving average decorrelation filter based on eigendecomposition of a Toeplitz covariance matrix.


\keepXColumns
\begin{tabularx}{\textwidth}{N T A}
\hline
Name & Type & Annotation\\
\hline
\hfuzz=500pt\includegraphics[width=1em]{element-mustset.pdf}~inputfileCovarianceFunction & \hfuzz=500pt filename & \hfuzz=500pt covariance function of time series\\
\hfuzz=500pt\includegraphics[width=1em]{element.pdf}~inFrequencyDomain & \hfuzz=500pt boolean & \hfuzz=500pt apply filter in frequency domain\\
\hfuzz=500pt\includegraphics[width=1em]{element-mustset.pdf}~padType & \hfuzz=500pt choice & \hfuzz=500pt \\
\hfuzz=500pt\includegraphics[width=1em]{connector.pdf}\includegraphics[width=1em]{element-mustset.pdf}~none & \hfuzz=500pt  & \hfuzz=500pt no padding is applied\\
\hfuzz=500pt\includegraphics[width=1em]{connector.pdf}\includegraphics[width=1em]{element-mustset.pdf}~zero & \hfuzz=500pt  & \hfuzz=500pt zero padding\\
\hfuzz=500pt\includegraphics[width=1em]{connector.pdf}\includegraphics[width=1em]{element-mustset.pdf}~constant & \hfuzz=500pt  & \hfuzz=500pt pad using first and last value\\
\hfuzz=500pt\includegraphics[width=1em]{connector.pdf}\includegraphics[width=1em]{element-mustset.pdf}~periodic & \hfuzz=500pt  & \hfuzz=500pt periodic continuation of matrix\\
\hfuzz=500pt\includegraphics[width=1em]{connector.pdf}\includegraphics[width=1em]{element-mustset.pdf}~symmetric & \hfuzz=500pt  & \hfuzz=500pt symmetric continuation around the matrix edges\\
\hline
\end{tabularx}


\subsection{TimeLag}
Lag operator in digital filter representation.


\keepXColumns
\begin{tabularx}{\textwidth}{N T A}
\hline
Name & Type & Annotation\\
\hline
\hfuzz=500pt\includegraphics[width=1em]{element-mustset.pdf}~lag & \hfuzz=500pt int & \hfuzz=500pt lag epochs: 1 (lag); -1 (lead)\\
\hfuzz=500pt\includegraphics[width=1em]{element.pdf}~inFrequencyDomain & \hfuzz=500pt boolean & \hfuzz=500pt apply filter in frequency domain\\
\hfuzz=500pt\includegraphics[width=1em]{element-mustset.pdf}~padType & \hfuzz=500pt choice & \hfuzz=500pt \\
\hfuzz=500pt\includegraphics[width=1em]{connector.pdf}\includegraphics[width=1em]{element-mustset.pdf}~none & \hfuzz=500pt  & \hfuzz=500pt no padding is applied\\
\hfuzz=500pt\includegraphics[width=1em]{connector.pdf}\includegraphics[width=1em]{element-mustset.pdf}~zero & \hfuzz=500pt  & \hfuzz=500pt zero padding\\
\hfuzz=500pt\includegraphics[width=1em]{connector.pdf}\includegraphics[width=1em]{element-mustset.pdf}~constant & \hfuzz=500pt  & \hfuzz=500pt pad using first and last value\\
\hfuzz=500pt\includegraphics[width=1em]{connector.pdf}\includegraphics[width=1em]{element-mustset.pdf}~periodic & \hfuzz=500pt  & \hfuzz=500pt periodic continuation of matrix\\
\hfuzz=500pt\includegraphics[width=1em]{connector.pdf}\includegraphics[width=1em]{element-mustset.pdf}~symmetric & \hfuzz=500pt  & \hfuzz=500pt symmetric continuation around the matrix edges\\
\hline
\end{tabularx}


\subsection{ReduceFilterOutput}
Removes the filtered signal from the input, i.e. the input is passed
through a \configClass{digitalFilter}{digitalFilterType} with a frequency response of $1-H(f)$.


\keepXColumns
\begin{tabularx}{\textwidth}{N T A}
\hline
Name & Type & Annotation\\
\hline
\hfuzz=500pt\includegraphics[width=1em]{element-mustset-unbounded.pdf}~filter & \hfuzz=500pt \hyperref[digitalFilterType]{digitalFilter} & \hfuzz=500pt remove filter output from input signal\\
\hline
\end{tabularx}

\clearpage
%==================================

\section{Doodson}\label{doodson}
This is a string which describes a tidal frequency either coded as Doodson number
or using Darwin´s name, e.g. \verb|255.555| or \verb|M2|.

The following names are defined:
\begin{itemize}
\item \verb|055.565|: \verb|om1|  \item \verb|055.575|: \verb|om2|     \item \verb|056.554|: \verb|sa|
\item \verb|056.555|: \verb|sa|   \item \verb|057.555|: \verb|ssa|     \item \verb|058.554|: \verb|sta|
\item \verb|063.655|: \verb|msm|  \item \verb|065.455|: \verb|mm|      \item \verb|073.555|: \verb|msf|
\item \verb|075.555|: \verb|mf|   \item \verb|083.655|: \verb|mstm|    \item \verb|085.455|: \verb|mtm|
\item \verb|093.555|: \verb|msq|  \item \verb|093.555|: \verb|msqm|    \item \verb|125.755|: \verb|2q1|
\item \verb|127.555|: \verb|sig1| \item \verb|127.555|: \verb|sigma1|  \item \verb|135.655|: \verb|q1|
\item \verb|137.455|: \verb|ro1|  \item \verb|137.455|: \verb|rho1|    \item \verb|145.555|: \verb|o1|
\item \verb|147.555|: \verb|tau1| \item \verb|155.655|: \verb|m1|      \item \verb|157.455|: \verb|chi1|
\item \verb|162.556|: \verb|pi1|  \item \verb|163.555|: \verb|p1|      \item \verb|164.555|: \verb|s1|
\item \verb|165.555|: \verb|k1|   \item \verb|166.554|: \verb|psi1|    \item \verb|167.555|: \verb|fi1|
\item \verb|167.555|: \verb|phi1| \item \verb|173.655|: \verb|the1|    \item \verb|173.655|: \verb|theta1|
\item \verb|175.455|: \verb|j1|   \item \verb|183.555|: \verb|so1|     \item \verb|185.555|: \verb|oo1|
\item \verb|195.455|: \verb|v1|   \item \verb|225.855|: \verb|3n2|     \item \verb|227.655|: \verb|eps2|
\item \verb|235.755|: \verb|2n2|  \item \verb|237.555|: \verb|mu2|     \item \verb|237.555|: \verb|mi2|
\item \verb|245.655|: \verb|n2|   \item \verb|247.455|: \verb|nu2|     \item \verb|247.455|: \verb|ni2|
\item \verb|253.755|: \verb|gam2| \item \verb|254.556|: \verb|alf2|    \item \verb|255.555|: \verb|m2|
\item \verb|256.554|: \verb|bet2| \item \verb|257.555|: \verb|dlt2|    \item \verb|263.655|: \verb|la2|
\item \verb|263.655|: \verb|lmb2| \item \verb|263.655|: \verb|lambda2| \item \verb|265.455|: \verb|l2|
\item \verb|271.557|: \verb|2t2|  \item \verb|272.556|: \verb|t2|      \item \verb|273.555|: \verb|s2|
\item \verb|274.554|: \verb|r2|   \item \verb|275.555|: \verb|k2|      \item \verb|283.655|: \verb|ksi2|
\item \verb|285.455|: \verb|eta2| \item \verb|355.555|: \verb|m3|      \item \verb|381.555|: \verb|t3|
\item \verb|382.555|: \verb|s3|   \item \verb|383.555|: \verb|r3|      \item \verb|435.755|: \verb|n4|
\item \verb|445.655|: \verb|mn4|  \item \verb|455.555|: \verb|m4|      \item \verb|473.555|: \verb|ms4|
\item \verb|491.555|: \verb|s4|   \item \verb|655.555|: \verb|m6|      \item \verb|855.555|: \verb|m8|
\end{itemize}

\clearpage
%==================================

\section{EarthRotation}\label{earthRotationType}
This class realize the transformation between a terestrial
reference frame (TRF) and a celestial reference frame (CRF).


\subsection{File}\label{earthRotationType:file}
This class realize the transformation by interpolation from file.
This file can be created with \program{EarthOrientationParameterTimeSeries}.


\keepXColumns
\begin{tabularx}{\textwidth}{N T A}
\hline
Name & Type & Annotation\\
\hline
\hfuzz=500pt\includegraphics[width=1em]{element-mustset.pdf}~inputfileEOP & \hfuzz=500pt filename & \hfuzz=500pt \\
\hfuzz=500pt\includegraphics[width=1em]{element.pdf}~interpolationDegree & \hfuzz=500pt uint & \hfuzz=500pt for polynomial interpolation\\
\hline
\end{tabularx}


\subsection{Iers2010}
This class realize the transformation according to the IERS2010 conventions
given by the \emph{International Earth Rotation and Reference Systems Service} (IERS).
A file with the earth orientation parameter is needed (\configFile{inputfileEOP}{earthOrientationParameter}).


\keepXColumns
\begin{tabularx}{\textwidth}{N T A}
\hline
Name & Type & Annotation\\
\hline
\hfuzz=500pt\includegraphics[width=1em]{element.pdf}~inputfileEOP & \hfuzz=500pt filename & \hfuzz=500pt \\
\hfuzz=500pt\includegraphics[width=1em]{element.pdf}~truncatedNutation & \hfuzz=500pt boolean & \hfuzz=500pt use truncated nutation model (IAU2006B)\\
\hline
\end{tabularx}


\subsection{Iers2010b}\label{earthRotationType:iers2010b}
This class realize the transformation according to the IERS2010 conventions
given by the \emph{International Earth Rotation and Reference Systems Service} (IERS).
A file with the earth orientation parameter is needed (\configFile{inputfileEOP}{earthOrientationParameter}).
Includes additional high-frequency EOP models (\configFile{inputfileDoodsonEOP}{doodsonEarthOrientationParameter}).


\keepXColumns
\begin{tabularx}{\textwidth}{N T A}
\hline
Name & Type & Annotation\\
\hline
\hfuzz=500pt\includegraphics[width=1em]{element.pdf}~inputfileEOP & \hfuzz=500pt filename & \hfuzz=500pt \\
\hfuzz=500pt\includegraphics[width=1em]{element.pdf}~inputfileDoodsonEOP & \hfuzz=500pt filename & \hfuzz=500pt \\
\hline
\end{tabularx}


\subsection{Iers2003}
This class realize the transformation according to IERS2003 conventions
given by the \emph{International Earth Rotation and Reference Systems Service} (IERS).
A file with the earth orientation parameter is needed (\configFile{inputfileEOP}{earthOrientationParameter}).

The following subroutines are used:
\begin{itemize}
\item BPN2000.f,
\item ERA2000.f,
\item pmsdnut.f,
\item POM2000.f,
\item SP2000.f,
\item T2C2000.f,
\item XYS2000A.f
\end{itemize}
from \url{ftp://maia.usno.navy.mil/conv2000/chapter5/} and
\begin{itemize}
\item orthoeop.f
\end{itemize}
from \url{ftp://maia.usno.navy.mil/conv2000/chapter8/}


\keepXColumns
\begin{tabularx}{\textwidth}{N T A}
\hline
Name & Type & Annotation\\
\hline
\hfuzz=500pt\includegraphics[width=1em]{element-mustset.pdf}~inputfileEOP & \hfuzz=500pt filename & \hfuzz=500pt \\
\hline
\end{tabularx}


\subsection{Iers1996}
Very old.


\keepXColumns
\begin{tabularx}{\textwidth}{N T A}
\hline
Name & Type & Annotation\\
\hline
\hfuzz=500pt\includegraphics[width=1em]{element.pdf}~inputfileEOP & \hfuzz=500pt filename & \hfuzz=500pt \\
\hfuzz=500pt\includegraphics[width=1em]{element-mustset.pdf}~inputfileNutation & \hfuzz=500pt filename & \hfuzz=500pt \\
\hline
\end{tabularx}


\subsection{Gmst}
The transformation is realized as rotation about the z-axis.
The angle ist given by the Greenwich Mean Siderial Time (GMST).
\begin{verbatim}
  Double Tu0 = (timeUTC.mjdInt()-51544.5)/36525.0;

  Double GMST0 = (6.0/24 + 41.0/(24*60) + 50.54841/(24*60*60))
               + (8640184.812866/(24*60*60))*Tu0
               + (0.093104/(24*60*60))*Tu0*Tu0
               + (-6.2e-6/(24*60*60))*Tu0*Tu0*Tu0;
  Double r = 1.002737909350795 + 5.9006e-11*Tu0 - 5.9e-15*Tu0*Tu0;
  GMST = fmod(2*PI*(GMST0 + r * timeUTC.mjdMod()), 2*PI);
\end{verbatim}


\subsection{Earth Rotation Angle (ERA)}
The transformation is realized as rotation about the z-axis.
The angle ist given by the Earth Rotation Angle (ERA).


\subsection{Z-Axis}
The transformation is realized as rotation about the z-axis.
You must specify the angle (\config{initialAngle}) at \config{time0} and
the angular velocity (\config{angularVelocity}).


\keepXColumns
\begin{tabularx}{\textwidth}{N T A}
\hline
Name & Type & Annotation\\
\hline
\hfuzz=500pt\includegraphics[width=1em]{element-mustset.pdf}~initialAngle & \hfuzz=500pt double & \hfuzz=500pt Angle at time0 [rad]\\
\hfuzz=500pt\includegraphics[width=1em]{element-mustset.pdf}~angularVelocity & \hfuzz=500pt double & \hfuzz=500pt [rad/s]\\
\hfuzz=500pt\includegraphics[width=1em]{element-mustset.pdf}~time0 & \hfuzz=500pt time & \hfuzz=500pt \\
\hline
\end{tabularx}


\subsection{StarCamera}
This class reads quaternions from an instrument file and interpolates to the given time stamp.


\keepXColumns
\begin{tabularx}{\textwidth}{N T A}
\hline
Name & Type & Annotation\\
\hline
\hfuzz=500pt\includegraphics[width=1em]{element-mustset.pdf}~inputfileStarCamera & \hfuzz=500pt filename & \hfuzz=500pt \\
\hfuzz=500pt\includegraphics[width=1em]{element.pdf}~interpolationDegree & \hfuzz=500pt uint & \hfuzz=500pt degree of interpolation polynomial\\
\hline
\end{tabularx}


\subsection{MoonRotation}
This class realizes the transformation between the moon-fixed system
(Principal Axis System (PA) or Mean Earth System (ME))
and the ICRS according to the JPL ephemeris file.


\keepXColumns
\begin{tabularx}{\textwidth}{N T A}
\hline
Name & Type & Annotation\\
\hline
\hfuzz=500pt\includegraphics[width=1em]{element.pdf}~inputfileEphemerides & \hfuzz=500pt filename & \hfuzz=500pt librations\\
\hfuzz=500pt\includegraphics[width=1em]{element-mustset.pdf}~moonfixedSystem & \hfuzz=500pt choice & \hfuzz=500pt \\
\hfuzz=500pt\includegraphics[width=1em]{connector.pdf}\includegraphics[width=1em]{element-mustset.pdf}~PA & \hfuzz=500pt  & \hfuzz=500pt Principal Axis System\\
\hfuzz=500pt\includegraphics[width=1em]{connector.pdf}\includegraphics[width=1em]{element-mustset.pdf}~ME & \hfuzz=500pt  & \hfuzz=500pt Mean Earth System\\
\hline
\end{tabularx}

\clearpage
%==================================

\section{Eclipse}\label{eclipseType}
Shadowing of satellites by moon and Earth provided as factor
between $[0,1]$ with 0: full shadow and 1: full sun light.


\subsection{Conical}
\fig{!hb}{0.8}{eclipseConical}{fig:eclipseConical}{Modelling umbra and penumbra.}


\subsection{SOLAARS}
Earth’s penumbra modeling with Solar radiation pressure with
Oblateness and Lower Atmospheric Absorption, Refraction, and Scattering (SOLAARS).
See Robertson, Robbie. (2015),
Highly Physical Solar Radiation Pressure Modeling During Penumbra Transitions (pp. 67-75).

\clearpage
%==================================

\section{Ephemerides}\label{ephemeridesType}
Ephemerides of Sun, Moon and planets.
The coordinate system is defined as center of \configClass{origin}{planetType}.


\section{JPL}\label{ephemeridesType:jpl}
Using \verb|DExxx| ephemerides from NASA Jet Propulsion Laboratory (JPL).


\keepXColumns
\begin{tabularx}{\textwidth}{N T A}
\hline
Name & Type & Annotation\\
\hline
\hfuzz=500pt\includegraphics[width=1em]{element-mustset.pdf}~inputfileEphemerides & \hfuzz=500pt filename & \hfuzz=500pt \\
\hfuzz=500pt\includegraphics[width=1em]{element.pdf}~origin & \hfuzz=500pt \hyperref[planetType]{planet} & \hfuzz=500pt center of coordinate system\\
\hline
\end{tabularx}

\clearpage
%==================================

\section{Forces}\label{forcesType}
This class provides the forces acting on a satellite.
This encompasses \configClass{gravityfield}{gravityfieldType}, \configClass{tides}{tidesType}
and \configClass{miscAccelerations}{miscAccelerationsType}.


\keepXColumns
\begin{tabularx}{\textwidth}{N T A}
\hline
Name & Type & Annotation\\
\hline
\hfuzz=500pt\includegraphics[width=1em]{element-mustset.pdf}~forcesType & \hfuzz=500pt sequence & \hfuzz=500pt \\
\hfuzz=500pt\includegraphics[width=1em]{connector.pdf}\includegraphics[width=1em]{element-unbounded.pdf}~gravityfield & \hfuzz=500pt \hyperref[gravityfieldType]{gravityfield} & \hfuzz=500pt \\
\hfuzz=500pt\includegraphics[width=1em]{connector.pdf}\includegraphics[width=1em]{element-unbounded.pdf}~tides & \hfuzz=500pt \hyperref[tidesType]{tides} & \hfuzz=500pt \\
\hfuzz=500pt\includegraphics[width=1em]{connector.pdf}\includegraphics[width=1em]{element-unbounded.pdf}~miscAccelerations & \hfuzz=500pt \hyperref[miscAccelerationsType]{miscAccelerations} & \hfuzz=500pt \\
\hline
\end{tabularx}

\clearpage
%==================================

\section{GnssAntennaDefintionList}\label{gnssAntennaDefintionListType}
Provides a list of GnssAntennaDefinitions as used in \program{GnssAntennaDefinitionCreate}.


\subsection{New}
Creates a new antenna.


\keepXColumns
\begin{tabularx}{\textwidth}{N T A}
\hline
Name & Type & Annotation\\
\hline
\hfuzz=500pt\includegraphics[width=1em]{element.pdf}~name & \hfuzz=500pt string & \hfuzz=500pt \\
\hfuzz=500pt\includegraphics[width=1em]{element.pdf}~serial & \hfuzz=500pt string & \hfuzz=500pt \\
\hfuzz=500pt\includegraphics[width=1em]{element.pdf}~radome & \hfuzz=500pt string & \hfuzz=500pt \\
\hfuzz=500pt\includegraphics[width=1em]{element.pdf}~comment & \hfuzz=500pt string & \hfuzz=500pt \\
\hfuzz=500pt\includegraphics[width=1em]{element-mustset-unbounded.pdf}~pattern & \hfuzz=500pt sequence & \hfuzz=500pt \\
\hfuzz=500pt\includegraphics[width=1em]{connector.pdf}\includegraphics[width=1em]{element-mustset.pdf}~type & \hfuzz=500pt \hyperref[gnssType]{gnssType} & \hfuzz=500pt pattern matching of observation types\\
\hfuzz=500pt\includegraphics[width=1em]{connector.pdf}\includegraphics[width=1em]{element.pdf}~offsetX & \hfuzz=500pt double & \hfuzz=500pt [m] antenna center offset\\
\hfuzz=500pt\includegraphics[width=1em]{connector.pdf}\includegraphics[width=1em]{element.pdf}~offsetY & \hfuzz=500pt double & \hfuzz=500pt [m] antenna center offset\\
\hfuzz=500pt\includegraphics[width=1em]{connector.pdf}\includegraphics[width=1em]{element.pdf}~offsetZ & \hfuzz=500pt double & \hfuzz=500pt [m] antenna center offset\\
\hfuzz=500pt\includegraphics[width=1em]{connector.pdf}\includegraphics[width=1em]{element-mustset.pdf}~deltaAzimuth & \hfuzz=500pt angle & \hfuzz=500pt [degree] step size\\
\hfuzz=500pt\includegraphics[width=1em]{connector.pdf}\includegraphics[width=1em]{element-mustset.pdf}~deltaZenith & \hfuzz=500pt angle & \hfuzz=500pt [degree] step size\\
\hfuzz=500pt\includegraphics[width=1em]{connector.pdf}\includegraphics[width=1em]{element.pdf}~maxZenith & \hfuzz=500pt angle & \hfuzz=500pt [degree]\\
\hfuzz=500pt\includegraphics[width=1em]{connector.pdf}\includegraphics[width=1em]{element.pdf}~values & \hfuzz=500pt expression & \hfuzz=500pt [m] expression (zenith, azimuth: variables)\\
\hline
\end{tabularx}


\subsection{FromFile}
Select all or the first antenna from an \file{antenna definition file}{gnssAntennaDefinition}
which matches the wildcards.


\keepXColumns
\begin{tabularx}{\textwidth}{N T A}
\hline
Name & Type & Annotation\\
\hline
\hfuzz=500pt\includegraphics[width=1em]{element-mustset.pdf}~inputfileAntennaDefinition & \hfuzz=500pt filename & \hfuzz=500pt \\
\hfuzz=500pt\includegraphics[width=1em]{element.pdf}~name & \hfuzz=500pt string & \hfuzz=500pt \\
\hfuzz=500pt\includegraphics[width=1em]{element.pdf}~serial & \hfuzz=500pt string & \hfuzz=500pt \\
\hfuzz=500pt\includegraphics[width=1em]{element.pdf}~radome & \hfuzz=500pt string & \hfuzz=500pt \\
\hfuzz=500pt\includegraphics[width=1em]{element.pdf}~onlyFirstMatch & \hfuzz=500pt boolean & \hfuzz=500pt otherwise all machting antennas included\\
\hline
\end{tabularx}


\subsection{FromStationInfo}
Select all antennas from an \file{antenna definition file}{gnssAntennaDefinition}
which are used by a station within a defined time interval.
With \config{specializeAntenna} an individual antenna is created for each different serial number
using the general type specific values from file.


\keepXColumns
\begin{tabularx}{\textwidth}{N T A}
\hline
Name & Type & Annotation\\
\hline
\hfuzz=500pt\includegraphics[width=1em]{element-mustset.pdf}~inputfileStationInfo & \hfuzz=500pt filename & \hfuzz=500pt \\
\hfuzz=500pt\includegraphics[width=1em]{element-mustset.pdf}~inputfileAntennaDefinition & \hfuzz=500pt filename & \hfuzz=500pt \\
\hfuzz=500pt\includegraphics[width=1em]{element.pdf}~timeStart & \hfuzz=500pt time & \hfuzz=500pt only antennas used in this time interval\\
\hfuzz=500pt\includegraphics[width=1em]{element.pdf}~timeEnd & \hfuzz=500pt time & \hfuzz=500pt only antennas used in this time interval\\
\hfuzz=500pt\includegraphics[width=1em]{element.pdf}~specializeAntenna & \hfuzz=500pt boolean & \hfuzz=500pt e.g. separate different serial numbers from stationInfo\\
\hline
\end{tabularx}


\subsection{Resample}
The azimuth and elevation dependent antenna center variations (patterns) of all \config{antenna}s
are resampled to a new resolution.


\keepXColumns
\begin{tabularx}{\textwidth}{N T A}
\hline
Name & Type & Annotation\\
\hline
\hfuzz=500pt\includegraphics[width=1em]{element-mustset-unbounded.pdf}~antenna & \hfuzz=500pt \hyperref[gnssAntennaDefintionListType]{gnssAntennaDefintionList} & \hfuzz=500pt \\
\hfuzz=500pt\includegraphics[width=1em]{element.pdf}~deltaAzimuth & \hfuzz=500pt angle & \hfuzz=500pt [degree] step size, empty: no change\\
\hfuzz=500pt\includegraphics[width=1em]{element.pdf}~deltaZenith & \hfuzz=500pt angle & \hfuzz=500pt [degree] step size, empty: no change\\
\hfuzz=500pt\includegraphics[width=1em]{element.pdf}~maxZenith & \hfuzz=500pt angle & \hfuzz=500pt [degree], empty: no change\\
\hline
\end{tabularx}


\subsection{Transform}
This class can be used to separate general antenna patterns for different \configClass{gnssType}{gnssType}s.
If the \config{antenna}s contain only one pattern for all GPS observations on the L1 frequency (\verb|*1*G**|),
the \config{patternTypes}=\verb|C1*G**| and \verb|L1*G**| create two patterns with the \verb|*1*G**| patterm as template.
The first matching pattern in the \config{antenna} is used as template.
Also new \config{additionalPattern} can be added (e.g. for \verb|*5*G**|).
With \config{addExistingPatterns} all already existing patterns that don't match completely to any of the above are added.


\keepXColumns
\begin{tabularx}{\textwidth}{N T A}
\hline
Name & Type & Annotation\\
\hline
\hfuzz=500pt\includegraphics[width=1em]{element-mustset-unbounded.pdf}~antenna & \hfuzz=500pt \hyperref[gnssAntennaDefintionListType]{gnssAntennaDefintionList} & \hfuzz=500pt \\
\hfuzz=500pt\includegraphics[width=1em]{element-unbounded.pdf}~patternTypes & \hfuzz=500pt \hyperref[gnssType]{gnssType} & \hfuzz=500pt gnssType for each pattern (first match is used)\\
\hfuzz=500pt\includegraphics[width=1em]{element-unbounded.pdf}~additionalPattern & \hfuzz=500pt sequence & \hfuzz=500pt additional new patterns\\
\hfuzz=500pt\includegraphics[width=1em]{connector.pdf}\includegraphics[width=1em]{element-mustset.pdf}~type & \hfuzz=500pt \hyperref[gnssType]{gnssType} & \hfuzz=500pt pattern matching of observation types\\
\hfuzz=500pt\includegraphics[width=1em]{connector.pdf}\includegraphics[width=1em]{element.pdf}~offsetX & \hfuzz=500pt double & \hfuzz=500pt [m] antenna center offset\\
\hfuzz=500pt\includegraphics[width=1em]{connector.pdf}\includegraphics[width=1em]{element.pdf}~offsetY & \hfuzz=500pt double & \hfuzz=500pt [m] antenna center offset\\
\hfuzz=500pt\includegraphics[width=1em]{connector.pdf}\includegraphics[width=1em]{element.pdf}~offsetZ & \hfuzz=500pt double & \hfuzz=500pt [m] antenna center offset\\
\hfuzz=500pt\includegraphics[width=1em]{connector.pdf}\includegraphics[width=1em]{element-mustset.pdf}~deltaAzimuth & \hfuzz=500pt angle & \hfuzz=500pt [degree] step size\\
\hfuzz=500pt\includegraphics[width=1em]{connector.pdf}\includegraphics[width=1em]{element-mustset.pdf}~deltaZenith & \hfuzz=500pt angle & \hfuzz=500pt [degree] step size\\
\hfuzz=500pt\includegraphics[width=1em]{connector.pdf}\includegraphics[width=1em]{element.pdf}~maxZenith & \hfuzz=500pt angle & \hfuzz=500pt [degree]\\
\hfuzz=500pt\includegraphics[width=1em]{connector.pdf}\includegraphics[width=1em]{element.pdf}~values & \hfuzz=500pt expression & \hfuzz=500pt [m] expression (zenith, azimuth: variables)\\
\hfuzz=500pt\includegraphics[width=1em]{element.pdf}~addExistingPatterns & \hfuzz=500pt boolean & \hfuzz=500pt add existing patterns that don't match completely any of the above\\
\hline
\end{tabularx}


\subsection{Rename}
Replaces parts of the descrption of \config{antenna}s.
The star "\verb|*|" left this part untouched.


\keepXColumns
\begin{tabularx}{\textwidth}{N T A}
\hline
Name & Type & Annotation\\
\hline
\hfuzz=500pt\includegraphics[width=1em]{element-mustset-unbounded.pdf}~antenna & \hfuzz=500pt \hyperref[gnssAntennaDefintionListType]{gnssAntennaDefintionList} & \hfuzz=500pt \\
\hfuzz=500pt\includegraphics[width=1em]{element.pdf}~name & \hfuzz=500pt string & \hfuzz=500pt *: left this part untouched\\
\hfuzz=500pt\includegraphics[width=1em]{element.pdf}~serial & \hfuzz=500pt string & \hfuzz=500pt *: left this part untouched\\
\hfuzz=500pt\includegraphics[width=1em]{element.pdf}~radome & \hfuzz=500pt string & \hfuzz=500pt *: left this part untouched\\
\hfuzz=500pt\includegraphics[width=1em]{element.pdf}~comment & \hfuzz=500pt string & \hfuzz=500pt *: left this part untouched\\
\hline
\end{tabularx}

\clearpage
%==================================

\section{GnssParametrization}\label{gnssParametrizationType}
This class defines the models and parameters of the linearized observation equations
for all phase and code measurements (see \program{GnssProcessing})
\begin{equation}\label{gnssParametrizationType:model}
  \M l - \M f(\M x_0) = \left.\frac{\partial \M f(\M x)}{\partial \M x}\right|_{\M x_0} \Delta\M x + \M\epsilon,
\end{equation}
where the left side is the observation vector minus the effects computed from the a priori models.
After each least squares adjustment
(see \configClass{GnssProcessing:processingStep:estimate}{gnssProcessingStepType:estimate})
the a priori parameters are updated
\begin{equation}\label{gnssParametrizationType:update}
  \M x_0 := \M x_0 + \Delta\hat{\M x}.
\end{equation}
The vector $\M x_0$ can be written with
\configClass{GnssProcessing:processingStep:writeAprioriSolution}{gnssProcessingStepType:writeAprioriSolution}.
Any \config{outputfiles} defined in the parametrizations are written with
\configClass{GnssProcessing:processingStep:writeResults}{gnssProcessingStepType:writeResults}.

Each parametrization (and possible constraint equations) has a \config{name} which enables
activating/deactivating the estimation of subsets of $\Delta\M x$ with
\configClass{GnssProcessing:processingStep:selectParametrizations}{gnssProcessingStepType:selectParametrizations}.
The a priori model $\M f(\M x_0)$ is unaffected and is always reduced.

The model for the different observation types can be described as
\begin{equation}\label{gnssParametrizationType:gnssFullModel}
\begin{split}
  f[\tau\nu a]_r^s(\M x) &= \text{geometry}(\M r_r^s) + \text{clock}^s(t) + \text{clock}_r(t) \\
               &+ \text{iono}([tn],t,\M r_r^s) + \text{tropo}(t,\M r_r^s) \\
               &+ \text{ant}[\tau\nu a]^s  + \text{ant}[\tau\nu a]_r \\
               &+ \text{bias}[\tau\nu a]^s + \text{bias}[\tau\nu a]_r
               + \lambda[Ln] N[Lna]_r^s + \text{other}(\ldots) + \epsilon[\tau\nu a]_r^s
\end{split}
\end{equation}
The notation $[\tau\nu a]_r^s$ describes the
attribution to a signal type $\tau$ (i.e., C or L), frequency $\nu$,
signal attribute $a$ (e.g., C, W, Q, X), transmitting satellite $s$, and observing receiver $r$.
It follows the \href{https://files.igs.org/pub/data/format/rinex305.pdf}{RINEX 3 definition},
see \reference{GnssType}{gnssType}.

See also \program{GnssProcessing}.


\subsection{IonosphereSTEC}\label{gnssParametrizationType:ionosphereSTEC}
The influence of the ionosphere is modelled by a STEC parameter (slant total electron content)
between each transmitter and receiver at each epoch. These parameters are pre-eliminated
from the observation equations before accumulating the normal equations.
This is similar to using the ionosphere-free linear combination as observations
but only one STEC parameter is needed for an arbitrary number of observation types.

The influcence to the code and phase observation is modelled as
\begin{equation}\label{gnssParametrizationType:IonosphereSTEC:STEC}
\begin{split}
f_C(STEC) &= \frac{40.3}{f^2}STEC + \frac{7525\M b^T\M k}{f^3}STEC +  \frac{r}{f^4}STEC^2 \\
f_L(STEC) &= -\frac{40.3}{f^2}STEC - \frac{7525\M b^T\M k}{2f^3}STEC - \frac{r}{3f^4}STEC^2 + \text{bending}(E)STEC^2
\end{split}
\end{equation}
The second order term depends on the \configClass{magnetosphere}{magnetosphereType} $\M b$
and the direction of the signal $\M k$.

If further information about the ionosphere is available
(in the form of a prior model or as additional parametrizations
such as \configClass{parametrization:ionosphereMap}{gnssParametrizationType:ionosphereMap} or
\configClass{parametrization:ionosphereVTEC}{gnssParametrizationType:ionosphereVTEC}) the STEC
parameters describe local and short–term scintillations. The STEC parameters are estimated
as additions to the model and it is advised to constrain them towards zero
with a standard deviation of \config{sigmaSTEC}.


\keepXColumns
\begin{tabularx}{\textwidth}{N T A}
\hline
Name & Type & Annotation\\
\hline
\hfuzz=500pt\includegraphics[width=1em]{element.pdf}~name & \hfuzz=500pt string & \hfuzz=500pt used for parameter selection\\
\hfuzz=500pt\includegraphics[width=1em]{element.pdf}~apply2ndOrderCorrection & \hfuzz=500pt boolean & \hfuzz=500pt apply ionospheric correction\\
\hfuzz=500pt\includegraphics[width=1em]{element.pdf}~apply3rdOrderCorrection & \hfuzz=500pt boolean & \hfuzz=500pt apply ionospheric correction\\
\hfuzz=500pt\includegraphics[width=1em]{element.pdf}~applyBendingCorrection & \hfuzz=500pt boolean & \hfuzz=500pt apply ionospheric correction\\
\hfuzz=500pt\includegraphics[width=1em]{element-mustset.pdf}~magnetosphere & \hfuzz=500pt \hyperref[magnetosphereType]{magnetosphere} & \hfuzz=500pt \\
\hfuzz=500pt\includegraphics[width=1em]{element.pdf}~nameConstraint & \hfuzz=500pt string & \hfuzz=500pt used for parameter selection\\
\hfuzz=500pt\includegraphics[width=1em]{element.pdf}~sigmaSTEC & \hfuzz=500pt double & \hfuzz=500pt (0 = unconstrained) sigma [TECU] for STEC constraint\\
\hline
\end{tabularx}


\subsection{IonosphereVTEC}\label{gnssParametrizationType:ionosphereVTEC}
The influence of the ionosphere is modelled by a VTEC parameter (vertical total electron content)
for every selected receiver each epoch. The slant TEC is computed
using the elevation $E$ dependent Modified Single-Layer Model (MSLM) mapping function
\begin{equation}\label{gnssParametrizationType:IonosphereVTEC:STEC}
  STEC = \frac{VTEC}{\cos z'}
  \qquad\text{with}\qquad
  \sin z'= \left(\frac{R}{R+H}\right)\sin\left(\alpha(\pi/2-E)\right)
\end{equation}
inserted into eq. \eqref{gnssParametrizationType:IonosphereSTEC:STEC}.

The result is written as a \file{times series file}{instrument} at epochs with observations
depending on \configClass{GnssProcessing:processingStep:selectEpochs}{gnssProcessingStepType:selectEpochs}.

This class provides a simplifed model of the ionosphere for single receivers
and enables the separation of the TEC and signal biases, meaning
\configClass{parametrization:tecBiases}{gnssParametrizationType:tecBiases} becomes estimable.
Local and short-term scintillations should be considered by adding loosely constrained
\configClass{parametrization:ionosphereSTEC}{gnssParametrizationType:ionosphereSTEC}.


\keepXColumns
\begin{tabularx}{\textwidth}{N T A}
\hline
Name & Type & Annotation\\
\hline
\hfuzz=500pt\includegraphics[width=1em]{element.pdf}~name & \hfuzz=500pt string & \hfuzz=500pt \\
\hfuzz=500pt\includegraphics[width=1em]{element-mustset-unbounded.pdf}~selectReceivers & \hfuzz=500pt \hyperref[platformSelectorType]{platformSelector} & \hfuzz=500pt \\
\hfuzz=500pt\includegraphics[width=1em]{element.pdf}~outputfileVTEC & \hfuzz=500pt filename & \hfuzz=500pt variable \{station\} available\\
\hfuzz=500pt\includegraphics[width=1em]{element.pdf}~mapR & \hfuzz=500pt double & \hfuzz=500pt constant of MSLM mapping function\\
\hfuzz=500pt\includegraphics[width=1em]{element.pdf}~mapH & \hfuzz=500pt double & \hfuzz=500pt constant of MSLM mapping function\\
\hfuzz=500pt\includegraphics[width=1em]{element.pdf}~mapAlpha & \hfuzz=500pt double & \hfuzz=500pt constant of MSLM mapping function\\
\hline
\end{tabularx}


\subsection{IonosphereMap}\label{gnssParametrizationType:ionosphereMap}
TODO: reading and writing ionosphere maps not implemented yet.
% A priori ionopshere model can be provided with \configFile{inputfileMap}{gnssIonosphereMaps}.

The ionosphere is parametrized as a \configClass{temporal}{parametrizationTemporalType}ly changing
(e.g. hourly linear splines)
spherical harmonics expansion
\begin{equation}
  VTEC(\lambda,\theta,t) = \sum_{n=0}^{n_{max}} \sum_{m=0}^n c_{nm}(t)C_{nm}(\lambda,\theta)+s_{nm}(t)S_{nm}(\lambda,\theta)
\end{equation}
up to \config{maxDegree}=\verb|15| in a solar-geomagentic frame defined
by \configClass{magnetosphere}{magnetosphereType}.
The VTEC values are mapped to STEC values via eq.~\eqref{gnssParametrizationType:IonosphereVTEC:STEC}.

Local and short-term scintillations can be considered by adding constrained
\configClass{parametrization:ionosphereSTEC}{gnssParametrizationType:ionosphereSTEC}.
To account for signal biases add
\configClass{parametrization:tecBiases}{gnssParametrizationType:tecBiases}.


\keepXColumns
\begin{tabularx}{\textwidth}{N T A}
\hline
Name & Type & Annotation\\
\hline
\hfuzz=500pt\includegraphics[width=1em]{element.pdf}~name & \hfuzz=500pt string & \hfuzz=500pt \\
\hfuzz=500pt\includegraphics[width=1em]{element-mustset-unbounded.pdf}~selectReceivers & \hfuzz=500pt \hyperref[platformSelectorType]{platformSelector} & \hfuzz=500pt \\
\hfuzz=500pt\includegraphics[width=1em]{element-mustset.pdf}~maxDegree & \hfuzz=500pt uint & \hfuzz=500pt spherical harmonics\\
\hfuzz=500pt\includegraphics[width=1em]{element-unbounded.pdf}~temporal & \hfuzz=500pt \hyperref[parametrizationTemporalType]{parametrizationTemporal} & \hfuzz=500pt temporal evolution of TEC values\\
\hfuzz=500pt\includegraphics[width=1em]{element.pdf}~mapR & \hfuzz=500pt double & \hfuzz=500pt [m] constant of MSLM mapping function\\
\hfuzz=500pt\includegraphics[width=1em]{element.pdf}~mapH & \hfuzz=500pt double & \hfuzz=500pt [m] constant of MSLM mapping function\\
\hfuzz=500pt\includegraphics[width=1em]{element.pdf}~mapAlpha & \hfuzz=500pt double & \hfuzz=500pt constant of MSLM mapping function\\
\hfuzz=500pt\includegraphics[width=1em]{element-mustset.pdf}~magnetosphere & \hfuzz=500pt \hyperref[magnetosphereType]{magnetosphere} & \hfuzz=500pt \\
\hline
\end{tabularx}


\subsection{Clocks}\label{gnssParametrizationType:clocks}
Clock errors are estimated epoch-wise for each \configClass{selectTransmitter/Receiver}{platformSelectorType}.
No clock errors are estimated if no valid observations are available (e.g. data gaps in the observations).

These parameters are lineary dependent and would lead to a rank deficiency in the normal equation
matrix. To circumvent this issue, the estimation requires an additional zero-mean constraint added in each epoch.
This is realized with an additional observation equation
\begin{equation}
 0 = \sum_i \delta t^{s_i} + \sum_k \delta t_{r_k}
\end{equation}
summed over all \configClass{selectTransmitters/ReceiversZeroMean}{platformSelectorType}
with a standard deviation of \config{sigmaZeroMeanConstraint}.


\keepXColumns
\begin{tabularx}{\textwidth}{N T A}
\hline
Name & Type & Annotation\\
\hline
\hfuzz=500pt\includegraphics[width=1em]{element.pdf}~name & \hfuzz=500pt string & \hfuzz=500pt used for parameter selection\\
\hfuzz=500pt\includegraphics[width=1em]{element-unbounded.pdf}~selectTransmitters & \hfuzz=500pt \hyperref[platformSelectorType]{platformSelector} & \hfuzz=500pt \\
\hfuzz=500pt\includegraphics[width=1em]{element-unbounded.pdf}~selectReceivers & \hfuzz=500pt \hyperref[platformSelectorType]{platformSelector} & \hfuzz=500pt \\
\hfuzz=500pt\includegraphics[width=1em]{element.pdf}~outputfileClockTransmitter & \hfuzz=500pt filename & \hfuzz=500pt variable \{prn\} available\\
\hfuzz=500pt\includegraphics[width=1em]{element.pdf}~outputfileClockReceiver & \hfuzz=500pt filename & \hfuzz=500pt variable \{station\} available\\
\hfuzz=500pt\includegraphics[width=1em]{element.pdf}~nameConstraint & \hfuzz=500pt string & \hfuzz=500pt used for parameter selection\\
\hfuzz=500pt\includegraphics[width=1em]{element-unbounded.pdf}~selectTransmittersZeroMean & \hfuzz=500pt \hyperref[platformSelectorType]{platformSelector} & \hfuzz=500pt \\
\hfuzz=500pt\includegraphics[width=1em]{element-unbounded.pdf}~selectReceiversZeroMean & \hfuzz=500pt \hyperref[platformSelectorType]{platformSelector} & \hfuzz=500pt \\
\hfuzz=500pt\includegraphics[width=1em]{element.pdf}~sigmaZeroMeanConstraint & \hfuzz=500pt double & \hfuzz=500pt (0 = unconstrained) sigma [m] for zero-mean constraint over all selected clocks\\
\hline
\end{tabularx}


\subsection{ClocksModel}\label{gnssParametrizationType:clocksModel}
This parametrization is an alternative to \configClass{parametrization:clocks}{gnssParametrizationType:clocks}.
Clock errors are estimated epoch-wise for each \configClass{selectTransmitter/Receiver}{platformSelectorType}
and, opposed to \configClass{parametrization:clocks}{gnssParametrizationType:clocks}, are also estimated for epochs
that have no valid observations available (e.g. data gaps).

The clock error of the an epoch can be predicted by the clock error
of the preceding epoch and an unknown clock drift
\begin{equation}
  \Delta t_{i+1} = \Delta t_{i} + t_{drift} dt + \epsilon_i.
\end{equation}
This equation is applied as an additional constraint equation in each epoch
\begin{equation}
   0 = \Delta t_{i+1} - \Delta t_{i} - t_{drift} dt + \epsilon_i.
\end{equation}
The variance $\sigma^2(\epsilon)$ is estimated iteratively by variance component estimation (VCE).
Clock jumps are treated as outliers and are automatically downweighted as described in
\configClass{GnssProcessing:processingStep:estimate}{gnssProcessingStepType:estimate}.

The absolute initial clock error and clock drift cannot be determined if all receiver
and transmitter clocks are estimated together due to their linear dependency.
This linear dependency would lead to a rank deficiency in the normal equation matrix in the same
manner as described in \configClass{parametrization:clocks}{gnssParametrizationType:clocks}.
To circumvent the rank deficiency additional zero-mean constraints are required for the first and last epoch.
The realization of the constraint is done as an additional observation equation in the form
\begin{equation}
 0 = \sum_i \delta t^{s_i} + \sum_k \delta t_{r_k}
\end{equation}
summed over all \configClass{selectTransmitters/ReceiversZeroMean}{platformSelectorType}
with a standard deviation of \config{sigmaZeroMeanConstraint}.


\keepXColumns
\begin{tabularx}{\textwidth}{N T A}
\hline
Name & Type & Annotation\\
\hline
\hfuzz=500pt\includegraphics[width=1em]{element.pdf}~name & \hfuzz=500pt string & \hfuzz=500pt used for parameter selection\\
\hfuzz=500pt\includegraphics[width=1em]{element-unbounded.pdf}~selectTransmitters & \hfuzz=500pt \hyperref[platformSelectorType]{platformSelector} & \hfuzz=500pt \\
\hfuzz=500pt\includegraphics[width=1em]{element-unbounded.pdf}~selectReceivers & \hfuzz=500pt \hyperref[platformSelectorType]{platformSelector} & \hfuzz=500pt \\
\hfuzz=500pt\includegraphics[width=1em]{element.pdf}~outputfileClockTransmitter & \hfuzz=500pt filename & \hfuzz=500pt variable \{prn\} available\\
\hfuzz=500pt\includegraphics[width=1em]{element.pdf}~outputfileClockReceiver & \hfuzz=500pt filename & \hfuzz=500pt variable \{station\} available\\
\hfuzz=500pt\includegraphics[width=1em]{element.pdf}~huber & \hfuzz=500pt double & \hfuzz=500pt clock jumps \$>\$ huber*sigma0 are downweighted\\
\hfuzz=500pt\includegraphics[width=1em]{element.pdf}~huberPower & \hfuzz=500pt double & \hfuzz=500pt clock jumps \$>\$ huber: sigma=(e/huber)\textasciicircum{}huberPower*sigma0\\
\hfuzz=500pt\includegraphics[width=1em]{element.pdf}~nameConstraint & \hfuzz=500pt string & \hfuzz=500pt used for parameter selection\\
\hfuzz=500pt\includegraphics[width=1em]{element-unbounded.pdf}~selectTransmittersZeroMean & \hfuzz=500pt \hyperref[platformSelectorType]{platformSelector} & \hfuzz=500pt use these transmitters for zero-mean constraint\\
\hfuzz=500pt\includegraphics[width=1em]{element-unbounded.pdf}~selectReceiversZeroMean & \hfuzz=500pt \hyperref[platformSelectorType]{platformSelector} & \hfuzz=500pt use these receivers for zero-mean constraint\\
\hfuzz=500pt\includegraphics[width=1em]{element.pdf}~sigmaZeroMeanConstraint & \hfuzz=500pt double & \hfuzz=500pt (0 = unconstrained) sigma [m] for zero-mean constraint over all selected clocks\\
\hline
\end{tabularx}


\subsection{SignalBiases}\label{gnssParametrizationType:signalBiases}
Each code and phase observation (e.g \verb|C1C| or \verb|L2W|) contains a bias at transmitter/receiver level
\begin{equation}
  [\tau\nu a]_r^s(t) = \dots + \text{bias}[\tau\nu a]^s + \text{bias}[\tau\nu a]_r + \dots
\end{equation}
This class provides the apriori model $\M f(\M x_0)$ of eq. \eqref{gnssParametrizationType:model} only.

The \configFile{inputfileSignalBiasTransmitter/Receiver}{gnssSignalBias} are read
for each receiver and transmitter. The file name is interpreted as a template with
the variables \verb|{prn}| and \verb|{station}| being replaced by the name.
(Infos regarding the variables \verb|{prn}| and \verb|{station}| can be found in
\configClass{gnssTransmitterGeneratorType}{gnssTransmitterGeneratorType} and
\configClass{gnssReceiverGeneratorType}{gnssReceiverGeneratorType} respectively). The files can
be converted with \program{GnssSinexBias2SignalBias}.

The estimation of the biases is complex due to different linear dependencies, which
result in rank deficiencies in the system of normal equations.
For simplification the parametrization for $\Delta\M x$ has been split into:
\configClass{parametrization:codeBiases}{gnssParametrizationType:codeBiases},
\configClass{parametrization:tecBiases}{gnssParametrizationType:tecBiases}, and
\configClass{parametrization:ambiguities}{gnssParametrizationType:ambiguities} (including phase biases).
The file handling on the other hand still remains within this class. Any prior
values for the receiver/transmitter biases are read with the respective \config{inputFileSignalBias}.
All biases for a receiver/transmitter are accumulated and written to the respective \config{outputFileSignalBias}.


\keepXColumns
\begin{tabularx}{\textwidth}{N T A}
\hline
Name & Type & Annotation\\
\hline
\hfuzz=500pt\includegraphics[width=1em]{element.pdf}~name & \hfuzz=500pt string & \hfuzz=500pt used for parameter selection\\
\hfuzz=500pt\includegraphics[width=1em]{element-unbounded.pdf}~selectTransmitters & \hfuzz=500pt \hyperref[platformSelectorType]{platformSelector} & \hfuzz=500pt \\
\hfuzz=500pt\includegraphics[width=1em]{element-unbounded.pdf}~selectReceivers & \hfuzz=500pt \hyperref[platformSelectorType]{platformSelector} & \hfuzz=500pt \\
\hfuzz=500pt\includegraphics[width=1em]{element.pdf}~outputfileSignalBiasTransmitter & \hfuzz=500pt filename & \hfuzz=500pt variable \{prn\} available\\
\hfuzz=500pt\includegraphics[width=1em]{element.pdf}~outputfileSignalBiasReceiver & \hfuzz=500pt filename & \hfuzz=500pt variable \{station\} available\\
\hfuzz=500pt\includegraphics[width=1em]{element.pdf}~inputfileSignalBiasTransmitter & \hfuzz=500pt filename & \hfuzz=500pt variable \{prn\} available\\
\hfuzz=500pt\includegraphics[width=1em]{element.pdf}~inputfileSignalBiasReceiver & \hfuzz=500pt filename & \hfuzz=500pt variable \{station\} available\\
\hline
\end{tabularx}


\subsection{Ambiguities}\label{gnssParametrizationType:ambiguities}
Sets up an ambiguity parameter for each track and phase observation type.
\begin{equation}
  [L\nu a]_r^s(t) = \dots + \text{bias}[L\nu a]^s + \text{bias}[L\nu a]_r + \lambda[L\nu] N[L\nu a]_r^s
\end{equation}
As the phase observations contain a float bias at transmitter/receiver level, not all ambiguities
are resolvable to integer values. The number of resolvable ambiguities can be increased with
known phase biases read from file via \configClass{parametrization:signalBiases}{gnssParametrizationType:signalBiases}.
In this case, \configClass{estimateTransmitter/ReceiverPhaseBiasTransmitter}{platformSelectorType} should
not be used for the corresponding transmitters and receivers.

In case of GLONASS, the phase biases at receiver level differ between different frequency channels
(frequency division multiple access, FDMA) and for each channel an extra float phase bias is estimated.
With \config{linearGlonassBias} a linear relationship between bias and frequency channel is assumed,
which reduces the number of float bias parameters and increases the number of resolvable integer ambiguities.

The integer ambiguities can be resolved and fixed in
\configClass{GnssProcessing:processingStep:resolveAmbiguities}{gnssProcessingStepType:resolveAmbiguities}.
Resolved integer ambiguities are not estimated as unknown parameters in
\configClass{gnssProcessingStepType:estimate}{gnssProcessingStepType} anymore
and are removed from the system of normal equations.

The estimated phase biases can be written to files in
\configClass{parametrization:signalBiases}{gnssParametrizationType:signalBiases}.


\keepXColumns
\begin{tabularx}{\textwidth}{N T A}
\hline
Name & Type & Annotation\\
\hline
\hfuzz=500pt\includegraphics[width=1em]{element.pdf}~name & \hfuzz=500pt string & \hfuzz=500pt used for parameter selection\\
\hfuzz=500pt\includegraphics[width=1em]{element-unbounded.pdf}~estimateTransmitterPhaseBias & \hfuzz=500pt \hyperref[platformSelectorType]{platformSelector} & \hfuzz=500pt \\
\hfuzz=500pt\includegraphics[width=1em]{element-unbounded.pdf}~estimateReceiverPhaseBias & \hfuzz=500pt \hyperref[platformSelectorType]{platformSelector} & \hfuzz=500pt \\
\hfuzz=500pt\includegraphics[width=1em]{element.pdf}~linearGlonassBias & \hfuzz=500pt boolean & \hfuzz=500pt bias depends linear on frequency channel number\\
\hline
\end{tabularx}


\subsection{CodeBiases}\label{gnssParametrizationType:codeBiases}
Each code observation (e.g \verb|C1C| or \verb|C2W|) contains a bias at transmitter/receiver level
\begin{equation}
  [C\nu a]_r^s(t) = \dots + \text{bias}[C\nu a]^s + \text{bias}[C\nu a]_r + \dots
\end{equation}
The code biases cannot be estimated together with clock errors and ionospheric delays in an absolute sense
as rank deficiencies will occur in the system of normal equations. Therefore, the biases are not initialized and set up
as parameters directly but only estimable linear combinations are parametrized.

The basic idea is to set up simplified normal equations with the biases,
clock and STEC parameters of one single receiver or transmitter,
eliminate clock and STEC parameters and perform an eigen value decomposition
of the normal equation matrix
\begin{equation}
  \M N = \M Q \M\Lambda \M Q^T.
\end{equation}
Instead of estimating the original bias parameter $\M x$ a transformed set $\bar{\M x}$
is introduced:
\begin{equation}
  \bar{\M x} = \M Q^T \M x.
\end{equation}
The new parameters corresponding to eigen values $\lambda>0$ are estimable,
the others are left out (set to zero). The missing linear combinations,
which depend on the STEC parameters, can be added with
\configClass{parametrization:tecBiases}{gnssParametrizationType:tecBiases}.

Additional rank deficiencies may also occur when biases of transmitters and receivers are estimated together.
The minimum norm nullspace (also via eigen value decomposition)
is formulated as zero constraint equations and added with a standard deviation of \config{sigmaZeroMeanConstraint}.

In case of GLONASS the code biases at receiver level can differ between different frequency channels
(frequency division multiple access, FDMA) and for each channel an extra code bias is estimated.
With \config{linearGlonassBias} a linear relationship between bias and frequency channel is assumed,
which reduces the number of bias parameters.

The estimated biases can be written to files in
\configClass{parametrization:signalBiases}{gnssParametrizationType:signalBiases}.


\keepXColumns
\begin{tabularx}{\textwidth}{N T A}
\hline
Name & Type & Annotation\\
\hline
\hfuzz=500pt\includegraphics[width=1em]{element.pdf}~name & \hfuzz=500pt string & \hfuzz=500pt used for parameter selection\\
\hfuzz=500pt\includegraphics[width=1em]{element-unbounded.pdf}~selectTransmitters & \hfuzz=500pt \hyperref[platformSelectorType]{platformSelector} & \hfuzz=500pt \\
\hfuzz=500pt\includegraphics[width=1em]{element-unbounded.pdf}~selectReceivers & \hfuzz=500pt \hyperref[platformSelectorType]{platformSelector} & \hfuzz=500pt \\
\hfuzz=500pt\includegraphics[width=1em]{element.pdf}~linearGlonassBias & \hfuzz=500pt boolean & \hfuzz=500pt bias depends linear on frequency channel number\\
\hfuzz=500pt\includegraphics[width=1em]{element.pdf}~nameConstraint & \hfuzz=500pt string & \hfuzz=500pt used for parameter selection\\
\hfuzz=500pt\includegraphics[width=1em]{element.pdf}~sigmaZeroMeanConstraint & \hfuzz=500pt double & \hfuzz=500pt (0 = unconstrained) sigma [m] for null space constraint\\
\hline
\end{tabularx}


\subsection{TecBiases}\label{gnssParametrizationType:tecBiases}
Each code observation (e.g \verb|C1C| or \verb|C2W|) contains a bias at transmitter/receiver level
\begin{equation}
  [C\nu a]_r^s(t) = \dots + \text{bias}[C\nu a]^s + \text{bias}[C\nu a]_r + \ldots
\end{equation}
This parametrization represents the linear combination of signal biases
which completely depend on the STEC parameters. Ignoring these bias combinations would result
in a biased STEC estimation (all other parameters are nearly unaffected).
To determine this part of the signal biases
the \configClass{parametrization:ionosphereSTEC}{gnssParametrizationType:ionosphereSTEC} should be constrained.
Furthermore, additional information about the ionosphere is required from
\configClass{parametrization:ionosphereVTEC}{gnssParametrizationType:ionosphereVTEC} or
\configClass{parametrization:ionosphereMap}{gnssParametrizationType:ionosphereMap}.

Rank deficiencies due to the signal bias parameters may occur if biases of
transmitters and receivers are estimated together.
The minimum norm nullspace is formulated as zero constraint equations and added with
a standard deviation of \config{sigmaZeroMeanConstraint}.

The accumulated estimated result can be written to files in
\configClass{parametrization:signalBiases}{gnssParametrizationType:signalBiases}.


\keepXColumns
\begin{tabularx}{\textwidth}{N T A}
\hline
Name & Type & Annotation\\
\hline
\hfuzz=500pt\includegraphics[width=1em]{element.pdf}~name & \hfuzz=500pt string & \hfuzz=500pt used for parameter selection\\
\hfuzz=500pt\includegraphics[width=1em]{element-unbounded.pdf}~selectTransmitters & \hfuzz=500pt \hyperref[platformSelectorType]{platformSelector} & \hfuzz=500pt \\
\hfuzz=500pt\includegraphics[width=1em]{element-unbounded.pdf}~selectReceivers & \hfuzz=500pt \hyperref[platformSelectorType]{platformSelector} & \hfuzz=500pt \\
\hfuzz=500pt\includegraphics[width=1em]{element.pdf}~linearGlonassBias & \hfuzz=500pt boolean & \hfuzz=500pt phase or code biases depend linear on frequency channel number\\
\hfuzz=500pt\includegraphics[width=1em]{element.pdf}~nameConstraint & \hfuzz=500pt string & \hfuzz=500pt used for parameter selection\\
\hfuzz=500pt\includegraphics[width=1em]{element.pdf}~sigmaZeroMeanConstraint & \hfuzz=500pt double & \hfuzz=500pt (0 = unconstrained) sigma [m] for null space constraint\\
\hline
\end{tabularx}


\subsection{TemporalBias}\label{gnssParametrizationType:temporalBias}
This parametrization resolves the issue of some phase observations suffering from time-variable biases.
Such a phenomenon has been found to affect GPS block IIF satellites on the L5 phase measurements
(see Montenbruck et al. 2011, DOI: \href{https://doi.org/10.1007/s10291-011-0232-x}{10.1007/s10291-011-0232-x}).

For these time-variable biases an appropriate temporal representation has to be defined in
\configClass{parametrizationTemporal}{parametrizationTemporalType}.
For example, time-variable biases for GPS block IIF L5 phase observations (\configClass{type}{gnssType}=\verb|L5*G|)
can be represented by a cubic spline with a nodal distance of one hour.

The result is written as a \file{times series file}{instrument} at the processing sampling
or the sampling set by \configClass{GnssProcessing:processingStep:selectEpochs}{gnssProcessingStepType:selectEpochs}).

This parametrization should be set up in addition to the constant
\configClass{parametrization:signalBiases}{gnssParametrizationType:signalBiases}.
Depending on the temporal representation a temporal zero-mean constraint is needed
to separate this parametrization from the constant component. The constraint equations are added with
a standard deviation of \config{sigmaZeroMeanConstraint}.


\keepXColumns
\begin{tabularx}{\textwidth}{N T A}
\hline
Name & Type & Annotation\\
\hline
\hfuzz=500pt\includegraphics[width=1em]{element.pdf}~name & \hfuzz=500pt string & \hfuzz=500pt used for parameter selection\\
\hfuzz=500pt\includegraphics[width=1em]{element-mustset-unbounded.pdf}~selectTransmitters & \hfuzz=500pt \hyperref[platformSelectorType]{platformSelector} & \hfuzz=500pt \\
\hfuzz=500pt\includegraphics[width=1em]{element.pdf}~outputfileBiasTimeSeries & \hfuzz=500pt filename & \hfuzz=500pt variable \{prn\} available\\
\hfuzz=500pt\includegraphics[width=1em]{element.pdf}~inputfileBiasTimeSeries & \hfuzz=500pt filename & \hfuzz=500pt variable \{prn\} available\\
\hfuzz=500pt\includegraphics[width=1em]{element-mustset.pdf}~type & \hfuzz=500pt \hyperref[gnssType]{gnssType} & \hfuzz=500pt \\
\hfuzz=500pt\includegraphics[width=1em]{element-unbounded.pdf}~parametrizationTemporal & \hfuzz=500pt \hyperref[parametrizationTemporalType]{parametrizationTemporal} & \hfuzz=500pt \\
\hfuzz=500pt\includegraphics[width=1em]{element.pdf}~nameConstraint & \hfuzz=500pt string & \hfuzz=500pt used for parameter selection\\
\hfuzz=500pt\includegraphics[width=1em]{element.pdf}~sigmaZeroMeanConstraint & \hfuzz=500pt double & \hfuzz=500pt (0 = unconstrained) sigma [m] for temporal zero-mean constraint\\
\hline
\end{tabularx}


\subsection{StaticPositions}\label{gnssParametrizationType:staticPositions}
Estimates a static position for all
\configClass{selectReceivers}{platformSelectorType} in the terrestrial frame.

No-net constraints can be applied for a subset of stations,
\configClass{selectNoNetReceivers}{platformSelectorType}, with a
standard deviation of \config{noNetTranslationSigma} and \config{noNetRotationSigma}.
If the template \configFile{inputfileNoNetPositions}{stringList} is provided
the constraints are applied relatively to these positions. Only stations with an existing position file
are considered. Without \configFile{inputfileNoNetPositions}{stringList}
the constraints are applied towards the apriori values from
\configClass{GnssProcessing:receiver}{gnssReceiverGeneratorType}.
As a single corrupted station position can disturb the no-net conditions,
the rotation/translation parameters are estimated in a
\reference{robust least squares adjustment}{fundamentals.robustLeastSquares}
beforehand. The computed weight matrix is used to downweight corrupted stations
in the constraint equations.

In case you want to align to an ITRF/IGS reference frame, precise coordinates can be
generated by combining \program{Sinex2StationPosition}, \program{Sinex2StationDiscontinuities},
and \program{Sinex2StationPostSeismicDeformation}.


\keepXColumns
\begin{tabularx}{\textwidth}{N T A}
\hline
Name & Type & Annotation\\
\hline
\hfuzz=500pt\includegraphics[width=1em]{element.pdf}~name & \hfuzz=500pt string & \hfuzz=500pt used for parameter selection\\
\hfuzz=500pt\includegraphics[width=1em]{element-mustset-unbounded.pdf}~selectReceivers & \hfuzz=500pt \hyperref[platformSelectorType]{platformSelector} & \hfuzz=500pt \\
\hfuzz=500pt\includegraphics[width=1em]{element.pdf}~outputfileGriddedPosition & \hfuzz=500pt filename & \hfuzz=500pt delta north east up for all stations\\
\hfuzz=500pt\includegraphics[width=1em]{element.pdf}~outputfilePosition & \hfuzz=500pt filename & \hfuzz=500pt variable \{station\} available, full estimated coordinates (in TRF)\\
\hfuzz=500pt\includegraphics[width=1em]{element.pdf}~nameConstraint & \hfuzz=500pt string & \hfuzz=500pt used for parameter selection\\
\hfuzz=500pt\includegraphics[width=1em]{element-unbounded.pdf}~selectNoNetReceivers & \hfuzz=500pt \hyperref[platformSelectorType]{platformSelector} & \hfuzz=500pt \\
\hfuzz=500pt\includegraphics[width=1em]{element.pdf}~inputfileNoNetPositions & \hfuzz=500pt filename & \hfuzz=500pt variable \{station\} available, precise coordinates used for no-net constraints (in TRF)\\
\hfuzz=500pt\includegraphics[width=1em]{element.pdf}~noNetTranslationSigma & \hfuzz=500pt double & \hfuzz=500pt (0 = unconstrained) sigma [m] for no-net translation constraint on station coordinates\\
\hfuzz=500pt\includegraphics[width=1em]{element.pdf}~noNetRotationSigma & \hfuzz=500pt double & \hfuzz=500pt (0 = unconstrained) sigma [m] at Earth's surface for no-net rotation constraint on station coordinates\\
\hfuzz=500pt\includegraphics[width=1em]{element.pdf}~huber & \hfuzz=500pt double & \hfuzz=500pt stations \$>\$ huber*sigma0 are downweighted in no-net constraint\\
\hfuzz=500pt\includegraphics[width=1em]{element.pdf}~huberPower & \hfuzz=500pt double & \hfuzz=500pt stations \$>\$ huber: sigma=(e/huber)\textasciicircum{}huberPower*sigma0\\
\hline
\end{tabularx}


\subsection{KinematicPositions}\label{gnssParametrizationType:kinematicPositions}
Estimates the epoch-wise \configFile{outputfilePositions}{instrument}
in an Earth-fixed frame (or in case of LEO satellites in an intertial frame).

The $3\times3$ epoch wise \configFile{outputfileCovarianceEpoch}{instrument}
are computed within
\configClass{GnssProcessing:processingStep:computeCovarianceMatrix}{gnssProcessingStepType:computeCovarianceMatrix}


\keepXColumns
\begin{tabularx}{\textwidth}{N T A}
\hline
Name & Type & Annotation\\
\hline
\hfuzz=500pt\includegraphics[width=1em]{element.pdf}~name & \hfuzz=500pt string & \hfuzz=500pt used for parameter selection\\
\hfuzz=500pt\includegraphics[width=1em]{element-mustset-unbounded.pdf}~selectReceivers & \hfuzz=500pt \hyperref[platformSelectorType]{platformSelector} & \hfuzz=500pt \\
\hfuzz=500pt\includegraphics[width=1em]{element.pdf}~outputfilePositions & \hfuzz=500pt filename & \hfuzz=500pt variable \{station\} available, estimated kinematic positions/orbit\\
\hfuzz=500pt\includegraphics[width=1em]{element.pdf}~outputfileCovarianceEpoch & \hfuzz=500pt filename & \hfuzz=500pt variable \{station\} available, 3x3 epoch covariances\\
\hline
\end{tabularx}


\subsection{LeoDynamicOrbits}\label{gnssParametrizationType:leoDynamicOrbits}
The estimation of (reduced) dynamic orbits is formulated as variational equations.
It is based on \configFile{inputfileVariational}{variationalEquation} calculated with \program{PreprocessingVariationalEquation}.
Necessary integrations are performed by integrating a moving interpolation polynomial of degree \config{integrationDegree}.
The \configClass{parametrizationAcceleration}{parametrizationAccelerationType} must include at least those
parameters that were estimated in \program{PreprocessingVariationalEquationOrbitFit}.
Additional \configClass{stochasticPulse}{timeSeriesType} parameters can be set up to reduce orbit mismodeling.
If not enough epochs with observations are available (\config{minEstimableEpochsRatio}) the LEO satellite is disabled.


\keepXColumns
\begin{tabularx}{\textwidth}{N T A}
\hline
Name & Type & Annotation\\
\hline
\hfuzz=500pt\includegraphics[width=1em]{element.pdf}~name & \hfuzz=500pt string & \hfuzz=500pt used for parameter selection\\
\hfuzz=500pt\includegraphics[width=1em]{element-mustset-unbounded.pdf}~selectReceivers & \hfuzz=500pt \hyperref[platformSelectorType]{platformSelector} & \hfuzz=500pt \\
\hfuzz=500pt\includegraphics[width=1em]{element.pdf}~outputfileOrbit & \hfuzz=500pt filename & \hfuzz=500pt variable \{station\} available\\
\hfuzz=500pt\includegraphics[width=1em]{element.pdf}~outputfileParameters & \hfuzz=500pt filename & \hfuzz=500pt variable \{station\} available\\
\hfuzz=500pt\includegraphics[width=1em]{element-mustset.pdf}~inputfileVariational & \hfuzz=500pt filename & \hfuzz=500pt variable \{station\} available\\
\hfuzz=500pt\includegraphics[width=1em]{element-unbounded.pdf}~stochasticPulse & \hfuzz=500pt \hyperref[timeSeriesType]{timeSeries} & \hfuzz=500pt [mu/s] parametrization of stochastic pulses\\
\hfuzz=500pt\includegraphics[width=1em]{element-unbounded.pdf}~parametrizationAcceleration & \hfuzz=500pt \hyperref[parametrizationAccelerationType]{parametrizationAcceleration} & \hfuzz=500pt orbit force parameters\\
\hfuzz=500pt\includegraphics[width=1em]{element-mustset.pdf}~ephemerides & \hfuzz=500pt \hyperref[ephemeridesType]{ephemerides} & \hfuzz=500pt \\
\hfuzz=500pt\includegraphics[width=1em]{element.pdf}~minEstimableEpochsRatio & \hfuzz=500pt double & \hfuzz=500pt drop satellites with lower ratio of estimable epochs to total epochs\\
\hfuzz=500pt\includegraphics[width=1em]{element.pdf}~integrationDegree & \hfuzz=500pt uint & \hfuzz=500pt integration of forces by polynomial approximation of degree n\\
\hfuzz=500pt\includegraphics[width=1em]{element.pdf}~interpolationDegree & \hfuzz=500pt uint & \hfuzz=500pt for orbit interpolation and velocity calculation\\
\hline
\end{tabularx}


\subsection{TransmitterDynamicOrbits}\label{gnssParametrizationType:transmitterDynamicOrbits}
Same as \configClass{leoDynamicOrbits}{gnssParametrizationType:leoDynamicOrbits} but
for transmitting GNSS satellites.
For more details see \reference{orbit integration}{cookbook.gnssNetwork:orbitIntegration}.


\keepXColumns
\begin{tabularx}{\textwidth}{N T A}
\hline
Name & Type & Annotation\\
\hline
\hfuzz=500pt\includegraphics[width=1em]{element.pdf}~name & \hfuzz=500pt string & \hfuzz=500pt used for parameter selection\\
\hfuzz=500pt\includegraphics[width=1em]{element-mustset-unbounded.pdf}~selectTransmitters & \hfuzz=500pt \hyperref[platformSelectorType]{platformSelector} & \hfuzz=500pt \\
\hfuzz=500pt\includegraphics[width=1em]{element.pdf}~outputfileOrbit & \hfuzz=500pt filename & \hfuzz=500pt variable \{prn\} available\\
\hfuzz=500pt\includegraphics[width=1em]{element.pdf}~outputfileParameters & \hfuzz=500pt filename & \hfuzz=500pt variable \{prn\} available\\
\hfuzz=500pt\includegraphics[width=1em]{element-mustset.pdf}~inputfileVariational & \hfuzz=500pt filename & \hfuzz=500pt variable \{prn\} available\\
\hfuzz=500pt\includegraphics[width=1em]{element-unbounded.pdf}~stochasticPulse & \hfuzz=500pt \hyperref[timeSeriesType]{timeSeries} & \hfuzz=500pt [mu/s] parametrization of stochastic pulses\\
\hfuzz=500pt\includegraphics[width=1em]{element-unbounded.pdf}~parametrizationAcceleration & \hfuzz=500pt \hyperref[parametrizationAccelerationType]{parametrizationAcceleration} & \hfuzz=500pt orbit force parameters\\
\hfuzz=500pt\includegraphics[width=1em]{element-mustset.pdf}~ephemerides & \hfuzz=500pt \hyperref[ephemeridesType]{ephemerides} & \hfuzz=500pt \\
\hfuzz=500pt\includegraphics[width=1em]{element.pdf}~minEstimableEpochsRatio & \hfuzz=500pt double & \hfuzz=500pt drop satellites with lower ratio of estimable epochs to total epochs\\
\hfuzz=500pt\includegraphics[width=1em]{element.pdf}~integrationDegree & \hfuzz=500pt uint & \hfuzz=500pt integration of forces by polynomial approximation of degree n\\
\hfuzz=500pt\includegraphics[width=1em]{element.pdf}~interpolationDegree & \hfuzz=500pt uint & \hfuzz=500pt for orbit interpolation and velocity calculation\\
\hline
\end{tabularx}


\subsection{Troposphere}\label{gnssParametrizationType:troposphere}
A priori tropospheric correction is handled by a \configClass{troposphere}{troposphereType} model (e.g. Vienna Mapping Functions 3).
Additional parameters for zenith wet delay and gradients can be set up via
\configClass{troposphereWetEstimation}{parametrizationTemporalType} (usually 2-hourly linear splines)
and \configClass{troposphereGradientEstimation}{parametrizationTemporalType} (usually a daily trend).
These parameters can be soft-constrained using
\configClass{parametrization:constraints}{gnssParametrizationType:constraints}
to avoid an unsolvable system of normal equations in case of data gaps.


\keepXColumns
\begin{tabularx}{\textwidth}{N T A}
\hline
Name & Type & Annotation\\
\hline
\hfuzz=500pt\includegraphics[width=1em]{element.pdf}~name & \hfuzz=500pt string & \hfuzz=500pt used for parameter selection\\
\hfuzz=500pt\includegraphics[width=1em]{element-mustset-unbounded.pdf}~selectReceivers & \hfuzz=500pt \hyperref[platformSelectorType]{platformSelector} & \hfuzz=500pt \\
\hfuzz=500pt\includegraphics[width=1em]{element.pdf}~outputfileTroposphere & \hfuzz=500pt filename & \hfuzz=500pt columns: MJD, ZHD, ZWD, dry north gradient, wet north gradient, dry east gradient, wet east gradient, ...\\
\hfuzz=500pt\includegraphics[width=1em]{element-mustset.pdf}~troposphere & \hfuzz=500pt \hyperref[troposphereType]{troposphere} & \hfuzz=500pt a priori troposphere model\\
\hfuzz=500pt\includegraphics[width=1em]{element-unbounded.pdf}~troposphereWetEstimation & \hfuzz=500pt \hyperref[parametrizationTemporalType]{parametrizationTemporal} & \hfuzz=500pt [m] parametrization of zenith wet delays\\
\hfuzz=500pt\includegraphics[width=1em]{element-unbounded.pdf}~troposphereGradientEstimation & \hfuzz=500pt \hyperref[parametrizationTemporalType]{parametrizationTemporal} & \hfuzz=500pt [degree] parametrization of north and east gradients\\
\hline
\end{tabularx}


\subsection{EarthRotation}\label{gnssParametrizationType:earthRotation}
Earth rotation parameters (ERPs) can be estimated by defining
\config{estimatePole} ($x_p$, $y_p$) and \config{estimateUT1} ($dUT1, LOD$).

Estimating length of day (LOD) with the sign according to IGS conventions requires a negative
value in \configClass{parametrizationTemporal:trend:timeStep}{parametrizationTemporalType:trend}.

Constraints on the defined parameters can be added via
\configClass{parametrization:constraints}{gnssParametrizationType:constraints}.
An example would be to set up \configClass{estimateUT1:constant}{parametrizationTemporalType:constant}
so the $dUT1$ parameter is included in the normal equation system . Since $dUT1$ cannot be
determined by GNSS, a hard constraint to its a priori value can then be added.


\keepXColumns
\begin{tabularx}{\textwidth}{N T A}
\hline
Name & Type & Annotation\\
\hline
\hfuzz=500pt\includegraphics[width=1em]{element.pdf}~name & \hfuzz=500pt string & \hfuzz=500pt used for parameter selection\\
\hfuzz=500pt\includegraphics[width=1em]{element.pdf}~outputfileEOP & \hfuzz=500pt filename & \hfuzz=500pt EOP time series (mjd, xp, yp, sp, dUT1, LOD, X, Y, S)\\
\hfuzz=500pt\includegraphics[width=1em]{element-unbounded.pdf}~estimatePole & \hfuzz=500pt \hyperref[parametrizationTemporalType]{parametrizationTemporal} & \hfuzz=500pt xp, yp [mas]\\
\hfuzz=500pt\includegraphics[width=1em]{element-unbounded.pdf}~estimateUT1 & \hfuzz=500pt \hyperref[parametrizationTemporalType]{parametrizationTemporal} & \hfuzz=500pt rotation angle [ms]\\
\hfuzz=500pt\includegraphics[width=1em]{element-unbounded.pdf}~estimateNutation & \hfuzz=500pt \hyperref[parametrizationTemporalType]{parametrizationTemporal} & \hfuzz=500pt dX, dY [mas]\\
\hline
\end{tabularx}


\subsection{ReceiverAntennas}\label{gnssParametrizationType:receiverAntennas}
This class is for parametrization the antenna for their antenna center offsets (ACO) and
antenna center variations (ACV) by \configClass{antennaCenterVariations}{parametrizationGnssAntennaType}.
The receivers to be estimated can be selected by \configClass{selectReceivers}{platformSelectorType}.

The amount of patterns to be estimated is configurable with a list of \configClass{patternTypes}{gnssType}.
For each added \configClass{patternTypes}{gnssType} a set of parameters will be evaluated. The observations
will be assigned to the first \configClass{patternTypes}{gnssType} that matches their own.
E.g. having the patterns: \verb|***G| and \verb|L1*| would lead to all GPS observations be assigned
to the observation equations of the first pattern. The patterntype \verb|L1*| would then consist
of all other GNSS L1 phase observations. \config{addNonMatchingTypes} will, if activated, create automatically patterns
for \configClass{observations}{gnssType} that are not selected within the list \configClass{patternTypes}{gnssType}.
Furthermore, it is possible to group same antenna build types from different receivers by \config{groupAntennas}.
The grouping by same antenna build ignores antenna serial numbers.

To estimate the antenna variation parameters, a longer period of observations might be necessary
for accurate estimations. Hence one should use this parametrization by
accumulating normal equations from several epochs.
This can be accomplished as the last steps in the \configClass{processing steps}{gnssProcessingStepType}
 by adding \configClass{ReceiverAntennas}{gnssParametrizationType:receiverAntennas}
to current selected parameters with \configClass{GnssProcessing:processingStep:selectParametrizations}{gnssProcessingStepType:selectParametrizations}
and write the normal equation matrix with \configClass{GnssProcessing:processingStep:writeNormalEquations}{gnssProcessingStepType:writeNormalEquations}.
The written normal equations can then be accumulated with \program{NormalsAccumulate} and solved by \program{NormalsSolverVCE}.
Further, one should apply constraints to the normal equations by \program{GnssAntennaNormalsConstraint} since the estimation
 of ACO and ACV can lead to rank deficiencies in the normal equation matrix.
Last the solved normal equation can be parsed to a \file{antenna definition file}{gnssAntennaDefinition}
 with the program \program{ParameterVector2GnssAntennaDefinition}.

As example referring to the cookbook \reference{GNSS satellite orbit determination and station network analysis}{cookbook.gnssNetwork},
one could add additionally \configClass{receiverAntennas}{gnssParametrizationType:receiverAntennas} as parametrization.
Since the estimations are done on a daily basis for each receiver we add an additional
\configClass{selectParametrizations}{gnssProcessingStepType:selectParametrizations} which
disables \verb|parameter.receiverAntenna|. After all stations are processed together with all parameters, one
adds \verb|parameter.receiverAntenna| with \configClass{selectParametrizations}{gnssProcessingStepType:selectParametrizations}
 to the current selected parametrizations.
The last \configClass{processingStep}{gnssProcessingStepType} is \configClass{GnssProcessing:processingStep:writeNormalEquations}{gnssProcessingStepType:writeNormalEquations}
to write the daily normal equations including the parametrization \configClass{receiverAntennas}{gnssParametrizationType:receiverAntennas} into files.
These normal equation files are then processed with the programs:

\begin{itemize}
  \item \program{NormalsAccumulate}: accumulates normal equations.
  \item \program{GnssAntennaNormalsConstraint}: apply constraint to the normal equations.
  \item \program{NormalsSolverVCE}: solves the normal equations.
  \item \program{ParameterVector2GnssAntennaDefinition}: writes the solution into a \file{antenna definition file}{gnssAntennaDefinition}
\end{itemize}

Note that the apriori value $\M x_0$ for this parametrization is always zero and never updated
according to eq.~\eqref{gnssParametrizationType:update}.


\keepXColumns
\begin{tabularx}{\textwidth}{N T A}
\hline
Name & Type & Annotation\\
\hline
\hfuzz=500pt\includegraphics[width=1em]{element.pdf}~name & \hfuzz=500pt string & \hfuzz=500pt used for parameter selection\\
\hfuzz=500pt\includegraphics[width=1em]{element-mustset-unbounded.pdf}~selectReceivers & \hfuzz=500pt \hyperref[platformSelectorType]{platformSelector} & \hfuzz=500pt \\
\hfuzz=500pt\includegraphics[width=1em]{element-mustset-unbounded.pdf}~antennaCenterVariations & \hfuzz=500pt \hyperref[parametrizationGnssAntennaType]{parametrizationGnssAntenna} & \hfuzz=500pt estimate antenna center variations\\
\hfuzz=500pt\includegraphics[width=1em]{element-unbounded.pdf}~patternTypes & \hfuzz=500pt \hyperref[gnssType]{gnssType} & \hfuzz=500pt gnssType for each pattern (first match is used)\\
\hfuzz=500pt\includegraphics[width=1em]{element.pdf}~addNonMatchingTypes & \hfuzz=500pt boolean & \hfuzz=500pt add patterns for additional observed gnssTypes that don't match any of the above\\
\hfuzz=500pt\includegraphics[width=1em]{element.pdf}~groupAntennas & \hfuzz=500pt boolean & \hfuzz=500pt common ACVs for same antenna build types (ignores antenna serial number)\\
\hline
\end{tabularx}


\subsection{TransmitterAntennas}\label{gnssParametrizationType:transmitterAntennas}
Same as \configClass{receiverAntennas}{gnssParametrizationType:receiverAntennas} but
for transmitting antennas (GNSS satellites).


\keepXColumns
\begin{tabularx}{\textwidth}{N T A}
\hline
Name & Type & Annotation\\
\hline
\hfuzz=500pt\includegraphics[width=1em]{element.pdf}~name & \hfuzz=500pt string & \hfuzz=500pt used for parameter selection\\
\hfuzz=500pt\includegraphics[width=1em]{element-mustset-unbounded.pdf}~selectTransmitters & \hfuzz=500pt \hyperref[platformSelectorType]{platformSelector} & \hfuzz=500pt \\
\hfuzz=500pt\includegraphics[width=1em]{element-mustset-unbounded.pdf}~antennaCenterVariations & \hfuzz=500pt \hyperref[parametrizationGnssAntennaType]{parametrizationGnssAntenna} & \hfuzz=500pt estimate antenna center variations\\
\hfuzz=500pt\includegraphics[width=1em]{element-unbounded.pdf}~patternTypes & \hfuzz=500pt \hyperref[gnssType]{gnssType} & \hfuzz=500pt gnssType for each pattern (first match is used)\\
\hfuzz=500pt\includegraphics[width=1em]{element.pdf}~addNonMatchingTypes & \hfuzz=500pt boolean & \hfuzz=500pt add patterns for additional observed gnssTypes that don't match any of the above\\
\hfuzz=500pt\includegraphics[width=1em]{element.pdf}~groupAntennas & \hfuzz=500pt boolean & \hfuzz=500pt common ACVs for same antenna build types (ignores antenna serial number)\\
\hline
\end{tabularx}


\subsection{Constraints}\label{gnssParametrizationType:constraints}
Add a pseudo observation equation (constraint)
for each selected \configClass{parameters}{parameterSelectorType}
\begin{equation}
  b-x_0 = 1 \cdot dx + \epsilon,
\end{equation}
where $b$ is the \config{bias} and $x_0$ is the a priori value of the parameter
if \config{relativeToApriori} is not set.
The standard deviation \config{sigma} is used to weight the observation equations.


\keepXColumns
\begin{tabularx}{\textwidth}{N T A}
\hline
Name & Type & Annotation\\
\hline
\hfuzz=500pt\includegraphics[width=1em]{element.pdf}~name & \hfuzz=500pt string & \hfuzz=500pt \\
\hfuzz=500pt\includegraphics[width=1em]{element-mustset-unbounded.pdf}~parameters & \hfuzz=500pt \hyperref[parameterSelectorType]{parameterSelector} & \hfuzz=500pt parameter to constrain\\
\hfuzz=500pt\includegraphics[width=1em]{element-mustset.pdf}~sigma & \hfuzz=500pt double & \hfuzz=500pt sigma of the constraint (same unit as parameter)\\
\hfuzz=500pt\includegraphics[width=1em]{element.pdf}~bias & \hfuzz=500pt double & \hfuzz=500pt constrain all selected parameters towards this value\\
\hfuzz=500pt\includegraphics[width=1em]{element.pdf}~relativeToApriori & \hfuzz=500pt boolean & \hfuzz=500pt constrain only dx and not full x=dx+x0\\
\hline
\end{tabularx}


\subsection{Group}\label{gnssParametrizationType:group}
Groups a set of parameters. This class can be used to structure complex parametrizations
and has no further effect itself.


\keepXColumns
\begin{tabularx}{\textwidth}{N T A}
\hline
Name & Type & Annotation\\
\hline
\hfuzz=500pt\includegraphics[width=1em]{element-mustset-unbounded.pdf}~parametrization & \hfuzz=500pt \hyperref[gnssParametrizationType]{gnssParametrization} & \hfuzz=500pt \\
\hline
\end{tabularx}

\clearpage
%==================================

\section{GnssProcessingStep}\label{gnssProcessingStepType}
Processing step in \program{GnssProcessing}.

Processing steps enable a dynamic definition of the consecutive steps performed during any kind of GNSS processing.
The most common steps are \configClass{estimate}{gnssProcessingStepType:estimate}, which performs an iterative least
squares adjustment, and \configClass{writeResults}{gnssProcessingStepType:writeResults}, which writes all output files
defined in \program{GnssProcessing} and is usually the last step.
Some steps such as \configClass{selectParametrizations}{gnssProcessingStepType:selectParametrizations},
\configClass{selectEpochs}{gnssProcessingStepType:selectEpochs},
\configClass{selectNormalsBlockStructure}{gnssProcessingStepType:selectNormalsBlockStructure}, and
\configClass{selectReceivers}{gnssProcessingStepType:selectReceivers} affect all subsequent steps.
In case these steps are used within a \configClass{group}{gnssProcessingStepType:group} or
\configClass{forEachReceiverSeparately}{gnssProcessingStepType:forEachReceiverSeparately} step,
they only affect the steps within this level.

For usage examples see cookbooks on \reference{GNSS satellite orbit determination and network analysis}{cookbook.gnssNetwork:processing}
or \reference{Kinematic orbit determination of LEO satellites}{cookbook.kinematicOrbit}.


\subsection{Estimate}\label{gnssProcessingStepType:estimate}
Iterative non-linear least squares adjustment.
In every iteration it accumulates the system of normal equations, solves the system and updates the estimated parameters.
The estimated parameters serve as a priori values in the next iteration and the following processing steps.
Iterates until either every single parameter update (converted to an influence in meter)
is below a \config{convergenceThreshold} or \config{maxIterationCount} is reached.

With \config{computeResiduals} the observation equations are computed
again after each update to compute the observation residuals.

The overall standard deviation of a single observation used for the weighting
is composed of several factors
\begin{equation}
  \hat{\sigma}_i = \hat{\sigma}_i^{huber} \hat{\sigma}_{[\tau\nu a]}^{recv} \sigma_{[\tau\nu a]}^{recv}(E,A),
\end{equation}
where $[\tau\nu a]$ is the signal type, the azmiuth and elevation dependent $\sigma_{[\tau\nu a]}^{recv}(E,A)$ is given by
\configFile{receiver:inputfileAccuracyDefinition}{gnssAntennaDefinition} and the other factors are
estimated iteratively from the residuals.

With \config{computeWeights} a standardized variance $\hat{s}_i^2$
for each residual $\hat{\epsilon}_i$ is computed
\begin{equation}
  \hat{s}_i^2 = \frac{1}{\hat{\sigma}_{[\tau\nu a]}^{recv} \sigma_{[\tau\nu a]}^{recv}(E,A)}\frac{\hat{\epsilon}_i^2}{r_i}
  \qquad\text{with}\qquad
  r_i = \left(\M A\left(\M A^T\M A\right)^{-1}\M A^T\right)_{ii}
\end{equation}
taking the redundancy $r_i$ into account. If $\hat{s}_i$ is above a threshold \config{huber}
the observation gets a higher standard deviation used for weighting according to
\begin{equation}
  \hat{\sigma}_i^{huber} =
  \left\{ \begin{array}{ll}
    1                              & s < huber,\\
    (\hat{s}_i/huber)^{huberPower} & s \ge huber
  \end{array} \right.,
\end{equation}
similar to \reference{robust least squares adjustment}{fundamentals.robustLeastSquares}.

With \config{adjustSigma0} individual variance factors can be computed
for each station and all phases of a system and each code observation \reference{type}{gnssType}
(e.g. for each \verb|L**G|, \verb|L**E|, \verb|C1CG|, \verb|C2WG|, \verb|C1CE|, \ldots)
separately
\begin{equation}
  \hat{\sigma}_{[\tau\nu a]}^{recv} = \sqrt{\frac{\hat{\M\epsilon}^T\M P\hat{\M\epsilon}}{r}}.
\end{equation}


\keepXColumns
\begin{tabularx}{\textwidth}{N T A}
\hline
Name & Type & Annotation\\
\hline
\hfuzz=500pt\includegraphics[width=1em]{element.pdf}~computeResiduals & \hfuzz=500pt boolean & \hfuzz=500pt \\
\hfuzz=500pt\includegraphics[width=1em]{element.pdf}~adjustSigma0 & \hfuzz=500pt boolean & \hfuzz=500pt adjust sigma0 by scale factor (per receiver and type)\\
\hfuzz=500pt\includegraphics[width=1em]{element.pdf}~computeWeights & \hfuzz=500pt boolean & \hfuzz=500pt downweight outliers\\
\hfuzz=500pt\includegraphics[width=1em]{element.pdf}~huber & \hfuzz=500pt double & \hfuzz=500pt residuals \$>\$ huber*sigma0 are downweighted\\
\hfuzz=500pt\includegraphics[width=1em]{element.pdf}~huberPower & \hfuzz=500pt double & \hfuzz=500pt residuals \$>\$ huber: sigma=(e/huber)\textasciicircum{}huberPower*sigma0\\
\hfuzz=500pt\includegraphics[width=1em]{element.pdf}~convergenceThreshold & \hfuzz=500pt double & \hfuzz=500pt [m] stop iteration once full convergence is reached\\
\hfuzz=500pt\includegraphics[width=1em]{element.pdf}~maxIterationCount & \hfuzz=500pt uint & \hfuzz=500pt maximum number of iterations\\
\hline
\end{tabularx}


\subsection{ResolveAmbiguities}\label{gnssProcessingStepType:resolveAmbiguities}
Performs a least squares adjustment like \configClass{processingStep:estimate}{gnssProcessingStepType:estimate}
but with additional integer phase ambiguity resolution.
After this step all resolved ambiguities are removed from the normal equation system.

Integer ambiguity resolution is performed based on the least squares ambiguity decorrelation adjustment
(LAMBDA) method (Teunissen 1995, DOI \href{https://doi.org/10.1007/BF00863419}{10.1007/BF00863419}), specifically
the modified algorithm (MLAMBDA) by Chang et al. (2005, DOI \href{https://doi.org/10.1007/s00190-005-0004-x}{10.1007/s00190-005-0004-x}).
First the covariance matrix of the integer ambiguity parameters is computed by eliminating all but those parameters
from the full normal equation matrix and inverting it. Then, a Z-transformation is performed as described by
Chang et al. (2005) to decorrelate the ambiguity parameters without losing their integer nature.

The search process follows MLAMBDA and uses integer minimization of the weighted sum of squared residuals.
It is computationally infeasible to search a hyper-ellipsoid with a dimension of ten thousand or more.
Instead, a blocked search algorithm is performed by moving a window with a length of, for example,
\config{searchBlockSize}=\verb|200| parameters over the decorrelated ambiguities, starting from the most accurate.
In each step, the window is moved by half of its length and the overlapping parts are compared to each other.
If all fixed ambiguities in the overlap agree, the algorithm continues.
Otherwise, both windows are combined and the search is repeated using the combined window, again comparing with the overlapping
part of the preceding window. If not all solutions could be checked for a block after \config{maxSearchSteps},
the selected \config{incompleteAction} is performed.
If the algorithm reaches ambiguities with a standard deviation higher than \config{sigmaMaxResolve},
ambiguity resolution stops and the remaining ambiguities are left as float values.
Otherwise, all ambiguity parameters are fixed to integer values.

In contrast to an integer least squares solution over the full ambiguity vector, it is not guaranteed that the resulting solution
is optimal in the sense of minimal variance with given covariance.
This trade-off is necessary to cope with large numbers of ambiguities.


\keepXColumns
\begin{tabularx}{\textwidth}{N T A}
\hline
Name & Type & Annotation\\
\hline
\hfuzz=500pt\includegraphics[width=1em]{element.pdf}~outputfileAmbiguities & \hfuzz=500pt filename & \hfuzz=500pt resolved ambiguities\\
\hfuzz=500pt\includegraphics[width=1em]{element.pdf}~sigmaMaxResolve & \hfuzz=500pt double & \hfuzz=500pt max. allowed std. dev. of ambiguity to resolve [cycles]\\
\hfuzz=500pt\includegraphics[width=1em]{element.pdf}~searchBlockSize & \hfuzz=500pt uint & \hfuzz=500pt block size for blocked integer search\\
\hfuzz=500pt\includegraphics[width=1em]{element.pdf}~maxSearchSteps & \hfuzz=500pt uint & \hfuzz=500pt max. steps of integer search for each block\\
\hfuzz=500pt\includegraphics[width=1em]{element-mustset.pdf}~incompleteAction & \hfuzz=500pt choice & \hfuzz=500pt if not all solutions tested after maxSearchSteps\\
\hfuzz=500pt\includegraphics[width=1em]{connector.pdf}\includegraphics[width=1em]{element-mustset.pdf}~stop & \hfuzz=500pt  & \hfuzz=500pt stop searching, ambiguities remain float in this block\\
\hfuzz=500pt\includegraphics[width=1em]{connector.pdf}\includegraphics[width=1em]{element-mustset.pdf}~resolve & \hfuzz=500pt  & \hfuzz=500pt use best integer solution found so far\\
\hfuzz=500pt\includegraphics[width=1em]{connector.pdf}\includegraphics[width=1em]{element-mustset.pdf}~shrinkBlockSize & \hfuzz=500pt  & \hfuzz=500pt try again with half block size\\
\hfuzz=500pt\includegraphics[width=1em]{connector.pdf}\includegraphics[width=1em]{element-mustset.pdf}~throwException & \hfuzz=500pt  & \hfuzz=500pt stop and throw an exception\\
\hfuzz=500pt\includegraphics[width=1em]{element.pdf}~computeResiduals & \hfuzz=500pt boolean & \hfuzz=500pt \\
\hfuzz=500pt\includegraphics[width=1em]{element.pdf}~adjustSigma0 & \hfuzz=500pt boolean & \hfuzz=500pt adjust sigma0 by scale factor (per receiver and type)\\
\hfuzz=500pt\includegraphics[width=1em]{element.pdf}~computeWeights & \hfuzz=500pt boolean & \hfuzz=500pt downweight outliers\\
\hfuzz=500pt\includegraphics[width=1em]{element.pdf}~huber & \hfuzz=500pt double & \hfuzz=500pt residuals \$>\$ huber*sigma0 are downweighted\\
\hfuzz=500pt\includegraphics[width=1em]{element.pdf}~huberPower & \hfuzz=500pt double & \hfuzz=500pt residuals \$>\$ huber: sigma=(e/huber)\textasciicircum{}huberPower*sigma0\\
\hline
\end{tabularx}


\subsection{ComputeCovarianceMatrix}\label{gnssProcessingStepType:computeCovarianceMatrix}
Accumulates the normal equations and computes the covariance matrix as inverse of the normal matrix.
It is not the full inverse but only the elements which are set in the normal matrix
(see  \configClass{gnssProcessingStep:selectNormalsBlockStructure}{gnssProcessingStepType:selectNormalsBlockStructure})
are computed. The matrix is passed to the \configClass{parametrizations}{gnssParametrizationType}.
Only used in \configClass{parametrizations:kinematicPositions}{gnssParametrizationType:kinematicPositions}
to get the epoch wise covariance information at the moment.


\subsection{WriteResults}\label{gnssProcessingStepType:writeResults}
In this step all \config{outputfiles} defined in \configClass{parametrizations}{gnssParametrizationType}
are written. It considers the settings of
\configClass{processingStep:selectParametrizations}{gnssProcessingStepType:selectParametrizations},
\configClass{processingStep:selectEpochs}{gnssProcessingStepType:selectEpochs}, and
\configClass{processingStep:selectReceivers}{gnssProcessingStepType:selectReceivers}.

It is usually the last processing step, but can also be used at other points in the
processing in combination with \config{suffix} to write intermediate results, for example
before \configClass{gnssProcessingStep:resolveAmbiguities}{gnssProcessingStepType:resolveAmbiguities} to
output the float solution.


\keepXColumns
\begin{tabularx}{\textwidth}{N T A}
\hline
Name & Type & Annotation\\
\hline
\hfuzz=500pt\includegraphics[width=1em]{element.pdf}~suffix & \hfuzz=500pt string & \hfuzz=500pt appended to every output file name (e.g. orbit.G01.suffix.dat)\\
\hline
\end{tabularx}


\subsection{WriteNormalEquations}\label{gnssProcessingStepType:writeNormalEquations}
Accumulates the normal equations matrix and writes it.
If \configClass{remainingParameters}{parameterSelectorType}
is set only the selected parameters are written to the normal equations
and all other parameters are eliminated beforehand (implicitly solved).

The solution of the normals would results in $\Delta\M x$
(see \configClass{parametrizations}{gnssParametrizationType}). To write the
appropriate apriori vector $\M x_0$ use
\configClass{processingStep:writeAprioriSolution}{gnssProcessingStepType:writeAprioriSolution}.


\keepXColumns
\begin{tabularx}{\textwidth}{N T A}
\hline
Name & Type & Annotation\\
\hline
\hfuzz=500pt\includegraphics[width=1em]{element-mustset.pdf}~outputfileNormalEquations & \hfuzz=500pt filename & \hfuzz=500pt normals\\
\hfuzz=500pt\includegraphics[width=1em]{element-unbounded.pdf}~remainingParameters & \hfuzz=500pt \hyperref[parameterSelectorType]{parameterSelector} & \hfuzz=500pt parameter order/selection of output normal equations\\
\hfuzz=500pt\includegraphics[width=1em]{element.pdf}~constraintsOnly & \hfuzz=500pt boolean & \hfuzz=500pt write only normals of constraints without observations\\
\hfuzz=500pt\includegraphics[width=1em]{element.pdf}~defaultNormalsBlockSize & \hfuzz=500pt uint & \hfuzz=500pt block size for distributing the normal equations, 0: one block, empty: original block size\\
\hline
\end{tabularx}


\subsection{WriteAprioriSolution}\label{gnssProcessingStepType:writeAprioriSolution}
Writes the current apriori vector $\M x_0$
(see \configClass{parametrizations}{gnssParametrizationType}).
If \configClass{remainingParameters}{parameterSelectorType}
is set only the selected parameters are written.


\keepXColumns
\begin{tabularx}{\textwidth}{N T A}
\hline
Name & Type & Annotation\\
\hline
\hfuzz=500pt\includegraphics[width=1em]{element.pdf}~outputfileAprioriSolution & \hfuzz=500pt filename & \hfuzz=500pt a priori parameters\\
\hfuzz=500pt\includegraphics[width=1em]{element.pdf}~outputfileParameterNames & \hfuzz=500pt filename & \hfuzz=500pt parameter names\\
\hfuzz=500pt\includegraphics[width=1em]{element-unbounded.pdf}~remainingParameters & \hfuzz=500pt \hyperref[parameterSelectorType]{parameterSelector} & \hfuzz=500pt parameter order/selection of output normal equations\\
\hline
\end{tabularx}


\subsection{WriteResiduals}\label{gnssProcessingStepType:writeResiduals}
Writes the \file{observation residuals}{instrument} for all
\configClass{selectReceivers}{platformSelectorType}.
For for each station a file is written. The file name is interpreted as
a template with the variable \verb|{station}| being replaced by the station name.


\keepXColumns
\begin{tabularx}{\textwidth}{N T A}
\hline
Name & Type & Annotation\\
\hline
\hfuzz=500pt\includegraphics[width=1em]{element-mustset-unbounded.pdf}~selectReceivers & \hfuzz=500pt \hyperref[platformSelectorType]{platformSelector} & \hfuzz=500pt subset of used stations\\
\hfuzz=500pt\includegraphics[width=1em]{element-mustset.pdf}~outputfileResiduals & \hfuzz=500pt filename & \hfuzz=500pt variable \{station\} available\\
\hline
\end{tabularx}


\subsection{WriteUsedStationList}\label{gnssProcessingStepType:writeUsedStationList}
Writes a \file{list}{stringList} of receivers (stations) which are used in the last step and
selected by \configClass{selectReceivers}{platformSelectorType}.


\keepXColumns
\begin{tabularx}{\textwidth}{N T A}
\hline
Name & Type & Annotation\\
\hline
\hfuzz=500pt\includegraphics[width=1em]{element-mustset-unbounded.pdf}~selectReceivers & \hfuzz=500pt \hyperref[platformSelectorType]{platformSelector} & \hfuzz=500pt subset of used stations\\
\hfuzz=500pt\includegraphics[width=1em]{element-mustset.pdf}~outputfileUsedStationList & \hfuzz=500pt filename & \hfuzz=500pt ascii file with names of used stations\\
\hline
\end{tabularx}


\subsection{WriteUsedTransmitterList}\label{gnssProcessingStepType:writeUsedTransmitterList}
Writes a \file{list}{stringList} of transmitters which are used in the last step and
selected by \configClass{selectTransmitters}{platformSelectorType}.


\keepXColumns
\begin{tabularx}{\textwidth}{N T A}
\hline
Name & Type & Annotation\\
\hline
\hfuzz=500pt\includegraphics[width=1em]{element-mustset-unbounded.pdf}~selectTransmitters & \hfuzz=500pt \hyperref[platformSelectorType]{platformSelector} & \hfuzz=500pt subset of used transmitters\\
\hfuzz=500pt\includegraphics[width=1em]{element-mustset.pdf}~outputfileUsedTransmitterList & \hfuzz=500pt filename & \hfuzz=500pt ascii file with PRNs\\
\hline
\end{tabularx}


\subsection{PrintResidualStatistics}\label{gnssProcessingStepType:printResidualStatistics}
Print residual statistics.
\begin{verbatim}
  areq: C1CG**: factor =  0.64, sigma0 = 1.00, count =  2748, outliers =    48 (1.75 \%)
  areq: C1WG**: factor =  0.50, sigma0 = 1.00, count =  2748, outliers =    43 (1.56 \%)
  areq: C2WG**: factor =  0.50, sigma0 = 1.00, count =  2748, outliers =    59 (2.15 \%)
  areq: C5XG**: factor =  0.46, sigma0 = 1.00, count =  1279, outliers =    23 (1.80 \%)
  areq: L1CG**: factor =  0.86, sigma0 = 0.96, count =  2748, outliers =    40 (1.46 \%)
  areq: L1WG**: factor =  0.86, sigma0 = 1.02, count =  2748, outliers =    40 (1.46 \%)
  areq: L2WG**: factor =  0.86, sigma0 = 0.96, count =  2748, outliers =    40 (1.46 \%)
  areq: L5XG**: factor =  0.86, sigma0 = 1.30, count =  1279, outliers =    14 (1.09 \%)
  areq: C1PR**: factor =  0.48, sigma0 = 1.00, count =  1713, outliers =    53 (3.09 \%)
  areq: C2PR**: factor =  0.55, sigma0 = 1.00, count =  1713, outliers =    51 (2.98 \%)
  areq: L1PR**: factor =  0.85, sigma0 = 1.09, count =  1713, outliers =    29 (1.69 \%)
  areq: L2PR**: factor =  0.85, sigma0 = 0.88, count =  1713, outliers =    29 (1.69 \%)
  areq: C1XE**: factor =  0.44, sigma0 = 1.00, count =  1264, outliers =    21 (1.66 \%)
  areq: C5XE**: factor =  0.33, sigma0 = 1.00, count =  1264, outliers =    27 (2.14 \%)
  areq: C7XE**: factor =  0.28, sigma0 = 1.00, count =  1264, outliers =    41 (3.24 \%)
  areq: L1XE**: factor =  0.82, sigma0 = 1.14, count =  1264, outliers =    15 (1.19 \%)
  areq: L5XE**: factor =  0.82, sigma0 = 0.84, count =  1264, outliers =    15 (1.19 \%)
  areq: L7XE**: factor =  0.82, sigma0 = 0.94, count =  1264, outliers =    15 (1.19 \%)
  badg: C1CG**: factor =  1.25, sigma0 = 1.00, count =  2564, outliers =    47 (1.83 \%)
  ...
\end{verbatim}


\subsection{SelectParametrizations}\label{gnssProcessingStepType:selectParametrizations}
Enable/disable parameter groups and constraint groups for subsequent steps,
e.g. \configClass{processingStep:estimate}{gnssProcessingStepType:estimate} or
\configClass{processingStep:writeResults}{gnssProcessingStepType:writeResults}.
The \config{name} and \config{nameConstraint} of these groups
are defined in \configClass{parametrizations}{gnssParametrizationType}.
Prior models or previously estimated parameters used as new apriori $\M x_0$ values are unaffected
and they are always reduced from the observations. This means all unselected parameters are kept fixed
to their last result.

An example would be to process at a 5-minute sampling using
\configClass{processingStep:selectEpochs}{gnssProcessingStepType:selectEpochs}
and then at the end to densify the clock parameters to the full 30-second observation sampling
while keeping all other parameters fixed
(\config{disable}=\verb|*|, \config{enable}=\verb|*.clock*|, \config{enable}=\verb|parameter.STEC|).


\keepXColumns
\begin{tabularx}{\textwidth}{N T A}
\hline
Name & Type & Annotation\\
\hline
\hfuzz=500pt\includegraphics[width=1em]{element-mustset-unbounded.pdf}~parametrization & \hfuzz=500pt choice & \hfuzz=500pt \\
\hfuzz=500pt\includegraphics[width=1em]{connector.pdf}\includegraphics[width=1em]{element-mustset.pdf}~enable & \hfuzz=500pt sequence & \hfuzz=500pt \\
\hfuzz=500pt\quad\includegraphics[width=1em]{connector.pdf}\includegraphics[width=1em]{element-mustset-unbounded.pdf}~name & \hfuzz=500pt string & \hfuzz=500pt wildcards: * and ?\\
\hfuzz=500pt\includegraphics[width=1em]{connector.pdf}\includegraphics[width=1em]{element-mustset.pdf}~disable & \hfuzz=500pt sequence & \hfuzz=500pt \\
\hfuzz=500pt\quad\includegraphics[width=1em]{connector.pdf}\includegraphics[width=1em]{element-mustset-unbounded.pdf}~name & \hfuzz=500pt string & \hfuzz=500pt wildcards: * and ?\\
\hline
\end{tabularx}


\subsection{SelectEpochs}\label{gnssProcessingStepType:selectEpochs}
Select epochs for subsequent steps. This step can be used to reduce the processing sampling
while keeping the original observation sampling for all preprocessing steps (e.g. outlier and cycle slip detection).
Another example is to process at a 5-minute sampling by setting \config{nthEpoch}=\verb|10| and then
at the end to densify only the clock parameters to the full 30-second observation sampling by
setting \config{nthEpoch}=\verb|1| while keeping all other parameters fixed
with \configClass{processingStep:selectParametrizations}{gnssProcessingStepType:selectParametrizations}.


\keepXColumns
\begin{tabularx}{\textwidth}{N T A}
\hline
Name & Type & Annotation\\
\hline
\hfuzz=500pt\includegraphics[width=1em]{element-mustset.pdf}~nthEpoch & \hfuzz=500pt uint & \hfuzz=500pt use only every nth epoch in all subsequent processing steps\\
\hline
\end{tabularx}


\subsection{SelectNormalsBlockStructure}\label{gnssProcessingStepType:selectNormalsBlockStructure}
Select block structure of sparse normal equations for subsequent steps.

This step can be used to define the structure of the different parts of the normal equation system,
which can have a major impact on computing performance and memory consumption depending on the processing setup.

\fig{!hb}{0.4}{gnss_normals_structure}{fig:gnss_normals_structure}{Structure of normal equations in GNSS processing}

The normal equation system is divided into three parts for epoch, interval, and ambiguity parameters.
The epoch part is subdivided further into one subpart per epoch. Each part is divided into blocks and only non-zero
blocks are stored in memory to reduce memory consumption and to prevent unnecessary matrix computations.
\config{defaultBlockSizeEpoch}, \config{defaultBlockSizeInterval}, and \config{defaultBlockSizeAmbiguity} control
the size of the blocks within each part of the normal equations. \config{defaultBlockReceiverCount} can be set to group
a number of receivers into one block within the epoch and interval parts.

If \config{keepEpochNormalsInMemory}=\verb|no| epoch blocks are eliminated after they are set up to reduce the number
of parameters in the normal equation system. \config{defaultBlockCountReduction} controls after how many epoch blocks
an elimination step is performed. For larger processing setups or high sampling rates epoch block elimination is recommended
as the large number of clock parameters require a lot of memory.


\keepXColumns
\begin{tabularx}{\textwidth}{N T A}
\hline
Name & Type & Annotation\\
\hline
\hfuzz=500pt\includegraphics[width=1em]{element.pdf}~defaultBlockSizeEpoch & \hfuzz=500pt uint & \hfuzz=500pt block size of epoch parameters, 0: one block\\
\hfuzz=500pt\includegraphics[width=1em]{element.pdf}~defaultBlockSizeInterval & \hfuzz=500pt uint & \hfuzz=500pt block size of interval parameters, 0: one block\\
\hfuzz=500pt\includegraphics[width=1em]{element.pdf}~defaultBlockSizeAmbiguity & \hfuzz=500pt uint & \hfuzz=500pt block size of ambiguity parameters, 0: one block\\
\hfuzz=500pt\includegraphics[width=1em]{element.pdf}~defaultBlockReceiverCount & \hfuzz=500pt uint & \hfuzz=500pt number of receivers to group into one block for epoch and interval\\
\hfuzz=500pt\includegraphics[width=1em]{element.pdf}~defaultBlockCountReduction & \hfuzz=500pt uint & \hfuzz=500pt minimum number of blocks for epoch reduction\\
\hfuzz=500pt\includegraphics[width=1em]{element.pdf}~keepEpochNormalsInMemory & \hfuzz=500pt boolean & \hfuzz=500pt speeds up processing but uses much more memory\\
\hfuzz=500pt\includegraphics[width=1em]{element.pdf}~accumulateEpochObservations & \hfuzz=500pt boolean & \hfuzz=500pt set up all observations per epoch and receiver at once\\
\hline
\end{tabularx}


\subsection{SelectReceivers}\label{gnssProcessingStepType:selectReceivers}
This step can be used to process only a subset of stations in subsequent processing steps.
The most common use is to start the processing with a well-distributed network of core stations as seen in
\reference{GNSS satellite orbit determination and network analysis}{cookbook.gnssNetwork:processing}.
To later process all other stations individually, use the processing step
\configClass{processingStep:forEachReceiverSeparately}{gnssProcessingStepType:forEachReceiverSeparately}
and select all stations excluding the core stations in that step.


\keepXColumns
\begin{tabularx}{\textwidth}{N T A}
\hline
Name & Type & Annotation\\
\hline
\hfuzz=500pt\includegraphics[width=1em]{element-mustset-unbounded.pdf}~selectReceivers & \hfuzz=500pt \hyperref[platformSelectorType]{platformSelector} & \hfuzz=500pt \\
\hline
\end{tabularx}


\subsection{ForEachReceiverSeparately}\label{gnssProcessingStepType:forEachReceiverSeparately}
Perform these processing steps for each \configClass{selectReceivers}{platformSelectorType} separately.
All non-receiver related parameters parameters are disabled in these processing steps (see .

This step can be used for individual precise point positioning (PPP) of all stations.
During \reference{GNSS satellite orbit determination and network analysis}{cookbook.gnssNetwork:processing} this step is used after the
initial processing of the core network to process all other stations individually. In that case provide the same station list as
\configFile{inputfileExcludeStationList}{stringList} in this step that was used as \configFile{inputfileStationList}{stringList} in the
\configClass{selectReceivers}{gnssProcessingStepType:selectReceivers} step where the core network was selected.


\keepXColumns
\begin{tabularx}{\textwidth}{N T A}
\hline
Name & Type & Annotation\\
\hline
\hfuzz=500pt\includegraphics[width=1em]{element-mustset-unbounded.pdf}~selectReceivers & \hfuzz=500pt \hyperref[platformSelectorType]{platformSelector} & \hfuzz=500pt \\
\hfuzz=500pt\includegraphics[width=1em]{element.pdf}~variableReceiver & \hfuzz=500pt string & \hfuzz=500pt variable is set for each receiver\\
\hfuzz=500pt\includegraphics[width=1em]{element-mustset-unbounded.pdf}~processingStep & \hfuzz=500pt \hyperref[gnssProcessingStepType]{gnssProcessingStep} & \hfuzz=500pt steps are processed consecutively\\
\hline
\end{tabularx}


\subsection{Group}\label{gnssProcessingStepType:group}
Perform these processing steps. This step can be used to structure complex processing flows.
The \configClass{select..}{gnssProcessingStepType:selectParametrizations} processing steps
defined within a group only affect the steps within this group.


\keepXColumns
\begin{tabularx}{\textwidth}{N T A}
\hline
Name & Type & Annotation\\
\hline
\hfuzz=500pt\includegraphics[width=1em]{element-mustset-unbounded.pdf}~processingStep & \hfuzz=500pt \hyperref[gnssProcessingStepType]{gnssProcessingStep} & \hfuzz=500pt steps are processed consecutively\\
\hline
\end{tabularx}


\subsection{DisableTransmitterShadowEpochs}\label{gnssProcessingStepType:disableTransmitterShadowEpochs}
Disable transmitter epochs during eclipse.
With proper attitude modeling (see \program{SimulateStarCameraGnss}) this is usually not necessary.


\keepXColumns
\begin{tabularx}{\textwidth}{N T A}
\hline
Name & Type & Annotation\\
\hline
\hfuzz=500pt\includegraphics[width=1em]{element-mustset-unbounded.pdf}~selectTransmitters & \hfuzz=500pt \hyperref[platformSelectorType]{platformSelector} & \hfuzz=500pt \\
\hfuzz=500pt\includegraphics[width=1em]{element.pdf}~disableShadowEpochs & \hfuzz=500pt boolean & \hfuzz=500pt disable epochs if satellite is in Earth's/Moon's shadow\\
\hfuzz=500pt\includegraphics[width=1em]{element.pdf}~disablePostShadowRecoveryEpochs & \hfuzz=500pt boolean & \hfuzz=500pt disable epochs if satellite is in post-shadow recovery maneuver for GPS block IIA\\
\hfuzz=500pt\includegraphics[width=1em]{element-mustset.pdf}~ephemerides & \hfuzz=500pt \hyperref[ephemeridesType]{ephemerides} & \hfuzz=500pt \\
\hfuzz=500pt\includegraphics[width=1em]{element-mustset.pdf}~eclipse & \hfuzz=500pt \hyperref[eclipseType]{eclipse} & \hfuzz=500pt eclipse model used to determine if a satellite is in Earth's shadow\\
\hline
\end{tabularx}

\clearpage
%==================================

\section{GnssReceiverGenerator}\label{gnssReceiverGeneratorType}
Definition and basic information of GNSS receivers.

Most of the input files are provided in GROOPS file formats at
\url{https://ftp.tugraz.at/outgoing/ITSG/groops} (marked with \textbf{*} below).
These files are regularly updated.
\begin{itemize}
  \item \configFile{inputfileStationInfo}{platform}\textbf{*}:
        Antenna and receiver information, antenna reference point offsets, antenna orientations.
        Created via \program{GnssStationLog2Platform} or \program{PlatformCreate}.
  \item \configFile{inputfileAntennaDefinition}{gnssAntennaDefinition}\textbf{*}:
        Antenna center offsets and variations.
        Created via \program{GnssAntex2AntennaDefinition} or \program{GnssAntennaDefinitionCreate}.
  \item \configFile{inputfileReceiverDefinition}{gnssReceiverDefinition}:
        Observed signal types (optional).
        Created via \program{GnssReceiverDefinitionCreate} in case you want to define which signal
        types a receiver model can observe.
  \item \configFile{inputfileAccuracyDefinition}{gnssAntennaDefinition}\textbf{*}:
        Elevation and azimuth dependent accuracy.
        Created via \program{GnssAntennaDefinitionCreate}.
  \item \configFile{inputfileObservation}{instrument}:
        Converted from RINEX observation files via \program{RinexObservation2GnssReceiver}.
\end{itemize}

It is possible to limit the observation types to be used in the processing by a list of \configClass{useType}{gnssType}
and any observation types not defined within the list are ignored and discarded.
Similarly observations defined in the list of \configClass{ignoreType}{gnssType} are ignored and discarded.
The codes used follow the \href{https://files.igs.org/pub/data/format/rinex305.pdf}{RINEX 3 definition}.

Each receiver goes through a \config{preprocessing} step individually, where observation outliers are removed or downweighted,
continuous tracks of phase observations are defined for ambiguity parametrization, cycle slips are detected, and receivers are
disabled if they do not fulfill certain requirements. The preprocessing step consists of an initial PPP estimation done by
\reference{robust least squares adjustment}{fundamentals.robustLeastSquares} and checks whether the position error
of the solutions exceeds \config{codeMaxPositionDiff}. If the error exceeds the threshold the receiver will be discarded.
The preprocessing also sets initial clock error values and removes tracks that stay below a certain elevation mask (\config{elevationTrackMinimum}).

See also \program{GnssProcessing} and \program{GnssSimulateReceiver}.


\subsection{StationNetwork}\label{gnssReceiverGeneratorType:stationNetwork}
A network of GNSS ground stations is defined via \configFile{inputfileStationList}{stringTable}.
Each line can contain more than one station. The first station in each line for which \configFile{inputfileObservations}{instrument}
exists and contains enough observations is used for the processing.
All input files except \configFile{inputfileAntennaDefinition}{gnssAntennaDefinition},
\configFile{inputfileReceiverDefinition}{gnssReceiverDefinition}, and
\configFile{inputfileAccuracyDefinition}{gnssAntennaDefinition} are read for each station.
The file name is interpreted as a template with the variable \verb|{station}| being replaced by the station name.

The effects of loading and tidal deformation on station positions can be corrected for
via \configClass{loadingDisplacement}{gravityfieldType} and
\configClass{tidalDisplacement}{tidesType}, respectively.
Tidal deformations typically include:
\begin{itemize}
  \item \configClass{earthTide}{tidesType:earthTide}: Earth tidal deformations (IERS conventions)
  \item \configClass{doodsonHarmonicTide}{tidesType:doodsonHarmonicTide}: ocean tidal deformations
        (e.g. fes2014b\_n720, \config{minDegree}=\verb|1|)
  \item \configClass{doodsonHarmonicTide}{tidesType:doodsonHarmonicTide}: atmospheric tidal deformation
        (e.g. AOD1B RL06, \config{minDegree}=\verb|1|)
  \item \configClass{poleTide}{tidesType:poleTide}: pole tidal deformations (IERS conventions)
  \item \configClass{poleOceanTide}{tidesType:oceanPoleTide}: ocean pole tidal deformations (IERS conventions)
\end{itemize}



\keepXColumns
\begin{tabularx}{\textwidth}{N T A}
\hline
Name & Type & Annotation\\
\hline
\hfuzz=500pt\includegraphics[width=1em]{element-mustset.pdf}~inputfileStationList & \hfuzz=500pt filename & \hfuzz=500pt ascii file with station names\\
\hfuzz=500pt\includegraphics[width=1em]{element.pdf}~maxStationCount & \hfuzz=500pt uint & \hfuzz=500pt maximum number of stations to be used\\
\hfuzz=500pt\includegraphics[width=1em]{element-mustset.pdf}~inputfileStationInfo & \hfuzz=500pt filename & \hfuzz=500pt variable \{station\} available. station metadata (antennas, receivers, ...)\\
\hfuzz=500pt\includegraphics[width=1em]{element-mustset.pdf}~inputfileAntennaDefinition & \hfuzz=500pt filename & \hfuzz=500pt antenna center offsets and variations\\
\hfuzz=500pt\includegraphics[width=1em]{element-mustset.pdf}~noAntennaPatternFound & \hfuzz=500pt choice & \hfuzz=500pt what should happen if no antenna pattern is found for an observation\\
\hfuzz=500pt\includegraphics[width=1em]{connector.pdf}\includegraphics[width=1em]{element-mustset.pdf}~ignoreObservation & \hfuzz=500pt  & \hfuzz=500pt ignore observation if no matching pattern is found\\
\hfuzz=500pt\includegraphics[width=1em]{connector.pdf}\includegraphics[width=1em]{element-mustset.pdf}~useNearestFrequency & \hfuzz=500pt  & \hfuzz=500pt use pattern of nearest frequency if no matching pattern is found\\
\hfuzz=500pt\includegraphics[width=1em]{connector.pdf}\includegraphics[width=1em]{element-mustset.pdf}~throwException & \hfuzz=500pt  & \hfuzz=500pt throw exception if no matching pattern is found\\
\hfuzz=500pt\includegraphics[width=1em]{element.pdf}~inputfileReceiverDefinition & \hfuzz=500pt filename & \hfuzz=500pt observed signal types\\
\hfuzz=500pt\includegraphics[width=1em]{element-mustset.pdf}~inputfileAccuracyDefinition & \hfuzz=500pt filename & \hfuzz=500pt elevation and azimuth dependent accuracy\\
\hfuzz=500pt\includegraphics[width=1em]{element.pdf}~inputfileStationPosition & \hfuzz=500pt filename & \hfuzz=500pt variable \{station\} available.\\
\hfuzz=500pt\includegraphics[width=1em]{element.pdf}~inputfileClock & \hfuzz=500pt filename & \hfuzz=500pt variable \{station\} available\\
\hfuzz=500pt\includegraphics[width=1em]{element.pdf}~inputfileObservations & \hfuzz=500pt filename & \hfuzz=500pt variable \{station\} available\\
\hfuzz=500pt\includegraphics[width=1em]{element-unbounded.pdf}~loadingDisplacement & \hfuzz=500pt \hyperref[gravityfieldType]{gravityfield} & \hfuzz=500pt loading deformation\\
\hfuzz=500pt\includegraphics[width=1em]{element-unbounded.pdf}~tidalDisplacement & \hfuzz=500pt \hyperref[tidesType]{tides} & \hfuzz=500pt tidal deformation\\
\hfuzz=500pt\includegraphics[width=1em]{element.pdf}~ephemerides & \hfuzz=500pt \hyperref[ephemeridesType]{ephemerides} & \hfuzz=500pt for tidal deformation\\
\hfuzz=500pt\includegraphics[width=1em]{element-mustset.pdf}~inputfileDeformationLoadLoveNumber & \hfuzz=500pt filename & \hfuzz=500pt \\
\hfuzz=500pt\includegraphics[width=1em]{element.pdf}~inputfilePotentialLoadLoveNumber & \hfuzz=500pt filename & \hfuzz=500pt if full potential is given and not only loading potential\\
\hfuzz=500pt\includegraphics[width=1em]{element-unbounded.pdf}~useType & \hfuzz=500pt \hyperref[gnssType]{gnssType} & \hfuzz=500pt only use observations that match any of these patterns\\
\hfuzz=500pt\includegraphics[width=1em]{element-unbounded.pdf}~ignoreType & \hfuzz=500pt \hyperref[gnssType]{gnssType} & \hfuzz=500pt ignore observations that match any of these patterns\\
\hfuzz=500pt\includegraphics[width=1em]{element.pdf}~elevationCutOff & \hfuzz=500pt angle & \hfuzz=500pt [degree] ignore observations below cutoff\\
\hfuzz=500pt\includegraphics[width=1em]{element.pdf}~elevationTrackMinimum & \hfuzz=500pt angle & \hfuzz=500pt [degree] ignore tracks that never exceed minimum elevation\\
\hfuzz=500pt\includegraphics[width=1em]{element.pdf}~minObsCountPerTrack & \hfuzz=500pt uint & \hfuzz=500pt tracks with less number of epochs with observations are dropped\\
\hfuzz=500pt\includegraphics[width=1em]{element.pdf}~minEstimableEpochsRatio & \hfuzz=500pt double & \hfuzz=500pt [0,1] drop stations with lower ratio of estimable epochs to total epochs\\
\hfuzz=500pt\includegraphics[width=1em]{element-mustset.pdf}~preprocessing & \hfuzz=500pt sequence & \hfuzz=500pt settings for preprocessing of observations/stations\\
\hfuzz=500pt\includegraphics[width=1em]{connector.pdf}\includegraphics[width=1em]{element.pdf}~printStatistics & \hfuzz=500pt boolean & \hfuzz=500pt print preprocesssing statistics for all receivers\\
\hfuzz=500pt\includegraphics[width=1em]{connector.pdf}\includegraphics[width=1em]{element.pdf}~huber & \hfuzz=500pt double & \hfuzz=500pt residuals \$>\$ huber*sigma0 are downweighted\\
\hfuzz=500pt\includegraphics[width=1em]{connector.pdf}\includegraphics[width=1em]{element.pdf}~huberPower & \hfuzz=500pt double & \hfuzz=500pt residuals \$>\$ huber: sigma=(e/huber)\textasciicircum{}huberPower*sigma0\\
\hfuzz=500pt\includegraphics[width=1em]{connector.pdf}\includegraphics[width=1em]{element.pdf}~codeMaxPositionDiff & \hfuzz=500pt double & \hfuzz=500pt [m] max. allowed position error by PPP code only clock error estimation\\
\hfuzz=500pt\includegraphics[width=1em]{connector.pdf}\includegraphics[width=1em]{element.pdf}~denoisingLambda & \hfuzz=500pt double & \hfuzz=500pt regularization parameter for total variation denoising used in cylce slip detection\\
\hfuzz=500pt\includegraphics[width=1em]{connector.pdf}\includegraphics[width=1em]{element.pdf}~tecWindowSize & \hfuzz=500pt uint & \hfuzz=500pt (0 = disabled) window size for TEC smoothness evaluation used in cycle slip detection\\
\hfuzz=500pt\includegraphics[width=1em]{connector.pdf}\includegraphics[width=1em]{element.pdf}~tecSigmaFactor & \hfuzz=500pt double & \hfuzz=500pt factor applied to moving standard deviation used as threshold in TEC smoothness evaluation during cycle slip detection\\
\hfuzz=500pt\includegraphics[width=1em]{connector.pdf}\includegraphics[width=1em]{element.pdf}~outputfileTrackBefore & \hfuzz=500pt filename & \hfuzz=500pt variables \{station\}, \{prn\}, \{trackTimeStart\}, \{trackTimeEnd\}, \{types\}, TEC and MW-like combinations in cycles for each track before cycle slip detection\\
\hfuzz=500pt\includegraphics[width=1em]{connector.pdf}\includegraphics[width=1em]{element.pdf}~outputfileTrackAfter & \hfuzz=500pt filename & \hfuzz=500pt variables \{station\}, \{prn\}, \{trackTimeStart\}, \{trackTimeEnd\}, \{types\}, TEC and MW-like combinations in cycles for each track after cycle slip detection\\
\hline
\end{tabularx}


\subsection{LowEarthOrbiter}\label{gnssReceiverGeneratorType:lowEarthOrbiter}
A single low-Earth orbiting (LEO) satellite with an onboard GNSS receiver.
An apriori orbit is needed as \configFile{inputfileOrbit}{instrument}.
Attitude data must be provided via \configFile{inputfileStarCamera}{instrument}.
If no attitude data is available from the satellite operator,
the star camera data can be simulated by using \program{SimulateStarCamera}.


\keepXColumns
\begin{tabularx}{\textwidth}{N T A}
\hline
Name & Type & Annotation\\
\hline
\hfuzz=500pt\includegraphics[width=1em]{element-mustset.pdf}~inputfileStationInfo & \hfuzz=500pt filename & \hfuzz=500pt satellite metadata (antenna, receiver, ...)\\
\hfuzz=500pt\includegraphics[width=1em]{element-mustset.pdf}~inputfileAntennaDefinition & \hfuzz=500pt filename & \hfuzz=500pt antenna center offsets and variations\\
\hfuzz=500pt\includegraphics[width=1em]{element-mustset.pdf}~noAntennaPatternFound & \hfuzz=500pt choice & \hfuzz=500pt what should happen if no antenna pattern is found for an observation\\
\hfuzz=500pt\includegraphics[width=1em]{connector.pdf}\includegraphics[width=1em]{element-mustset.pdf}~ignoreObservation & \hfuzz=500pt  & \hfuzz=500pt ignore observation if no matching pattern is found\\
\hfuzz=500pt\includegraphics[width=1em]{connector.pdf}\includegraphics[width=1em]{element-mustset.pdf}~useNearestFrequency & \hfuzz=500pt  & \hfuzz=500pt use pattern of nearest frequency if no matching pattern is found\\
\hfuzz=500pt\includegraphics[width=1em]{connector.pdf}\includegraphics[width=1em]{element-mustset.pdf}~throwException & \hfuzz=500pt  & \hfuzz=500pt throw exception if no matching pattern is found\\
\hfuzz=500pt\includegraphics[width=1em]{element.pdf}~inputfileReceiverDefinition & \hfuzz=500pt filename & \hfuzz=500pt observed signal types\\
\hfuzz=500pt\includegraphics[width=1em]{element-mustset.pdf}~inputfileAccuracyDefinition & \hfuzz=500pt filename & \hfuzz=500pt elevation and azimut dependent accuracy\\
\hfuzz=500pt\includegraphics[width=1em]{element.pdf}~inputfileObservations & \hfuzz=500pt filename & \hfuzz=500pt \\
\hfuzz=500pt\includegraphics[width=1em]{element-mustset.pdf}~inputfileOrbit & \hfuzz=500pt filename & \hfuzz=500pt approximate positions\\
\hfuzz=500pt\includegraphics[width=1em]{element-mustset.pdf}~inputfileStarCamera & \hfuzz=500pt filename & \hfuzz=500pt satellite attitude\\
\hfuzz=500pt\includegraphics[width=1em]{element.pdf}~sigmaFactorPhase & \hfuzz=500pt expression & \hfuzz=500pt PHASE: factor = f(FREQ, ELE, SNR, ROTI, dTEc, IONOINDEX)\\
\hfuzz=500pt\includegraphics[width=1em]{element.pdf}~sigmaFactorCode & \hfuzz=500pt expression & \hfuzz=500pt CODE: factor = f(FREQ, ELE, SNR, ROTI, dTEc, IONOINDEX)\\
\hfuzz=500pt\includegraphics[width=1em]{element.pdf}~supportsIntegerAmbiguities & \hfuzz=500pt boolean & \hfuzz=500pt receiver tracks full cycle integer ambiguities\\
\hfuzz=500pt\includegraphics[width=1em]{element.pdf}~wavelengthFactor & \hfuzz=500pt double & \hfuzz=500pt factor to account for half-wavelength observations (collected by codeless squaring techniques)\\
\hfuzz=500pt\includegraphics[width=1em]{element-unbounded.pdf}~useType & \hfuzz=500pt \hyperref[gnssType]{gnssType} & \hfuzz=500pt only use observations that match any of these patterns\\
\hfuzz=500pt\includegraphics[width=1em]{element-unbounded.pdf}~ignoreType & \hfuzz=500pt \hyperref[gnssType]{gnssType} & \hfuzz=500pt ignore observations that match any of these patterns\\
\hfuzz=500pt\includegraphics[width=1em]{element.pdf}~elevationCutOff & \hfuzz=500pt angle & \hfuzz=500pt [degree] ignore observations below cutoff\\
\hfuzz=500pt\includegraphics[width=1em]{element.pdf}~minObsCountPerTrack & \hfuzz=500pt uint & \hfuzz=500pt tracks with less number of epochs with observations are dropped\\
\hfuzz=500pt\includegraphics[width=1em]{element-mustset.pdf}~preprocessing & \hfuzz=500pt sequence & \hfuzz=500pt settings for preprocessing of observations/stations\\
\hfuzz=500pt\includegraphics[width=1em]{connector.pdf}\includegraphics[width=1em]{element.pdf}~printStatistics & \hfuzz=500pt boolean & \hfuzz=500pt print preprocesssing statistics for all receivers\\
\hfuzz=500pt\includegraphics[width=1em]{connector.pdf}\includegraphics[width=1em]{element.pdf}~huber & \hfuzz=500pt double & \hfuzz=500pt residuals \$>\$ huber*sigma0 are downweighted\\
\hfuzz=500pt\includegraphics[width=1em]{connector.pdf}\includegraphics[width=1em]{element.pdf}~huberPower & \hfuzz=500pt double & \hfuzz=500pt residuals \$>\$ huber: sigma=(e/huber)\textasciicircum{}huberPower*sigma0\\
\hfuzz=500pt\includegraphics[width=1em]{connector.pdf}\includegraphics[width=1em]{element.pdf}~codeMaxPositionDiff & \hfuzz=500pt double & \hfuzz=500pt [m] max. allowed position error by PPP code only clock error estimation\\
\hfuzz=500pt\includegraphics[width=1em]{connector.pdf}\includegraphics[width=1em]{element.pdf}~denoisingLambda & \hfuzz=500pt double & \hfuzz=500pt regularization parameter for total variation denoising used in cylce slip detection\\
\hfuzz=500pt\includegraphics[width=1em]{connector.pdf}\includegraphics[width=1em]{element.pdf}~tecWindowSize & \hfuzz=500pt uint & \hfuzz=500pt (0 = disabled) window size for TEC smoothness evaluation used in cycle slip detection\\
\hfuzz=500pt\includegraphics[width=1em]{connector.pdf}\includegraphics[width=1em]{element.pdf}~tecSigmaFactor & \hfuzz=500pt double & \hfuzz=500pt factor applied to moving standard deviation used as threshold in TEC smoothness evaluation during cycle slip detection\\
\hfuzz=500pt\includegraphics[width=1em]{connector.pdf}\includegraphics[width=1em]{element.pdf}~outputfileTrackBefore & \hfuzz=500pt filename & \hfuzz=500pt variables \{station\}, \{prn\}, \{timeStart\}, \{timeEnd\}, \{types\}, TEC and MW-like combinations in cycles for each track before cycle slip detection\\
\hfuzz=500pt\includegraphics[width=1em]{connector.pdf}\includegraphics[width=1em]{element.pdf}~outputfileTrackAfter & \hfuzz=500pt filename & \hfuzz=500pt variables \{station\}, \{prn\}, \{timeStart\}, \{timeEnd\}, \{types\}, TEC and MW-like combinations in cycles for each track after cycle slip detection\\
\hline
\end{tabularx}

\clearpage
%==================================

\section{GnssTransmitterGenerator}\label{gnssTransmitterGeneratorType}
Definition and basic information of GNSS transmitters.

See also \program{GnssProcessing} and \program{GnssSimulateReceiver}.


\subsection{GNSS}\label{gnssTransmitterGeneratorType:gnss}
A list of satellite PRNs (i.e for GPS: G01, G02, G03, ...) must be provided via
\configFile{inputfileTransmitterList}{stringList}. Satellite system codes follow the
\href{https://files.igs.org/pub/data/format/rinex305.pdf}{RINEX 3 definition}, see \reference{GnssType}{gnssType}.
All input files except \configFile{inputfileAntennaDefinition}{gnssAntennaDefinition},
and \configFile{inputfileReceiverDefinition}{gnssReceiverDefinition} are read for each satellite.
The file name is interpreted as a template with the variable \verb|{prn}| being replaced by the satellite PRN.

Metadata input files (marked with \textbf{*} below) are provided in GROOPS file formats at
\url{https://ftp.tugraz.at/outgoing/ITSG/groops}. These files are regularly updated.
\begin{itemize}
  \item \configFile{inputfileTransmitterInfo}{platform}\textbf{*}:
        PRN-SVN mapping, antenna offsets and orientations.
        Created via \program{GnssAntex2AntennaDefinition} or \program{PlatformCreate}.
  \item \configFile{inputfileAntennaDefinition}{gnssAntennaDefinition}\textbf{*}:
        Antenna center variations.
        Created via \program{GnssAntex2AntennaDefinition} or \program{GnssAntennaDefinitionCreate}.
  \item \configFile{inputfileReceiverDefinition}{gnssReceiverDefinition}\textbf{*}:
        Transmitted signal types.
        Created via \program{GnssReceiverDefinitionCreate} in case you want to define which signal
        types a satellite transmits.
  \item \configFile{inputfileOrbit}{instrument}: Converted via \program{Sp3Format2Orbit} or
        output of \program{GnssProcessing}.
  \item \configFile{inputfileAttitude}{instrument}:
        Rotation from body frame to CRF. Created via \program{SimulateStarCameraGnss} or converted via \program{GnssOrbex2StarCamera}.
  \item \configFile{inputfileClock}{instrument}:
        Converted via \program{GnssClockRinex2InstrumentClock} or \program{GnssRinexNavigation2OrbitClock} or
        output of \program{GnssProcessing}.
\end{itemize}


\keepXColumns
\begin{tabularx}{\textwidth}{N T A}
\hline
Name & Type & Annotation\\
\hline
\hfuzz=500pt\includegraphics[width=1em]{element-mustset-unbounded.pdf}~inputfileTransmitterList & \hfuzz=500pt filename & \hfuzz=500pt ascii file with transmitter PRNs, used to loop variable \{prn\}\\
\hfuzz=500pt\includegraphics[width=1em]{element-mustset.pdf}~inputfileTransmitterInfo & \hfuzz=500pt filename & \hfuzz=500pt variable \{prn\} available\\
\hfuzz=500pt\includegraphics[width=1em]{element-mustset.pdf}~inputfileAntennaDefintion & \hfuzz=500pt filename & \hfuzz=500pt phase centers and variations (ANTEX like)\\
\hfuzz=500pt\includegraphics[width=1em]{element-mustset.pdf}~noAntennaPatternFound & \hfuzz=500pt choice & \hfuzz=500pt what should happen is no antenna pattern is found for an observation\\
\hfuzz=500pt\includegraphics[width=1em]{connector.pdf}\includegraphics[width=1em]{element-mustset.pdf}~ignoreObservation & \hfuzz=500pt  & \hfuzz=500pt ignore observation if no matching pattern is found\\
\hfuzz=500pt\includegraphics[width=1em]{connector.pdf}\includegraphics[width=1em]{element-mustset.pdf}~useNearestFrequency & \hfuzz=500pt  & \hfuzz=500pt use pattern of nearest frequency if no matching pattern is found\\
\hfuzz=500pt\includegraphics[width=1em]{connector.pdf}\includegraphics[width=1em]{element-mustset.pdf}~throwException & \hfuzz=500pt  & \hfuzz=500pt throw exception if no matching pattern is found\\
\hfuzz=500pt\includegraphics[width=1em]{element.pdf}~inputfileSignalDefintion & \hfuzz=500pt filename & \hfuzz=500pt transmitted signal types\\
\hfuzz=500pt\includegraphics[width=1em]{element-mustset.pdf}~inputfileOrbit & \hfuzz=500pt filename & \hfuzz=500pt variable \{prn\} available\\
\hfuzz=500pt\includegraphics[width=1em]{element-mustset.pdf}~inputfileAttitude & \hfuzz=500pt filename & \hfuzz=500pt variable \{prn\} available\\
\hfuzz=500pt\includegraphics[width=1em]{element-mustset.pdf}~inputfileClock & \hfuzz=500pt filename & \hfuzz=500pt variable \{prn\} available\\
\hfuzz=500pt\includegraphics[width=1em]{element.pdf}~interpolationDegree & \hfuzz=500pt uint & \hfuzz=500pt for orbit interpolation and velocity calculation\\
\hline
\end{tabularx}

\clearpage
%==================================

\section{GnssType}\label{gnssType}
A GnssType string consists of six parts (type, frequency, attribute, system, PRN, frequency number)
represented by seven characters.
\begin{itemize}
\item The first three characters (representing type, frequency, and attribute) correspond to the observation codes of the
      \href{https://files.igs.org/pub/data/format/rinex305.pdf}{RINEX 3 definition}.
\item The satellite system character also follows the RINEX 3 definition:
      \begin{itemize}
        \item \verb|G| = GPS
        \item \verb|R| = GLONASS
        \item \verb|E| = Galileo
        \item \verb|C| = BeiDou
        \item \verb|S| = SBAS
        \item \verb|J| = QZSS
        \item \verb|I| = IRNSS
      \end{itemize}
\item PRN is a two-digit number identifying a satellite.
\item Frequency number is only used for GLONASS, where the range -7 to 14 is represented by letters starting with A.
\end{itemize}

Each part of a GnssType string can be replaced by a wildcard '\verb|*|', enabling the use of these strings as patterns,
for example to select a subset of observations (e.g. \verb|C**G**| matches all GPS code/range observations).
Trailing wildcards are optional, meaning \verb|L1*R| is automatically expanded to \verb|L1*R***|.
For some RINEX 2 types (e.g. Galileo L5) the RINEX 3 attribute is unknown/undefined and can be replaced by \verb|?|,
for example \verb|L5?E01|.

Examples:
\begin{itemize}
\item \verb|C1CG23| = code/range observation, L1 frequency, derived from C/A code, GPS, PRN 23
\item \verb|L2PR05B| = phase observation, G2 frequency, derived from P code, GLONASS, PRN 05, frequency number -6
\item \verb|*5*E**| = all observation types, E5a frequency, all attributes, Galileo, all PRNs
\end{itemize}

\clearpage
%==================================

\section{Gravityfield}\label{gravityfieldType}
This class computes functionals of the time depending gravity field,
e.g potential, gravity anomalies or gravity gradients.

If several instances of the class are given the results are summed up.
Before summation every single result is multiplicated by a \config{factor}.
To subtract a normal field like GRS80 from a potential
to get the disturbance potential you must choose one factor by 1
and the other by -1. To get the mean of two fields just set each factor to 0.5.

Some of the instances gives also information about the accuracy.
The variance of the result (sum) is computed by means of variance propagation.


\subsection{PotentialCoefficients}\label{gravityfieldType:potentialCoefficients}
Reads coefficients of a spherical harmonics expansion from file.
The potential is given by
\begin{equation}
V(\lambda,\vartheta,r) = \frac{GM}{R}\sum_{n=0}^\infty \sum_{m=0}^n \left(\frac{R}{r}\right)^{n+1}
  \left(c_{nm} C_{nm}(\lambda,\vartheta) + s_{nm} S_{nm}(\lambda,\vartheta)\right).
\end{equation}
If set the expansion is limited in the range between \config{minDegree}
and \config{maxDegree} inclusivly. The computed result
is multiplied with \config{factor}. If \config{setSigmasToZero} is true
the variances are set to zero. This option is only important for variance propagation
and does not change the result of the gravity field functionals.


\keepXColumns
\begin{tabularx}{\textwidth}{N T A}
\hline
Name & Type & Annotation\\
\hline
\hfuzz=500pt\includegraphics[width=1em]{element-mustset.pdf}~inputfilePotentialCoefficients & \hfuzz=500pt filename & \hfuzz=500pt \\
\hfuzz=500pt\includegraphics[width=1em]{element.pdf}~minDegree & \hfuzz=500pt uint & \hfuzz=500pt \\
\hfuzz=500pt\includegraphics[width=1em]{element.pdf}~maxDegree & \hfuzz=500pt uint & \hfuzz=500pt \\
\hfuzz=500pt\includegraphics[width=1em]{element.pdf}~factor & \hfuzz=500pt double & \hfuzz=500pt the result is multiplied by this factor, set -1 to subtract the field\\
\hfuzz=500pt\includegraphics[width=1em]{element.pdf}~setSigmasToZero & \hfuzz=500pt boolean & \hfuzz=500pt set variances to zero, should be used by adding back reference fields\\
\hline
\end{tabularx}


\subsection{PotentialCoefficientsInterior}
Reads coefficients of a spherical harmonics expansion (for inner space) from file.
If set the expansion is limited in the range between \config{minDegree}
and \config{maxDegree} inclusivly. The computed result is multiplied with \config{factor}.
If \config{setSigmasToZero} is true the variances are set to zero.
This option is only important for error propagation
and does not change the result of the gravity field functionals.


\keepXColumns
\begin{tabularx}{\textwidth}{N T A}
\hline
Name & Type & Annotation\\
\hline
\hfuzz=500pt\includegraphics[width=1em]{element-mustset.pdf}~inputfilePotentialCoefficients & \hfuzz=500pt filename & \hfuzz=500pt \\
\hfuzz=500pt\includegraphics[width=1em]{element.pdf}~minDegree & \hfuzz=500pt uint & \hfuzz=500pt \\
\hfuzz=500pt\includegraphics[width=1em]{element.pdf}~maxDegree & \hfuzz=500pt uint & \hfuzz=500pt \\
\hfuzz=500pt\includegraphics[width=1em]{element.pdf}~factor & \hfuzz=500pt double & \hfuzz=500pt the result is multiplied by this factor, set -1 to subtract the field\\
\hfuzz=500pt\includegraphics[width=1em]{element.pdf}~setSigmasToZero & \hfuzz=500pt boolean & \hfuzz=500pt set variances to zero, should be used by adding back reference fields\\
\hline
\end{tabularx}


\subsection{FromParametrization}\label{gravityfieldType:fromParametrization}
Reads a solution vector from file \configFile{inputfileSolution}{matrix}
which may be computed by a least squares adjustment (e.g. by \program{NormalsSolverVCE}).
The coefficients of the vector are interpreted from position \config{indexStart}
(counting from zero) with help of \configClass{parametrizationGravity}{parametrizationGravityType}.
If the solution file contains solution of several right hand sides you can choose
one with number \config{rightSide} (counting from zero).
You can also read a vector from file \configFile{inputfileSigmax}{matrix}
containing the accuracies of the coefficients.

The computed result is multiplied with \config{factor}.


\keepXColumns
\begin{tabularx}{\textwidth}{N T A}
\hline
Name & Type & Annotation\\
\hline
\hfuzz=500pt\includegraphics[width=1em]{element-mustset-unbounded.pdf}~parametrization & \hfuzz=500pt \hyperref[parametrizationGravityType]{parametrizationGravity} & \hfuzz=500pt \\
\hfuzz=500pt\includegraphics[width=1em]{element-mustset.pdf}~inputfileSolution & \hfuzz=500pt filename & \hfuzz=500pt solution vector\\
\hfuzz=500pt\includegraphics[width=1em]{element.pdf}~inputfileSigmax & \hfuzz=500pt filename & \hfuzz=500pt standards deviations or covariance matrix of the solution\\
\hfuzz=500pt\includegraphics[width=1em]{element.pdf}~indexStart & \hfuzz=500pt uint & \hfuzz=500pt position in the solution vector\\
\hfuzz=500pt\includegraphics[width=1em]{element.pdf}~rightSide & \hfuzz=500pt uint & \hfuzz=500pt if solution contains several right hand sides, select one\\
\hfuzz=500pt\includegraphics[width=1em]{element.pdf}~factor & \hfuzz=500pt double & \hfuzz=500pt the result is multiplied by this factor, set -1 to subtract the field\\
\hline
\end{tabularx}


\subsection{TimeSplines}\label{gravityfieldType:timeSplines}
Read a time variable gravity field from file
\configFile{inputfileTimeSplinesGravityfield}{timeSplinesGravityField}
represented by a spherical harmonics expansion in the spatial domain and spline functions
in the time domain. If set the expansion is limited in the range between
\config{minDegree} and \config{maxDegree} inclusivly.

This file can be created for example by \program{Gravityfield2TimeSplines} or
\program{PotentialCoefficients2BlockMeanTimeSplines}.

The computed result is multiplied with \config{factor}.


\keepXColumns
\begin{tabularx}{\textwidth}{N T A}
\hline
Name & Type & Annotation\\
\hline
\hfuzz=500pt\includegraphics[width=1em]{element-mustset.pdf}~inputfileTimeSplinesGravityfield & \hfuzz=500pt filename & \hfuzz=500pt \\
\hfuzz=500pt\includegraphics[width=1em]{element.pdf}~inputfileTimeSplinesCovariance & \hfuzz=500pt filename & \hfuzz=500pt \\
\hfuzz=500pt\includegraphics[width=1em]{element.pdf}~minDegree & \hfuzz=500pt uint & \hfuzz=500pt \\
\hfuzz=500pt\includegraphics[width=1em]{element.pdf}~maxDegree & \hfuzz=500pt uint & \hfuzz=500pt \\
\hfuzz=500pt\includegraphics[width=1em]{element.pdf}~factor & \hfuzz=500pt double & \hfuzz=500pt the result is multiplied by this factor, set -1 to subtract the field\\
\hline
\end{tabularx}


\subsection{Trend}\label{gravityfieldType:trend}
The given \configClass{gravityfield}{gravityfieldType} is interpreted
as trend function and the result is computed at time $t$ as follows
\begin{equation}
V(\M x,t) = \frac{t-t_0}{\Delta t}V(\M x),
\end{equation}
with $t_0$ is \config{timeStart} and $\Delta t$ is \config{timeStep}.


\keepXColumns
\begin{tabularx}{\textwidth}{N T A}
\hline
Name & Type & Annotation\\
\hline
\hfuzz=500pt\includegraphics[width=1em]{element-mustset-unbounded.pdf}~gravityfield & \hfuzz=500pt \hyperref[gravityfieldType]{gravityfield} & \hfuzz=500pt this field is multiplicated by (time-time0)/timeStep\\
\hfuzz=500pt\includegraphics[width=1em]{element-mustset.pdf}~timeStart & \hfuzz=500pt time & \hfuzz=500pt reference time\\
\hfuzz=500pt\includegraphics[width=1em]{element-mustset.pdf}~timeStep & \hfuzz=500pt time & \hfuzz=500pt \\
\hline
\end{tabularx}


\subsection{Oscillation}\label{gravityfieldType:oscillation}
The given \configClass{gravityfield}{gravityfieldType} is interpreted
as oscillation function and the result is computed at time $t$ as follows
\begin{equation}
V(\M x,t) = \cos(\omega)V_{cos}(\M x)+\sin(\omega)V_{sin}(\M x),
\end{equation}
with $\omega=\frac{2\pi}{T}(t-t_0)$.


\keepXColumns
\begin{tabularx}{\textwidth}{N T A}
\hline
Name & Type & Annotation\\
\hline
\hfuzz=500pt\includegraphics[width=1em]{element-mustset-unbounded.pdf}~gravityfieldCos & \hfuzz=500pt \hyperref[gravityfieldType]{gravityfield} & \hfuzz=500pt multiplicated by cos(2pi/T(time-time0))\\
\hfuzz=500pt\includegraphics[width=1em]{element-mustset-unbounded.pdf}~gravityfieldSin & \hfuzz=500pt \hyperref[gravityfieldType]{gravityfield} & \hfuzz=500pt multiplicated by sin(2pi/T(time-time0))\\
\hfuzz=500pt\includegraphics[width=1em]{element-mustset.pdf}~time0 & \hfuzz=500pt time & \hfuzz=500pt reference time\\
\hfuzz=500pt\includegraphics[width=1em]{element-mustset.pdf}~period & \hfuzz=500pt time & \hfuzz=500pt [day]\\
\hline
\end{tabularx}


\subsection{InInterval}
A \configClass{gravityfield}{gravityfieldType} is only evaluated in the interval between
\config{timeStart} inclusively and \config{timeEnd} exclusively.
Outside the interval the result is zero.

This class is useful to get a time series of monthly mean GRACE gravity field solutions.
In each month another file of potentialCoefficients is valid.
This can easily be created with \configClass{loop}{loopType}.


\keepXColumns
\begin{tabularx}{\textwidth}{N T A}
\hline
Name & Type & Annotation\\
\hline
\hfuzz=500pt\includegraphics[width=1em]{element-mustset-unbounded.pdf}~gravityfield & \hfuzz=500pt \hyperref[gravityfieldType]{gravityfield} & \hfuzz=500pt \\
\hfuzz=500pt\includegraphics[width=1em]{element-mustset.pdf}~timeStart & \hfuzz=500pt time & \hfuzz=500pt first point in time\\
\hfuzz=500pt\includegraphics[width=1em]{element-mustset.pdf}~timeEnd & \hfuzz=500pt time & \hfuzz=500pt last point in time will be less or equal timeEnd\\
\hline
\end{tabularx}


\subsection{Tides}\label{gravityfieldType:tides}
Treat \configClass{tides}{tidesType} as gravitational forces.
The tides need a realization of \configClass{earthRotation}{earthRotationType}
to transform between the CRF and TRF and to compute rotational deformation
from polar motion.
It also needs \configClass{ephemerides}{ephemeridesType} from Sun, moon, and planets.


\keepXColumns
\begin{tabularx}{\textwidth}{N T A}
\hline
Name & Type & Annotation\\
\hline
\hfuzz=500pt\includegraphics[width=1em]{element-mustset-unbounded.pdf}~tides & \hfuzz=500pt \hyperref[tidesType]{tides} & \hfuzz=500pt \\
\hfuzz=500pt\includegraphics[width=1em]{element-mustset.pdf}~earthRotation & \hfuzz=500pt \hyperref[earthRotationType]{earthRotation} & \hfuzz=500pt \\
\hfuzz=500pt\includegraphics[width=1em]{element.pdf}~ephemerides & \hfuzz=500pt \hyperref[ephemeridesType]{ephemerides} & \hfuzz=500pt \\
\hline
\end{tabularx}


\subsection{Topography}\label{gravityfieldType:topography}
The gravity is integrated from a topographic mass distribution.
For each grid point in \configFile{inputfileGridRectangular}{griddedData} a prisma with
\config{density} is assumed. The horizontal extension is computed from the grid spacing
and the vertical extension is given by \config{radialLowerBound}
and \config{radialUpperBound} above ellipsoid. All values are expressions and computed
for each point with given data in the grid file. The standard variables for grids
are available, see~\reference{dataVariables}{general.parser:dataVariables}.

Example: The grid file contains the orthometric height of the topography in the first
column, the geoid height in the second and the mean density of each prism in the third
column. In this case the following settings should be used:
\begin{itemize}
\item \config{radialUpperBound} = \verb|data0+data1|,
\item \config{radialLowerBound} = \verb|data1|,
\item \config{density} = \verb|data2|.
\end{itemize}

As the prim computation is time consuming a maximum distance around the evaluation point
can defined with \config{distancePrism}. Afterwards a simplified radial line
(the prism mass is concentrated to a line in the center) is used up to
a distance of \config{distanceLine}. At last the prim is approximated by a point mass
in the center up to a distance \config{distanceMax} (if set). Prisms nearby the evaluation
point can be excluded with \config{distanceMin}.


\keepXColumns
\begin{tabularx}{\textwidth}{N T A}
\hline
Name & Type & Annotation\\
\hline
\hfuzz=500pt\includegraphics[width=1em]{element-mustset.pdf}~inputfileGridRectangular & \hfuzz=500pt filename & \hfuzz=500pt Digital Terrain Model\\
\hfuzz=500pt\includegraphics[width=1em]{element.pdf}~density & \hfuzz=500pt expression & \hfuzz=500pt expression [kg/m**3]\\
\hfuzz=500pt\includegraphics[width=1em]{element.pdf}~radialUpperBound & \hfuzz=500pt expression & \hfuzz=500pt expression (variables 'height', 'data', 'L', 'B' and, 'area' are taken from the gridded data\\
\hfuzz=500pt\includegraphics[width=1em]{element.pdf}~radialLowerBound & \hfuzz=500pt expression & \hfuzz=500pt expression (variables 'height', 'data', 'L', 'B' and, 'area' are taken from the gridded data\\
\hfuzz=500pt\includegraphics[width=1em]{element.pdf}~distanceMin & \hfuzz=500pt double & \hfuzz=500pt [km] min. influence distance (ignore near zone)\\
\hfuzz=500pt\includegraphics[width=1em]{element.pdf}~distancePrism & \hfuzz=500pt double & \hfuzz=500pt [km] max. distance for prism formular\\
\hfuzz=500pt\includegraphics[width=1em]{element.pdf}~distanceLine & \hfuzz=500pt double & \hfuzz=500pt [km] max. distance for radial integration\\
\hfuzz=500pt\includegraphics[width=1em]{element.pdf}~distanceMax & \hfuzz=500pt double & \hfuzz=500pt [km] max. influence distance (ignore far zone)\\
\hfuzz=500pt\includegraphics[width=1em]{element.pdf}~factor & \hfuzz=500pt double & \hfuzz=500pt the result is multiplied by this factor, set -1 to subtract the field\\
\hline
\end{tabularx}


\subsection{EarthquakeOscillation}
The given \configClass{gravityfield}{gravityfieldType} is interpreted as an oscillation function
in the gravitational potential field, caused by large earthquakes.
The result is computed at time $t$ as follows:
\begin{equation}
C_{lm}(\M t) = \sum_{n=0}^NC_{nlm}(1-\cos(\omega)\exp(\frac{-\omega}{2Q_{nlm}})),
\end{equation}
with $\omega=\frac{2\pi}{T_{nlm}}(t-t_0)$. In this equation, $Q_{nlm}$ is the attenuation factor,
$n$ is the overtone factor, $m$ is degree, $l$ is order, and $t$ is time in second.
$T_{nlm}$ and $Q_{nlm}$ are computed with the elastic Earth model or observed from the long
period record of superconducting gravimeter measurements after the earthquakes.


\keepXColumns
\begin{tabularx}{\textwidth}{N T A}
\hline
Name & Type & Annotation\\
\hline
\hfuzz=500pt\includegraphics[width=1em]{element-mustset.pdf}~inputCoefficientMatrix & \hfuzz=500pt filename & \hfuzz=500pt oscillation model parameters\\
\hfuzz=500pt\includegraphics[width=1em]{element-mustset.pdf}~time0 & \hfuzz=500pt time & \hfuzz=500pt the time earthquake happened\\
\hfuzz=500pt\includegraphics[width=1em]{element.pdf}~minDegree & \hfuzz=500pt uint & \hfuzz=500pt \\
\hfuzz=500pt\includegraphics[width=1em]{element.pdf}~maxDegree & \hfuzz=500pt uint & \hfuzz=500pt \\
\hfuzz=500pt\includegraphics[width=1em]{element.pdf}~GM & \hfuzz=500pt double & \hfuzz=500pt Geocentric gravitational constant\\
\hfuzz=500pt\includegraphics[width=1em]{element.pdf}~R & \hfuzz=500pt double & \hfuzz=500pt reference radius\\
\hline
\end{tabularx}


\subsection{Filter}
Convert \configClass{gravityfield}{gravityfieldType} to spherical harmonics
and \configClass{filter}{sphericalHarmonicsFilterType} the coefficients.


\keepXColumns
\begin{tabularx}{\textwidth}{N T A}
\hline
Name & Type & Annotation\\
\hline
\hfuzz=500pt\includegraphics[width=1em]{element-mustset-unbounded.pdf}~gravityfield & \hfuzz=500pt \hyperref[gravityfieldType]{gravityfield} & \hfuzz=500pt \\
\hfuzz=500pt\includegraphics[width=1em]{element-mustset-unbounded.pdf}~filter & \hfuzz=500pt \hyperref[sphericalHarmonicsFilterType]{sphericalHarmonicsFilter} & \hfuzz=500pt \\
\hline
\end{tabularx}

\clearpage
%==================================

\section{Grid}\label{gridType}
This class generates a set of grid points. In a first step, the grid
is always generated globally, with \configClass{border}{borderType} a regional
subset of points can be extracted from the global grid. The parameters
\config{R} and \config{inverseFlattening} define the shape of the ellipsoid
on which the grid is generated. In case \config{inverseFlattening} is
chosen as zero, a sphere is used. With \config{height} the distance of
the points above the ellipsoid can be defined. In addition to the location
of the points, weights are assigned to each of the points. These weights
can be regarded as the surface element associated with each grid point.


\subsection{Geograph}
The geographical grid is an equal-angular point distribution with points
located along meridians and along circles of latitude. \config{deltaLambda}
denotes the angular difference between adjacent points along meridians and
\config{deltaPhi} describes the angular difference between adjacent points
along circles of latitude. The point setting results as follows:
\begin{equation}
\lambda_i=\frac{\Delta\lambda}{2}+i\cdot\Delta\lambda\qquad\mbox{with}\qquad 0\leq i< \frac{360^\circ}{\Delta\lambda},
\end{equation}
\begin{equation}
\varphi_j=-90^\circ+\frac{\Delta\varphi}{2}+j\cdot\Delta\varphi\qquad\mbox{with}\qquad 0\leq j<\frac{180^\circ}{\Delta\varphi}.
\end{equation}
The number of grid points can be determined by
\begin{equation}
I=\frac{360^\circ}{\Delta\lambda}\cdot\frac{180^\circ}{\Delta\varphi}.
\end{equation}
The weights are calculated according to
\begin{equation}
w_i=\int\limits_{\lambda_i-\frac{\Delta\lambda}{2}}^{\lambda_i+\frac{\Delta\lambda}{2}}\int\limits_{\vartheta_i-\frac{\Delta\vartheta}{2}}^{\vartheta_i+\frac{\Delta\vartheta}{2}}=2\cdot\Delta\lambda\sin(\Delta\vartheta)\sin(\vartheta_i).
\end{equation}


\keepXColumns
\begin{tabularx}{\textwidth}{N T A}
\hline
Name & Type & Annotation\\
\hline
\hfuzz=500pt\includegraphics[width=1em]{element-mustset.pdf}~deltaLambda & \hfuzz=500pt angle & \hfuzz=500pt \\
\hfuzz=500pt\includegraphics[width=1em]{element-mustset.pdf}~deltaPhi & \hfuzz=500pt angle & \hfuzz=500pt \\
\hfuzz=500pt\includegraphics[width=1em]{element.pdf}~height & \hfuzz=500pt double & \hfuzz=500pt ellipsoidal height expression (variables 'height', 'L', 'B')\\
\hfuzz=500pt\includegraphics[width=1em]{element.pdf}~R & \hfuzz=500pt double & \hfuzz=500pt major axsis of the ellipsoid/sphere\\
\hfuzz=500pt\includegraphics[width=1em]{element.pdf}~inverseFlattening & \hfuzz=500pt double & \hfuzz=500pt flattening of the ellipsoid, 0: sphere\\
\hfuzz=500pt\includegraphics[width=1em]{element-unbounded.pdf}~border & \hfuzz=500pt \hyperref[borderType]{border} & \hfuzz=500pt \\
\hline
\end{tabularx}


\subsection{TriangleVertex}
The zeroth level of densification
coincides with the 12 icosahedron vertices, as displayed in the upper left part
of Fig.~\ref{fig:triangle_grid}. Then, depending on the envisaged densification,
each triangle edge is divided into $n$ parts, illustrated in the upper right
part of Fig.~\ref{fig:triangle_grid}. The new nodes on the edges are then connected
by arcs of great circles parallel to the triangle edges. The intersections of
each three corresponding parallel lines become nodes of the densified grid as well.
As in case of a spherical triangle those three connecting lines do not exactly
intersect in one point, the center of the resulting triangle is used as location
for the new node (lower left part of Fig.~\ref{fig:triangle_grid}). The lower right
side of Fig.~\ref{fig:triangle_grid} finally shows the densified triangle vertex
grid for a level of $n=3$. The number of grid points in dependence of the chosen
level of densification can be calculated by
\begin{equation}\label{eq:numberVertex}
I=10\cdot(n+1)^2+2.
\end{equation}

\fig{!hb}{0.6}{icogrid}{fig:triangle_grid}{TriangleVertex grid.}


\keepXColumns
\begin{tabularx}{\textwidth}{N T A}
\hline
Name & Type & Annotation\\
\hline
\hfuzz=500pt\includegraphics[width=1em]{element-mustset.pdf}~level & \hfuzz=500pt uint & \hfuzz=500pt division of icosahedron, point count = 10*(n+1)**2+2\\
\hfuzz=500pt\includegraphics[width=1em]{element.pdf}~R & \hfuzz=500pt double & \hfuzz=500pt major axsis of the ellipsoid/sphere\\
\hfuzz=500pt\includegraphics[width=1em]{element.pdf}~inverseFlattening & \hfuzz=500pt double & \hfuzz=500pt flattening of the ellipsoid, 0: sphere\\
\hfuzz=500pt\includegraphics[width=1em]{element-unbounded.pdf}~border & \hfuzz=500pt \hyperref[borderType]{border} & \hfuzz=500pt \\
\hline
\end{tabularx}


\subsection{TriangleCenter}
The points of the zeroth level are located at the centers of the icosahedron triangles.
To achieve a finer grid, each of the triangles is divided into four smaller triangles by
connecting the midpoints of the triangle edges. The refined grid points are again located
at the center of the triangles. Subsequently, the triangles can be further densified up to
the desired level of densification $n$, which is defined by \config{level}.

The number of global grid points for a certain level can be determined by
\begin{equation}\label{eq:numberCenter}
I=20\cdot 4^n.
\end{equation}
Thus the quantity of grid points depends exponentially on the level $n$, as with
every additional level the number of grid points quadruplicates.


\keepXColumns
\begin{tabularx}{\textwidth}{N T A}
\hline
Name & Type & Annotation\\
\hline
\hfuzz=500pt\includegraphics[width=1em]{element-mustset.pdf}~level & \hfuzz=500pt uint & \hfuzz=500pt division of icosahedron, point count = 5*4**(n+1)\\
\hfuzz=500pt\includegraphics[width=1em]{element.pdf}~R & \hfuzz=500pt double & \hfuzz=500pt major axsis of the ellipsoid/sphere\\
\hfuzz=500pt\includegraphics[width=1em]{element.pdf}~inverseFlattening & \hfuzz=500pt double & \hfuzz=500pt flattening of the ellipsoid, 0: sphere\\
\hfuzz=500pt\includegraphics[width=1em]{element-unbounded.pdf}~border & \hfuzz=500pt \hyperref[borderType]{border} & \hfuzz=500pt \\
\hline
\end{tabularx}


\subsection{Gauss}
 The grid features equiangular spacing along circles of latitude with
 \config{parallelsCount} defining the number $L$ of the parallels.
\begin{equation}
\Delta\lambda=\frac{\pi}{L}\qquad\Rightarrow\qquad\lambda_i=\frac{\Delta\lambda}{2}+i\cdot\Delta\lambda\qquad\mbox{with}\qquad 0\leq i< 2L.
\end{equation}
Along the meridians the points are located at $L$ parallels at
the $L$ zeros $\vartheta_j$ of the Legendre polynomial of degree~$L$,
\begin{equation}
P_L(\cos\vartheta_j)=0.
\end{equation}
Consequently, the number of grid points sums up to
\begin{equation}
I=2\cdot L^2.
\end{equation}
The weights can be calculated according to
\begin{equation}
w_i(L)=\Delta\lambda\frac{2}{(1-t_i^2)(P'_{L}(\cos(\vartheta _i)))^2},\label{weights}
\end{equation}


\keepXColumns
\begin{tabularx}{\textwidth}{N T A}
\hline
Name & Type & Annotation\\
\hline
\hfuzz=500pt\includegraphics[width=1em]{element-mustset.pdf}~parallelsCount & \hfuzz=500pt uint & \hfuzz=500pt \\
\hfuzz=500pt\includegraphics[width=1em]{element.pdf}~R & \hfuzz=500pt double & \hfuzz=500pt major axsis of the ellipsoid/sphere\\
\hfuzz=500pt\includegraphics[width=1em]{element.pdf}~inverseFlattening & \hfuzz=500pt double & \hfuzz=500pt flattening of the ellipsoid, 0: sphere\\
\hfuzz=500pt\includegraphics[width=1em]{element-unbounded.pdf}~border & \hfuzz=500pt \hyperref[borderType]{border} & \hfuzz=500pt \\
\hline
\end{tabularx}


\subsection{Reuter}
The Reuter grid features equi-distant spacing along the meridians determined
by the control parameter~$\gamma$ according to
\begin{equation}
\Delta\vartheta=\frac{\pi}{\gamma}\qquad\Rightarrow\vartheta_j=j\Delta\vartheta,\qquad\mbox{with}\qquad 1\leq j\leq \gamma-1.
\end{equation}
Thus $\gamma+1$ denotes the number of points per meridian, as the two poles
are included in the point distribution as well. Along the circles of latitude,
the number of grid points decreases with increasing latitude in order to achieve
an evenly distributed point pattern. This number is chosen, so that the points
along each circle of latitude have the same spherical distance as two adjacent
latitudes. The resulting relationship is given by
\begin{equation}\label{eq:sphericalDistance}
\Delta\vartheta=\arccos\left( \cos^2\vartheta_j+\sin^2\vartheta_j\cos\Delta\lambda_j\right).
\end{equation}
The left hand side of this equation is the spherical distance between adjacent
latitudes, the right hand side stands for the spherical distance between two points
with the same polar distance $\vartheta_j$ and a longitudinal difference of
$\Delta\lambda_i$. This longitudinal distance can be adjusted depending on
$\vartheta_j$ to fulfill Eq.~\eqref{eq:sphericalDistance}. The resulting
formula for $\Delta\lambda_i$ is
\begin{equation}\label{eq:deltaLambdai}
\Delta\lambda_j=\arccos\left( \frac{\sin\Delta\vartheta -\cos^2\vartheta_j}{\sin^2\vartheta_j}\right).
\end{equation}
The number of points~$\gamma_j$ for each circle of latitude can then be determined by
\begin{equation}\label{eq:gammai}
\gamma_j=\left[ \frac{2\pi}{\Delta\lambda_j}\right] .
\end{equation}
Here the Gauss bracket $[x]$ specifies the largest integer equal to or less than $x$.
The longitudes are subsequently determined by
\begin{equation}
\lambda_{ij}=\frac{\Delta\lambda_j}{2}+i\cdot(2\pi/\gamma_j),\qquad \mbox{with\qquad}0\leq i< \gamma_j.
\end{equation}
The number of grid points can be estimated by
\begin{equation}\label{eq:numberReuter}
I=\leq 2+\frac{4}{\pi}\gamma^2,
\end{equation}
The $\leq$ results from the fact that the $\gamma_j$ are restricted to integer values.


\keepXColumns
\begin{tabularx}{\textwidth}{N T A}
\hline
Name & Type & Annotation\\
\hline
\hfuzz=500pt\includegraphics[width=1em]{element-mustset.pdf}~gamma & \hfuzz=500pt uint & \hfuzz=500pt number of parallels\\
\hfuzz=500pt\includegraphics[width=1em]{element.pdf}~height & \hfuzz=500pt double & \hfuzz=500pt ellipsoidal height\\
\hfuzz=500pt\includegraphics[width=1em]{element.pdf}~R & \hfuzz=500pt double & \hfuzz=500pt major axsis of the ellipsoid/sphere\\
\hfuzz=500pt\includegraphics[width=1em]{element.pdf}~inverseFlattening & \hfuzz=500pt double & \hfuzz=500pt flattening of the ellipsoid, 0: sphere\\
\hfuzz=500pt\includegraphics[width=1em]{element-unbounded.pdf}~border & \hfuzz=500pt \hyperref[borderType]{border} & \hfuzz=500pt \\
\hline
\end{tabularx}


\subsection{Corput}
This kind of grid distributes an arbitrarily chosen number of $I$ points
(defined by \config{globalPointsCount}) following a recursive, quasi random sequence.
In longitudinal direction the pattern follows
\begin{equation}
\Delta\lambda=\frac{2\pi}{I}\qquad\Rightarrow\qquad\frac{\Delta\lambda}{2}+\lambda_i=i\cdot\Delta\lambda\qquad\mbox{with}\qquad 1\leq i\leq I.
\end{equation}
This implies that every grid point features a unique longitude, with equi-angular
longitudinal differences.

The polar distance in the form $t_i=\cos\vartheta_i$ for each point is determined
by the following recursive sequence:
\begin{itemize}
\item Starting from an interval $t\in[-1,1]$.
\item If $I=1$, then the midpoint of the interval is returned as result of
the sequence, and the sequence is terminated.
\item If the number of points is uneven, the  midpoint is included into the list of $t_i$.
\item Subsequently, the interval is bisected into an upper and lower half,
       and the sequence is called for both halves.
\item $t$ from upper and lower half are alternately sorted into the list of $t_i$.
\item The polar distances are calculated by
\begin{equation}
\vartheta_i=\arccos\, t_i.
\end{equation}
\end{itemize}


\keepXColumns
\begin{tabularx}{\textwidth}{N T A}
\hline
Name & Type & Annotation\\
\hline
\hfuzz=500pt\includegraphics[width=1em]{element-mustset.pdf}~globalPointsCount & \hfuzz=500pt uint & \hfuzz=500pt \\
\hfuzz=500pt\includegraphics[width=1em]{element.pdf}~height & \hfuzz=500pt double & \hfuzz=500pt ellipsoidal height\\
\hfuzz=500pt\includegraphics[width=1em]{element.pdf}~R & \hfuzz=500pt double & \hfuzz=500pt major axsis of the ellipsoid/sphere\\
\hfuzz=500pt\includegraphics[width=1em]{element.pdf}~inverseFlattening & \hfuzz=500pt double & \hfuzz=500pt flattening of the ellipsoid, 0: sphere\\
\hfuzz=500pt\includegraphics[width=1em]{element-unbounded.pdf}~border & \hfuzz=500pt \hyperref[borderType]{border} & \hfuzz=500pt \\
\hline
\end{tabularx}


\subsection{Driscoll}
The Driscoll-Healy grid, has equiangular spacing along the meridians as well
as along the circles of latitude. In longitudinal direction (along the parallels),
these angular differences for a given \config{dimension} $L$ coincide with those
described for the corresponding geographical grid and Gauss grid. Along the meridians,
the size of the latitudinal differences is half the size compared to the geographical
grid. This results in the following point pattern,
\begin{equation}
\begin{split}
\Delta\lambda=\frac{\pi}{L}\qquad&\Rightarrow\qquad\lambda_i=\frac{\Delta\lambda}{2}+i\cdot\Delta\lambda\qquad&\mbox{with}\qquad 0\leq i< 2L, \\
\Delta\vartheta=\frac{\pi}{2L}\qquad&\Rightarrow\qquad\vartheta_j=j\cdot\Delta\vartheta\qquad&\mbox{with}\qquad 1\leq j\leq 2L.
\end{split}
\end{equation}
Consequently, the number of grid points is
\begin{equation}
I=4\cdot L^2.
\end{equation}
The weights are given by
\begin{equation}
w_i=\Delta\lambda\frac{4}{2L}\sin(\vartheta_i)\sum_{l=0}^{L-1}\frac{\sin\left[ (2l+1)\;\vartheta_i\right] }{2l+1}.
\end{equation}


\keepXColumns
\begin{tabularx}{\textwidth}{N T A}
\hline
Name & Type & Annotation\\
\hline
\hfuzz=500pt\includegraphics[width=1em]{element-mustset.pdf}~dimension & \hfuzz=500pt uint & \hfuzz=500pt number of parallels = 2*dimension\\
\hfuzz=500pt\includegraphics[width=1em]{element.pdf}~height & \hfuzz=500pt double & \hfuzz=500pt ellipsoidal height\\
\hfuzz=500pt\includegraphics[width=1em]{element.pdf}~R & \hfuzz=500pt double & \hfuzz=500pt major axsis of the ellipsoid/sphere\\
\hfuzz=500pt\includegraphics[width=1em]{element.pdf}~inverseFlattening & \hfuzz=500pt double & \hfuzz=500pt flattening of the ellipsoid, 0: sphere\\
\hfuzz=500pt\includegraphics[width=1em]{element-unbounded.pdf}~border & \hfuzz=500pt \hyperref[borderType]{border} & \hfuzz=500pt \\
\hline
\end{tabularx}


\subsection{SinglePoint}
Creates one single point.


\keepXColumns
\begin{tabularx}{\textwidth}{N T A}
\hline
Name & Type & Annotation\\
\hline
\hfuzz=500pt\includegraphics[width=1em]{element-mustset.pdf}~L & \hfuzz=500pt angle & \hfuzz=500pt longitude\\
\hfuzz=500pt\includegraphics[width=1em]{element-mustset.pdf}~B & \hfuzz=500pt angle & \hfuzz=500pt latitude\\
\hfuzz=500pt\includegraphics[width=1em]{element.pdf}~height & \hfuzz=500pt double & \hfuzz=500pt ellipsoidal height\\
\hfuzz=500pt\includegraphics[width=1em]{element.pdf}~area & \hfuzz=500pt double & \hfuzz=500pt associated area element on unit sphere\\
\hfuzz=500pt\includegraphics[width=1em]{element.pdf}~R & \hfuzz=500pt double & \hfuzz=500pt major axsis of the ellipsoid/sphere\\
\hfuzz=500pt\includegraphics[width=1em]{element.pdf}~inverseFlattening & \hfuzz=500pt double & \hfuzz=500pt flattening of the ellipsoid, 0: sphere\\
\hline
\end{tabularx}


\subsection{SinglePointCartesian}
Creates one single point.


\keepXColumns
\begin{tabularx}{\textwidth}{N T A}
\hline
Name & Type & Annotation\\
\hline
\hfuzz=500pt\includegraphics[width=1em]{element-mustset.pdf}~x & \hfuzz=500pt double & \hfuzz=500pt [m]\\
\hfuzz=500pt\includegraphics[width=1em]{element-mustset.pdf}~y & \hfuzz=500pt double & \hfuzz=500pt [m]\\
\hfuzz=500pt\includegraphics[width=1em]{element-mustset.pdf}~z & \hfuzz=500pt double & \hfuzz=500pt [m]\\
\hfuzz=500pt\includegraphics[width=1em]{element.pdf}~area & \hfuzz=500pt double & \hfuzz=500pt associated area element on unit sphere\\
\hline
\end{tabularx}


\subsection{File}\label{gridType:file}
In this class grid is read from a file, which is given by \configFile{inputfileGrid}{griddedData}.
A corresponding file can be generated with \program{GriddedDataCreate} or with \program{Matrix2GriddedData}.


\keepXColumns
\begin{tabularx}{\textwidth}{N T A}
\hline
Name & Type & Annotation\\
\hline
\hfuzz=500pt\includegraphics[width=1em]{element-mustset.pdf}~inputfileGrid & \hfuzz=500pt filename & \hfuzz=500pt \\
\hfuzz=500pt\includegraphics[width=1em]{element-unbounded.pdf}~border & \hfuzz=500pt \hyperref[borderType]{border} & \hfuzz=500pt \\
\hline
\end{tabularx}

\clearpage
%==================================

\section{InstrumentType}\label{instrumentTypeType}
Defines the type of an \file{instrument file}{instrument}.


\keepXColumns
\begin{tabularx}{\textwidth}{N T A}
\hline
Name & Type & Annotation\\
\hline
\hfuzz=500pt\includegraphics[width=1em]{element-mustset.pdf}~instrumentTypeType & \hfuzz=500pt choice & \hfuzz=500pt instrument type\\
\hfuzz=500pt\includegraphics[width=1em]{connector.pdf}\includegraphics[width=1em]{element-mustset.pdf}~INSTRUMENTTIME & \hfuzz=500pt  & \hfuzz=500pt time without data\\
\hfuzz=500pt\includegraphics[width=1em]{connector.pdf}\includegraphics[width=1em]{element-mustset.pdf}~MISCVALUE & \hfuzz=500pt  & \hfuzz=500pt single value\\
\hfuzz=500pt\includegraphics[width=1em]{connector.pdf}\includegraphics[width=1em]{element-mustset.pdf}~MISCVALUES & \hfuzz=500pt  & \hfuzz=500pt multiple values\\
\hfuzz=500pt\includegraphics[width=1em]{connector.pdf}\includegraphics[width=1em]{element-mustset.pdf}~VECTOR3D & \hfuzz=500pt  & \hfuzz=500pt x, y, z\\
\hfuzz=500pt\includegraphics[width=1em]{connector.pdf}\includegraphics[width=1em]{element-mustset.pdf}~COVARIANCE3D & \hfuzz=500pt  & \hfuzz=500pt xx, yy, zz, xy, xz, yz\\
\hfuzz=500pt\includegraphics[width=1em]{connector.pdf}\includegraphics[width=1em]{element-mustset.pdf}~ORBIT & \hfuzz=500pt  & \hfuzz=500pt position [m], velocity [m/s], acceleration [m/s\textasciicircum{}2] (each x, y, z)\\
\hfuzz=500pt\includegraphics[width=1em]{connector.pdf}\includegraphics[width=1em]{element-mustset.pdf}~STARCAMERA & \hfuzz=500pt  & \hfuzz=500pt quaternions (q0, qx, qy, qz)\\
\hfuzz=500pt\includegraphics[width=1em]{connector.pdf}\includegraphics[width=1em]{element-mustset.pdf}~ACCELEROMETER & \hfuzz=500pt  & \hfuzz=500pt x, y, z [m/s\textasciicircum{}2]\\
\hfuzz=500pt\includegraphics[width=1em]{connector.pdf}\includegraphics[width=1em]{element-mustset.pdf}~SATELLITETRACKING & \hfuzz=500pt  & \hfuzz=500pt range [m], range rate [m/s], range acceleration [m/s\textasciicircum{}2]\\
\hfuzz=500pt\includegraphics[width=1em]{connector.pdf}\includegraphics[width=1em]{element-mustset.pdf}~GRADIOMETER & \hfuzz=500pt  & \hfuzz=500pt xx, yy, zz, xy, xz, yz [1/s\textasciicircum{}2]\\
\hfuzz=500pt\includegraphics[width=1em]{connector.pdf}\includegraphics[width=1em]{element-mustset.pdf}~GNSSRECEIVER & \hfuzz=500pt  & \hfuzz=500pt GNSS phase/code observations [m]\\
\hfuzz=500pt\includegraphics[width=1em]{connector.pdf}\includegraphics[width=1em]{element-mustset.pdf}~OBSERVATIONSIGMA & \hfuzz=500pt  & \hfuzz=500pt accuracy\\
\hfuzz=500pt\includegraphics[width=1em]{connector.pdf}\includegraphics[width=1em]{element-mustset.pdf}~MASS & \hfuzz=500pt  & \hfuzz=500pt \\
\hfuzz=500pt\includegraphics[width=1em]{connector.pdf}\includegraphics[width=1em]{element-mustset.pdf}~THRUSTER & \hfuzz=500pt  & \hfuzz=500pt \\
\hfuzz=500pt\includegraphics[width=1em]{connector.pdf}\includegraphics[width=1em]{element-mustset.pdf}~MAGNETOMETER & \hfuzz=500pt  & \hfuzz=500pt \\
\hfuzz=500pt\includegraphics[width=1em]{connector.pdf}\includegraphics[width=1em]{element-mustset.pdf}~ACCHOUSEKEEPING & \hfuzz=500pt  & \hfuzz=500pt \\
\hline
\end{tabularx}

\clearpage
%==================================

\section{InterpolatorTimeSeries}\label{interpolatorTimeSeriesType}
This class resamples data of a times series to new poins in time.


\subsection{Polynomial}
Polynomial prediction using a moving polynomial of \config{polynomialDegree}.
The optimal polynomial is chosen based on the centricity of the data points around the resampling
point and the distance to all polynomial data points. All polynomial data points must be within
\config{maxDataPointRange}. Resampling points within \config{maxExtrapolationDistance} of the
polynomial will be extrapolated. The elements \config{maxDataPointRange} and \config{maxExtrapolationDistance}
are given in the unit of seconds. If negative values are used, the unit is relative to the median input sampling.

\fig{!hb}{0.5}{instrumentResample_polynomial}{fig:instrumentResample_polynomial}{Example of polynomial prediction when resampling from 5 to 1 minute sampling}


\keepXColumns
\begin{tabularx}{\textwidth}{N T A}
\hline
Name & Type & Annotation\\
\hline
\hfuzz=500pt\includegraphics[width=1em]{element-mustset.pdf}~polynomialDegree & \hfuzz=500pt uint & \hfuzz=500pt degree of the moving polynomial\\
\hfuzz=500pt\includegraphics[width=1em]{element-mustset.pdf}~maxDataPointRange & \hfuzz=500pt double & \hfuzz=500pt [seconds] all degree+1 data points must be within this range for a valid polynomial\\
\hfuzz=500pt\includegraphics[width=1em]{element.pdf}~maxExtrapolationDistance & \hfuzz=500pt double & \hfuzz=500pt [seconds] resampling points within this range of the polynomial will be extrapolated\\
\hline
\end{tabularx}


\subsection{Least squares polynomial fit}
A polynomial of \config{polynomialDegree} is estimated using all data points within
\config{maxDataPointDistance} of the resampling point. This polynomial is then used
to predict the resampling point. A resampling point will be extrapolated if there are
only data points before/after as long as the closest one is within \config{maxExtrapolationDistance}.
The elements \config{maxDataPointDistance} and \config{maxExtrapolationDistance} are given
in the unit of seconds. If negative values are used, the unit is relative to the median input sampling.

\fig{!hb}{0.5}{instrumentResample_leastSquares}{fig:instrumentResample_leastSquares}{Example of least squares polynomial fit when resampling from 5 to 1 minute sampling}


\keepXColumns
\begin{tabularx}{\textwidth}{N T A}
\hline
Name & Type & Annotation\\
\hline
\hfuzz=500pt\includegraphics[width=1em]{element-mustset.pdf}~polynomialDegree & \hfuzz=500pt uint & \hfuzz=500pt degree of the estimated polynomial\\
\hfuzz=500pt\includegraphics[width=1em]{element-mustset.pdf}~maxDataPointDistance & \hfuzz=500pt double & \hfuzz=500pt [seconds] all data points within this distance around the resampling point will be used\\
\hfuzz=500pt\includegraphics[width=1em]{element.pdf}~maxExtrapolationDistance & \hfuzz=500pt double & \hfuzz=500pt [seconds] resampling points within this range of the polynomial will be extrapolated\\
\hline
\end{tabularx}


\subsection{Fill gaps with least squares polynomial fit}


\keepXColumns
\begin{tabularx}{\textwidth}{N T A}
\hline
Name & Type & Annotation\\
\hline
\hfuzz=500pt\includegraphics[width=1em]{element.pdf}~polynomialDegree & \hfuzz=500pt uint & \hfuzz=500pt degree of the estimated polynomial\\
\hfuzz=500pt\includegraphics[width=1em]{element.pdf}~maxDataGap & \hfuzz=500pt double & \hfuzz=500pt [seconds] max data gap to interpolate\\
\hfuzz=500pt\includegraphics[width=1em]{element.pdf}~maxDataSpan & \hfuzz=500pt double & \hfuzz=500pt [seconds] time span on each side used for least squares fit\\
\hfuzz=500pt\includegraphics[width=1em]{element.pdf}~margin & \hfuzz=500pt double & \hfuzz=500pt [seconds] margin for identical times\\
\hline
\end{tabularx}

\clearpage
%==================================

\section{Kernel}\label{kernelType}
Kernel defines harmonic isotropic integral kernels $K$.
\begin{equation}
T(P) = \frac{1}{4\pi}\int_\Omega K(P,Q)\cdot f(Q)\,d\Omega(Q),
\end{equation}
where $T$ is the (disturbance)potential and $f$ is a functional on the spherical surface~$\Omega$.
The Kernel can be exapanded into a series of (fully normalized) legendre polynomials
\begin{equation}\label{eq.kernel}
K(\cos\psi,r,R) = \sum_n \left(\frac{R}{r}\right)^{n+1}
k_n\sqrt{2n+1}\bar{P}_n(\cos\psi).
\end{equation}
On the one hand the kernel defines the type of the functionals~$f$ that are measured
or have to be computed, e.g. gravity anomalies given by the Stokes-kernel.
On the other hand the kernel functions can be used as basis functions to represent
the gravity field, e.g. as spline functions or wavelets.


\subsection{GeoidHeight}\label{kernelType:geoidHeight}
The geoid height is defined by Bruns formula
\begin{equation}
N = \frac{1}{\gamma}T
\end{equation}
with $T$ the disturbance potential and the normal gravity
\begin{equation}\label{normalgravity}
\gamma  = \gamma_0 - 0.30877\cdot 10^{-5}/s^2(1-0.00142\sin^2(B))h
\end{equation}
and
\begin{equation}
\gamma_0 = 9.780327\,m/s^2(1+0.0053024\sin^2(B)-0.0000058\sin^2(2B))
\end{equation}
where $h$ is the ellipsoidal height in meter and $B$ the longitude.

The kernel is given by
\begin{equation}
K(\cos\psi,r,R) = \gamma\frac{R(r^2-R^2)}{l^3},
\end{equation}
and the coefficients in \eqref{eq.kernel} are
\begin{equation}
k_n = \gamma.
\end{equation}


\subsection{Anomalies}
Gravity anomalies in linearized form are defined by
\begin{equation}
\Delta g = -\frac{\partial T}{\partial r}-\frac{2}{r}T.
\end{equation}
The Stokes kernel is given by
\begin{equation}
K(\cos\psi,r,R) = \frac{2R^2}{l}-3\frac{Rl}{r^2}-\frac{R^2}{r^2}\cos\psi
\left(5+3\ln\frac{l+r-R\cos\psi}{2r}\right),
\end{equation}
and the coefficients in \eqref{eq.kernel} are
\begin{equation}
k_n = \frac{R}{n-1}.
\end{equation}


\subsection{Disturbance}\label{kernelType:disturbance}
Gravity disturbances in linearized form are defined by
\begin{equation}
\delta g = -\frac{dT}{dr}.
\end{equation}
The Hotine kernel is given by
\begin{equation}
K(\cos\psi,r,R) = \frac{2R^2}{l}-R\ln\frac{l+R-r\cos\psi}{r(1-\cos\psi)},
\end{equation}
and the coefficients in \eqref{eq.kernel} are
\begin{equation}
k_n = \frac{R}{n+1}.
\end{equation}


\subsection{Potential}
The Abel-Poisson kernel is given by
\begin{equation}
K(\cos\psi,r,R) = \frac{R(r^2-R^2)}{l^3},
\end{equation}
and the coefficients in \eqref{eq.kernel} are
\begin{equation}
k_n = 1.
\end{equation}


\subsection{Density}
This kernel defines a point mass or mass on a single layer ($1/l$-kernel)
taking the effect of the loading into account.

The coefficients of the kernel defined in \eqref{eq.kernel} are
\begin{equation}
k_n = 4\pi G R\frac{1+k_n'}{2n+1},
\end{equation}
where $G$ is the gravitational constant and $k_n'$ are the load Love numbers.


\keepXColumns
\begin{tabularx}{\textwidth}{N T A}
\hline
Name & Type & Annotation\\
\hline
\hfuzz=500pt\includegraphics[width=1em]{element.pdf}~inputfileLoadingLoveNumber & \hfuzz=500pt filename & \hfuzz=500pt \\
\hline
\end{tabularx}


\subsection{WaterHeight}\label{kernelType:waterHeight}
Height of equivalent water columns taking the effect of the loading into account.

The coefficients of the kernel defined in \eqref{eq.kernel} are
\begin{equation}
k_n = 4\pi G \rho R\frac{1+k_n'}{2n+1},
\end{equation}
where $G$ is the gravitational constant, $\rho$ is the \config{density} of water
and $k_n'$ are the load Love numbers.


\keepXColumns
\begin{tabularx}{\textwidth}{N T A}
\hline
Name & Type & Annotation\\
\hline
\hfuzz=500pt\includegraphics[width=1em]{element-mustset.pdf}~density & \hfuzz=500pt double & \hfuzz=500pt [kg/m**3]\\
\hfuzz=500pt\includegraphics[width=1em]{element.pdf}~inputfileLoadingLoveNumber & \hfuzz=500pt filename & \hfuzz=500pt \\
\hline
\end{tabularx}


\subsection{BottomPressure}
Ocean bottom pressure caused by water and atmosphere masses columns taking the effect of the loading into account.

The coefficients of the kernel defined in \eqref{eq.kernel} are
\begin{equation}
k_n = \frac{4\pi G R }{\gamma}\frac{1+k_n'}{2n+1},
\end{equation}
where $G$ is the gravitational constant, $\gamma$ is the normal gravity and $k_n'$ are the load Love numbers.


\keepXColumns
\begin{tabularx}{\textwidth}{N T A}
\hline
Name & Type & Annotation\\
\hline
\hfuzz=500pt\includegraphics[width=1em]{element.pdf}~inputfileLoadingLoveNumber & \hfuzz=500pt filename & \hfuzz=500pt \\
\hline
\end{tabularx}


\subsection{Deformation}
Computes the radial deformation caused by loading.

The coefficients of the kernel defined in \eqref{eq.kernel} are
\begin{equation}
k_n = \gamma\frac{1+k_n'}{h_n'},
\end{equation}
where $\gamma$ is the normal gravity defined in \eqref{normalgravity},
$h_n'$ and $k_n'$ are the load Love numbers and the load deformation Love numbers.


\keepXColumns
\begin{tabularx}{\textwidth}{N T A}
\hline
Name & Type & Annotation\\
\hline
\hfuzz=500pt\includegraphics[width=1em]{element-mustset.pdf}~inputfileDeformationLoadLoveNumber & \hfuzz=500pt filename & \hfuzz=500pt \\
\hfuzz=500pt\includegraphics[width=1em]{element.pdf}~inputfilePotentialLoadLoveNumber & \hfuzz=500pt filename & \hfuzz=500pt if full potential is given and not only loading potential\\
\hline
\end{tabularx}


\subsection{RadialGradient}
This kernel defines the second radial derivative of the (disturbance) potential.
\begin{equation}
T_{rr} = \frac{\partial^2 T}{\partial r^2}.
\end{equation}
The coefficients of the kernel defined in \eqref{eq.kernel} are
\begin{equation}
k_n = \frac{r^2}{(n+1)(n+2)}.
\end{equation}


\subsection{Coefficients}\label{kernelType:coefficients}
The kernel is defined by the coefficients $k_n$ given by file.


\keepXColumns
\begin{tabularx}{\textwidth}{N T A}
\hline
Name & Type & Annotation\\
\hline
\hfuzz=500pt\includegraphics[width=1em]{element-mustset.pdf}~inputfileCoefficients & \hfuzz=500pt filename & \hfuzz=500pt \\
\hline
\end{tabularx}


\subsection{FilterGauss}
Another \configClass{kernel}{kernelType} is smoothed by a gauss filter
which is defined by
\begin{equation}
F(\cos\psi) = \frac{b\cdot e^{-b(1-\cos\psi)}}{1-e^{-2b}}
\end{equation}
with $b = \frac{ln(2)}{1-\cos(r/R)}$ where $r$ is the given
smoothing \config{radius} in km and $R=6378.1366$~km is the
Earth radius.
The coefficients~$k_n$ of the \config{kernel} are multiplicated by
\begin{equation}
f_n = \frac{1}{2n+1} \int_{-1}^1 F(t)\cdot \bar{P}_n(t)\,dt.
\end{equation}


\keepXColumns
\begin{tabularx}{\textwidth}{N T A}
\hline
Name & Type & Annotation\\
\hline
\hfuzz=500pt\includegraphics[width=1em]{element-mustset.pdf}~kernel & \hfuzz=500pt \hyperref[kernelType]{kernel} & \hfuzz=500pt \\
\hfuzz=500pt\includegraphics[width=1em]{element-mustset.pdf}~radius & \hfuzz=500pt double & \hfuzz=500pt filter radius [km]\\
\hline
\end{tabularx}


\subsection{BlackmanLowpass}
Another \configClass{kernel}{kernelType} is smoothed by a Blackman low-pass filter. The filter is
defined through the beginning and end of the transition from pass-band to stop-band. This
transition band is specified by \config{startDegreeTransition} ($n_1$) and \config{stopDegreeTransition} ($n_2$).

The coefficients of this kernel are defined as
\begin{equation}
\begin{cases}
1 & \text{for } n < n_1 \\
A_n^2 & \text{for } n_1\leq n \leq n_2 \\
0 & \text{for } n > n_2 \\
\end{cases}
\end{equation}
with
\begin{equation}
A_n = 0.42 + 0.5\cos(\pi \frac{n-n_1}{n_2-n_1}) + 0.08 \cos(2\pi\frac{n-n_1}{n_2-n_1}).
\end{equation}


\keepXColumns
\begin{tabularx}{\textwidth}{N T A}
\hline
Name & Type & Annotation\\
\hline
\hfuzz=500pt\includegraphics[width=1em]{element-mustset.pdf}~kernel & \hfuzz=500pt \hyperref[kernelType]{kernel} & \hfuzz=500pt \\
\hfuzz=500pt\includegraphics[width=1em]{element-mustset.pdf}~startDegreeTransition & \hfuzz=500pt uint & \hfuzz=500pt minimum degree in transition band\\
\hfuzz=500pt\includegraphics[width=1em]{element-mustset.pdf}~stopDegreeTransition & \hfuzz=500pt uint & \hfuzz=500pt maximum degree in transition band\\
\hline
\end{tabularx}


\subsection{Truncation}
Another \configClass{kernel}{kernelType} is truncated before \config{minDegree} and after \config{maxDegree}.
The coefficients of this kernel are defined as
\begin{equation}
  k_n =
  \begin{cases}
  1 & \text{for } n_{\text{minDegree}} \leq n \leq n_{\text{maxDegree}}\\
  0 & \text{else.} \\
  \end{cases}
\end{equation}


\keepXColumns
\begin{tabularx}{\textwidth}{N T A}
\hline
Name & Type & Annotation\\
\hline
\hfuzz=500pt\includegraphics[width=1em]{element-mustset.pdf}~kernel & \hfuzz=500pt \hyperref[kernelType]{kernel} & \hfuzz=500pt \\
\hfuzz=500pt\includegraphics[width=1em]{element.pdf}~minDegree & \hfuzz=500pt uint & \hfuzz=500pt truncate before minDegree\\
\hfuzz=500pt\includegraphics[width=1em]{element-mustset.pdf}~maxDegree & \hfuzz=500pt uint & \hfuzz=500pt truncate after maxDegree\\
\hline
\end{tabularx}


\subsection{SelenoidHeight}
The selenoid height is defined by Bruns formula
\begin{equation}
N = \frac{1}{\gamma}T
\end{equation}
with $T$ the disturbance potential and the normal gravity $\gamma=\frac{GM}{R^2}$ of the moon.

The kernel is given by
\begin{equation}
K(\cos\psi,r,R) = \gamma\frac{R(r^2-R^2)}{l^3},
\end{equation}
and the coefficients in \eqref{eq.kernel} are
\begin{equation}
k_n = \gamma.
\end{equation}

\clearpage
%==================================

\section{Loop}\label{loopType}
Generates a sequence with variables to loop over.
The variable names can be set with \config{variableLoop...} and
the current values are assigned to the variables for each loop step.
See \reference{Loop and conditions}{general.loopsAndConditions} for usage.


\subsection{TimeSeries}
Loop over points in time.


\keepXColumns
\begin{tabularx}{\textwidth}{N T A}
\hline
Name & Type & Annotation\\
\hline
\hfuzz=500pt\includegraphics[width=1em]{element-mustset-unbounded.pdf}~timeSeries & \hfuzz=500pt \hyperref[timeSeriesType]{timeSeries} & \hfuzz=500pt loop is called for every point in time\\
\hfuzz=500pt\includegraphics[width=1em]{element.pdf}~variableLoopTime & \hfuzz=500pt string & \hfuzz=500pt variable with time of each loop\\
\hfuzz=500pt\includegraphics[width=1em]{element.pdf}~variableLoopIndex & \hfuzz=500pt string & \hfuzz=500pt variable with index of current iteration (starts with zero)\\
\hfuzz=500pt\includegraphics[width=1em]{element.pdf}~variableLoopCount & \hfuzz=500pt string & \hfuzz=500pt variable with total number of iterations\\
\hline
\end{tabularx}


\subsection{TimeIntervals}
Loop over the intervals between points in time.


\keepXColumns
\begin{tabularx}{\textwidth}{N T A}
\hline
Name & Type & Annotation\\
\hline
\hfuzz=500pt\includegraphics[width=1em]{element-mustset-unbounded.pdf}~timeIntervals & \hfuzz=500pt \hyperref[timeSeriesType]{timeSeries} & \hfuzz=500pt loop is called for every interval\\
\hfuzz=500pt\includegraphics[width=1em]{element.pdf}~variableLoopTimeStart & \hfuzz=500pt string & \hfuzz=500pt variable with starting time of each interval\\
\hfuzz=500pt\includegraphics[width=1em]{element.pdf}~variableLoopTimeEnd & \hfuzz=500pt string & \hfuzz=500pt variable with ending time of each interval\\
\hfuzz=500pt\includegraphics[width=1em]{element.pdf}~variableLoopIndex & \hfuzz=500pt string & \hfuzz=500pt variable with index of current iteration (starts with zero)\\
\hfuzz=500pt\includegraphics[width=1em]{element.pdf}~variableLoopCount & \hfuzz=500pt string & \hfuzz=500pt variable with total number of iterations\\
\hline
\end{tabularx}


\subsection{ManualList}\label{loopType:manualList}
Loop over list of strings.


\keepXColumns
\begin{tabularx}{\textwidth}{N T A}
\hline
Name & Type & Annotation\\
\hline
\hfuzz=500pt\includegraphics[width=1em]{element-mustset-unbounded.pdf}~string & \hfuzz=500pt string & \hfuzz=500pt explicit list of strings\\
\hfuzz=500pt\includegraphics[width=1em]{element.pdf}~variableLoopString & \hfuzz=500pt string & \hfuzz=500pt name of the variable to be replaced\\
\hfuzz=500pt\includegraphics[width=1em]{element.pdf}~variableLoopIndex & \hfuzz=500pt string & \hfuzz=500pt variable with index of current iteration (starts with zero)\\
\hfuzz=500pt\includegraphics[width=1em]{element.pdf}~variableLoopCount & \hfuzz=500pt string & \hfuzz=500pt variable with total number of iterations\\
\hline
\end{tabularx}


\subsection{ManualTable}
Loop over rows of a table containing strings.
The table can be defined \config{rowWise} or \config{columnWise}.
Each row/column must have the same number of cells.


\keepXColumns
\begin{tabularx}{\textwidth}{N T A}
\hline
Name & Type & Annotation\\
\hline
\hfuzz=500pt\includegraphics[width=1em]{element-mustset.pdf}~table & \hfuzz=500pt choice & \hfuzz=500pt define table by rows/columns\\
\hfuzz=500pt\includegraphics[width=1em]{connector.pdf}\includegraphics[width=1em]{element-mustset.pdf}~rowWise & \hfuzz=500pt sequence & \hfuzz=500pt define table by rows\\
\hfuzz=500pt\quad\includegraphics[width=1em]{connector.pdf}\includegraphics[width=1em]{element-mustset-unbounded.pdf}~row & \hfuzz=500pt sequence & \hfuzz=500pt define table by rows\\
\hfuzz=500pt\quad\quad\includegraphics[width=1em]{connector.pdf}\includegraphics[width=1em]{element-mustset-unbounded.pdf}~cell & \hfuzz=500pt string & \hfuzz=500pt explicit list of cells in row/column\\
\hfuzz=500pt\includegraphics[width=1em]{connector.pdf}\includegraphics[width=1em]{element-mustset.pdf}~columnWise & \hfuzz=500pt sequence & \hfuzz=500pt define table by columns\\
\hfuzz=500pt\quad\includegraphics[width=1em]{connector.pdf}\includegraphics[width=1em]{element-mustset-unbounded.pdf}~column & \hfuzz=500pt sequence & \hfuzz=500pt define table by columns\\
\hfuzz=500pt\quad\quad\includegraphics[width=1em]{connector.pdf}\includegraphics[width=1em]{element-mustset-unbounded.pdf}~cell & \hfuzz=500pt string & \hfuzz=500pt explicit list of cells in row/column\\
\hfuzz=500pt\includegraphics[width=1em]{element-mustset-unbounded.pdf}~variableLoopString & \hfuzz=500pt string & \hfuzz=500pt 1. variable name for the 1. column, next variable name for the 2. column, ... \\
\hfuzz=500pt\includegraphics[width=1em]{element.pdf}~variableLoopIndex & \hfuzz=500pt string & \hfuzz=500pt variable with index of current iteration (starts with zero)\\
\hfuzz=500pt\includegraphics[width=1em]{element.pdf}~variableLoopCount & \hfuzz=500pt string & \hfuzz=500pt variable with total number of iterations\\
\hline
\end{tabularx}


\subsection{FileAscii}
Loop over list of strings from files.


\keepXColumns
\begin{tabularx}{\textwidth}{N T A}
\hline
Name & Type & Annotation\\
\hline
\hfuzz=500pt\includegraphics[width=1em]{element-mustset-unbounded.pdf}~inputfile & \hfuzz=500pt filename & \hfuzz=500pt simple ASCII file with strings (separated by whitespace)\\
\hfuzz=500pt\includegraphics[width=1em]{element.pdf}~sort & \hfuzz=500pt boolean & \hfuzz=500pt sort entries alphabetically (ascending)\\
\hfuzz=500pt\includegraphics[width=1em]{element.pdf}~removeDuplicates & \hfuzz=500pt boolean & \hfuzz=500pt remove duplicate entries (order is preserved)\\
\hfuzz=500pt\includegraphics[width=1em]{element.pdf}~startIndex & \hfuzz=500pt uint & \hfuzz=500pt start at element startIndex (counting from 0)\\
\hfuzz=500pt\includegraphics[width=1em]{element.pdf}~count & \hfuzz=500pt uint & \hfuzz=500pt use count elements (default: use all)\\
\hfuzz=500pt\includegraphics[width=1em]{element.pdf}~variableLoopString & \hfuzz=500pt string & \hfuzz=500pt name of the variable to be replaced\\
\hfuzz=500pt\includegraphics[width=1em]{element.pdf}~variableLoopIndex & \hfuzz=500pt string & \hfuzz=500pt variable with index of current iteration (starts with zero)\\
\hfuzz=500pt\includegraphics[width=1em]{element.pdf}~variableLoopCount & \hfuzz=500pt string & \hfuzz=500pt variable with total number of iterations\\
\hline
\end{tabularx}


\subsection{FileAsciiTable}
Loop over rows of a table containing strings.
Each row must have the same number of columns.


\keepXColumns
\begin{tabularx}{\textwidth}{N T A}
\hline
Name & Type & Annotation\\
\hline
\hfuzz=500pt\includegraphics[width=1em]{element-mustset-unbounded.pdf}~inputfile & \hfuzz=500pt filename & \hfuzz=500pt simple ASCII file with multiple columns (separated by whitespace)\\
\hfuzz=500pt\includegraphics[width=1em]{element.pdf}~startLine & \hfuzz=500pt uint & \hfuzz=500pt start at line startLine (counting from 0)\\
\hfuzz=500pt\includegraphics[width=1em]{element.pdf}~countLines & \hfuzz=500pt uint & \hfuzz=500pt read count lines (default: all)\\
\hfuzz=500pt\includegraphics[width=1em]{element-mustset-unbounded.pdf}~variableLoopString & \hfuzz=500pt string & \hfuzz=500pt 1. variable name for the 1. column, next variable name for the 2. column, ... \\
\hfuzz=500pt\includegraphics[width=1em]{element.pdf}~variableLoopIndex & \hfuzz=500pt string & \hfuzz=500pt variable with index of current iteration (starts with zero)\\
\hfuzz=500pt\includegraphics[width=1em]{element.pdf}~variableLoopCount & \hfuzz=500pt string & \hfuzz=500pt variable with total number of iterations\\
\hline
\end{tabularx}


\subsection{Matrix}
Loop over rows of a matrix. To define the loop variables the standard
data variables of the matrix are available, see~\reference{dataVariables}{general.parser:dataVariables}.


\keepXColumns
\begin{tabularx}{\textwidth}{N T A}
\hline
Name & Type & Annotation\\
\hline
\hfuzz=500pt\includegraphics[width=1em]{element-mustset.pdf}~inputfile & \hfuzz=500pt filename & \hfuzz=500pt \\
\hfuzz=500pt\includegraphics[width=1em]{element.pdf}~transpose & \hfuzz=500pt boolean & \hfuzz=500pt effectively loop over columns\\
\hfuzz=500pt\includegraphics[width=1em]{element.pdf}~startRow & \hfuzz=500pt expression & \hfuzz=500pt start at this row (variable: rows)\\
\hfuzz=500pt\includegraphics[width=1em]{element.pdf}~countRows & \hfuzz=500pt expression & \hfuzz=500pt use this many rows (variable: rows)\\
\hfuzz=500pt\includegraphics[width=1em]{element-mustset-unbounded.pdf}~variableLoop & \hfuzz=500pt expression & \hfuzz=500pt define a variable by name = expression (input columns are named data0, data1, ...)\\
\hfuzz=500pt\includegraphics[width=1em]{element.pdf}~variableLoopIndex & \hfuzz=500pt string & \hfuzz=500pt variable with index of current iteration (starts with zero)\\
\hfuzz=500pt\includegraphics[width=1em]{element.pdf}~variableLoopCount & \hfuzz=500pt string & \hfuzz=500pt variable with total number of iterations\\
\hline
\end{tabularx}


\subsection{UniformSampling}
Loop over sequence of numbers.


\keepXColumns
\begin{tabularx}{\textwidth}{N T A}
\hline
Name & Type & Annotation\\
\hline
\hfuzz=500pt\includegraphics[width=1em]{element-mustset.pdf}~rangeStart & \hfuzz=500pt double & \hfuzz=500pt start of range\\
\hfuzz=500pt\includegraphics[width=1em]{element-mustset.pdf}~rangeEnd & \hfuzz=500pt double & \hfuzz=500pt end of range (inclusive)\\
\hfuzz=500pt\includegraphics[width=1em]{element-mustset.pdf}~sampling & \hfuzz=500pt double & \hfuzz=500pt sampling\\
\hfuzz=500pt\includegraphics[width=1em]{element.pdf}~variableLoopNumber & \hfuzz=500pt string & \hfuzz=500pt name of the variable to be replaced\\
\hfuzz=500pt\includegraphics[width=1em]{element.pdf}~variableLoopIndex & \hfuzz=500pt string & \hfuzz=500pt variable with index of current iteration (starts with zero)\\
\hfuzz=500pt\includegraphics[width=1em]{element.pdf}~variableLoopCount & \hfuzz=500pt string & \hfuzz=500pt variable with total number of iterations\\
\hline
\end{tabularx}


\subsection{CommandOutput}\label{loopType:commandOutput}
Loop over lines of command output.


\keepXColumns
\begin{tabularx}{\textwidth}{N T A}
\hline
Name & Type & Annotation\\
\hline
\hfuzz=500pt\includegraphics[width=1em]{element-mustset-unbounded.pdf}~command & \hfuzz=500pt filename & \hfuzz=500pt each output line becomes a loop iteration\\
\hfuzz=500pt\includegraphics[width=1em]{element.pdf}~silently & \hfuzz=500pt boolean & \hfuzz=500pt without showing the output.\\
\hfuzz=500pt\includegraphics[width=1em]{element.pdf}~variableLoopString & \hfuzz=500pt string & \hfuzz=500pt name of the variable to be replaced\\
\hfuzz=500pt\includegraphics[width=1em]{element.pdf}~variableLoopIndex & \hfuzz=500pt string & \hfuzz=500pt variable with index of current iteration (starts with zero)\\
\hfuzz=500pt\includegraphics[width=1em]{element.pdf}~variableLoopCount & \hfuzz=500pt string & \hfuzz=500pt variable with total number of iterations\\
\hline
\end{tabularx}


\subsection{Loop}
Loop over nested loops. First \config{loop} is outermost loop, every subsequent \config{loop} is one level below the previous \config{loop}.


\keepXColumns
\begin{tabularx}{\textwidth}{N T A}
\hline
Name & Type & Annotation\\
\hline
\hfuzz=500pt\includegraphics[width=1em]{element-mustset-unbounded.pdf}~loop & \hfuzz=500pt \hyperref[loopType]{loop} & \hfuzz=500pt subloop\\
\hfuzz=500pt\includegraphics[width=1em]{element.pdf}~variableLoopIndex & \hfuzz=500pt string & \hfuzz=500pt variable with index of current iteration (starts with zero)\\
\hline
\end{tabularx}


\subsection{PlatformEquipment}
Loop over specific equipment of a \file{platform file}{platform}.


\keepXColumns
\begin{tabularx}{\textwidth}{N T A}
\hline
Name & Type & Annotation\\
\hline
\hfuzz=500pt\includegraphics[width=1em]{element-mustset.pdf}~inputfilePlatform & \hfuzz=500pt filename & \hfuzz=500pt platform info file\\
\hfuzz=500pt\includegraphics[width=1em]{element-mustset.pdf}~equipmentType & \hfuzz=500pt choice & \hfuzz=500pt equipment type to loop over\\
\hfuzz=500pt\includegraphics[width=1em]{connector.pdf}\includegraphics[width=1em]{element-mustset.pdf}~all & \hfuzz=500pt  & \hfuzz=500pt loop over all types\\
\hfuzz=500pt\includegraphics[width=1em]{connector.pdf}\includegraphics[width=1em]{element-mustset.pdf}~gnssAntenna & \hfuzz=500pt  & \hfuzz=500pt loop over antennas\\
\hfuzz=500pt\includegraphics[width=1em]{connector.pdf}\includegraphics[width=1em]{element-mustset.pdf}~gnssReceiver & \hfuzz=500pt  & \hfuzz=500pt loop over receivers\\
\hfuzz=500pt\includegraphics[width=1em]{connector.pdf}\includegraphics[width=1em]{element-mustset.pdf}~other & \hfuzz=500pt  & \hfuzz=500pt loop over other types\\
\hfuzz=500pt\includegraphics[width=1em]{element.pdf}~variableLoopName & \hfuzz=500pt string & \hfuzz=500pt variable with name\\
\hfuzz=500pt\includegraphics[width=1em]{element.pdf}~variableLoopSerial & \hfuzz=500pt string & \hfuzz=500pt variable with serial\\
\hfuzz=500pt\includegraphics[width=1em]{element.pdf}~variableLoopInfo & \hfuzz=500pt string & \hfuzz=500pt variable with radome (antenna) or version (receiver)\\
\hfuzz=500pt\includegraphics[width=1em]{element.pdf}~variableLoopTimeStart & \hfuzz=500pt string & \hfuzz=500pt variable with start time\\
\hfuzz=500pt\includegraphics[width=1em]{element.pdf}~variableLoopTimeEnd & \hfuzz=500pt string & \hfuzz=500pt variable with end time\\
\hfuzz=500pt\includegraphics[width=1em]{element.pdf}~variableLoopPositionX & \hfuzz=500pt string & \hfuzz=500pt variable with position x\\
\hfuzz=500pt\includegraphics[width=1em]{element.pdf}~variableLoopPositionY & \hfuzz=500pt string & \hfuzz=500pt variable with position y\\
\hfuzz=500pt\includegraphics[width=1em]{element.pdf}~variableLoopPositionY & \hfuzz=500pt string & \hfuzz=500pt variable with position z\\
\hfuzz=500pt\includegraphics[width=1em]{element.pdf}~variableLoopIndex & \hfuzz=500pt string & \hfuzz=500pt variable with index of current iteration (starts with zero)\\
\hfuzz=500pt\includegraphics[width=1em]{element.pdf}~variableLoopCount & \hfuzz=500pt string & \hfuzz=500pt variable with total number of iterations\\
\hline
\end{tabularx}


\subsection{FileGnssStationInfo}
DEPRECATDED. Use LoopPlatformEquipment instead.


\keepXColumns
\begin{tabularx}{\textwidth}{N T A}
\hline
Name & Type & Annotation\\
\hline
\hfuzz=500pt\includegraphics[width=1em]{element-mustset.pdf}~inputfileGnssStationInfo & \hfuzz=500pt filename & \hfuzz=500pt station/transmitter info file\\
\hfuzz=500pt\includegraphics[width=1em]{element-mustset.pdf}~infoType & \hfuzz=500pt choice & \hfuzz=500pt info to loop over\\
\hfuzz=500pt\includegraphics[width=1em]{connector.pdf}\includegraphics[width=1em]{element-mustset.pdf}~antenna & \hfuzz=500pt  & \hfuzz=500pt loop over antennas\\
\hfuzz=500pt\includegraphics[width=1em]{connector.pdf}\includegraphics[width=1em]{element-mustset.pdf}~receiver & \hfuzz=500pt  & \hfuzz=500pt loop over receivers\\
\hfuzz=500pt\includegraphics[width=1em]{element.pdf}~variableLoopName & \hfuzz=500pt string & \hfuzz=500pt variable with antenna/receiver name\\
\hfuzz=500pt\includegraphics[width=1em]{element.pdf}~variableLoopSerial & \hfuzz=500pt string & \hfuzz=500pt variable with antenna/receiver serial\\
\hfuzz=500pt\includegraphics[width=1em]{element.pdf}~variableLoopInfo & \hfuzz=500pt string & \hfuzz=500pt variable with radome (antenna) or version (receiver)\\
\hfuzz=500pt\includegraphics[width=1em]{element.pdf}~variableLoopTimeStart & \hfuzz=500pt string & \hfuzz=500pt variable with antenna/receiver start time\\
\hfuzz=500pt\includegraphics[width=1em]{element.pdf}~variableLoopTimeEnd & \hfuzz=500pt string & \hfuzz=500pt variable with antenna/receiver end time\\
\hfuzz=500pt\includegraphics[width=1em]{element.pdf}~variableLoopIndex & \hfuzz=500pt string & \hfuzz=500pt variable with index of current iteration (starts with zero)\\
\hfuzz=500pt\includegraphics[width=1em]{element.pdf}~variableLoopCount & \hfuzz=500pt string & \hfuzz=500pt variable with total number of iterations\\
\hline
\end{tabularx}

\clearpage
%==================================

\section{Magnetosphere}\label{magnetosphereType}
This class provides functions of the magnetic field of the Earth.


\subsection{IGRF}
International Geomagnetic Reference Field.


\keepXColumns
\begin{tabularx}{\textwidth}{N T A}
\hline
Name & Type & Annotation\\
\hline
\hfuzz=500pt\includegraphics[width=1em]{element.pdf}~inputfileMagneticNorthPole & \hfuzz=500pt filename & \hfuzz=500pt time series of north pole\\
\hline
\end{tabularx}

\clearpage
%==================================

\section{MatrixGenerator}\label{matrixGeneratorType}
This class provides a matrix used e.g. by \program{MatrixCalculate}.
If multiple matrices are given the resulting matrix is the sum all
and the size is exandeded to fit all matrices. Before the computation of each submatrix
the variables \verb|rowsBefore| and \verb|columnsBefore| with current size of the overall matrix
are set. As all matrices can be manipulated before, complex matrix operations are possible.


\subsection{File}
Matrix from \file{file}{matrix}.


\keepXColumns
\begin{tabularx}{\textwidth}{N T A}
\hline
Name & Type & Annotation\\
\hline
\hfuzz=500pt\includegraphics[width=1em]{element-mustset.pdf}~inputfileMatrix & \hfuzz=500pt filename & \hfuzz=500pt \\
\hfuzz=500pt\includegraphics[width=1em]{element.pdf}~factor & \hfuzz=500pt double & \hfuzz=500pt \\
\hline
\end{tabularx}


\subsection{Normals file}
Matrix from a \file{normal equation file}{normalEquation}. The symmetric normal matrix,
the right hand side vector, the lPl vector, or the observation count $(1\times1)$ can be selected.


\keepXColumns
\begin{tabularx}{\textwidth}{N T A}
\hline
Name & Type & Annotation\\
\hline
\hfuzz=500pt\includegraphics[width=1em]{element-mustset.pdf}~inputfileNormalEquation & \hfuzz=500pt filename & \hfuzz=500pt \\
\hfuzz=500pt\includegraphics[width=1em]{element-mustset.pdf}~type & \hfuzz=500pt choice & \hfuzz=500pt \\
\hfuzz=500pt\includegraphics[width=1em]{connector.pdf}\includegraphics[width=1em]{element-mustset.pdf}~normalMatrix & \hfuzz=500pt  & \hfuzz=500pt \\
\hfuzz=500pt\includegraphics[width=1em]{connector.pdf}\includegraphics[width=1em]{element-mustset.pdf}~rightHandSide & \hfuzz=500pt  & \hfuzz=500pt \\
\hfuzz=500pt\includegraphics[width=1em]{connector.pdf}\includegraphics[width=1em]{element-mustset.pdf}~lPl & \hfuzz=500pt  & \hfuzz=500pt \\
\hfuzz=500pt\includegraphics[width=1em]{connector.pdf}\includegraphics[width=1em]{element-mustset.pdf}~observationCount & \hfuzz=500pt  & \hfuzz=500pt \\
\hfuzz=500pt\includegraphics[width=1em]{element.pdf}~factor & \hfuzz=500pt double & \hfuzz=500pt \\
\hline
\end{tabularx}


\subsection{Expression}
Matrix filled by an expression. For each element of the new matrix the variables
\verb|row| and \verb|column| are set and the expression \config{element} is evaluated.

Excample: The \config{element}=\verb|if(row==column,1,0)| generates an identity matrix.


\keepXColumns
\begin{tabularx}{\textwidth}{N T A}
\hline
Name & Type & Annotation\\
\hline
\hfuzz=500pt\includegraphics[width=1em]{element-mustset.pdf}~rows & \hfuzz=500pt expression & \hfuzz=500pt (variables: rowsBefore, columnsBefore)\\
\hfuzz=500pt\includegraphics[width=1em]{element-mustset.pdf}~columns & \hfuzz=500pt expression & \hfuzz=500pt (variables: rowsBefore, columnsBefore)\\
\hfuzz=500pt\includegraphics[width=1em]{element-mustset.pdf}~element & \hfuzz=500pt expression & \hfuzz=500pt for each element of matrix (variables: row, column, rows, columns, rowsBefore, columnsBefore)\\
\hline
\end{tabularx}


\subsection{Element manipulation}
The elements of a matrix are replaced an expression.
For each element of the matrix the variables \verb|data|, \verb|row|, \verb|column|
are set and the expression \config{element} is evaluated and replaces the element.
Additionally the standard data variables are available (assigned each row),
see~\reference{dataVariables}{general.parser:dataVariables}.


\keepXColumns
\begin{tabularx}{\textwidth}{N T A}
\hline
Name & Type & Annotation\\
\hline
\hfuzz=500pt\includegraphics[width=1em]{element-mustset-unbounded.pdf}~matrix & \hfuzz=500pt \hyperref[matrixGeneratorType]{matrixGenerator} & \hfuzz=500pt \\
\hfuzz=500pt\includegraphics[width=1em]{element-mustset.pdf}~element & \hfuzz=500pt expression & \hfuzz=500pt for each element of matrix (variables: data, row, column, rows, columns, rowsBefore, columnsBefore)\\
\hline
\end{tabularx}


\subsection{ElementWiseOperation}
Given two matrices $\mathbf{A}$ and $\mathbf{B}$ this class computes $c_{ij} = f(a_{ij}, b_{ij})$,
where $f$ is an expression (for example \verb|data0*data1|).
For each element of the matrix the variables \verb|data0|, \verb|data1|, \verb|row|, \verb|column|
are set and the expression \config{element} is evaluated.


\keepXColumns
\begin{tabularx}{\textwidth}{N T A}
\hline
Name & Type & Annotation\\
\hline
\hfuzz=500pt\includegraphics[width=1em]{element-mustset-unbounded.pdf}~matrix1 & \hfuzz=500pt \hyperref[matrixGeneratorType]{matrixGenerator} & \hfuzz=500pt \\
\hfuzz=500pt\includegraphics[width=1em]{element-mustset-unbounded.pdf}~matrix2 & \hfuzz=500pt \hyperref[matrixGeneratorType]{matrixGenerator} & \hfuzz=500pt \\
\hfuzz=500pt\includegraphics[width=1em]{element.pdf}~expression & \hfuzz=500pt expression & \hfuzz=500pt for each element of matrix (variables: data0, data1, row, column, rows, columns, rowsBefore, columnsBefore)\\
\hline
\end{tabularx}


\subsection{Append}
Append matrix to the right (first row) or bottom (first column).


\keepXColumns
\begin{tabularx}{\textwidth}{N T A}
\hline
Name & Type & Annotation\\
\hline
\hfuzz=500pt\includegraphics[width=1em]{element-mustset-unbounded.pdf}~matrix & \hfuzz=500pt \hyperref[matrixGeneratorType]{matrixGenerator} & \hfuzz=500pt \\
\hfuzz=500pt\includegraphics[width=1em]{element-mustset.pdf}~side & \hfuzz=500pt choice & \hfuzz=500pt \\
\hfuzz=500pt\includegraphics[width=1em]{connector.pdf}\includegraphics[width=1em]{element-mustset.pdf}~right & \hfuzz=500pt  & \hfuzz=500pt \\
\hfuzz=500pt\includegraphics[width=1em]{connector.pdf}\includegraphics[width=1em]{element-mustset.pdf}~bottom & \hfuzz=500pt  & \hfuzz=500pt \\
\hfuzz=500pt\includegraphics[width=1em]{connector.pdf}\includegraphics[width=1em]{element-mustset.pdf}~diagonal & \hfuzz=500pt  & \hfuzz=500pt \\
\hline
\end{tabularx}


\subsection{Shift}
Shift start row and start column of a matrix.
In other words: zero lines and columns are inserted at the beginning of the matrix.


\keepXColumns
\begin{tabularx}{\textwidth}{N T A}
\hline
Name & Type & Annotation\\
\hline
\hfuzz=500pt\includegraphics[width=1em]{element-mustset-unbounded.pdf}~matrix & \hfuzz=500pt \hyperref[matrixGeneratorType]{matrixGenerator} & \hfuzz=500pt \\
\hfuzz=500pt\includegraphics[width=1em]{element.pdf}~startRow & \hfuzz=500pt expression & \hfuzz=500pt start row (variables: rowsBefore, columnsBefore, rows, columns)\\
\hfuzz=500pt\includegraphics[width=1em]{element.pdf}~startColumn & \hfuzz=500pt expression & \hfuzz=500pt start column (variables: rowsBefore, columnsBefore, rows, columns)\\
\hline
\end{tabularx}


\subsection{Slice}
Slice of a matrix.


\keepXColumns
\begin{tabularx}{\textwidth}{N T A}
\hline
Name & Type & Annotation\\
\hline
\hfuzz=500pt\includegraphics[width=1em]{element-mustset-unbounded.pdf}~matrix & \hfuzz=500pt \hyperref[matrixGeneratorType]{matrixGenerator} & \hfuzz=500pt \\
\hfuzz=500pt\includegraphics[width=1em]{element.pdf}~startRow & \hfuzz=500pt expression & \hfuzz=500pt start row of matrix (variables: rowsBefore, columnsBefore, rows, columns)\\
\hfuzz=500pt\includegraphics[width=1em]{element.pdf}~startColumn & \hfuzz=500pt expression & \hfuzz=500pt start column of matrix (variables: rowsBefore, columnsBefore, rows, columns)\\
\hfuzz=500pt\includegraphics[width=1em]{element.pdf}~rows & \hfuzz=500pt expression & \hfuzz=500pt 0: until end (variables: rowsBefore, columnsBefore, rows, columns)\\
\hfuzz=500pt\includegraphics[width=1em]{element.pdf}~columns & \hfuzz=500pt expression & \hfuzz=500pt 0: until end (variables: rowsBefore, columnsBefore, rows, columns)\\
\hline
\end{tabularx}


\subsection{Reshape}
Matrix reshaped columnwise to new row and columns.


\keepXColumns
\begin{tabularx}{\textwidth}{N T A}
\hline
Name & Type & Annotation\\
\hline
\hfuzz=500pt\includegraphics[width=1em]{element-mustset-unbounded.pdf}~matrix & \hfuzz=500pt \hyperref[matrixGeneratorType]{matrixGenerator} & \hfuzz=500pt \\
\hfuzz=500pt\includegraphics[width=1em]{element-mustset.pdf}~rows & \hfuzz=500pt expression & \hfuzz=500pt 0: auto-determine rows, (variables: rowsBefore, columnsBefore)\\
\hfuzz=500pt\includegraphics[width=1em]{element-mustset.pdf}~columns & \hfuzz=500pt expression & \hfuzz=500pt 0: auto-determine columns (variables: rowsBefore, columnsBefore)\\
\hline
\end{tabularx}


\subsection{Reorder}\label{matrixGeneratorType:reorder}
Reorder rows or columns of a matrix by an index vectors.
The index vector can be created with \program{ParameterSelection2IndexVector}.


\keepXColumns
\begin{tabularx}{\textwidth}{N T A}
\hline
Name & Type & Annotation\\
\hline
\hfuzz=500pt\includegraphics[width=1em]{element-mustset-unbounded.pdf}~matrix & \hfuzz=500pt \hyperref[matrixGeneratorType]{matrixGenerator} & \hfuzz=500pt \\
\hfuzz=500pt\includegraphics[width=1em]{element.pdf}~inputfileIndexVectorRow & \hfuzz=500pt filename & \hfuzz=500pt index in input matrix or -1 for new parameter.\\
\hfuzz=500pt\includegraphics[width=1em]{element.pdf}~inputfileIndexVectorColumn & \hfuzz=500pt filename & \hfuzz=500pt index in input matrix or -1 for new parameter.\\
\hline
\end{tabularx}


\subsection{Sort}
Sort matrix by \config{column} in ascending order by default or in \config{descending} order.


\keepXColumns
\begin{tabularx}{\textwidth}{N T A}
\hline
Name & Type & Annotation\\
\hline
\hfuzz=500pt\includegraphics[width=1em]{element-mustset-unbounded.pdf}~matrix & \hfuzz=500pt \hyperref[matrixGeneratorType]{matrixGenerator} & \hfuzz=500pt \\
\hfuzz=500pt\includegraphics[width=1em]{element-mustset-unbounded.pdf}~column & \hfuzz=500pt uint & \hfuzz=500pt sort by column, top = highest priority\\
\hfuzz=500pt\includegraphics[width=1em]{element.pdf}~descending & \hfuzz=500pt boolean & \hfuzz=500pt \\
\hline
\end{tabularx}


\subsection{Transpose}
Transposed of a matrix $\M A^T$.


\keepXColumns
\begin{tabularx}{\textwidth}{N T A}
\hline
Name & Type & Annotation\\
\hline
\hfuzz=500pt\includegraphics[width=1em]{element-mustset-unbounded.pdf}~matrix & \hfuzz=500pt \hyperref[matrixGeneratorType]{matrixGenerator} & \hfuzz=500pt \\
\hline
\end{tabularx}


\subsection{Multiplication}
Multiplication of matrices.


\keepXColumns
\begin{tabularx}{\textwidth}{N T A}
\hline
Name & Type & Annotation\\
\hline
\hfuzz=500pt\includegraphics[width=1em]{element-mustset-unbounded.pdf}~matrix1 & \hfuzz=500pt \hyperref[matrixGeneratorType]{matrixGenerator} & \hfuzz=500pt \\
\hfuzz=500pt\includegraphics[width=1em]{element-mustset-unbounded.pdf}~matrix2 & \hfuzz=500pt \hyperref[matrixGeneratorType]{matrixGenerator} & \hfuzz=500pt \\
\hfuzz=500pt\includegraphics[width=1em]{element.pdf}~factor & \hfuzz=500pt double & \hfuzz=500pt \\
\hline
\end{tabularx}


\subsection{Inverse}
Inverse of a matrix $\M A^{-1}$.


\keepXColumns
\begin{tabularx}{\textwidth}{N T A}
\hline
Name & Type & Annotation\\
\hline
\hfuzz=500pt\includegraphics[width=1em]{element-mustset-unbounded.pdf}~matrix & \hfuzz=500pt \hyperref[matrixGeneratorType]{matrixGenerator} & \hfuzz=500pt \\
\hfuzz=500pt\includegraphics[width=1em]{element.pdf}~pseudoInverse & \hfuzz=500pt boolean & \hfuzz=500pt compute pseudo inverse instead of regular one\\
\hline
\end{tabularx}


\subsection{Cholesky}
Upper triangular natrix of the cholesky decomposition of a symmetric matrix $\M A=\M W^T\M W$.


\keepXColumns
\begin{tabularx}{\textwidth}{N T A}
\hline
Name & Type & Annotation\\
\hline
\hfuzz=500pt\includegraphics[width=1em]{element-mustset-unbounded.pdf}~matrix & \hfuzz=500pt \hyperref[matrixGeneratorType]{matrixGenerator} & \hfuzz=500pt \\
\hline
\end{tabularx}


\subsection{RankKUpdate}
Symmetric matrix from rank k update: $\M A^T\M A$.


\keepXColumns
\begin{tabularx}{\textwidth}{N T A}
\hline
Name & Type & Annotation\\
\hline
\hfuzz=500pt\includegraphics[width=1em]{element-mustset-unbounded.pdf}~matrix & \hfuzz=500pt \hyperref[matrixGeneratorType]{matrixGenerator} & \hfuzz=500pt \\
\hfuzz=500pt\includegraphics[width=1em]{element.pdf}~factor & \hfuzz=500pt double & \hfuzz=500pt \\
\hline
\end{tabularx}


\subsection{EigenValues}
Computes the eigenvalues of a square matrix and gives a vector of eigenvalues for symmetric matrices
or a matrix with 2 columns with real and imaginary parts in general case.


\keepXColumns
\begin{tabularx}{\textwidth}{N T A}
\hline
Name & Type & Annotation\\
\hline
\hfuzz=500pt\includegraphics[width=1em]{element-mustset-unbounded.pdf}~matrix & \hfuzz=500pt \hyperref[matrixGeneratorType]{matrixGenerator} & \hfuzz=500pt \\
\hfuzz=500pt\includegraphics[width=1em]{element.pdf}~eigenVectors & \hfuzz=500pt boolean & \hfuzz=500pt return eigen vectors instead of eigen values\\
\hline
\end{tabularx}


\subsection{Diagonal}
Extract the diagonal or subdiagnoal ($n\times 1$ vector) of a matrix.
The zero \config{diagonal} means the main diagonal, a positive value the superdiagonal,
and a negative the subdiagonal.


\keepXColumns
\begin{tabularx}{\textwidth}{N T A}
\hline
Name & Type & Annotation\\
\hline
\hfuzz=500pt\includegraphics[width=1em]{element-mustset-unbounded.pdf}~matrix & \hfuzz=500pt \hyperref[matrixGeneratorType]{matrixGenerator} & \hfuzz=500pt \\
\hfuzz=500pt\includegraphics[width=1em]{element.pdf}~diagonal & \hfuzz=500pt int & \hfuzz=500pt zero: main diagonal, positive: superdiagonal, negative: subdiagonal\\
\hline
\end{tabularx}


\subsection{FromDiagonal}
Generate a matrix from a diagonal vector.


\keepXColumns
\begin{tabularx}{\textwidth}{N T A}
\hline
Name & Type & Annotation\\
\hline
\hfuzz=500pt\includegraphics[width=1em]{element-mustset-unbounded.pdf}~matrix & \hfuzz=500pt \hyperref[matrixGeneratorType]{matrixGenerator} & \hfuzz=500pt (nx1) or (1xn) diagonal vector\\
\hfuzz=500pt\includegraphics[width=1em]{element.pdf}~diagonal & \hfuzz=500pt int & \hfuzz=500pt zero: main diagonal, positive: superdiagonal, negative: subdiagonal\\
\hline
\end{tabularx}


\subsection{Set type}
Set type (matrix, matrixSymmetricUpper, matrixSymmetricLower, matrixTriangularUpper, matrixTriangularLower)
of a matrix. If the type is not matrix, the matrix must be quadratic. Symmetric matrices are filled symmetric
and for triangular matrix the other triangle is set to zero.


\keepXColumns
\begin{tabularx}{\textwidth}{N T A}
\hline
Name & Type & Annotation\\
\hline
\hfuzz=500pt\includegraphics[width=1em]{element-mustset-unbounded.pdf}~matrix & \hfuzz=500pt \hyperref[matrixGeneratorType]{matrixGenerator} & \hfuzz=500pt \\
\hfuzz=500pt\includegraphics[width=1em]{element-mustset.pdf}~type & \hfuzz=500pt choice & \hfuzz=500pt \\
\hfuzz=500pt\includegraphics[width=1em]{connector.pdf}\includegraphics[width=1em]{element-mustset.pdf}~matrix & \hfuzz=500pt  & \hfuzz=500pt \\
\hfuzz=500pt\includegraphics[width=1em]{connector.pdf}\includegraphics[width=1em]{element-mustset.pdf}~matrixSymmetricUpper & \hfuzz=500pt  & \hfuzz=500pt \\
\hfuzz=500pt\includegraphics[width=1em]{connector.pdf}\includegraphics[width=1em]{element-mustset.pdf}~matrixSymmetricLower & \hfuzz=500pt  & \hfuzz=500pt \\
\hfuzz=500pt\includegraphics[width=1em]{connector.pdf}\includegraphics[width=1em]{element-mustset.pdf}~matrixTriangularUpper & \hfuzz=500pt  & \hfuzz=500pt \\
\hfuzz=500pt\includegraphics[width=1em]{connector.pdf}\includegraphics[width=1em]{element-mustset.pdf}~matrixTriangularLower & \hfuzz=500pt  & \hfuzz=500pt \\
\hline
\end{tabularx}

\clearpage
%==================================

\section{MiscAccelerations}\label{miscAccelerationsType}
This class gives the non conservative forces acting on satellites.


\subsection{Relativistic effect}\label{miscAccelerationsType:relativisticEffect}
The relativistic effect to the acceleration of an artificial Earth satellite
according to IERS2010 conventions.


\keepXColumns
\begin{tabularx}{\textwidth}{N T A}
\hline
Name & Type & Annotation\\
\hline
\hfuzz=500pt\includegraphics[width=1em]{element.pdf}~beta & \hfuzz=500pt double & \hfuzz=500pt PPN (parameterized post-Newtonian) parameter\\
\hfuzz=500pt\includegraphics[width=1em]{element.pdf}~gamma & \hfuzz=500pt double & \hfuzz=500pt PPN (parameterized post-Newtonian) parameter\\
\hfuzz=500pt\includegraphics[width=1em]{element.pdf}~J & \hfuzz=500pt double & \hfuzz=500pt Earth’s angular momentum per unit mass [m**2/s]\\
\hfuzz=500pt\includegraphics[width=1em]{element.pdf}~GM & \hfuzz=500pt double & \hfuzz=500pt Geocentric gravitational constant\\
\hfuzz=500pt\includegraphics[width=1em]{element.pdf}~factor & \hfuzz=500pt double & \hfuzz=500pt the result is multiplied by this factor\\
\hline
\end{tabularx}


\subsection{RadiationPressure}\label{miscAccelerationsType:RadiationPressure}
This class computes acceleration acting on a satellite caused by Solar and Earth radiation pressure
and thermal radiation.

Solar radiation pressure: The solar constant at 1~AU can be set via \config{solarFlux}.
The \config{factorSolarRadation} can be used to scale the computed acceleration of the direct solar radiation.

Earth radiation pressure:
Input are a time series of gridded albedo values (unitless) as \configFile{inputfileAlbedoTimeSeries}{griddedDataTimeSeries}
and a time series of gridded longwave flux (W/m$^2$) as \configFile{inputfileLongwaveFluxTimeSeries}{griddedDataTimeSeries}.
Both files are optional and if not specified, the respective effect on the acceleration is not computed.
The \config{factorEarthRadation} can be used to scale the computed acceleration of the earth radiation.

The thermal radiation (TRP) of the satellite itself is either computed as direct re-emission or
based on the actual temperature of the satellite surfaces, depending on the seetings of the
\file{satellite macro model}{satelliteModel}. The second one uses a transient temperature model
with a temporal differential equation which disallows parallel computing.
The \config{factorThermalRadiation} can be used to scale the computed acceleration of the TRP.

The algorithms are described in:

Woeske et. al. (2019), GRACE accelerometer calibration by high precision non-gravitational force modeling,
Advances in Space Research, \url{https://doi.org/10.1016/j.asr.2018.10.025}.


\keepXColumns
\begin{tabularx}{\textwidth}{N T A}
\hline
Name & Type & Annotation\\
\hline
\hfuzz=500pt\includegraphics[width=1em]{element.pdf}~solarflux & \hfuzz=500pt double & \hfuzz=500pt solar flux constant in 1 AU [W/m\textasciicircum{}2]\\
\hfuzz=500pt\includegraphics[width=1em]{element-mustset.pdf}~eclipse & \hfuzz=500pt \hyperref[eclipseType]{eclipse} & \hfuzz=500pt \\
\hfuzz=500pt\includegraphics[width=1em]{element.pdf}~inputfileAlbedoTimeSeries & \hfuzz=500pt filename & \hfuzz=500pt GriddedDataTimeSeries of albedo values (unitless)\\
\hfuzz=500pt\includegraphics[width=1em]{element.pdf}~inputfileLongwaveFluxTimeSeries & \hfuzz=500pt filename & \hfuzz=500pt GriddedDataTimeSeries of longwave flux values [W/m\textasciicircum{}2]\\
\hfuzz=500pt\includegraphics[width=1em]{element.pdf}~factorSolarRadation & \hfuzz=500pt double & \hfuzz=500pt Solar radiation pressure is multiplied by this factor\\
\hfuzz=500pt\includegraphics[width=1em]{element.pdf}~factorEarthRadation & \hfuzz=500pt double & \hfuzz=500pt Earth radiation preussure is multiplied by this factor\\
\hfuzz=500pt\includegraphics[width=1em]{element.pdf}~factorThermalRadiation & \hfuzz=500pt double & \hfuzz=500pt Thermal (re-)radiation is multiplied by this factor\\
\hline
\end{tabularx}


\subsection{AtmosphericDrag}\label{miscAccelerationsType:atmosphericDrag}
Atmospheric drag model.
Algorithm for the atmospheric drag modelling is based on the free molecule flow
theory by Sentman 1961. An analytical expression of this treatise is given in
Moe and Moe 2005.

Sentman L. (1961), Free molecule flow theory and its application to the determination
of aerodynamic forces, Technical report.

Moe K., Moe M. M. (2005), Gas-surface interactions and satellite drag coefficients,
Planetary and Space Science 53(8), 793-801, doi:10.1016/j.pss.2005.03.005.

Optional determination steps:
Turn temperature on or off.
In the first case, the model mentioned above is applied, which estimates variable drag
and lift coefficients - in the latter case a constant drag coefficient can be specified.

Turn wind on/off:
It enables the usage of the Horizontal Wind Model 2014 to add additional thermospheric
winds in the calculation process.


\keepXColumns
\begin{tabularx}{\textwidth}{N T A}
\hline
Name & Type & Annotation\\
\hline
\hfuzz=500pt\includegraphics[width=1em]{element-mustset.pdf}~thermosphere & \hfuzz=500pt \hyperref[thermosphereType]{thermosphere} & \hfuzz=500pt \\
\hfuzz=500pt\includegraphics[width=1em]{element.pdf}~earthRotation & \hfuzz=500pt double & \hfuzz=500pt [rad/s]\\
\hfuzz=500pt\includegraphics[width=1em]{element.pdf}~considerTemperature & \hfuzz=500pt boolean & \hfuzz=500pt compute drag and lift, otherwise simple drag coefficient is used\\
\hfuzz=500pt\includegraphics[width=1em]{element.pdf}~considerWind & \hfuzz=500pt boolean & \hfuzz=500pt \\
\hfuzz=500pt\includegraphics[width=1em]{element.pdf}~factor & \hfuzz=500pt double & \hfuzz=500pt the result is multiplied by this factor\\
\hline
\end{tabularx}


\subsection{AtmosphericDragFromDensityFile}\label{miscAccelerationsType:atmosphericDragFromDensityFile}
Atmospheric drag computed from thermospheric density along the orbit
(\configFile{inputfileDensity}{instrument}, MISCVALUE). The \configClass{thermosphere}{thermosphereType}
is used to to compute temperature and wind.
For further details see \configClass{atmosphericDrag}{miscAccelerationsType:atmosphericDrag}.


\keepXColumns
\begin{tabularx}{\textwidth}{N T A}
\hline
Name & Type & Annotation\\
\hline
\hfuzz=500pt\includegraphics[width=1em]{element-mustset.pdf}~inputfileDensity & \hfuzz=500pt filename & \hfuzz=500pt density along orbit, MISCVALUE (kg/m\textasciicircum{}3)\\
\hfuzz=500pt\includegraphics[width=1em]{element-mustset.pdf}~thermosphere & \hfuzz=500pt \hyperref[thermosphereType]{thermosphere} & \hfuzz=500pt used to compute temperature and wind\\
\hfuzz=500pt\includegraphics[width=1em]{element.pdf}~earthRotation & \hfuzz=500pt double & \hfuzz=500pt [rad/s]\\
\hfuzz=500pt\includegraphics[width=1em]{element.pdf}~considerTemperature & \hfuzz=500pt boolean & \hfuzz=500pt compute drag and lift, otherwise simple drag coefficient is used\\
\hfuzz=500pt\includegraphics[width=1em]{element.pdf}~considerWind & \hfuzz=500pt boolean & \hfuzz=500pt \\
\hfuzz=500pt\includegraphics[width=1em]{element.pdf}~factor & \hfuzz=500pt double & \hfuzz=500pt the result is multiplied by this factor\\
\hline
\end{tabularx}


\subsection{Antenna thrust}\label{miscAccelerationsType:antennaThrust}
The thrust (acceleration) in the opposite direction the antenna is facing
which is generated by satellite antenna broadcasts.
The thrust is defined in the satellite macro model.


\keepXColumns
\begin{tabularx}{\textwidth}{N T A}
\hline
Name & Type & Annotation\\
\hline
\hfuzz=500pt\includegraphics[width=1em]{element.pdf}~factor & \hfuzz=500pt double & \hfuzz=500pt the result is multiplied by this factor\\
\hline
\end{tabularx}


\subsection{FromParametrization}\label{miscAccelerationsType:fromParametrization}
Reads a solution vector from file \configFile{inputfileSolution}{matrix}
which may be computed by a least squares adjustment (e.g. by \program{NormalsSolverVCE}).
The coefficients of the vector are interpreted from position \config{indexStart}
(counting from zero) with help of \configClass{parametrization}{parametrizationAccelerationType}.
If the solution file contains solution of several right hand sides you can choose
one with number \config{rightSide} (counting from zero).

The computed result is multiplied with \config{factor}.


\keepXColumns
\begin{tabularx}{\textwidth}{N T A}
\hline
Name & Type & Annotation\\
\hline
\hfuzz=500pt\includegraphics[width=1em]{element-mustset-unbounded.pdf}~parametrization & \hfuzz=500pt \hyperref[parametrizationAccelerationType]{parametrizationAcceleration} & \hfuzz=500pt \\
\hfuzz=500pt\includegraphics[width=1em]{element-mustset.pdf}~inputfileSolution & \hfuzz=500pt filename & \hfuzz=500pt solution vector\\
\hfuzz=500pt\includegraphics[width=1em]{element.pdf}~indexStart & \hfuzz=500pt uint & \hfuzz=500pt position in the solution vector\\
\hfuzz=500pt\includegraphics[width=1em]{element.pdf}~rightSide & \hfuzz=500pt uint & \hfuzz=500pt if solution contains several right hand sides, select one\\
\hfuzz=500pt\includegraphics[width=1em]{element.pdf}~factor & \hfuzz=500pt double & \hfuzz=500pt the result is multiplied by this factor, set -1 to subtract the field\\
\hline
\end{tabularx}


\subsection{SolarRadiationPressure}\label{miscAccelerationsType:solarRadiationPressure}
DEPRECATED. Use radiationPressure instead.


\keepXColumns
\begin{tabularx}{\textwidth}{N T A}
\hline
Name & Type & Annotation\\
\hline
\hfuzz=500pt\includegraphics[width=1em]{element.pdf}~solarflux & \hfuzz=500pt double & \hfuzz=500pt solar flux constant in 1 AU [W/m**2]\\
\hfuzz=500pt\includegraphics[width=1em]{element-mustset.pdf}~eclipse & \hfuzz=500pt \hyperref[eclipseType]{eclipse} & \hfuzz=500pt \\
\hfuzz=500pt\includegraphics[width=1em]{element.pdf}~factor & \hfuzz=500pt double & \hfuzz=500pt the result is multiplied by this factor, set -1 to subtract the field\\
\hline
\end{tabularx}


\subsection{Albedo}\label{miscAccelerationsType:albedo}
DEPRECATED. Use radiationPressure instead.


\keepXColumns
\begin{tabularx}{\textwidth}{N T A}
\hline
Name & Type & Annotation\\
\hline
\hfuzz=500pt\includegraphics[width=1em]{element.pdf}~inputfileReflectivity & \hfuzz=500pt filename & \hfuzz=500pt \\
\hfuzz=500pt\includegraphics[width=1em]{element.pdf}~inputfileEmissivity & \hfuzz=500pt filename & \hfuzz=500pt \\
\hfuzz=500pt\includegraphics[width=1em]{element.pdf}~solarflux & \hfuzz=500pt double & \hfuzz=500pt solar flux constant in 1 AU [W/m**2]\\
\hfuzz=500pt\includegraphics[width=1em]{element.pdf}~factor & \hfuzz=500pt double & \hfuzz=500pt the result is multiplied by this factor, set -1 to subtract the field\\
\hline
\end{tabularx}

\clearpage
%==================================

\section{NoiseGenerator}\label{noiseGeneratorType}
This class implements the generation of different types of noise.
It provides a generic interface that can be implemented by different
types of generators. The characteristics of the generated noise
is determined by the generators. See the appropriate documentation
for more information.


\subsection{White}
The noise is Gaussian with a standard deviation \config{sigma}.
The noise is computed via a pseudo random sequence with a start value given
by \config{initRandom}. The same value always yields the same sequence.
Be careful in \reference{parallel}{general.parallelization} mode
as all nodes generates the same pseudo random sequence.
If this value is set to zero a real random value is used as starting value.


\keepXColumns
\begin{tabularx}{\textwidth}{N T A}
\hline
Name & Type & Annotation\\
\hline
\hfuzz=500pt\includegraphics[width=1em]{element-mustset.pdf}~sigma & \hfuzz=500pt double & \hfuzz=500pt standard deviation\\
\hfuzz=500pt\includegraphics[width=1em]{element.pdf}~initRandom & \hfuzz=500pt uint & \hfuzz=500pt start value for pseudo random sequence, 0: real random\\
\hline
\end{tabularx}


\subsection{ExpressionPSD}
This generator creates noise defined by a one sided PSD.
The \config{psd} is an expression controlled by the variable 'freq'.
To determine the frequency \config{sampling} must be given.


\keepXColumns
\begin{tabularx}{\textwidth}{N T A}
\hline
Name & Type & Annotation\\
\hline
\hfuzz=500pt\includegraphics[width=1em]{element-mustset-unbounded.pdf}~noise & \hfuzz=500pt \hyperref[noiseGeneratorType]{noiseGenerator} & \hfuzz=500pt Basis noise\\
\hfuzz=500pt\includegraphics[width=1em]{element-mustset.pdf}~psd & \hfuzz=500pt expression & \hfuzz=500pt one sided PSD (variable: freq [Hz]) [unit\textasciicircum{}2/Hz]\\
\hfuzz=500pt\includegraphics[width=1em]{element-mustset.pdf}~sampling & \hfuzz=500pt double & \hfuzz=500pt to determine frequency [seconds]\\
\hline
\end{tabularx}


\subsection{Filter}
Generated noise \configClass{noise}{noiseGeneratorType} is
filtered by a \configClass{filter}{digitalFilterType}.


\keepXColumns
\begin{tabularx}{\textwidth}{N T A}
\hline
Name & Type & Annotation\\
\hline
\hfuzz=500pt\includegraphics[width=1em]{element-mustset-unbounded.pdf}~filter & \hfuzz=500pt \hyperref[digitalFilterType]{digitalFilter} & \hfuzz=500pt digital filter\\
\hfuzz=500pt\includegraphics[width=1em]{element-mustset-unbounded.pdf}~noise & \hfuzz=500pt \hyperref[noiseGeneratorType]{noiseGenerator} & \hfuzz=500pt Basis noise\\
\hfuzz=500pt\includegraphics[width=1em]{element.pdf}~warmupEpochCount & \hfuzz=500pt uint & \hfuzz=500pt number of additional epochs at before start and after end\\
\hfuzz=500pt\includegraphics[width=1em]{element.pdf}~overSamplingFactor & \hfuzz=500pt uint & \hfuzz=500pt noise with multiple higher sampling -\$>\$ filter -\$>\$ decimate\\
\hline
\end{tabularx}


\subsection{PowerLaw}
This generator creates noise that conforms to a power law relationship, where the power
of the noise at a frequency is proportional to $1/f^\alpha$, with a typically between -2 and 2.


\keepXColumns
\begin{tabularx}{\textwidth}{N T A}
\hline
Name & Type & Annotation\\
\hline
\hfuzz=500pt\includegraphics[width=1em]{element-mustset-unbounded.pdf}~noise & \hfuzz=500pt \hyperref[noiseGeneratorType]{noiseGenerator} & \hfuzz=500pt Basis noise\\
\hfuzz=500pt\includegraphics[width=1em]{element-mustset.pdf}~alpha & \hfuzz=500pt double & \hfuzz=500pt Exponent of the power law relationship 1/f\textasciicircum{}alpha\\
\hline
\end{tabularx}

\clearpage
%==================================

\section{NormalEquation}\label{normalEquationType}
This class provides a system of normal equations.
This total system is the weighted sum of individual normals.
\begin{equation}
 \M N_{total} =  \sum_{k=1} \frac{1}{\sigma_k^2}\M N_k
 \qquad\text{and}\qquad
\M n_{total} = \sum_{k=1} \frac{1}{\sigma_k^2} \M n_k.
\end{equation}
The normals do not need to have the same dimension. The dimension
of the total combined system is chosen to cover all individual systems.
For each normal a \config{startIndex} is required which indicates
the position of the first unknown of the individual normal within the
combined parameter vector.

The $\sigma_k$ of the relative weights are defined by \config{aprioriSigma}
in a first step. If an apriori solution \configFile{inputfileApproxSolution}{matrix} is
given or the normals are solved iteratively the weights are determined by means
of variance compoment estimation (VCE), see \program{NormalsSolverVCE}:
\begin{equation}
\sigma_k^2 =
\frac{\M e_k^T\M P\M e_k}
{n_k-\frac{1}{\sigma_k^2}\text{trace}\left(\M N_k\M N_{total}^{-1}\right)},
\end{equation}
where $n_k$ is the number of observations. The square sum of the residuals
is calculated by
\begin{equation}
\M e_k^T\M P\M e_k = \M x^T\M N_k\M x - 2\M n_k^T\M x + \M l_k^T\M P_k\M l_k.
\end{equation}
The system of normal equations can be solved with several right hand sides at once. But
only one right hand side, which can be selected with the index \config{rightHandSide}
(counting from zero), can be used to compute the variance factors.
The combined normal $\M N_{total}$ and the solution $\M x$ are taken from the previous
iteration step. In case of \configClass{DesignVCE}{normalEquationType:designVCE} the algorithm
is a little bit different as described below.


\subsection{Design}\label{normalEquationType:design}
This class acculumates normal equations from observation equations.
The class \configClass{observation}{observationType} computes
the linearized and decorrelated equation system for each arc $i$:
\begin{equation}
\M l_i  = \M A_i \M x + \M B_i \M y_i + \M e_i.
\end{equation}
The arc depending parameters~$\M y_i$ are eliminated and the system of normal
equations is acculumated according to
\begin{equation}
 \M N = \sum_{i=1}^m \M A_i^T  \M A_i
 \qquad\text{and}\qquad
\M n = \sum_{i=1}^m \M A_i^T \M l_i.
\end{equation}


\keepXColumns
\begin{tabularx}{\textwidth}{N T A}
\hline
Name & Type & Annotation\\
\hline
\hfuzz=500pt\includegraphics[width=1em]{element-mustset.pdf}~observation & \hfuzz=500pt \hyperref[observationType]{observation} & \hfuzz=500pt \\
\hfuzz=500pt\includegraphics[width=1em]{element.pdf}~aprioriSigma & \hfuzz=500pt double & \hfuzz=500pt \\
\hfuzz=500pt\includegraphics[width=1em]{element.pdf}~startIndex & \hfuzz=500pt uint & \hfuzz=500pt add this normals at index of total matrix (counting from 0)\\
\hfuzz=500pt\includegraphics[width=1em]{element.pdf}~inputfileArcList & \hfuzz=500pt filename & \hfuzz=500pt to accelerate computation\\
\hline
\end{tabularx}


\subsection{DesignVCE}\label{normalEquationType:designVCE}
This class acculumates normal equations from observation equations.
The class \configClass{observation}{observationType} computes
the linearized and decorrelated equation system for each arc $i$:
\begin{equation}
\M l_i  = \M A_i \M x + \M B_i \M y_i + \M e_i.
\end{equation}
The arc depending parameters~$\M y_i$ are eliminated and the system of normal
equations is acculumated according to
\begin{equation}
 \M N =  \sum_{i=1} \frac{1}{\sigma_i^2}\M A_i^T  \M A_i
 \qquad\text{and}\qquad
\M n = \sum_{i=1} \frac{1}{\sigma_i^2} \M A_i^T \M l_i.
\end{equation}
The variance $\sigma_i^2$ of each individual arc is determined by
\begin{equation}
\sigma_i^2 =
\frac{(\M l_i-\M A_i\M x)^T(\M l_i-\M A_i\M x)}
{n_i-\frac{1}{\sigma_i^2}\text{trace}\left(\M A_i^T  \M A_i\M N_{total}^{-1}\right)},
\end{equation}
where $n_i$ is the number of observations. If an apriori solution is not given at the first
iteration step a zero vector is assumed.


\keepXColumns
\begin{tabularx}{\textwidth}{N T A}
\hline
Name & Type & Annotation\\
\hline
\hfuzz=500pt\includegraphics[width=1em]{element-mustset.pdf}~observation & \hfuzz=500pt \hyperref[observationType]{observation} & \hfuzz=500pt \\
\hfuzz=500pt\includegraphics[width=1em]{element.pdf}~startIndex & \hfuzz=500pt uint & \hfuzz=500pt add this normals at index of total matrix (counting from 0)\\
\hfuzz=500pt\includegraphics[width=1em]{element.pdf}~inputfileArcList & \hfuzz=500pt filename & \hfuzz=500pt to accelerate computation\\
\hline
\end{tabularx}


\subsection{File}\label{normalEquationType:file}
Reads a system of normal equations from file \configFile{inputfileNormalEquation}{normalEquation}
as generated by e.g. \program{NormalsBuild}.


\keepXColumns
\begin{tabularx}{\textwidth}{N T A}
\hline
Name & Type & Annotation\\
\hline
\hfuzz=500pt\includegraphics[width=1em]{element-mustset.pdf}~inputfileNormalEquation & \hfuzz=500pt filename & \hfuzz=500pt \\
\hfuzz=500pt\includegraphics[width=1em]{element.pdf}~aprioriSigma & \hfuzz=500pt double & \hfuzz=500pt \\
\hfuzz=500pt\includegraphics[width=1em]{element.pdf}~startIndex & \hfuzz=500pt uint & \hfuzz=500pt add this normals at index of total matrix (counting from 0)\\
\hline
\end{tabularx}


\subsection{Regularization}\label{normalEquationType:regularization}
Set up a system of normal equations
\begin{equation}
\M N = \M R
\qquad\text{and}\qquad
\M n = \M R \M b,
\end{equation}
where $\M R$ is a diagonal matrix whose elements are given as a vector by
\configFile{inputfileDiagonalMatrix}{matrix} and $\M b$ is the right hand side towards which will
be regularized. It can be given by \configFile{inputfileBiasVector}{matrix}.
The diagonal matrix can be generated with \program{NormalsRegularizationBorders},
\program{NormalsRegularizationSphericalHarmonics}, or \program{MatrixCalculate}.
If $\M R$ is not given a unit matrix is assumed.
The right hand side $\M b$ may be generated with \program{Gravityfield2SphericalHarmonicsVector}.
If $\M b$ is not given a zero vector is assumed.


\keepXColumns
\begin{tabularx}{\textwidth}{N T A}
\hline
Name & Type & Annotation\\
\hline
\hfuzz=500pt\includegraphics[width=1em]{element.pdf}~inputfileDiagonalMatrix & \hfuzz=500pt filename & \hfuzz=500pt Vector with the diagonal elements of the weight matrix\\
\hfuzz=500pt\includegraphics[width=1em]{element.pdf}~inputfileBias & \hfuzz=500pt filename & \hfuzz=500pt Matrix with right hand sides\\
\hfuzz=500pt\includegraphics[width=1em]{element.pdf}~aprioriSigma & \hfuzz=500pt double & \hfuzz=500pt \\
\hfuzz=500pt\includegraphics[width=1em]{element.pdf}~startIndex & \hfuzz=500pt uint & \hfuzz=500pt regularization of parameters starts at this index (counting from 0)\\
\hline
\end{tabularx}


\subsection{RegularizationGeneralized}

Generalized regularization which is represented by the observation equation
\begin{equation}
\mathbf{x}_0 = \mathbf{I} \mathbf{x} + \mathbf{v}, \mathbf{v} \sim \mathcal{N}(0, \sum_k \sigma^2_k \mathbf{V}_k).
\end{equation}

There are no requirements for partial covariance matrices $\mathbf{V}_k$ except for them being symmetric.
The accumulated covariance matrix $\sum_k \sigma^2_k \mathbf{V}_k$ must be positive definite however.
The variance components $\sigma^2_k$ are estimated during the adjustment process and are assumed to be positive.
All \configFile{inputfilePartialCovarianceMatrix}{matrix} must be of same size
and must match the dimension of \configFile{inputfileBiasMatrix}{matrix}
(if provided, otherwise a zero vector of appropriate dimensions is created).

The parameter \config{aprioriSigma} determines the initial variance factor for the partial covariance matrices. Either one $\sigma_0$ can be
supplied or one for each $\mathbf{V}_k$.

The regularization matrix can be applied to a subset of parameters by adjusting \config{startIndex}.


\keepXColumns
\begin{tabularx}{\textwidth}{N T A}
\hline
Name & Type & Annotation\\
\hline
\hfuzz=500pt\includegraphics[width=1em]{element-mustset-unbounded.pdf}~inputfilePartialCovarianceMatrix & \hfuzz=500pt filename & \hfuzz=500pt symmetric matrix (sum of all matrices must be positive definite)\\
\hfuzz=500pt\includegraphics[width=1em]{element.pdf}~inputfileBiasMatrix & \hfuzz=500pt filename & \hfuzz=500pt bias vector (default: zero vector)\\
\hfuzz=500pt\includegraphics[width=1em]{element-mustset-unbounded.pdf}~aprioriSigma & \hfuzz=500pt double & \hfuzz=500pt apriori sigmas for initial iteration (default: 1.0)\\
\hfuzz=500pt\includegraphics[width=1em]{element.pdf}~startIndex & \hfuzz=500pt uint & \hfuzz=500pt regularization of parameters starts at this index (counting from 0)\\
\hline
\end{tabularx}

\clearpage
%==================================

\section{Observation}\label{observationType}
This class set up the oberservation equations in linearized Gauss-Makoff model
\begin{equation}\label{gmm}
\M l  = \M A \M x + \M e\qquad\text{and}\qquad\mathcal{C}(\M e) = \sigma^2\M P^{-1}.
\end{equation}
The observations are divided into short data blocks which can computed independently
and so easily can be parallelized. Usually this data blocks are short arcs of a
satellites orbit. In most cases the unknown parameter vector contains coefficients
of a gravity field parametrization given by \configClass{parametrizationGravity}{parametrizationGravityType}.
Additional parameters like instrument calibrations parameters are appended at the
end of the vector~$\M x$.
It is possible to give several observation vectors in one model.

The observations within each arc are decorrelated in the following way:
In a first step a cholesky decomposition of the covariance matrix is performed
\begin{equation}
\M P^{-1} = \M W^T\M W,
\end{equation}
where $\M W$ is an upper regular triangular matrix.
In a second step the transformation
\begin{equation}\label{dekorrelierung}
\bar{\M A} = \M W^{-T}\M A\qquad\text{and}\qquad \bar{\M l} = \M W^{-T}\M l
\end{equation}
gives an estimation from decorrelated observations with equal variance
\begin{equation}\label{normal.GMM}
\bar{\M l} = \bar{\M A} \M x + \bar{\M e}
\qquad\text{and}\qquad
\mathcal{C}(\bar{\M e})= \sigma^2 \M I.
\end{equation}
Usually the arc depending parameters are eliminated in the next step.


\subsection{PodVariational}\label{observationType:podVariational}
The observation equations for precise orbit data (POD) are formulated as variational equations.
It is based on \file{inputfileVariational}{variationalEquation} calculated with \program{PreprocessingVariationalEquation}.
Necessary integrations are performed by integrating a moving interpolation polynomial of degree \config{integrationDegree}.

The kinematic positions as pseudo observations are taken from
\config{rightHandSide} and should not given equally spaced in time. The observation
equations are interpolated to these times by a moving polynomial of degree \config{interpolationDegree}.

The accuracy or the full covariance matrix of the precise orbit data is provided in
\configClass{covariancePod}{covariancePodType} and can be estimated with \program{PreprocessingPod}.

\config{accelerateComputation}: In the event that the sampling of the kinematic orbit is much higher than the sampling
of the variational equations (e.g. 1 second vs. 5 seconds) the accumulation of the observation equations
can be accelerated by transforming the observation equations
\begin{equation}
  \M l = \M J \M A \M x + \M e,
\end{equation}
where $\M J$ describes the interpolation of the sampling of the variational design matrix~$\M A$
to the sampling of the observations $\M l$ with more rows than columns. The QR decomposition
\begin{equation}
  \M J = \begin{pmatrix} \M Q_1 & \M Q_2 \end{pmatrix}
         \begin{pmatrix} \M R \\ \M 0 \end{pmatrix}.
\end{equation}
can be used to transform the observation equations
\begin{equation}
  \begin{pmatrix} \M Q_1^T \M l \\ \M Q_2^T \M l \end{pmatrix} =
  \begin{pmatrix} \M Q_1^T \M R \\ \M 0 \end{pmatrix} \M A \M x +
  \begin{pmatrix} \M Q_1^T \M e \\ \M Q_2^T \M e \end{pmatrix}.
\end{equation}
As the zero lines should not be considered the computational time for the accumulation is reduced.
This option is not meaningful for evaluating the residuals such in \program{PreprocessingPod}.


\keepXColumns
\begin{tabularx}{\textwidth}{N T A}
\hline
Name & Type & Annotation\\
\hline
\hfuzz=500pt\includegraphics[width=1em]{element-mustset.pdf}~rightHandSide & \hfuzz=500pt sequence & \hfuzz=500pt input for observation vectors\\
\hfuzz=500pt\includegraphics[width=1em]{connector.pdf}\includegraphics[width=1em]{element-mustset.pdf}~inputfileOrbit & \hfuzz=500pt filename & \hfuzz=500pt kinematic positions as observations\\
\hfuzz=500pt\includegraphics[width=1em]{element-mustset.pdf}~inputfileVariational & \hfuzz=500pt filename & \hfuzz=500pt approximate position and integrated state matrix\\
\hfuzz=500pt\includegraphics[width=1em]{element.pdf}~ephemerides & \hfuzz=500pt \hyperref[ephemeridesType]{ephemerides} & \hfuzz=500pt \\
\hfuzz=500pt\includegraphics[width=1em]{element-unbounded.pdf}~parametrizationGravity & \hfuzz=500pt \hyperref[parametrizationGravityType]{parametrizationGravity} & \hfuzz=500pt gravity field parametrization\\
\hfuzz=500pt\includegraphics[width=1em]{element-unbounded.pdf}~parametrizationAcceleration & \hfuzz=500pt \hyperref[parametrizationAccelerationType]{parametrizationAcceleration} & \hfuzz=500pt orbit force parameters\\
\hfuzz=500pt\includegraphics[width=1em]{element.pdf}~integrationDegree & \hfuzz=500pt uint & \hfuzz=500pt integration of forces by polynomial approximation of degree n\\
\hfuzz=500pt\includegraphics[width=1em]{element.pdf}~interpolationDegree & \hfuzz=500pt uint & \hfuzz=500pt orbit interpolation by polynomial approximation of degree n\\
\hfuzz=500pt\includegraphics[width=1em]{element.pdf}~accelerateComputation & \hfuzz=500pt boolean & \hfuzz=500pt acceleration of computation by transforming the observations\\
\hfuzz=500pt\includegraphics[width=1em]{element.pdf}~covariancePod & \hfuzz=500pt \hyperref[covariancePodType]{covariancePod} & \hfuzz=500pt covariance matrix of kinematic orbits\\
\hline
\end{tabularx}


\subsection{PodIntegral}\label{observationType:podIntegral}
The observation equations for precise orbit data (POD) of short arcs are given by
\begin{equation}
  {\M r}_\epsilon(\tau) = {\M r}_A(1-\tau) + {\M r}_B\tau - T^2\int_0^1 K(\tau,\tau')
  \left(\M f_0(\tau')+\nabla V(\tau')\right)\,d\tau'
\end{equation}
with the integral kernel
\begin{equation}
  K(\tau,\tau') = \begin{cases} \tau'(1-\tau) & \text{for }\tau'\le\tau \\
  \tau(1-\tau') & \text{for }\tau'>\tau \end{cases},
\end{equation}
and the normalized time
\begin{equation}
  \tau = \frac{t-t_A}{T}\qquad\text{with}\qquad T=t_B-t_A.
\end{equation}
The kinematic positions~${\M r}_\epsilon(\tau)$ as pseudo observations are taken from
\configClass{rightHandSide}{podRightSideType}. From these positions the influence of the reference forces $\M f_0(\tau)$
is subtracted which are computed with the background models in \configClass{rightHandSide}{podRightSideType}.
The integral is solved by the integration of a moving interpolation polynomial of degree \config{integrationDegree}.
The boundary values ${\M r}_A$ and ${\M r}_B$ (satellite's state vector) are estimated per arc
and are usually directly eliminated if \config{keepSatelliteStates} is not set.

The unknown gravity field $\nabla V(\M r, t)$ parametrized by \configClass{parametrizationGravity}{parametrizationGravityType}
is not evaluated at the observed positions but at the orbit given by \configFile{inputfileOrbit}{instrument}.
The same is true for the reference forces. The linearized effect of the gravity field change by the position
adjustment is taken into account by \config{gradientfield}. This may be a low order field up to a
spherical harmonics degree of $n=2$ or $n=3$.

The \configFile{inputfileOrbit}{instrument}, \configFile{inputfileStarCamera}{instrument}, and \configFile{inputfileAccelerometer}{instrument}
must be synchronous and must be given with a constant sampling and without any gaps in each short arc
(see \program{InstrumentSynchronize}).
The kinematic positions~${\M r}_\epsilon(\tau)$ should not given equally spaced in time
but must be divided into the same arcs as the other instrument data.
The observation equations are interpolated to this time by a polynomial interpolation
with degree \config{interpolationDegree}.

The accuracy or the full covariance matrix of the precise orbit data is provided in
\configClass{covariancePod}{covariancePodType} and can be estimated with \program{PreprocessingPod}.

For \config{accelerateComputation} see \configClass{observation:podVariational}{observationType:podVariational}.


\keepXColumns
\begin{tabularx}{\textwidth}{N T A}
\hline
Name & Type & Annotation\\
\hline
\hfuzz=500pt\includegraphics[width=1em]{element.pdf}~inputfileSatelliteModel & \hfuzz=500pt filename & \hfuzz=500pt satellite macro model\\
\hfuzz=500pt\includegraphics[width=1em]{element-mustset-unbounded.pdf}~rightHandSide & \hfuzz=500pt \hyperref[podRightSideType]{podRightSide} & \hfuzz=500pt input for the reduced observation vector\\
\hfuzz=500pt\includegraphics[width=1em]{element-mustset.pdf}~inputfileOrbit & \hfuzz=500pt filename & \hfuzz=500pt used to evaluate the observation equations, not used as observations\\
\hfuzz=500pt\includegraphics[width=1em]{element-mustset.pdf}~inputfileStarCamera & \hfuzz=500pt filename & \hfuzz=500pt \\
\hfuzz=500pt\includegraphics[width=1em]{element-mustset.pdf}~earthRotation & \hfuzz=500pt \hyperref[earthRotationType]{earthRotation} & \hfuzz=500pt \\
\hfuzz=500pt\includegraphics[width=1em]{element.pdf}~ephemerides & \hfuzz=500pt \hyperref[ephemeridesType]{ephemerides} & \hfuzz=500pt \\
\hfuzz=500pt\includegraphics[width=1em]{element-unbounded.pdf}~gradientfield & \hfuzz=500pt \hyperref[gravityfieldType]{gravityfield} & \hfuzz=500pt low order field to estimate the change of the gravity by position adjustement\\
\hfuzz=500pt\includegraphics[width=1em]{element-unbounded.pdf}~parametrizationGravity & \hfuzz=500pt \hyperref[parametrizationGravityType]{parametrizationGravity} & \hfuzz=500pt gravity field parametrization\\
\hfuzz=500pt\includegraphics[width=1em]{element-unbounded.pdf}~parametrizationAcceleration & \hfuzz=500pt \hyperref[parametrizationAccelerationType]{parametrizationAcceleration} & \hfuzz=500pt orbit force parameters\\
\hfuzz=500pt\includegraphics[width=1em]{element.pdf}~keepSatelliteStates & \hfuzz=500pt boolean & \hfuzz=500pt set boundary values of each arc global\\
\hfuzz=500pt\includegraphics[width=1em]{element.pdf}~integrationDegree & \hfuzz=500pt uint & \hfuzz=500pt integration of forces by polynomial approximation of degree n\\
\hfuzz=500pt\includegraphics[width=1em]{element.pdf}~interpolationDegree & \hfuzz=500pt uint & \hfuzz=500pt orbit interpolation by polynomial approximation of degree n\\
\hfuzz=500pt\includegraphics[width=1em]{element.pdf}~accelerateComputation & \hfuzz=500pt boolean & \hfuzz=500pt acceleration of computation by transforming the observations\\
\hfuzz=500pt\includegraphics[width=1em]{element.pdf}~covariancePod & \hfuzz=500pt \hyperref[covariancePodType]{covariancePod} & \hfuzz=500pt covariance matrix of kinematic orbits\\
\hline
\end{tabularx}


\subsection{PodAcceleration}\label{observationType:podAcceleration}
The observation equations for precise orbit data (POD) are given by
\begin{equation}
\ddot{\M r}(t) - \M g_0(t) = \nabla V(\M r, t),
\end{equation}
where the accelerations of the satellite $\ddot{\M r}(t)$ are derived from the kinematic positions
in \configClass{rightHandSide}{podRightSideType}. The orbit differentation is performed by a moving
polynomial interpolation or approximation with degree \config{interpolationDegree}
and number of used epochs \config{numberOfEpochs}. The reference forces $\M g_0(t)$ are computed
with the background models in \configClass{rightHandSide}{podRightSideType}.

All instrument data \configFile{inputfileOrbit}{instrument}, \configFile{inputfileStarCamera}{instrument},
and \configFile{inputfileAccelerometer}{instrument} must be synchronous and be given
with a constant sampling without any gaps in each short arc (see \program{InstrumentSynchronize}).

The unknown gravity field $\nabla V(\M r, t)$ parametrized by \configClass{parametrizationGravity}{parametrizationGravityType}
is not evaluated at the observed positions but at the orbit given by \configFile{inputfileOrbit}{instrument}.
The same is true for the reference forces. This orbit may be a more accurate dynamical orbit but
in most cases the kinematic orbit provides good results.

The accuracy or the full covariance matrix of the precise orbit data is provided in
\configClass{covariancePod}{covariancePodType} and can be estimated with \program{PreprocessingPod}.


\keepXColumns
\begin{tabularx}{\textwidth}{N T A}
\hline
Name & Type & Annotation\\
\hline
\hfuzz=500pt\includegraphics[width=1em]{element.pdf}~inputfileSatelliteModel & \hfuzz=500pt filename & \hfuzz=500pt satellite macro model\\
\hfuzz=500pt\includegraphics[width=1em]{element-mustset-unbounded.pdf}~rightHandSide & \hfuzz=500pt \hyperref[podRightSideType]{podRightSide} & \hfuzz=500pt input for the reduced observation vector\\
\hfuzz=500pt\includegraphics[width=1em]{element-mustset.pdf}~inputfileOrbit & \hfuzz=500pt filename & \hfuzz=500pt used to evaluate the observation equations, not used as observations\\
\hfuzz=500pt\includegraphics[width=1em]{element-mustset.pdf}~inputfileStarCamera & \hfuzz=500pt filename & \hfuzz=500pt \\
\hfuzz=500pt\includegraphics[width=1em]{element-mustset.pdf}~earthRotation & \hfuzz=500pt \hyperref[earthRotationType]{earthRotation} & \hfuzz=500pt \\
\hfuzz=500pt\includegraphics[width=1em]{element.pdf}~ephemerides & \hfuzz=500pt \hyperref[ephemeridesType]{ephemerides} & \hfuzz=500pt \\
\hfuzz=500pt\includegraphics[width=1em]{element-mustset-unbounded.pdf}~parametrizationGravity & \hfuzz=500pt \hyperref[parametrizationGravityType]{parametrizationGravity} & \hfuzz=500pt gravity field parametrization\\
\hfuzz=500pt\includegraphics[width=1em]{element-unbounded.pdf}~parametrizationAcceleration & \hfuzz=500pt \hyperref[parametrizationAccelerationType]{parametrizationAcceleration} & \hfuzz=500pt orbit force parameters\\
\hfuzz=500pt\includegraphics[width=1em]{element.pdf}~interpolationDegree & \hfuzz=500pt uint & \hfuzz=500pt orbit differentation  by polynomial approximation of degree n\\
\hfuzz=500pt\includegraphics[width=1em]{element.pdf}~numberOfEpochs & \hfuzz=500pt uint & \hfuzz=500pt number of used Epochs for polynom computation\\
\hfuzz=500pt\includegraphics[width=1em]{element.pdf}~covariancePod & \hfuzz=500pt \hyperref[covariancePodType]{covariancePod} & \hfuzz=500pt covariance matrix of kinematic orbits\\
\hline
\end{tabularx}


\subsection{PodEnergy}\label{observationType:podEnergy}
The observation equations for precise orbit data (POD) are given by
\begin{equation}
  \frac{1}{2}\dot{\M r}^2
  -\dot{\M r} \cdot (\M\Omega\times\M r)
  +\int_{t_0}^t(\dot{\M\Omega}\times\M r)\cdot \dot{\M r}\,dt
  - \int_{t_0}^t \M g_0 \cdot\dot{\M r}'\,dt
  = V + E.
\end{equation}
where the velocities of the satellite $\ddot{\M r}(t)$ are derived from
the kinematic positions in \configClass{rightHandSide}{podRightSideType} and the Earth's rotation vector~$\M\Omega(t)$ is modeled
within \configClass{earthRotation}{earthRotationType}. The orbit differentation is
performed by a polynomial interpolation with degree \config{interpolationDegree}.
The integrals are solved a polynomial interpolation with degree \config{integrationDegree}.
The reference forces $\M g_0(t)$ are computed with the background models in \configClass{rightHandSide}{podRightSideType}.

All instrument data \configFile{inputfileOrbit}{instrument}, \configFile{inputfileStarCamera}{instrument}, and \configFile{inputfileAccelerometer}{instrument}
must be synchronous and be given with a constant sampling without any gaps in each short arc
(see \program{InstrumentSynchronize}).

The unknown gravity potential $V(\M r)$ parametrized by \configClass{parametrizationGravity}{parametrizationGravityType}
is not evaluated at the observed positions but at the orbit given by \configFile{inputfileOrbit}{instrument}.
The same is true for the reference forces. This orbit may be a more accurate dynamical orbit but
in most cases the kinematic orbit provides good results.

An unknown energy bias~$E$ per arc is parametrized by \configClass{parametrizationBias}{parametrizationTemporalType}
and should be a constant in theory but temporal changes might help to absorb other unmodelled effects.

The accuracy or the full covariance matrix of the precise orbit data is provided in
\configClass{covariancePod}{covariancePodType} and can be estimated with \program{PreprocessingPod}.


\keepXColumns
\begin{tabularx}{\textwidth}{N T A}
\hline
Name & Type & Annotation\\
\hline
\hfuzz=500pt\includegraphics[width=1em]{element.pdf}~inputfileSatelliteModel & \hfuzz=500pt filename & \hfuzz=500pt satellite macro model\\
\hfuzz=500pt\includegraphics[width=1em]{element-mustset-unbounded.pdf}~rightHandSide & \hfuzz=500pt \hyperref[podRightSideType]{podRightSide} & \hfuzz=500pt input for the reduced observation vector\\
\hfuzz=500pt\includegraphics[width=1em]{element-mustset.pdf}~inputfileOrbit & \hfuzz=500pt filename & \hfuzz=500pt used to evaluate the observation equations, not used as observations\\
\hfuzz=500pt\includegraphics[width=1em]{element-mustset.pdf}~inputfileStarCamera & \hfuzz=500pt filename & \hfuzz=500pt \\
\hfuzz=500pt\includegraphics[width=1em]{element-mustset.pdf}~earthRotation & \hfuzz=500pt \hyperref[earthRotationType]{earthRotation} & \hfuzz=500pt \\
\hfuzz=500pt\includegraphics[width=1em]{element.pdf}~ephemerides & \hfuzz=500pt \hyperref[ephemeridesType]{ephemerides} & \hfuzz=500pt \\
\hfuzz=500pt\includegraphics[width=1em]{element-mustset-unbounded.pdf}~parametrizationGravity & \hfuzz=500pt \hyperref[parametrizationGravityType]{parametrizationGravity} & \hfuzz=500pt gravity field parametrization (potential)\\
\hfuzz=500pt\includegraphics[width=1em]{element-mustset-unbounded.pdf}~parametrizationBias & \hfuzz=500pt \hyperref[parametrizationTemporalType]{parametrizationTemporal} & \hfuzz=500pt unknown total energy per arc\\
\hfuzz=500pt\includegraphics[width=1em]{element.pdf}~interpolationDegree & \hfuzz=500pt uint & \hfuzz=500pt orbit differentation  by polynomial approximation of degree n\\
\hfuzz=500pt\includegraphics[width=1em]{element.pdf}~integrationDegree & \hfuzz=500pt uint & \hfuzz=500pt integration of forces by polynomial approximation of degree n\\
\hfuzz=500pt\includegraphics[width=1em]{element.pdf}~covariancePod & \hfuzz=500pt \hyperref[covariancePodType]{covariancePod} & \hfuzz=500pt covariance matrix of kinematic orbits\\
\hline
\end{tabularx}


\subsection{SstVariational}\label{observationType:sstVariational}
Like \configClass{observation:podVariational}{observationType:podVariational} (see there for details)
but with two satellites and additional satellite-to-satellite (SST) observations.

If multiple \configFile{inputfileSatelliteTracking}{instrument} are given
all data are add together. So corrections in extra files like the light time correction
can easily be added. Empirical parameters for the SST observations can be setup with
\configClass{parametrizationSst}{parametrizationSatelliteTrackingType}.
The accuracy or the full covariance matrix of SST is provided in
\configClass{covarianceSst}{covarianceSstType}.


\keepXColumns
\begin{tabularx}{\textwidth}{N T A}
\hline
Name & Type & Annotation\\
\hline
\hfuzz=500pt\includegraphics[width=1em]{element-mustset.pdf}~rightHandSide & \hfuzz=500pt sequence & \hfuzz=500pt input for observation vectors\\
\hfuzz=500pt\includegraphics[width=1em]{connector.pdf}\includegraphics[width=1em]{element-mustset-unbounded.pdf}~inputfileSatelliteTracking & \hfuzz=500pt filename & \hfuzz=500pt ranging observations and corrections\\
\hfuzz=500pt\includegraphics[width=1em]{connector.pdf}\includegraphics[width=1em]{element.pdf}~inputfileOrbit1 & \hfuzz=500pt filename & \hfuzz=500pt kinematic positions of satellite A as observations\\
\hfuzz=500pt\includegraphics[width=1em]{connector.pdf}\includegraphics[width=1em]{element.pdf}~inputfileOrbit2 & \hfuzz=500pt filename & \hfuzz=500pt kinematic positions of satellite B as observations\\
\hfuzz=500pt\includegraphics[width=1em]{element-mustset.pdf}~sstType & \hfuzz=500pt choice & \hfuzz=500pt \\
\hfuzz=500pt\includegraphics[width=1em]{connector.pdf}\includegraphics[width=1em]{element-mustset.pdf}~range & \hfuzz=500pt  & \hfuzz=500pt \\
\hfuzz=500pt\includegraphics[width=1em]{connector.pdf}\includegraphics[width=1em]{element-mustset.pdf}~rangeRate & \hfuzz=500pt  & \hfuzz=500pt \\
\hfuzz=500pt\includegraphics[width=1em]{connector.pdf}\includegraphics[width=1em]{element-mustset.pdf}~none & \hfuzz=500pt  & \hfuzz=500pt \\
\hfuzz=500pt\includegraphics[width=1em]{element-mustset.pdf}~inputfileVariational1 & \hfuzz=500pt filename & \hfuzz=500pt approximate position and integrated state matrix\\
\hfuzz=500pt\includegraphics[width=1em]{element-mustset.pdf}~inputfileVariational2 & \hfuzz=500pt filename & \hfuzz=500pt approximate position and integrated state matrix\\
\hfuzz=500pt\includegraphics[width=1em]{element.pdf}~ephemerides & \hfuzz=500pt \hyperref[ephemeridesType]{ephemerides} & \hfuzz=500pt \\
\hfuzz=500pt\includegraphics[width=1em]{element-unbounded.pdf}~parametrizationGravity & \hfuzz=500pt \hyperref[parametrizationGravityType]{parametrizationGravity} & \hfuzz=500pt gravity field parametrization\\
\hfuzz=500pt\includegraphics[width=1em]{element-unbounded.pdf}~parametrizationAcceleration1 & \hfuzz=500pt \hyperref[parametrizationAccelerationType]{parametrizationAcceleration} & \hfuzz=500pt orbit1 force parameters\\
\hfuzz=500pt\includegraphics[width=1em]{element-unbounded.pdf}~parametrizationAcceleration2 & \hfuzz=500pt \hyperref[parametrizationAccelerationType]{parametrizationAcceleration} & \hfuzz=500pt orbit2 force parameters\\
\hfuzz=500pt\includegraphics[width=1em]{element-unbounded.pdf}~parametrizationSst & \hfuzz=500pt \hyperref[parametrizationSatelliteTrackingType]{parametrizationSatelliteTracking} & \hfuzz=500pt satellite tracking parameter\\
\hfuzz=500pt\includegraphics[width=1em]{element.pdf}~integrationDegree & \hfuzz=500pt uint & \hfuzz=500pt integration of forces by polynomial approximation of degree n\\
\hfuzz=500pt\includegraphics[width=1em]{element.pdf}~interpolationDegree & \hfuzz=500pt uint & \hfuzz=500pt orbit interpolation by polynomial approximation of degree n\\
\hfuzz=500pt\includegraphics[width=1em]{element-mustset.pdf}~covarianceSst & \hfuzz=500pt \hyperref[covarianceSstType]{covarianceSst} & \hfuzz=500pt covariance matrix of satellite to satellite tracking observations\\
\hfuzz=500pt\includegraphics[width=1em]{element-mustset.pdf}~covariancePod1 & \hfuzz=500pt \hyperref[covariancePodType]{covariancePod} & \hfuzz=500pt covariance matrix of kinematic orbits (satellite 1)\\
\hfuzz=500pt\includegraphics[width=1em]{element-mustset.pdf}~covariancePod2 & \hfuzz=500pt \hyperref[covariancePodType]{covariancePod} & \hfuzz=500pt covariance matrix of kinematic orbits (satellite 2)\\
\hline
\end{tabularx}


\subsection{SstIntegral}\label{observationType:sstIntegral}
Like \configClass{observation:podIntegral}{observationType:podIntegral} (see there for details)
but with two satellites and additional satellite-to-satellite (SST) observations.

If multiple \configFile{inputfileSatelliteTracking}{instrument} are given
all data are add together. So corrections in extra files like the light time correction
can easily be added. Empirical parameters for the SST observations can be setup with
\configClass{parametrizationSst}{parametrizationSatelliteTrackingType}.
The accuracy or the full covariance matrix of SST is provided in
\configClass{covarianceSst}{covarianceSstType}.


\keepXColumns
\begin{tabularx}{\textwidth}{N T A}
\hline
Name & Type & Annotation\\
\hline
\hfuzz=500pt\includegraphics[width=1em]{element.pdf}~inputfileSatelliteModel1 & \hfuzz=500pt filename & \hfuzz=500pt satellite macro model\\
\hfuzz=500pt\includegraphics[width=1em]{element.pdf}~inputfileSatelliteModel2 & \hfuzz=500pt filename & \hfuzz=500pt satellite macro model\\
\hfuzz=500pt\includegraphics[width=1em]{element-mustset-unbounded.pdf}~rightHandSide & \hfuzz=500pt \hyperref[sstRightSideType]{sstRightSide} & \hfuzz=500pt input for the reduced observation vector\\
\hfuzz=500pt\includegraphics[width=1em]{element-mustset.pdf}~sstType & \hfuzz=500pt choice & \hfuzz=500pt \\
\hfuzz=500pt\includegraphics[width=1em]{connector.pdf}\includegraphics[width=1em]{element-mustset.pdf}~range & \hfuzz=500pt  & \hfuzz=500pt \\
\hfuzz=500pt\includegraphics[width=1em]{connector.pdf}\includegraphics[width=1em]{element-mustset.pdf}~rangeRate & \hfuzz=500pt  & \hfuzz=500pt \\
\hfuzz=500pt\includegraphics[width=1em]{connector.pdf}\includegraphics[width=1em]{element-mustset.pdf}~rangeAcceleration & \hfuzz=500pt  & \hfuzz=500pt \\
\hfuzz=500pt\includegraphics[width=1em]{connector.pdf}\includegraphics[width=1em]{element-mustset.pdf}~none & \hfuzz=500pt  & \hfuzz=500pt \\
\hfuzz=500pt\includegraphics[width=1em]{element-mustset.pdf}~inputfileOrbit1 & \hfuzz=500pt filename & \hfuzz=500pt used to evaluate the observation equations, not used as observations\\
\hfuzz=500pt\includegraphics[width=1em]{element-mustset.pdf}~inputfileOrbit2 & \hfuzz=500pt filename & \hfuzz=500pt used to evaluate the observation equations, not used as observations\\
\hfuzz=500pt\includegraphics[width=1em]{element-mustset.pdf}~inputfileStarCamera1 & \hfuzz=500pt filename & \hfuzz=500pt \\
\hfuzz=500pt\includegraphics[width=1em]{element-mustset.pdf}~inputfileStarCamera2 & \hfuzz=500pt filename & \hfuzz=500pt \\
\hfuzz=500pt\includegraphics[width=1em]{element-mustset.pdf}~earthRotation & \hfuzz=500pt \hyperref[earthRotationType]{earthRotation} & \hfuzz=500pt \\
\hfuzz=500pt\includegraphics[width=1em]{element.pdf}~ephemerides & \hfuzz=500pt \hyperref[ephemeridesType]{ephemerides} & \hfuzz=500pt \\
\hfuzz=500pt\includegraphics[width=1em]{element-unbounded.pdf}~gradientfield & \hfuzz=500pt \hyperref[gravityfieldType]{gravityfield} & \hfuzz=500pt low order field to estimate the change of the gravity by position adjustement\\
\hfuzz=500pt\includegraphics[width=1em]{element-unbounded.pdf}~parametrizationGravity & \hfuzz=500pt \hyperref[parametrizationGravityType]{parametrizationGravity} & \hfuzz=500pt gravity field parametrization\\
\hfuzz=500pt\includegraphics[width=1em]{element-unbounded.pdf}~parametrizationAcceleration1 & \hfuzz=500pt \hyperref[parametrizationAccelerationType]{parametrizationAcceleration} & \hfuzz=500pt orbit1 force parameters\\
\hfuzz=500pt\includegraphics[width=1em]{element-unbounded.pdf}~parametrizationAcceleration2 & \hfuzz=500pt \hyperref[parametrizationAccelerationType]{parametrizationAcceleration} & \hfuzz=500pt orbit2 force parameters\\
\hfuzz=500pt\includegraphics[width=1em]{element-unbounded.pdf}~parametrizationSst & \hfuzz=500pt \hyperref[parametrizationSatelliteTrackingType]{parametrizationSatelliteTracking} & \hfuzz=500pt satellite tracking parameter\\
\hfuzz=500pt\includegraphics[width=1em]{element.pdf}~keepSatelliteStates & \hfuzz=500pt boolean & \hfuzz=500pt set boundary values of each arc global\\
\hfuzz=500pt\includegraphics[width=1em]{element.pdf}~integrationDegree & \hfuzz=500pt uint & \hfuzz=500pt integration of forces by polynomial approximation of degree n\\
\hfuzz=500pt\includegraphics[width=1em]{element.pdf}~interpolationDegree & \hfuzz=500pt uint & \hfuzz=500pt orbit interpolation by polynomial approximation of degree n\\
\hfuzz=500pt\includegraphics[width=1em]{element-mustset.pdf}~covarianceSst & \hfuzz=500pt \hyperref[covarianceSstType]{covarianceSst} & \hfuzz=500pt covariance matrix of satellite to satellite tracking observations\\
\hfuzz=500pt\includegraphics[width=1em]{element-mustset.pdf}~covariancePod1 & \hfuzz=500pt \hyperref[covariancePodType]{covariancePod} & \hfuzz=500pt covariance matrix of kinematic orbits (satellite 1)\\
\hfuzz=500pt\includegraphics[width=1em]{element-mustset.pdf}~covariancePod2 & \hfuzz=500pt \hyperref[covariancePodType]{covariancePod} & \hfuzz=500pt covariance matrix of kinematic orbits (satellite 2)\\
\hline
\end{tabularx}


\subsection{DualSstVariational}\label{observationType:dualSstVariational}
Like \configClass{observation:sstVariational}{observationType:sstVariational} (see there for details)
but with two simultaneous satellite-to-satellite (SST) observations.

This class reads two SST observation files (\configFile{inputfileSatelliteTracking1}{instrument} and
\configFile{inputfileSatelliteTracking2}{instrument}).
Empirical parameters for the SST observations can be setup independently for both SST observation
types with \configClass{parametrizationSst1}{parametrizationSatelliteTrackingType} and
\configClass{parametrizationSst2}{parametrizationSatelliteTrackingType}.

Both SST observation types are reduced by the same background models and the same impact
of accelerometer measurements. The covariance matrix of the reduced observations should not consider
the the instrument noise only (\configClass{covarianceSst1/2}{covarianceSstType}) but must
take the cross correlations \configClass{covarianceAcc}{covarianceSstType} into account.
The covariance matrix of the reduced observations is given by
\begin{equation}
  \M\Sigma(\begin{bmatrix} \Delta l_{SST1} \\ \Delta l_{SST2} \end{bmatrix})
  = \begin{bmatrix} \M\Sigma_{SST1} + \M\Sigma_{ACC} & \M\Sigma_{ACC} \\
                   \M\Sigma_{ACC} & \M\Sigma_{SST2} + \M\Sigma_{ACC}
    \end{bmatrix}.
\end{equation}


\keepXColumns
\begin{tabularx}{\textwidth}{N T A}
\hline
Name & Type & Annotation\\
\hline
\hfuzz=500pt\includegraphics[width=1em]{element-mustset.pdf}~rightHandSide & \hfuzz=500pt sequence & \hfuzz=500pt input for observation vectors\\
\hfuzz=500pt\includegraphics[width=1em]{connector.pdf}\includegraphics[width=1em]{element-mustset-unbounded.pdf}~inputfileSatelliteTracking1 & \hfuzz=500pt filename & \hfuzz=500pt ranging observations and corrections\\
\hfuzz=500pt\includegraphics[width=1em]{connector.pdf}\includegraphics[width=1em]{element-mustset-unbounded.pdf}~inputfileSatelliteTracking2 & \hfuzz=500pt filename & \hfuzz=500pt ranging observations and corrections\\
\hfuzz=500pt\includegraphics[width=1em]{connector.pdf}\includegraphics[width=1em]{element.pdf}~inputfileOrbit1 & \hfuzz=500pt filename & \hfuzz=500pt kinematic positions of satellite A as observations\\
\hfuzz=500pt\includegraphics[width=1em]{connector.pdf}\includegraphics[width=1em]{element.pdf}~inputfileOrbit2 & \hfuzz=500pt filename & \hfuzz=500pt kinematic positions of satellite B as observations\\
\hfuzz=500pt\includegraphics[width=1em]{element-mustset.pdf}~sstType & \hfuzz=500pt choice & \hfuzz=500pt \\
\hfuzz=500pt\includegraphics[width=1em]{connector.pdf}\includegraphics[width=1em]{element-mustset.pdf}~range & \hfuzz=500pt  & \hfuzz=500pt \\
\hfuzz=500pt\includegraphics[width=1em]{connector.pdf}\includegraphics[width=1em]{element-mustset.pdf}~rangeRate & \hfuzz=500pt  & \hfuzz=500pt \\
\hfuzz=500pt\includegraphics[width=1em]{connector.pdf}\includegraphics[width=1em]{element-mustset.pdf}~none & \hfuzz=500pt  & \hfuzz=500pt \\
\hfuzz=500pt\includegraphics[width=1em]{element-mustset.pdf}~inputfileVariational1 & \hfuzz=500pt filename & \hfuzz=500pt approximate position and integrated state matrix\\
\hfuzz=500pt\includegraphics[width=1em]{element-mustset.pdf}~inputfileVariational2 & \hfuzz=500pt filename & \hfuzz=500pt approximate position and integrated state matrix\\
\hfuzz=500pt\includegraphics[width=1em]{element.pdf}~ephemerides & \hfuzz=500pt \hyperref[ephemeridesType]{ephemerides} & \hfuzz=500pt \\
\hfuzz=500pt\includegraphics[width=1em]{element-unbounded.pdf}~parametrizationGravity & \hfuzz=500pt \hyperref[parametrizationGravityType]{parametrizationGravity} & \hfuzz=500pt gravity field parametrization\\
\hfuzz=500pt\includegraphics[width=1em]{element-unbounded.pdf}~parametrizationAcceleration1 & \hfuzz=500pt \hyperref[parametrizationAccelerationType]{parametrizationAcceleration} & \hfuzz=500pt orbit1 force parameters\\
\hfuzz=500pt\includegraphics[width=1em]{element-unbounded.pdf}~parametrizationAcceleration2 & \hfuzz=500pt \hyperref[parametrizationAccelerationType]{parametrizationAcceleration} & \hfuzz=500pt orbit2 force parameters\\
\hfuzz=500pt\includegraphics[width=1em]{element-unbounded.pdf}~parametrizationSst1 & \hfuzz=500pt \hyperref[parametrizationSatelliteTrackingType]{parametrizationSatelliteTracking} & \hfuzz=500pt satellite tracking parameter for first ranging observations\\
\hfuzz=500pt\includegraphics[width=1em]{element-unbounded.pdf}~parametrizationSst2 & \hfuzz=500pt \hyperref[parametrizationSatelliteTrackingType]{parametrizationSatelliteTracking} & \hfuzz=500pt satellite tracking parameter for second ranging observations\\
\hfuzz=500pt\includegraphics[width=1em]{element.pdf}~integrationDegree & \hfuzz=500pt uint & \hfuzz=500pt integration of forces by polynomial approximation of degree n\\
\hfuzz=500pt\includegraphics[width=1em]{element.pdf}~interpolationDegree & \hfuzz=500pt uint & \hfuzz=500pt orbit interpolation by polynomial approximation of degree n\\
\hfuzz=500pt\includegraphics[width=1em]{element-mustset.pdf}~covarianceSst1 & \hfuzz=500pt \hyperref[covarianceSstType]{covarianceSst} & \hfuzz=500pt covariance matrix of first satellite to satellite tracking observations\\
\hfuzz=500pt\includegraphics[width=1em]{element-mustset.pdf}~covarianceSst2 & \hfuzz=500pt \hyperref[covarianceSstType]{covarianceSst} & \hfuzz=500pt covariance matrix of second satellite to satellite tracking observations\\
\hfuzz=500pt\includegraphics[width=1em]{element.pdf}~covarianceAcc & \hfuzz=500pt \hyperref[covarianceSstType]{covarianceSst} & \hfuzz=500pt common covariance matrix of reduced satellite to satellite tracking observations\\
\hfuzz=500pt\includegraphics[width=1em]{element-mustset.pdf}~covariancePod1 & \hfuzz=500pt \hyperref[covariancePodType]{covariancePod} & \hfuzz=500pt covariance matrix of kinematic orbits (satellite 1)\\
\hfuzz=500pt\includegraphics[width=1em]{element-mustset.pdf}~covariancePod2 & \hfuzz=500pt \hyperref[covariancePodType]{covariancePod} & \hfuzz=500pt covariance matrix of kinematic orbits (satellite 2)\\
\hline
\end{tabularx}


\subsection{Gradiometer}\label{observationType:gradiometer}
Observation equations for satellite gravity gradiometry (SGG)
\begin{equation}
  \nabla\nabla V(\M r) =
  \begin{pmatrix}
    \frac{\partial^2 V}{\partial x^2}         & \frac{\partial^2 V}{\partial x\partial y} & \frac{\partial^2 V}{\partial x\partial z} \\
    \frac{\partial^2 V}{\partial y\partial x} & \frac{\partial^2 V}{\partial y^2}         & \frac{\partial^2 V}{\partial y\partial z} \\
    \frac{\partial^2 V}{\partial z\partial x} & \frac{\partial^2 V}{\partial z\partial y} & \frac{\partial^2 V}{\partial z^2}
  \end{pmatrix}.
\end{equation}
From the \configFile{inputfileGradiometer}{instrument} observations precomputed \configFile{inputfileReferenceGradiometer}{instrument}
together with other background models are reduced, all given in \configClass{rightHandSide}{sggRightSideType}.

All instrument data \configFile{inputfileGradiometer}{instrument}, \configFile{inputfileOrbit}{instrument},
and \configFile{inputfileStarCamera}{instrument} must be synchronous and be diveded
into each short arcs (see \program{InstrumentSynchronize}).

Additional to the \configClass{parametrizationGravity}{parametrizationGravityType}
an (temporal changing) bias for each gradiometer component and arc can be estimated with
\configClass{parametrizationBias}{parametrizationTemporalType}.

The accuracy or the full covariance matrix of the gradiometer is provided in
\config{covarianceSgg} and can be estimated with \program{PreprocessingGradiometer}.



\keepXColumns
\begin{tabularx}{\textwidth}{N T A}
\hline
Name & Type & Annotation\\
\hline
\hfuzz=500pt\includegraphics[width=1em]{element-mustset-unbounded.pdf}~rightHandSide & \hfuzz=500pt \hyperref[sggRightSideType]{sggRightSide} & \hfuzz=500pt input for the observation vector\\
\hfuzz=500pt\includegraphics[width=1em]{element-mustset.pdf}~inputfileOrbit & \hfuzz=500pt filename & \hfuzz=500pt \\
\hfuzz=500pt\includegraphics[width=1em]{element-mustset.pdf}~inputfileStarCamera & \hfuzz=500pt filename & \hfuzz=500pt \\
\hfuzz=500pt\includegraphics[width=1em]{element-mustset.pdf}~earthRotation & \hfuzz=500pt \hyperref[earthRotationType]{earthRotation} & \hfuzz=500pt \\
\hfuzz=500pt\includegraphics[width=1em]{element.pdf}~ephemerides & \hfuzz=500pt \hyperref[ephemeridesType]{ephemerides} & \hfuzz=500pt \\
\hfuzz=500pt\includegraphics[width=1em]{element-mustset-unbounded.pdf}~parametrizationGravity & \hfuzz=500pt \hyperref[parametrizationGravityType]{parametrizationGravity} & \hfuzz=500pt \\
\hfuzz=500pt\includegraphics[width=1em]{element-unbounded.pdf}~parametrizationBias & \hfuzz=500pt \hyperref[parametrizationTemporalType]{parametrizationTemporal} & \hfuzz=500pt per arc\\
\hfuzz=500pt\includegraphics[width=1em]{element.pdf}~useXX & \hfuzz=500pt boolean & \hfuzz=500pt \\
\hfuzz=500pt\includegraphics[width=1em]{element.pdf}~useYY & \hfuzz=500pt boolean & \hfuzz=500pt \\
\hfuzz=500pt\includegraphics[width=1em]{element.pdf}~useZZ & \hfuzz=500pt boolean & \hfuzz=500pt \\
\hfuzz=500pt\includegraphics[width=1em]{element.pdf}~useXY & \hfuzz=500pt boolean & \hfuzz=500pt \\
\hfuzz=500pt\includegraphics[width=1em]{element.pdf}~useXZ & \hfuzz=500pt boolean & \hfuzz=500pt \\
\hfuzz=500pt\includegraphics[width=1em]{element.pdf}~useYZ & \hfuzz=500pt boolean & \hfuzz=500pt \\
\hfuzz=500pt\includegraphics[width=1em]{element-mustset.pdf}~covarianceSgg & \hfuzz=500pt sequence & \hfuzz=500pt \\
\hfuzz=500pt\includegraphics[width=1em]{connector.pdf}\includegraphics[width=1em]{element.pdf}~sigma & \hfuzz=500pt double & \hfuzz=500pt general variance factor\\
\hfuzz=500pt\includegraphics[width=1em]{connector.pdf}\includegraphics[width=1em]{element.pdf}~inputfileSigmasPerArc & \hfuzz=500pt filename & \hfuzz=500pt different accuaries for each arc (multplicated with sigma)\\
\hfuzz=500pt\includegraphics[width=1em]{connector.pdf}\includegraphics[width=1em]{element.pdf}~inputfileCovarianceFunction & \hfuzz=500pt filename & \hfuzz=500pt covariance function in time\\
\hline
\end{tabularx}


\subsection{Terrestrial}\label{observationType:terrestrial}
The gravity field is estimated from point wise measurements.
The gravity field parametrization is given by \configClass{parametrizationGravity}{parametrizationGravityType}.
There is no need to have the data regular distributed or given on a sphere or ellipsoid.
The type of the gridded data (e.g gravity anomalies or geoid heights)
must be set with \configClass{kernel}{kernelType}.
A \configClass{referencefield}{gravityfieldType} can be reduced beforehand.

The observations at given positions are calculated from
\configFile{inputfileGriddedData}{griddedData}.
The input columns are enumerated by \verb|data0|,~\verb|data1|,~\ldots,
see~\reference{dataVariables}{general.parser:dataVariables}.

The observations can be divided into small blocks for parallelization.
With \config{blockingSize} set the maximum count of observations in each block.


\keepXColumns
\begin{tabularx}{\textwidth}{N T A}
\hline
Name & Type & Annotation\\
\hline
\hfuzz=500pt\includegraphics[width=1em]{element-mustset.pdf}~rightHandSide & \hfuzz=500pt sequence & \hfuzz=500pt input for observation vectors\\
\hfuzz=500pt\includegraphics[width=1em]{connector.pdf}\includegraphics[width=1em]{element-mustset.pdf}~inputfileGriddedData & \hfuzz=500pt filename & \hfuzz=500pt \\
\hfuzz=500pt\includegraphics[width=1em]{connector.pdf}\includegraphics[width=1em]{element-mustset.pdf}~observation & \hfuzz=500pt expression & \hfuzz=500pt [SI units]\\
\hfuzz=500pt\includegraphics[width=1em]{connector.pdf}\includegraphics[width=1em]{element.pdf}~sigma & \hfuzz=500pt expression & \hfuzz=500pt accuracy, 1/sigma used as weighting\\
\hfuzz=500pt\includegraphics[width=1em]{connector.pdf}\includegraphics[width=1em]{element-unbounded.pdf}~referencefield & \hfuzz=500pt \hyperref[gravityfieldType]{gravityfield} & \hfuzz=500pt \\
\hfuzz=500pt\includegraphics[width=1em]{element-mustset.pdf}~kernel & \hfuzz=500pt \hyperref[kernelType]{kernel} & \hfuzz=500pt type of observations\\
\hfuzz=500pt\includegraphics[width=1em]{element-mustset-unbounded.pdf}~parametrizationGravity & \hfuzz=500pt \hyperref[parametrizationGravityType]{parametrizationGravity} & \hfuzz=500pt \\
\hfuzz=500pt\includegraphics[width=1em]{element.pdf}~time & \hfuzz=500pt time & \hfuzz=500pt for reference field and parametrization\\
\hfuzz=500pt\includegraphics[width=1em]{element.pdf}~blockingSize & \hfuzz=500pt uint & \hfuzz=500pt segementation of the obervations if designmatrix can't be build at once\\
\hline
\end{tabularx}


\subsection{Deflections}\label{observationType:deflections}
The gravity field parametrized by \configClass{parametrizationGravity}{parametrizationGravityType}
is estimated from deflections of the vertical measurements.
A \configClass{referencefield}{gravityfieldType} can be reduced beforehand.

The observations $\xi$ in north direction and $\eta$ in east direction
at given positions are calculated from
\configFile{inputfileGriddedData}{griddedData}.
The input columns are enumerated by \verb|data0|,~\verb|data1|,~\ldots,
see~\reference{dataVariables}{general.parser:dataVariables}.

The ellipsoid parameters \config{R} and \config{inverseFlattening} are used
to define the local normal direction.

The observations can be divided into small blocks for parallelization.
With \config{blockingSize} set the maximum count of observations in each block.


\keepXColumns
\begin{tabularx}{\textwidth}{N T A}
\hline
Name & Type & Annotation\\
\hline
\hfuzz=500pt\includegraphics[width=1em]{element-mustset.pdf}~rightHandSide & \hfuzz=500pt sequence & \hfuzz=500pt input for observation vectors\\
\hfuzz=500pt\includegraphics[width=1em]{connector.pdf}\includegraphics[width=1em]{element-mustset.pdf}~inputfileGriddedData & \hfuzz=500pt filename & \hfuzz=500pt \\
\hfuzz=500pt\includegraphics[width=1em]{connector.pdf}\includegraphics[width=1em]{element-mustset.pdf}~observationXi & \hfuzz=500pt expression & \hfuzz=500pt North-South Deflections of the Vertical [rad]\\
\hfuzz=500pt\includegraphics[width=1em]{connector.pdf}\includegraphics[width=1em]{element-mustset.pdf}~observationEta & \hfuzz=500pt expression & \hfuzz=500pt East-West Deflections of the Vertical  [rad]\\
\hfuzz=500pt\includegraphics[width=1em]{connector.pdf}\includegraphics[width=1em]{element.pdf}~sigmaXi & \hfuzz=500pt expression & \hfuzz=500pt accuracy, 1/sigma used as weighting\\
\hfuzz=500pt\includegraphics[width=1em]{connector.pdf}\includegraphics[width=1em]{element.pdf}~sigmaEta & \hfuzz=500pt expression & \hfuzz=500pt accuracy, 1/sigma used as weighting\\
\hfuzz=500pt\includegraphics[width=1em]{connector.pdf}\includegraphics[width=1em]{element-unbounded.pdf}~referencefield & \hfuzz=500pt \hyperref[gravityfieldType]{gravityfield} & \hfuzz=500pt \\
\hfuzz=500pt\includegraphics[width=1em]{element-mustset-unbounded.pdf}~parametrizationGravity & \hfuzz=500pt \hyperref[parametrizationGravityType]{parametrizationGravity} & \hfuzz=500pt \\
\hfuzz=500pt\includegraphics[width=1em]{element.pdf}~time & \hfuzz=500pt time & \hfuzz=500pt for reference field and parametrization\\
\hfuzz=500pt\includegraphics[width=1em]{element.pdf}~R & \hfuzz=500pt double & \hfuzz=500pt reference radius for ellipsoid\\
\hfuzz=500pt\includegraphics[width=1em]{element.pdf}~inverseFlattening & \hfuzz=500pt double & \hfuzz=500pt reference flattening for ellipsoid, 0: sphere\\
\hfuzz=500pt\includegraphics[width=1em]{element.pdf}~blockingSize & \hfuzz=500pt uint & \hfuzz=500pt segementation of the obervations if designmatrix can't be build at once\\
\hline
\end{tabularx}


\subsection{StationLoading}\label{observationType:stationLoading}
Observation equations for displacements of a list of stations
due to the effect of time variable loading masses. The displacement~$\M u$ of a station is calculated according to
\begin{equation}
\M u(\M r) = \frac{1}{\gamma}\sum_{n=0}^\infty \left[\frac{h_n}{1+k_n}V_n(\M r)\,\M e_{up}
+ R\frac{l_n}{1+k_n}\left(
 \frac{\partial V_n(\M r)}{\partial \M e_{north}}\M e_{north}
+\frac{\partial V_n(\M r)}{\partial \M e_{east}} \M e_{east}\right)\right],
\end{equation}
where $\gamma$ is the normal gravity, the load Love and Shida numbers $h_n,l_n$ are given by
\configFile{inputfileDeformationLoadLoveNumber}{matrix} and the load Love numbers $k_n$ are given by
\configFile{inputfilePotentialLoadLoveNumber}{matrix}.
The $V_n$ are the spherical harmonics expansion of degree $n$ of the full time variable
gravitational potential (potential of the loading mass + deformation potential)
parametrized by \configClass{parametrizationGravity}{parametrizationGravityType}.
Additional parameters can be setup to estimate the realization of the reference frame
of the station coordinates (\config{estimateTranslation},
\config{estimateRotation}, and \config{estimateScale}).

The observations at stations coordinates are calculated from
\configFile{inputfileGriddedData}{griddedData}.
The input columns are enumerated by \verb|data0|,~\verb|data1|,~\ldots,
see~\reference{dataVariables}{general.parser:dataVariables}.

The ellipsoid parameters \config{R} and \config{inverseFlattening} are used
to define the local frame (north, east, up).

See also \program{Gravityfield2DisplacementTimeSeries}.

Reference:
Rietbroek (2014): Retrieval of Sea Level and Surface Loading Variations from Geodetic Observations
and Model Simulations: an Integrated Approach, Bonn, 2014. - Dissertation,
\url{https://nbn-resolving.org/urn:nbn:de:hbz:5n-35460}


\keepXColumns
\begin{tabularx}{\textwidth}{N T A}
\hline
Name & Type & Annotation\\
\hline
\hfuzz=500pt\includegraphics[width=1em]{element-mustset.pdf}~rightHandSide & \hfuzz=500pt sequence & \hfuzz=500pt input for observation vectors\\
\hfuzz=500pt\includegraphics[width=1em]{connector.pdf}\includegraphics[width=1em]{element-mustset.pdf}~inputfileGriddedData & \hfuzz=500pt filename & \hfuzz=500pt station positions with displacement data\\
\hfuzz=500pt\includegraphics[width=1em]{connector.pdf}\includegraphics[width=1em]{element-mustset.pdf}~observationNorth & \hfuzz=500pt expression & \hfuzz=500pt displacement [m]\\
\hfuzz=500pt\includegraphics[width=1em]{connector.pdf}\includegraphics[width=1em]{element-mustset.pdf}~observationEast & \hfuzz=500pt expression & \hfuzz=500pt displacement [m]\\
\hfuzz=500pt\includegraphics[width=1em]{connector.pdf}\includegraphics[width=1em]{element-mustset.pdf}~observationUp & \hfuzz=500pt expression & \hfuzz=500pt displacement [m]\\
\hfuzz=500pt\includegraphics[width=1em]{connector.pdf}\includegraphics[width=1em]{element.pdf}~sigmaNorth & \hfuzz=500pt expression & \hfuzz=500pt accuracy, 1/sigma used as weighting\\
\hfuzz=500pt\includegraphics[width=1em]{connector.pdf}\includegraphics[width=1em]{element.pdf}~sigmaEast & \hfuzz=500pt expression & \hfuzz=500pt accuracy, 1/sigma used as weighting\\
\hfuzz=500pt\includegraphics[width=1em]{connector.pdf}\includegraphics[width=1em]{element.pdf}~sigmaUp & \hfuzz=500pt expression & \hfuzz=500pt accuracy, 1/sigma used as weighting\\
\hfuzz=500pt\includegraphics[width=1em]{connector.pdf}\includegraphics[width=1em]{element.pdf}~inGlobalFrame & \hfuzz=500pt boolean & \hfuzz=500pt obs/sigmas given in global x,y,z frame instead of north,east,up\\
\hfuzz=500pt\includegraphics[width=1em]{connector.pdf}\includegraphics[width=1em]{element-unbounded.pdf}~referencefield & \hfuzz=500pt \hyperref[gravityfieldType]{gravityfield} & \hfuzz=500pt \\
\hfuzz=500pt\includegraphics[width=1em]{element.pdf}~time & \hfuzz=500pt time & \hfuzz=500pt for reference field and parametrization\\
\hfuzz=500pt\includegraphics[width=1em]{element-mustset-unbounded.pdf}~parametrizationGravity & \hfuzz=500pt \hyperref[parametrizationGravityType]{parametrizationGravity} & \hfuzz=500pt of loading (+defo) potential\\
\hfuzz=500pt\includegraphics[width=1em]{element.pdf}~estimateTranslation & \hfuzz=500pt boolean & \hfuzz=500pt coordinate center\\
\hfuzz=500pt\includegraphics[width=1em]{element.pdf}~estimateScale & \hfuzz=500pt boolean & \hfuzz=500pt scale factor of position\\
\hfuzz=500pt\includegraphics[width=1em]{element.pdf}~estimateRotation & \hfuzz=500pt boolean & \hfuzz=500pt rotation\\
\hfuzz=500pt\includegraphics[width=1em]{element-mustset.pdf}~inputfileDeformationLoadLoveNumber & \hfuzz=500pt filename & \hfuzz=500pt \\
\hfuzz=500pt\includegraphics[width=1em]{element.pdf}~inputfilePotentialLoadLoveNumber & \hfuzz=500pt filename & \hfuzz=500pt if full potential is given and not only loading potential\\
\hfuzz=500pt\includegraphics[width=1em]{element.pdf}~R & \hfuzz=500pt double & \hfuzz=500pt reference radius for ellipsoid\\
\hfuzz=500pt\includegraphics[width=1em]{element.pdf}~inverseFlattening & \hfuzz=500pt double & \hfuzz=500pt reference flattening for ellipsoid, 0: sphere\\
\hline
\end{tabularx}

\clearpage
%==================================

\section{OrbitPropagator}\label{orbitPropagatorType}
Implements the propagation of a satellite orbit under
the influence of \configClass{forces}{forcesType} as
used in \program{SimulateOrbit}
(dynamic orbits from numerical orbit integration).


\subsection{Euler}
This class implements Euler's method to propagate a satellite orbit under the influence of \configClass{Forces}{forcesType}.
Satellite is assumed to be oriented along-track.


\subsection{RungeKutta4}
This class implements the classical Runge-Kutta 4 method of orbit propagation
for satellite orbit under the influence of \configClass{Forces}{forcesType}.
No step-width control or other advanced features are implemented.
Satellite is assumed to be oriented along-track.
See: Montenbruck, Oliver, and Eberhard Gill. 2000. Satellite Orbits


\subsection{AdamsBashforthMoulton}
This class implements the Adams-Moulton class of predictor-corrector orbit propagators
for a satellite orbit under the influence of \configClass{Forces}{forcesType} using an implicit
Adams-Bashforth corrector. The coefficients for the propagator are derived using the equations
given in section 4.2.3 of [1]. Satellite is assumed to be oriented along-track.
[1] Montenbruck, Oliver, and Eberhard Gill. 2000. Satellite Orbits


\keepXColumns
\begin{tabularx}{\textwidth}{N T A}
\hline
Name & Type & Annotation\\
\hline
\hfuzz=500pt\includegraphics[width=1em]{element-mustset.pdf}~order & \hfuzz=500pt uint & \hfuzz=500pt Order of the Adams-Bashforth type propagator.\\
\hfuzz=500pt\includegraphics[width=1em]{element.pdf}~applyMoultonCorrector & \hfuzz=500pt boolean & \hfuzz=500pt Corrector step after Adams-Bashforth predcition.\\
\hfuzz=500pt\includegraphics[width=1em]{element-mustset.pdf}~warmup & \hfuzz=500pt \hyperref[orbitPropagatorType]{orbitPropagator} & \hfuzz=500pt \\
\hline
\end{tabularx}


\subsection{StoermerCowell}
This class implements the Stoermer-Cowell class of predictor-corrector orbit propagators for a satellite orbit
under the influence of \configClass{Forces}{forcesType}. The coefficients for the Stoermer predictor and Cowell corrector
are derived using the equations given in section 4.2.6 of [1]. Stoermer-Cowell is a double integration algorithm,
yielding positions directly from accelertions. It does not produce velocities. The velocities are derived using
Adams-type propagators as suggested in [2]. Satellite is assumed to be oriented along-track.
[1] Montenbruck, Oliver, and Eberhard Gill. 2000. Satellite Orbits
[2] Berry, Matthew M., and Liam M. Healy. 2004. “Implementation of Gauss-Jackson Integration for Orbit Propagation.”


\keepXColumns
\begin{tabularx}{\textwidth}{N T A}
\hline
Name & Type & Annotation\\
\hline
\hfuzz=500pt\includegraphics[width=1em]{element-mustset.pdf}~order & \hfuzz=500pt uint & \hfuzz=500pt Order of the Stoermer-Cowell type propagator.\\
\hfuzz=500pt\includegraphics[width=1em]{element-mustset.pdf}~warmup & \hfuzz=500pt \hyperref[orbitPropagatorType]{orbitPropagator} & \hfuzz=500pt \\
\hline
\end{tabularx}


\subsection{GaussJackson}
This class implements the Gauss-Jackson multi-step predictor-corrector method to
propagate a satellite orbit under the influence of \configClass{Forces}{forcesType}.
Satellite is assumed to be oriented along-track. Implementation is based on [1].
[1] Berry, Matthew M., and Liam M. Healy. 2004. “Implementation of Gauss-Jackson Integration for Orbit Propagation.”


\keepXColumns
\begin{tabularx}{\textwidth}{N T A}
\hline
Name & Type & Annotation\\
\hline
\hfuzz=500pt\includegraphics[width=1em]{element-mustset.pdf}~order & \hfuzz=500pt uint & \hfuzz=500pt of Gauss-Jackson method.\\
\hfuzz=500pt\includegraphics[width=1em]{element-mustset.pdf}~warmup & \hfuzz=500pt \hyperref[orbitPropagatorType]{orbitPropagator} & \hfuzz=500pt \\
\hfuzz=500pt\includegraphics[width=1em]{element.pdf}~correctorIterations & \hfuzz=500pt uint & \hfuzz=500pt Maximum number of iterations to run the corrector step for.\\
\hfuzz=500pt\includegraphics[width=1em]{element.pdf}~epsilon & \hfuzz=500pt double & \hfuzz=500pt Convergence criteria for position, velocity, and acceleration tests.\\
\hline
\end{tabularx}


\subsection{Polynomial}\label{orbitPropagatorType:Polynomial}
This class implements an integration Polynomial method to propagate a satellite orbit under
the influence of \configClass{Forces}{forcesType}. Satellite is assumed to be oriented along-track.
Implementation is based on code by Torsten Mayer-Gürr.


\keepXColumns
\begin{tabularx}{\textwidth}{N T A}
\hline
Name & Type & Annotation\\
\hline
\hfuzz=500pt\includegraphics[width=1em]{element-mustset.pdf}~degree & \hfuzz=500pt uint & \hfuzz=500pt polynomial degree to integrate accelerations\\
\hfuzz=500pt\includegraphics[width=1em]{element-mustset.pdf}~shift & \hfuzz=500pt int & \hfuzz=500pt shift polynomial in future (predicted accelerations)\\
\hfuzz=500pt\includegraphics[width=1em]{element-mustset.pdf}~epsilon & \hfuzz=500pt double & \hfuzz=500pt [m] max. position change to recompute forces\\
\hfuzz=500pt\includegraphics[width=1em]{element-mustset.pdf}~warmup & \hfuzz=500pt \hyperref[orbitPropagatorType]{orbitPropagator} & \hfuzz=500pt to compute epochs before start epoch\\
\hfuzz=500pt\includegraphics[width=1em]{element.pdf}~corrector & \hfuzz=500pt boolean & \hfuzz=500pt apply corrector iteration if position change is larger than epsilon\\
\hline
\end{tabularx}


\subsection{File}
Reads an orbit from file. If the needed epochs are not given an exception is thrown.


\keepXColumns
\begin{tabularx}{\textwidth}{N T A}
\hline
Name & Type & Annotation\\
\hline
\hfuzz=500pt\includegraphics[width=1em]{element-mustset.pdf}~inputfileOrbit & \hfuzz=500pt filename & \hfuzz=500pt epoch at timeStart is not used\\
\hfuzz=500pt\includegraphics[width=1em]{element-mustset.pdf}~margin & \hfuzz=500pt double & \hfuzz=500pt [seconds] to find identical times\\
\hfuzz=500pt\includegraphics[width=1em]{element.pdf}~recomputeForces & \hfuzz=500pt boolean & \hfuzz=500pt \\
\hline
\end{tabularx}

\clearpage
%==================================

\section{ParameterNames}\label{parameterNamesType}
Generate a list of parameter names. All parameters are appended.


\subsection{Name}
The parameter is given by explicitly by four parts:
\begin{enumerate}
\item object: Object this parameter refers to, e.g. \verb|graceA|, \verb|G023|, \verb|earth|, \ldots
\item type: Type of this parameter, e.g. \verb|accBias|, \verb|position.x|, \ldots
\item temporal: Temporal representation of this parameter, e.g. \verb|trend|, \verb|polynomial.degree1|, \ldots
\item interval: Interval/epoch this parameter represents, e.g. \verb|2017-01-01_00-00-00_2017-01-02_00-00-00|, \verb|2018-01-01_00-00-00|.
\end{enumerate}


\keepXColumns
\begin{tabularx}{\textwidth}{N T A}
\hline
Name & Type & Annotation\\
\hline
\hfuzz=500pt\includegraphics[width=1em]{element.pdf}~object & \hfuzz=500pt string & \hfuzz=500pt object this parameter refers to, e.g. graceA, G023, earth\\
\hfuzz=500pt\includegraphics[width=1em]{element.pdf}~type & \hfuzz=500pt string & \hfuzz=500pt type of this parameter, e.g. accBias, position.x\\
\hfuzz=500pt\includegraphics[width=1em]{element.pdf}~temporal & \hfuzz=500pt string & \hfuzz=500pt temporal representation of this parameter, e.g. trend, polynomial.degree1\\
\hfuzz=500pt\includegraphics[width=1em]{element.pdf}~interval & \hfuzz=500pt string & \hfuzz=500pt interval/epoch this parameter refers to, e.g. 2017-01-01\_00-00-00\_2017-01-02\_00-00-00, 2008-01-01\_00-00-00\\
\hline
\end{tabularx}


\subsection{File}
Read parameter names from \file{file}{parameterName}.


\keepXColumns
\begin{tabularx}{\textwidth}{N T A}
\hline
Name & Type & Annotation\\
\hline
\hfuzz=500pt\includegraphics[width=1em]{element-mustset.pdf}~inputfileParameterNames & \hfuzz=500pt filename & \hfuzz=500pt file with parameter names\\
\hline
\end{tabularx}


\subsection{Gravity}
Parameter names of gravity \configClass{parametrization}{parametrizationGravityType}.
An additional \config{object} name can be included in the parameter names.


\keepXColumns
\begin{tabularx}{\textwidth}{N T A}
\hline
Name & Type & Annotation\\
\hline
\hfuzz=500pt\includegraphics[width=1em]{element.pdf}~object & \hfuzz=500pt string & \hfuzz=500pt object these parameters refers to, e.g. earth\\
\hfuzz=500pt\includegraphics[width=1em]{element-mustset-unbounded.pdf}~parametrization & \hfuzz=500pt \hyperref[parametrizationGravityType]{parametrizationGravity} & \hfuzz=500pt \\
\hline
\end{tabularx}


\subsection{Acceleration}
Parameter names of satellite acceleration \configClass{parametrization}{parametrizationAccelerationType}.
Arc related parameters are appended if an \configFile{inputfileInstrument}{instrument} is provided which
defines the arc structure.
An additional \config{object} name can be included in the parameter names.


\keepXColumns
\begin{tabularx}{\textwidth}{N T A}
\hline
Name & Type & Annotation\\
\hline
\hfuzz=500pt\includegraphics[width=1em]{element.pdf}~object & \hfuzz=500pt string & \hfuzz=500pt object these parameters refers to, e.g. graceA, G023\\
\hfuzz=500pt\includegraphics[width=1em]{element-mustset-unbounded.pdf}~parameterization & \hfuzz=500pt \hyperref[parametrizationAccelerationType]{parametrizationAcceleration} & \hfuzz=500pt \\
\hfuzz=500pt\includegraphics[width=1em]{element.pdf}~inputfileInstrument & \hfuzz=500pt filename & \hfuzz=500pt defines the arc structure for arc related parameters\\
\hline
\end{tabularx}


\subsection{SatelliteTracking}
Parameter names of satellite tracking \configClass{parametrization}{parametrizationSatelliteTrackingType}.
An additional \config{object} name can be included in the parameter names.


\keepXColumns
\begin{tabularx}{\textwidth}{N T A}
\hline
Name & Type & Annotation\\
\hline
\hfuzz=500pt\includegraphics[width=1em]{element.pdf}~object & \hfuzz=500pt string & \hfuzz=500pt object these parameters refers to, e.g. grace1.grace2\\
\hfuzz=500pt\includegraphics[width=1em]{element-mustset-unbounded.pdf}~parameterization & \hfuzz=500pt \hyperref[parametrizationSatelliteTrackingType]{parametrizationSatelliteTracking} & \hfuzz=500pt \\
\hline
\end{tabularx}


\subsection{Temporal}
Parameter names from temporal parametrization.
It is possible to setup the temporal parameters for each \configClass{parameterNameBase}{parameterNamesType}.


\keepXColumns
\begin{tabularx}{\textwidth}{N T A}
\hline
Name & Type & Annotation\\
\hline
\hfuzz=500pt\includegraphics[width=1em]{element-unbounded.pdf}~parameterNameBase & \hfuzz=500pt \hyperref[parameterNamesType]{parameterNames} & \hfuzz=500pt \\
\hfuzz=500pt\includegraphics[width=1em]{element-mustset-unbounded.pdf}~parametrizationTemporal & \hfuzz=500pt \hyperref[parametrizationTemporalType]{parametrizationTemporal} & \hfuzz=500pt \\
\hline
\end{tabularx}


\subsection{GnssAntenna}
Parameter names of GNSS antenna center variation \configClass{parametrization}{parametrizationGnssAntennaType}.
An additional \config{object} name (antenna name) can be included in the parameter names.
It is possible to setup the parameters for each \configClass{gnssType}{gnssType}.


\keepXColumns
\begin{tabularx}{\textwidth}{N T A}
\hline
Name & Type & Annotation\\
\hline
\hfuzz=500pt\includegraphics[width=1em]{element.pdf}~object & \hfuzz=500pt string & \hfuzz=500pt antenna name\\
\hfuzz=500pt\includegraphics[width=1em]{element-mustset-unbounded.pdf}~parametrization & \hfuzz=500pt \hyperref[parametrizationGnssAntennaType]{parametrizationGnssAntenna} & \hfuzz=500pt \\
\hfuzz=500pt\includegraphics[width=1em]{element-unbounded.pdf}~gnssType & \hfuzz=500pt \hyperref[gnssType]{gnssType} & \hfuzz=500pt e.g. C1CG**\\
\hline
\end{tabularx}


\subsection{Observation}
Parameter names used in \configClass{observation equations}{observationType}.


\keepXColumns
\begin{tabularx}{\textwidth}{N T A}
\hline
Name & Type & Annotation\\
\hline
\hfuzz=500pt\includegraphics[width=1em]{element-mustset.pdf}~observation & \hfuzz=500pt \hyperref[observationType]{observation} & \hfuzz=500pt \\
\hline
\end{tabularx}


\subsection{Rename}
Replaces parts of \configClass{parameterName}{parameterNamesType}s.
The star "\verb|*|" left this part untouched.


\keepXColumns
\begin{tabularx}{\textwidth}{N T A}
\hline
Name & Type & Annotation\\
\hline
\hfuzz=500pt\includegraphics[width=1em]{element-mustset-unbounded.pdf}~parameterName & \hfuzz=500pt \hyperref[parameterNamesType]{parameterNames} & \hfuzz=500pt \\
\hfuzz=500pt\includegraphics[width=1em]{element.pdf}~object & \hfuzz=500pt string & \hfuzz=500pt *: left this part untouched, object\\
\hfuzz=500pt\includegraphics[width=1em]{element.pdf}~type & \hfuzz=500pt string & \hfuzz=500pt *: left this part untouched, type\\
\hfuzz=500pt\includegraphics[width=1em]{element.pdf}~temporal & \hfuzz=500pt string & \hfuzz=500pt *: left this part untouched, temporal representation\\
\hfuzz=500pt\includegraphics[width=1em]{element.pdf}~interval & \hfuzz=500pt string & \hfuzz=500pt *: left this part untouched, interval/epoch\\
\hline
\end{tabularx}


\subsection{Selection}
Select a subset of \configClass{parameterName}{parameterNamesType}s
using \configClass{parameterSelection}{parameterSelectorType}.


\keepXColumns
\begin{tabularx}{\textwidth}{N T A}
\hline
Name & Type & Annotation\\
\hline
\hfuzz=500pt\includegraphics[width=1em]{element-mustset-unbounded.pdf}~parameterName & \hfuzz=500pt \hyperref[parameterNamesType]{parameterNames} & \hfuzz=500pt \\
\hfuzz=500pt\includegraphics[width=1em]{element-mustset-unbounded.pdf}~parameterSelection & \hfuzz=500pt \hyperref[parameterSelectorType]{parameterSelector} & \hfuzz=500pt parameter order/selection\\
\hline
\end{tabularx}


\subsection{WithoutDuplicates}
Removes all duplicate names (keep first) from \configClass{parameterName}{parameterNamesType}.


\keepXColumns
\begin{tabularx}{\textwidth}{N T A}
\hline
Name & Type & Annotation\\
\hline
\hfuzz=500pt\includegraphics[width=1em]{element-mustset-unbounded.pdf}~parameterName & \hfuzz=500pt \hyperref[parameterNamesType]{parameterNames} & \hfuzz=500pt \\
\hline
\end{tabularx}

\clearpage
%==================================

\section{ParameterSelector}\label{parameterSelectorType}
This class provides an index vector from selected parameters,
which can be used e.g. to reorder a normal equation matrix.
The size of the index vector determines the size of the new matrix.
Entries are the indices of the selected parameters in the provided
parameter list or NULLINDEX for zero/new parameters.


\subsection{Wildcard}\label{parameterSelectorType:wildcard}
Parameter index vector from name. Name matching supports wildcards * for any number of characters and ? for exactly one character.
Does not add zero/empty parameters if there are no matches.


\keepXColumns
\begin{tabularx}{\textwidth}{N T A}
\hline
Name & Type & Annotation\\
\hline
\hfuzz=500pt\includegraphics[width=1em]{element.pdf}~object & \hfuzz=500pt string & \hfuzz=500pt object this parameter refers to, e.g. graceA, G023, earth (wildcards: * and ?)\\
\hfuzz=500pt\includegraphics[width=1em]{element.pdf}~type & \hfuzz=500pt string & \hfuzz=500pt type of this parameter, e.g. accBias, position.x (wildcards: * and ?)\\
\hfuzz=500pt\includegraphics[width=1em]{element.pdf}~temporal & \hfuzz=500pt string & \hfuzz=500pt temporal representation of this parameter, e.g. trend, polynomial.degree1 (wildcards: * and ?)\\
\hfuzz=500pt\includegraphics[width=1em]{element.pdf}~interval & \hfuzz=500pt string & \hfuzz=500pt interval/epoch this parameter refers to, e.g. 2017-01-01\_00-00-00\_2017-01-02\_00-00-00, 2008-01-01\_00-00-00 (wildcards: * and ?)\\
\hline
\end{tabularx}


\subsection{Names}
Parameter index vector from list of parameter names.


\keepXColumns
\begin{tabularx}{\textwidth}{N T A}
\hline
Name & Type & Annotation\\
\hline
\hfuzz=500pt\includegraphics[width=1em]{element-mustset-unbounded.pdf}~parameterName & \hfuzz=500pt \hyperref[parameterNamesType]{parameterNames} & \hfuzz=500pt \\
\hline
\end{tabularx}


\subsection{Range}
Parameter index vector from range.


\keepXColumns
\begin{tabularx}{\textwidth}{N T A}
\hline
Name & Type & Annotation\\
\hline
\hfuzz=500pt\includegraphics[width=1em]{element-mustset.pdf}~start & \hfuzz=500pt expression & \hfuzz=500pt start at this index (variables: length)\\
\hfuzz=500pt\includegraphics[width=1em]{element.pdf}~count & \hfuzz=500pt expression & \hfuzz=500pt count of parameters, default: all parameters to the end (variables: length)\\
\hline
\end{tabularx}


\subsection{Matrix}
Parameter index vector from matrix.


\keepXColumns
\begin{tabularx}{\textwidth}{N T A}
\hline
Name & Type & Annotation\\
\hline
\hfuzz=500pt\includegraphics[width=1em]{element-mustset.pdf}~inputfileMatrix & \hfuzz=500pt filename & \hfuzz=500pt index in old parameter list or -1 for new parameter\\
\hfuzz=500pt\includegraphics[width=1em]{element.pdf}~column & \hfuzz=500pt expression & \hfuzz=500pt use this column (counting from 0, variables: columns)\\
\hfuzz=500pt\includegraphics[width=1em]{element.pdf}~startRow & \hfuzz=500pt expression & \hfuzz=500pt start at this row (counting from 0, variables: rows)\\
\hfuzz=500pt\includegraphics[width=1em]{element.pdf}~countRows & \hfuzz=500pt expression & \hfuzz=500pt use these many rows (default: use all, variables: rows)\\
\hline
\end{tabularx}


\subsection{Zeros}
Expand parameter index vector by adding zero parameters.


\keepXColumns
\begin{tabularx}{\textwidth}{N T A}
\hline
Name & Type & Annotation\\
\hline
\hfuzz=500pt\includegraphics[width=1em]{element-mustset.pdf}~count & \hfuzz=500pt expression & \hfuzz=500pt count of zero parameters (variables: length)\\
\hline
\end{tabularx}


\subsection{Complement}\label{parameterSelectorType:complement}
Parameter index vector from a complement of other parameter selector(s).


\keepXColumns
\begin{tabularx}{\textwidth}{N T A}
\hline
Name & Type & Annotation\\
\hline
\hfuzz=500pt\includegraphics[width=1em]{element-mustset-unbounded.pdf}~parameterSelection & \hfuzz=500pt \hyperref[parameterSelectorType]{parameterSelector} & \hfuzz=500pt parameter order/selection\\
\hline
\end{tabularx}

\clearpage
%==================================

\section{ParametrizationAcceleration}\label{parametrizationAccelerationType}
This class defines parameters of satellite accelerations.
It will be used to set up the design matrix in a least squares adjustment.
If multiple parametrizations are given the coefficients in the parameter vector
are sequently appended.


\subsection{PerRevolution}\label{parametrizationAccelerationType:perRevolution}
Oscillation per revolution.


\keepXColumns
\begin{tabularx}{\textwidth}{N T A}
\hline
Name & Type & Annotation\\
\hline
\hfuzz=500pt\includegraphics[width=1em]{element-mustset.pdf}~order & \hfuzz=500pt uint & \hfuzz=500pt once, twice, ...\\
\hfuzz=500pt\includegraphics[width=1em]{element.pdf}~estimateX & \hfuzz=500pt boolean & \hfuzz=500pt along\\
\hfuzz=500pt\includegraphics[width=1em]{element.pdf}~estimateY & \hfuzz=500pt boolean & \hfuzz=500pt cross\\
\hfuzz=500pt\includegraphics[width=1em]{element.pdf}~estimateZ & \hfuzz=500pt boolean & \hfuzz=500pt radial\\
\hfuzz=500pt\includegraphics[width=1em]{element-unbounded.pdf}~interval & \hfuzz=500pt \hyperref[timeSeriesType]{timeSeries} & \hfuzz=500pt setup new parameters each interval\\
\hfuzz=500pt\includegraphics[width=1em]{element.pdf}~perArc & \hfuzz=500pt boolean & \hfuzz=500pt \\
\hline
\end{tabularx}


\subsection{AccBias}\label{parametrizationAccelerationType:accBias}
Temporal changing accelerometer bias per axis.


\keepXColumns
\begin{tabularx}{\textwidth}{N T A}
\hline
Name & Type & Annotation\\
\hline
\hfuzz=500pt\includegraphics[width=1em]{element.pdf}~estimateX & \hfuzz=500pt boolean & \hfuzz=500pt along\\
\hfuzz=500pt\includegraphics[width=1em]{element.pdf}~estimateY & \hfuzz=500pt boolean & \hfuzz=500pt cross\\
\hfuzz=500pt\includegraphics[width=1em]{element.pdf}~estimateZ & \hfuzz=500pt boolean & \hfuzz=500pt radial\\
\hfuzz=500pt\includegraphics[width=1em]{element-mustset-unbounded.pdf}~temporal & \hfuzz=500pt \hyperref[parametrizationTemporalType]{parametrizationTemporal} & \hfuzz=500pt \\
\hfuzz=500pt\includegraphics[width=1em]{element.pdf}~perArc & \hfuzz=500pt boolean & \hfuzz=500pt \\
\hline
\end{tabularx}


\subsection{AccelerometerScaleFactors}\label{parametrizationAccelerationType:accelerometerScaleFactors}
Accelerometer scale factor per axis.


\keepXColumns
\begin{tabularx}{\textwidth}{N T A}
\hline
Name & Type & Annotation\\
\hline
\hfuzz=500pt\includegraphics[width=1em]{element-mustset.pdf}~inputfileAccelerometer & \hfuzz=500pt filename & \hfuzz=500pt \\
\hfuzz=500pt\includegraphics[width=1em]{element.pdf}~estimateX & \hfuzz=500pt boolean & \hfuzz=500pt along\\
\hfuzz=500pt\includegraphics[width=1em]{element.pdf}~estimateY & \hfuzz=500pt boolean & \hfuzz=500pt cross\\
\hfuzz=500pt\includegraphics[width=1em]{element.pdf}~estimateZ & \hfuzz=500pt boolean & \hfuzz=500pt radial\\
\hfuzz=500pt\includegraphics[width=1em]{element.pdf}~estimateCrossTalk & \hfuzz=500pt boolean & \hfuzz=500pt non-orthognality of axes\\
\hfuzz=500pt\includegraphics[width=1em]{element.pdf}~estimateRotation & \hfuzz=500pt boolean & \hfuzz=500pt misalignment\\
\hfuzz=500pt\includegraphics[width=1em]{element-mustset-unbounded.pdf}~temporal & \hfuzz=500pt \hyperref[parametrizationTemporalType]{parametrizationTemporal} & \hfuzz=500pt \\
\hfuzz=500pt\includegraphics[width=1em]{element.pdf}~perArc & \hfuzz=500pt boolean & \hfuzz=500pt \\
\hline
\end{tabularx}


\subsection{GnssSolarRadiation}\label{parametrizationAccelerationType:gnssSolarRadiation}
GNSS solar radiation pressure model.


\keepXColumns
\begin{tabularx}{\textwidth}{N T A}
\hline
Name & Type & Annotation\\
\hline
\hfuzz=500pt\includegraphics[width=1em]{element.pdf}~estimateD0 & \hfuzz=500pt boolean & \hfuzz=500pt constant term along D-axis (sat-sun vector)\\
\hfuzz=500pt\includegraphics[width=1em]{element.pdf}~estimateD2 & \hfuzz=500pt boolean & \hfuzz=500pt 2-per-rev terms along D-axis\\
\hfuzz=500pt\includegraphics[width=1em]{element.pdf}~estimateD4 & \hfuzz=500pt boolean & \hfuzz=500pt 4-per-rev terms along D-axis\\
\hfuzz=500pt\includegraphics[width=1em]{element.pdf}~estimateY0 & \hfuzz=500pt boolean & \hfuzz=500pt constant term along Y-axis (solar panel axis)\\
\hfuzz=500pt\includegraphics[width=1em]{element.pdf}~estimateB0 & \hfuzz=500pt boolean & \hfuzz=500pt constant term along B-axis (cross product D x Y)\\
\hfuzz=500pt\includegraphics[width=1em]{element.pdf}~estimateB1 & \hfuzz=500pt boolean & \hfuzz=500pt 1-per-rev terms along B-axis\\
\hfuzz=500pt\includegraphics[width=1em]{element.pdf}~estimateB3 & \hfuzz=500pt boolean & \hfuzz=500pt 3-per-rev terms along B-axis\\
\hfuzz=500pt\includegraphics[width=1em]{element.pdf}~perArc & \hfuzz=500pt boolean & \hfuzz=500pt \\
\hfuzz=500pt\includegraphics[width=1em]{element-mustset.pdf}~eclipse & \hfuzz=500pt \hyperref[eclipseType]{eclipse} & \hfuzz=500pt \\
\hline
\end{tabularx}


\subsection{ThermosphericDensity}\label{parametrizationAccelerationType:thermosphericDensity}
Estimate the thermospheric density along the orbit using a satllite macro model.
An optional thermospheric model can be used to compute temperature and wind.
The temperature is used to estimate variable drag and lift coefficients, otherwise a constant drag coefficient is used.


\keepXColumns
\begin{tabularx}{\textwidth}{N T A}
\hline
Name & Type & Annotation\\
\hline
\hfuzz=500pt\includegraphics[width=1em]{element.pdf}~thermosphere & \hfuzz=500pt \hyperref[thermosphereType]{thermosphere} & \hfuzz=500pt for wind and temperature\\
\hfuzz=500pt\includegraphics[width=1em]{element.pdf}~earthRotation & \hfuzz=500pt double & \hfuzz=500pt [rad/s]\\
\hfuzz=500pt\includegraphics[width=1em]{element.pdf}~considerTemperature & \hfuzz=500pt boolean & \hfuzz=500pt compute drag and lift, otherwise simple drag coefficient is used\\
\hfuzz=500pt\includegraphics[width=1em]{element.pdf}~considerWind & \hfuzz=500pt boolean & \hfuzz=500pt \\
\hfuzz=500pt\includegraphics[width=1em]{element-mustset-unbounded.pdf}~temporalDensity & \hfuzz=500pt \hyperref[parametrizationTemporalType]{parametrizationTemporal} & \hfuzz=500pt parameters along orbit\\
\hfuzz=500pt\includegraphics[width=1em]{element.pdf}~perArc & \hfuzz=500pt boolean & \hfuzz=500pt \\
\hline
\end{tabularx}


\subsection{ModelScale}\label{parametrizationAccelerationType:modelScale}
Estimate a scale factor for a given model.


\keepXColumns
\begin{tabularx}{\textwidth}{N T A}
\hline
Name & Type & Annotation\\
\hline
\hfuzz=500pt\includegraphics[width=1em]{element-mustset-unbounded.pdf}~miscAccelerations & \hfuzz=500pt \hyperref[miscAccelerationsType]{miscAccelerations} & \hfuzz=500pt \\
\hfuzz=500pt\includegraphics[width=1em]{element-mustset-unbounded.pdf}~temporal & \hfuzz=500pt \hyperref[parametrizationTemporalType]{parametrizationTemporal} & \hfuzz=500pt \\
\hfuzz=500pt\includegraphics[width=1em]{element.pdf}~perArc & \hfuzz=500pt boolean & \hfuzz=500pt \\
\hline
\end{tabularx}

\clearpage
%==================================

\section{ParametrizationGnssAntenna}\label{parametrizationGnssAntennaType}
Parametrization of antenna center variations. It will be used to set up the design matrix in a least squares adjustment.
Usually the parametrization is setup separately for different \configClass{gnssType}{gnssType}.

If multiple parametrizations are given the parameters are sequently appended in the design matrix and parameter vector.


\subsection{Center}\label{parametrizationGnssAntennaType:center}
Antenna center or, if setup for a specific \configClass{gnssType}{gnssType},
phase/code center offset (e.g. \verb|*1*G| for GPS L1 phase center offset).


\keepXColumns
\begin{tabularx}{\textwidth}{N T A}
\hline
Name & Type & Annotation\\
\hline
\hfuzz=500pt\includegraphics[width=1em]{element.pdf}~estimateX & \hfuzz=500pt boolean & \hfuzz=500pt \\
\hfuzz=500pt\includegraphics[width=1em]{element.pdf}~estimateY & \hfuzz=500pt boolean & \hfuzz=500pt \\
\hfuzz=500pt\includegraphics[width=1em]{element.pdf}~estimateZ & \hfuzz=500pt boolean & \hfuzz=500pt \\
\hline
\end{tabularx}


\subsection{SphericalHarmonics}
Parametrization of antenna center variations in terms of spherical harmonics.
As usually only data above the horizon are observed only the even spherical harmonics
(degree/order $m+n$ even), which are symmetric to the equator, are setup.


\keepXColumns
\begin{tabularx}{\textwidth}{N T A}
\hline
Name & Type & Annotation\\
\hline
\hfuzz=500pt\includegraphics[width=1em]{element-mustset.pdf}~minDegree & \hfuzz=500pt uint & \hfuzz=500pt min degree\\
\hfuzz=500pt\includegraphics[width=1em]{element-mustset.pdf}~maxDegree & \hfuzz=500pt uint & \hfuzz=500pt max degree\\
\hline
\end{tabularx}


\subsection{RadialBasis}
Parametrization of antenna center variations with radial basis functions
\begin{equation}
  ACV(\M x(A, E)) = \sum_i a_i \Phi(\M x\cdot\M x_i)
\end{equation}
where $a_i$ the coefficients which has to be estimated and $\Phi$ are the basis
functions
\begin{equation}
  \Phi(\cos\psi) = \sum_n \sqrt{2n+1}P_n(\cos\psi).
\end{equation}

\fig{!hb}{0.4}{parametrizationGnssAntennaRadialBasis}{fig:parametrizationGnssAntennaRadialBasis}{Nodal points of the basis functions
using a Reuter grid for transmitting satellites (view angle of 18 deg). The red line indicates the view angle of 14 deg of ground stations.}


\keepXColumns
\begin{tabularx}{\textwidth}{N T A}
\hline
Name & Type & Annotation\\
\hline
\hfuzz=500pt\includegraphics[width=1em]{element-mustset-unbounded.pdf}~grid & \hfuzz=500pt \hyperref[gridType]{grid} & \hfuzz=500pt nodal points of shannon kernels\\
\hfuzz=500pt\includegraphics[width=1em]{element-mustset.pdf}~minDegree & \hfuzz=500pt uint & \hfuzz=500pt min degree of shannon kernel\\
\hfuzz=500pt\includegraphics[width=1em]{element-mustset.pdf}~maxDegree & \hfuzz=500pt uint & \hfuzz=500pt max degree of shannon kernel\\
\hline
\end{tabularx}

\clearpage
%==================================

\section{ParametrizationGravity}\label{parametrizationGravityType}
This class gives a parametrization of the time depending gravity field.
Together with the class \configClass{oberservation}{observationType} it will be used
to set up the design matrix in a least squares adjustment.
If multiple parametrizations are given the coefficients in the parameter vector
are sequently appended.


\subsection{SphericalHarmonics}\label{parametrizationGravityType:sphericalHarmonics}
The potential~$V$ is parametrized by a expansion of (fully normalized) spherical harmonics
\begin{equation}
V(\lambda,\vartheta,r) = \frac{GM}{R}\sum_{n=0}^\infty \sum_{m=0}^n \left(\frac{R}{r}\right)^{n+1}
  \left(c_{nm} C_{nm}(\lambda,\vartheta) + s_{nm} S_{nm}(\lambda,\vartheta)\right).
\end{equation}
You can set the range of degree~$n$ with \config{minDegree} and \config{maxDegree}.
The sorting sequence of the potential coefficients in the parameter vector can be defined by
\configClass{numbering}{sphericalHarmonicsNumberingType}.

The total count of parameters is $(n_{max}+1)^2-n_{min}^2$.


\keepXColumns
\begin{tabularx}{\textwidth}{N T A}
\hline
Name & Type & Annotation\\
\hline
\hfuzz=500pt\includegraphics[width=1em]{element-mustset.pdf}~minDegree & \hfuzz=500pt uint & \hfuzz=500pt \\
\hfuzz=500pt\includegraphics[width=1em]{element-mustset.pdf}~maxDegree & \hfuzz=500pt uint & \hfuzz=500pt \\
\hfuzz=500pt\includegraphics[width=1em]{element.pdf}~GM & \hfuzz=500pt double & \hfuzz=500pt Geocentric gravitational constant\\
\hfuzz=500pt\includegraphics[width=1em]{element.pdf}~R & \hfuzz=500pt double & \hfuzz=500pt reference radius\\
\hfuzz=500pt\includegraphics[width=1em]{element-mustset.pdf}~numbering & \hfuzz=500pt \hyperref[sphericalHarmonicsNumberingType]{sphericalHarmonicsNumbering} & \hfuzz=500pt numbering scheme\\
\hline
\end{tabularx}


\subsection{RadialBasis}\label{parametrizationGravityType:radialBasis}
The potential~$V$ is represented by a sum of space localizing basis functions
\begin{equation}
  V(\M x) = \sum_i a_i \Phi(\M x, \M x_i)
\end{equation}
where $a_i$ the coefficients which has to be estimated and $\Phi$ are the basis
functions given by isotropic radial \configClass{kernel}{kernelType} functions
\begin{equation}
  \Phi(\cos\psi,r,R) = \sum_n \left(\frac{R}{r}\right)^{n+1} k_n\sqrt{2n+1}\bar{P}_n(\cos\psi).
\end{equation}
The basis functions are located on a grid~$\M x_i$ given by \configClass{grid}{gridType}.
This class can also be used to estimate point masses if \configClass{kernel}{kernelType} is set to density.


\keepXColumns
\begin{tabularx}{\textwidth}{N T A}
\hline
Name & Type & Annotation\\
\hline
\hfuzz=500pt\includegraphics[width=1em]{element-mustset.pdf}~kernel & \hfuzz=500pt \hyperref[kernelType]{kernel} & \hfuzz=500pt shape of the radial basis function\\
\hfuzz=500pt\includegraphics[width=1em]{element-mustset-unbounded.pdf}~grid & \hfuzz=500pt \hyperref[gridType]{grid} & \hfuzz=500pt nodal point distribution\\
\hline
\end{tabularx}


\subsection{Temporal}
The time variable potential is given by
\begin{equation}
  V(\M x,t) = \sum_i V_i(\M x)\Psi_i(t),
\end{equation}
wehre $V_i(\M x)$ is the spatial parametrization of the gravity field
and can be choosen with \configClass{parametrizationGravity}{parametrizationGravityType}.
The parametrization in time domain $\Psi_i(t)$ is selected by
\configClass{parametrizationTemporal}{parametrizationTemporalType}.
The total parameter count is the parameter count of \configClass{parametrizationTemporal}{parametrizationTemporalType}
times the parameter count of \configClass{parametrizationGravity}{parametrizationGravityType}.


\keepXColumns
\begin{tabularx}{\textwidth}{N T A}
\hline
Name & Type & Annotation\\
\hline
\hfuzz=500pt\includegraphics[width=1em]{element-mustset-unbounded.pdf}~parametrizationGravity & \hfuzz=500pt \hyperref[parametrizationGravityType]{parametrizationGravity} & \hfuzz=500pt \\
\hfuzz=500pt\includegraphics[width=1em]{element-mustset-unbounded.pdf}~parametrizationTemporal & \hfuzz=500pt \hyperref[parametrizationTemporalType]{parametrizationTemporal} & \hfuzz=500pt \\
\hline
\end{tabularx}


\subsection{LinearTransformation}
Parametrization of the gravity field on the basis of a linear transformation of a source parametrization.
The linear transformation changes the original solution space represented by
\configClass{pararametrizationGravitySource}{parametrizationGravityType} from
\begin{equation}
  \mathbf{l} = \mathbf{A}\mathbf{x} + \mathbf{e}
\end{equation}
to
\begin{equation}
  \mathbf{l} = \mathbf{A}\mathbf{F}\mathbf{y} + \mathbf{e}
\end{equation}
through the linear transformation $\mathbf{x}=\mathbf{F}\mathbf{y}$.
It follows that the rows of the matrix $\mathbf{F}$ in \file{inputfileTransformationMatrix}{matrix} coincides with
the number of parameters in \configClass{pararametrizationGravitySource}{parametrizationGravityType}.
The new parameter count is given by the number of columns in $\mathbf{F}$ and may be smaller, equal or larger
than the original parameter count.


\keepXColumns
\begin{tabularx}{\textwidth}{N T A}
\hline
Name & Type & Annotation\\
\hline
\hfuzz=500pt\includegraphics[width=1em]{element-mustset-unbounded.pdf}~parametrizationGravitySource & \hfuzz=500pt \hyperref[parametrizationGravityType]{parametrizationGravity} & \hfuzz=500pt \\
\hfuzz=500pt\includegraphics[width=1em]{element-mustset.pdf}~inputfileTransformationMatrix & \hfuzz=500pt filename & \hfuzz=500pt transformation matrix from target to source parametrization (rows of this matrix must coincide with the parameter count of the source parametrization)\\
\hline
\end{tabularx}


\subsection{EarthquakeOscillation}
This class is used to estimate the earthquake oscillation function parameters,
i.e. $C_{nlm}$, $\omega_{nlm}$, and $P_{nlm}$.
The results describes the variation in the gravitational potential field caused by large earthquakes.
\begin{equation}
C_{lm}(\M t) = \sum_{n=0}^NC_{nlm}(1-\cos(\omega_{nlm}d\M t)\exp(P_{nlm}\omega_{nlm}d\M t)),
\end{equation}
with $\omega_{nlm}=\frac{2\pi}{T_{nlm}}$ and $P_{nlm}=\frac{-1}{2Q_{nlm}}$ . In this equation, $Q_{nlm}$ is the attenuation factor,
$n$ is the overtone factor, $m$ is degree, $l$ is order, and $t$ is time after earthquake in second.


\keepXColumns
\begin{tabularx}{\textwidth}{N T A}
\hline
Name & Type & Annotation\\
\hline
\hfuzz=500pt\includegraphics[width=1em]{element-mustset.pdf}~inputInitialCoefficient & \hfuzz=500pt filename & \hfuzz=500pt initial values for oscillation parameters\\
\hfuzz=500pt\includegraphics[width=1em]{element-mustset.pdf}~time0 & \hfuzz=500pt time & \hfuzz=500pt the time earthquake happened\\
\hfuzz=500pt\includegraphics[width=1em]{element-mustset.pdf}~minDegree & \hfuzz=500pt uint & \hfuzz=500pt \\
\hfuzz=500pt\includegraphics[width=1em]{element-mustset.pdf}~maxDegree & \hfuzz=500pt uint & \hfuzz=500pt \\
\hfuzz=500pt\includegraphics[width=1em]{element.pdf}~GM & \hfuzz=500pt double & \hfuzz=500pt Geocentric gravitational constant\\
\hfuzz=500pt\includegraphics[width=1em]{element.pdf}~R & \hfuzz=500pt double & \hfuzz=500pt reference radius\\
\hfuzz=500pt\includegraphics[width=1em]{element-mustset.pdf}~numbering & \hfuzz=500pt \hyperref[sphericalHarmonicsNumberingType]{sphericalHarmonicsNumbering} & \hfuzz=500pt numbering scheme\\
\hline
\end{tabularx}

\clearpage
%==================================

\section{ParametrizationSatelliteTracking}\label{parametrizationSatelliteTrackingType}
This class defines parameters of Satellite-to-Satellite tracking observations.
It will be used to set up the design matrix in a least squares adjustment.
If multiple parametrizations are given the coefficients in the parameter vector
are sequently appended.


\subsection{AntennaCenter}\label{parametrizationSatelliteTrackingType:antennaCenter}
Estimate the KBR antenna phase centre (APC) coordinates for each spacecraft in satellite reference frame (SRF)
as constant per axis, once per month. The observation equations are computed by taking the derivative
of the antenna offset correction equation w.r.t. the KBR APC coordinates.


\keepXColumns
\begin{tabularx}{\textwidth}{N T A}
\hline
Name & Type & Annotation\\
\hline
\hfuzz=500pt\includegraphics[width=1em]{element.pdf}~estimate1X & \hfuzz=500pt boolean & \hfuzz=500pt along (satellite 1)\\
\hfuzz=500pt\includegraphics[width=1em]{element.pdf}~estimate1Y & \hfuzz=500pt boolean & \hfuzz=500pt cross (satellite 1)\\
\hfuzz=500pt\includegraphics[width=1em]{element.pdf}~estimate1Z & \hfuzz=500pt boolean & \hfuzz=500pt nadir (satellite 1)\\
\hfuzz=500pt\includegraphics[width=1em]{element.pdf}~estimate2X & \hfuzz=500pt boolean & \hfuzz=500pt along (satellite 2)\\
\hfuzz=500pt\includegraphics[width=1em]{element.pdf}~estimate2Y & \hfuzz=500pt boolean & \hfuzz=500pt cross (satellite 2)\\
\hfuzz=500pt\includegraphics[width=1em]{element.pdf}~estimate2Z & \hfuzz=500pt boolean & \hfuzz=500pt nadir (satellite 2)\\
\hfuzz=500pt\includegraphics[width=1em]{element.pdf}~interpolationDegree & \hfuzz=500pt uint & \hfuzz=500pt differentiation by polynomial approximation of degree n\\
\hline
\end{tabularx}


\subsection{Bias}\label{parametrizationSatelliteTrackingType:bias}
Estimate bias for SST observations. The temporal variation is defined by \configClass{parametrizationTemporal}{parametrizationTemporalType}.


\keepXColumns
\begin{tabularx}{\textwidth}{N T A}
\hline
Name & Type & Annotation\\
\hline
\hfuzz=500pt\includegraphics[width=1em]{element-mustset-unbounded.pdf}~temporal & \hfuzz=500pt \hyperref[parametrizationTemporalType]{parametrizationTemporal} & \hfuzz=500pt \\
\hfuzz=500pt\includegraphics[width=1em]{element.pdf}~perArc & \hfuzz=500pt boolean & \hfuzz=500pt \\
\hline
\end{tabularx}


\subsection{Scale}\label{parametrizationSatelliteTrackingType:scale}
Estimate scale factor for SST observations with respect to reference SST data \configFile{inputfileSatelliteTracking}{instrument}. The temporal variation is defined by \configClass{parametrizationTemporal}{parametrizationTemporalType}.


\keepXColumns
\begin{tabularx}{\textwidth}{N T A}
\hline
Name & Type & Annotation\\
\hline
\hfuzz=500pt\includegraphics[width=1em]{element-mustset.pdf}~inputfileSatelliteTracking & \hfuzz=500pt filename & \hfuzz=500pt \\
\hfuzz=500pt\includegraphics[width=1em]{element-mustset-unbounded.pdf}~temporal & \hfuzz=500pt \hyperref[parametrizationTemporalType]{parametrizationTemporal} & \hfuzz=500pt \\
\hfuzz=500pt\includegraphics[width=1em]{element.pdf}~perArc & \hfuzz=500pt boolean & \hfuzz=500pt \\
\hline
\end{tabularx}


\subsection{TimeBias}\label{parametrizationSatelliteTrackingType:timeBias}
Estimate time shift in seconds in SST observations, with defined temporal variation by \configClass{parametrizationTemporal}{parametrizationTemporalType}. The design matrix is computed by taking the derivative of the ranging data w.r.t. time.


\keepXColumns
\begin{tabularx}{\textwidth}{N T A}
\hline
Name & Type & Annotation\\
\hline
\hfuzz=500pt\includegraphics[width=1em]{element-mustset.pdf}~polynomialDegree & \hfuzz=500pt uint & \hfuzz=500pt polynomial degree\\
\hfuzz=500pt\includegraphics[width=1em]{element-mustset-unbounded.pdf}~temporal & \hfuzz=500pt \hyperref[parametrizationTemporalType]{parametrizationTemporal} & \hfuzz=500pt \\
\hfuzz=500pt\includegraphics[width=1em]{element.pdf}~perArc & \hfuzz=500pt boolean & \hfuzz=500pt \\
\hline
\end{tabularx}


\subsection{ScaleModel}\label{parametrizationSatelliteTrackingType:scaleModel}
Estimate scale factors for deterministic signal models from satellite tracking instrument file \configFile{inputfileSatelliteTracking}{instrument}, see \program{EnsembleAveragingScaleModel}.
Amplitude variation of model waveforms is defined by \configClass{parametrizationTemporal}{parametrizationTemporalType}.


\keepXColumns
\begin{tabularx}{\textwidth}{N T A}
\hline
Name & Type & Annotation\\
\hline
\hfuzz=500pt\includegraphics[width=1em]{element-mustset.pdf}~inputfileSatelliteTracking & \hfuzz=500pt filename & \hfuzz=500pt \\
\hfuzz=500pt\includegraphics[width=1em]{element-mustset-unbounded.pdf}~temporal & \hfuzz=500pt \hyperref[parametrizationTemporalType]{parametrizationTemporal} & \hfuzz=500pt \\
\hfuzz=500pt\includegraphics[width=1em]{element.pdf}~perArc & \hfuzz=500pt boolean & \hfuzz=500pt \\
\hline
\end{tabularx}


\subsection{SpecialEffect}\label{parametrizationSatelliteTrackingType:specialEffect}
Estimate deterministic signals in the GRACE K-Band measurements caused by Sun intrusions
into the star camera baffles of GRACE-A and eclipse transits of the satellites.
These events can be time-indexed beforehand using satellite position and orientation,
see \program{GraceSstSpecialEvents}. The shape of this short-period waveform is nearly
constant within one month and can be approximated by a polynomial.
The amplitude variation of the waveform can also be taken into account
by \configClass{parametrizationTemporal}{parametrizationTemporalType}.


\keepXColumns
\begin{tabularx}{\textwidth}{N T A}
\hline
Name & Type & Annotation\\
\hline
\hfuzz=500pt\includegraphics[width=1em]{element-mustset.pdf}~inputfileEvents & \hfuzz=500pt filename & \hfuzz=500pt instrument with GRACE events\\
\hfuzz=500pt\includegraphics[width=1em]{element-mustset.pdf}~type & \hfuzz=500pt choice & \hfuzz=500pt \\
\hfuzz=500pt\includegraphics[width=1em]{connector.pdf}\includegraphics[width=1em]{element-mustset.pdf}~eclipse1 & \hfuzz=500pt  & \hfuzz=500pt \\
\hfuzz=500pt\includegraphics[width=1em]{connector.pdf}\includegraphics[width=1em]{element-mustset.pdf}~eclipse2 & \hfuzz=500pt  & \hfuzz=500pt \\
\hfuzz=500pt\includegraphics[width=1em]{connector.pdf}\includegraphics[width=1em]{element-mustset.pdf}~starCameraBox1 & \hfuzz=500pt  & \hfuzz=500pt \\
\hfuzz=500pt\includegraphics[width=1em]{connector.pdf}\includegraphics[width=1em]{element-mustset.pdf}~starCameraBox2 & \hfuzz=500pt  & \hfuzz=500pt \\
\hfuzz=500pt\includegraphics[width=1em]{connector.pdf}\includegraphics[width=1em]{element-mustset.pdf}~starCameraBox3 & \hfuzz=500pt  & \hfuzz=500pt \\
\hfuzz=500pt\includegraphics[width=1em]{connector.pdf}\includegraphics[width=1em]{element-mustset.pdf}~starCameraBox4 & \hfuzz=500pt  & \hfuzz=500pt \\
\hfuzz=500pt\includegraphics[width=1em]{connector.pdf}\includegraphics[width=1em]{element-mustset.pdf}~starCameraBox5 & \hfuzz=500pt  & \hfuzz=500pt \\
\hfuzz=500pt\includegraphics[width=1em]{connector.pdf}\includegraphics[width=1em]{element-mustset.pdf}~starCameraBox6 & \hfuzz=500pt  & \hfuzz=500pt \\
\hfuzz=500pt\includegraphics[width=1em]{element-mustset.pdf}~marginLeft & \hfuzz=500pt double & \hfuzz=500pt margin size (on both sides) [seconds]\\
\hfuzz=500pt\includegraphics[width=1em]{element-mustset.pdf}~marginRight & \hfuzz=500pt double & \hfuzz=500pt margin size (on both sides) [seconds]\\
\hfuzz=500pt\includegraphics[width=1em]{element.pdf}~minNumberOfEvents & \hfuzz=500pt uint & \hfuzz=500pt min. number of events to setup parameters\\
\hfuzz=500pt\includegraphics[width=1em]{element-mustset.pdf}~polynomialDegree & \hfuzz=500pt uint & \hfuzz=500pt polynomial degree\\
\hfuzz=500pt\includegraphics[width=1em]{element-mustset-unbounded.pdf}~temporal & \hfuzz=500pt \hyperref[parametrizationTemporalType]{parametrizationTemporal} & \hfuzz=500pt \\
\hline
\end{tabularx}

\clearpage
%==================================

\section{ParametrizationTemporal}\label{parametrizationTemporalType}
This class gives a parametrization of time depending parameters (gravity field, positions, ...).
It will be used to set up the design matrix in a least squares adjustment.
If multiple parametrizations are given the coefficients in the parameter vector
are sequently appended.

Useally time intervals are defined half open meaning the last time belongs not to the interval.
This behaviour can be changed for the last interval with \config{includeLastTime}.


\subsection{Constant}\label{parametrizationTemporalType:constant}
Represents a parameter being constant in time in each \config{interval}.


\keepXColumns
\begin{tabularx}{\textwidth}{N T A}
\hline
Name & Type & Annotation\\
\hline
\hfuzz=500pt\includegraphics[width=1em]{element-unbounded.pdf}~interval & \hfuzz=500pt \hyperref[timeSeriesType]{timeSeries} & \hfuzz=500pt \\
\hfuzz=500pt\includegraphics[width=1em]{element.pdf}~includeLastTime & \hfuzz=500pt boolean & \hfuzz=500pt \\
\hline
\end{tabularx}


\subsection{Trend}\label{parametrizationTemporalType:trend}
A time variable function is given by a linear trend
\begin{equation}
f(x,t) = \frac{1}{T}(t-t_0) \cdot f_t(x),
\end{equation}
with $t_0$ is \config{timeStart} and $T$ is \config{timeStep} in days.
A constant term is not included and must added separately.


\keepXColumns
\begin{tabularx}{\textwidth}{N T A}
\hline
Name & Type & Annotation\\
\hline
\hfuzz=500pt\includegraphics[width=1em]{element-mustset.pdf}~timeStart & \hfuzz=500pt time & \hfuzz=500pt reference time\\
\hfuzz=500pt\includegraphics[width=1em]{element-mustset.pdf}~timeStep & \hfuzz=500pt time & \hfuzz=500pt \\
\hline
\end{tabularx}


\subsection{Splines}\label{parametrizationTemporalType:splines}
A time variable function is given by
\begin{equation}
f(x,t) =  \sum_i f_i(x)\Psi_i(t),
\end{equation}
with the (spatial) coefficients $f_i(x)$ as parameters and the temporal basis functions~$\Psi_i(t)$.
Basis splines are defined as polynomials of degree~$n$ in intervals between nodal points in time $t_i$,
for details see~\reference{basis splines}{fundamentals.basisSplines}.

The parameters are ordered timewise. First all parameters of $f_{i=1}(x)$ then
$f_{i=2}(x)$ and so on. The total parameter count in each \config{interval} is $N=N_t+d-1$,
where $N_t$ is the count of time points from \configClass{timeSeries}{timeSeriesType} in each interval and $d$
is the \config{degree}.


\keepXColumns
\begin{tabularx}{\textwidth}{N T A}
\hline
Name & Type & Annotation\\
\hline
\hfuzz=500pt\includegraphics[width=1em]{element-mustset.pdf}~degree & \hfuzz=500pt uint & \hfuzz=500pt degree of splines\\
\hfuzz=500pt\includegraphics[width=1em]{element-mustset-unbounded.pdf}~timeSeries & \hfuzz=500pt \hyperref[timeSeriesType]{timeSeries} & \hfuzz=500pt nodal points in time domain\\
\hfuzz=500pt\includegraphics[width=1em]{element-unbounded.pdf}~intervals & \hfuzz=500pt \hyperref[timeSeriesType]{timeSeries} & \hfuzz=500pt \\
\hfuzz=500pt\includegraphics[width=1em]{element.pdf}~includeLastTime & \hfuzz=500pt boolean & \hfuzz=500pt \\
\hline
\end{tabularx}


\subsection{Polynomial}
A time variable function is represented by Legendre polynomials in each \config{interval}.
The time is normed to $[-1,1)$ in each interval.

The total parameter count is $(N+1)M$,
where $N$ is the polynmial degree and $M$ the number of intervals.


\keepXColumns
\begin{tabularx}{\textwidth}{N T A}
\hline
Name & Type & Annotation\\
\hline
\hfuzz=500pt\includegraphics[width=1em]{element-mustset.pdf}~polynomialDegree & \hfuzz=500pt uint & \hfuzz=500pt polynomial degree\\
\hfuzz=500pt\includegraphics[width=1em]{element-unbounded.pdf}~interval & \hfuzz=500pt \hyperref[timeSeriesType]{timeSeries} & \hfuzz=500pt intervals of polynomials\\
\hfuzz=500pt\includegraphics[width=1em]{element.pdf}~includeLastTime & \hfuzz=500pt boolean & \hfuzz=500pt \\
\hline
\end{tabularx}


\subsection{Oscillation}
A time variable function is given by a oscillation
\begin{equation}
f(x,t) = f^c(\M x)\cos(\omega_i(t)) + f^s(\M x)\sin(\omega_i(t))
\end{equation}
with $\omega_i=\frac{2\pi}{T_i}(t-t_0)$,
$t_0$ is \config{timeStart} and $T$ is \config{timePeriod} in days.


\keepXColumns
\begin{tabularx}{\textwidth}{N T A}
\hline
Name & Type & Annotation\\
\hline
\hfuzz=500pt\includegraphics[width=1em]{element-mustset-unbounded.pdf}~period & \hfuzz=500pt time & \hfuzz=500pt [day]\\
\hfuzz=500pt\includegraphics[width=1em]{element-mustset.pdf}~time0 & \hfuzz=500pt time & \hfuzz=500pt reference time\\
\hline
\end{tabularx}


\subsection{Fourier}
A time variable function is given by a fourier expansion
\begin{equation}
f(x,t) = \sum_{m=1}^M f_m^c(\M x)\cos(2\pi m \tau) + f_m^s(\M x)\sin(2\pi m \tau)
\end{equation}
with the normalized time
\begin{equation}
\tau = \frac{t-t_A}{t_B-t_A},
\end{equation}
and $t_A$ is \config{timeStart}, $t_B$ is \config{timeEnd} in each \config{interval}
and $M$ is the \config{fourierDegree}.

The total parameter count is $2MN$, where $N$ is the number of intervals.
The parameters are sorted in following order: $f_1^c, f_1^s, f_2^c, \ldots$.


\keepXColumns
\begin{tabularx}{\textwidth}{N T A}
\hline
Name & Type & Annotation\\
\hline
\hfuzz=500pt\includegraphics[width=1em]{element-mustset.pdf}~fourierDegree & \hfuzz=500pt uint & \hfuzz=500pt \\
\hfuzz=500pt\includegraphics[width=1em]{element-unbounded.pdf}~interval & \hfuzz=500pt \hyperref[timeSeriesType]{timeSeries} & \hfuzz=500pt intervals of fourier series\\
\hfuzz=500pt\includegraphics[width=1em]{element.pdf}~includeLastTime & \hfuzz=500pt boolean & \hfuzz=500pt \\
\hline
\end{tabularx}


\subsection{DoodsonHarmonic}
The time variable function is given by a fourier expansion
\begin{equation}
  f(x,t) = \sum_{i} f_i^c(x)\cos(\Theta_i(t)) + f_i^s(x)\sin(\Theta_i(t)),
\end{equation}
where $\Theta_i(t)$ are the arguments of the tide constituents $i$
\begin{equation}
  \Theta_i(t) = \sum_{k=1}^6 n_i^k\beta_k(t),
\end{equation}
where $\beta_k(t)$ are the Doodson's fundamental arguments ($\tau,s,h,p,N',p_s$) and $n_i^k$
are the Doodson multipliers for the term at frequency~$i$.
The multipliers must be given by \configClass{doodson}{doodson} coded as Doodson number
(e.g. 255.555) or as names intoduced by Darwin (e.g. M2).

The major constituents given by \configClass{doodson}{doodson} can be used to interpolate minor tidal constituents
using the file \configFile{inputfileAdmittance}{admittance}. This file can be created with
\program{DoodsonHarmonicsCalculateAdmittance}.

The total parameter count is $2N$ with $N$ the number of doodson frequencies.
The parameters are sorted in following order: $f_1^c, f_1^s, f_2^c, \ldots$.


\keepXColumns
\begin{tabularx}{\textwidth}{N T A}
\hline
Name & Type & Annotation\\
\hline
\hfuzz=500pt\includegraphics[width=1em]{element-mustset-unbounded.pdf}~doodson & \hfuzz=500pt \hyperref[doodson]{doodson} & \hfuzz=500pt code number (e.g. 255.555) or darwin name (e.g. M2)\\
\hfuzz=500pt\includegraphics[width=1em]{element.pdf}~inputfileAdmittance & \hfuzz=500pt filename & \hfuzz=500pt interpolation of minor constituents\\
\hline
\end{tabularx}

\clearpage
%==================================

\section{Planet}\label{planetType}
Defines the planet to compute the \configClass{ephemeris}{ephemeridesType}.


\keepXColumns
\begin{tabularx}{\textwidth}{N T A}
\hline
Name & Type & Annotation\\
\hline
\hfuzz=500pt\includegraphics[width=1em]{element-mustset.pdf}~planetType & \hfuzz=500pt choice & \hfuzz=500pt planet\\
\hfuzz=500pt\includegraphics[width=1em]{connector.pdf}\includegraphics[width=1em]{element-mustset.pdf}~earth & \hfuzz=500pt  & \hfuzz=500pt \\
\hfuzz=500pt\includegraphics[width=1em]{connector.pdf}\includegraphics[width=1em]{element-mustset.pdf}~sun & \hfuzz=500pt  & \hfuzz=500pt \\
\hfuzz=500pt\includegraphics[width=1em]{connector.pdf}\includegraphics[width=1em]{element-mustset.pdf}~moon & \hfuzz=500pt  & \hfuzz=500pt \\
\hfuzz=500pt\includegraphics[width=1em]{connector.pdf}\includegraphics[width=1em]{element-mustset.pdf}~mercury & \hfuzz=500pt  & \hfuzz=500pt \\
\hfuzz=500pt\includegraphics[width=1em]{connector.pdf}\includegraphics[width=1em]{element-mustset.pdf}~venus & \hfuzz=500pt  & \hfuzz=500pt \\
\hfuzz=500pt\includegraphics[width=1em]{connector.pdf}\includegraphics[width=1em]{element-mustset.pdf}~mars & \hfuzz=500pt  & \hfuzz=500pt \\
\hfuzz=500pt\includegraphics[width=1em]{connector.pdf}\includegraphics[width=1em]{element-mustset.pdf}~jupiter & \hfuzz=500pt  & \hfuzz=500pt \\
\hfuzz=500pt\includegraphics[width=1em]{connector.pdf}\includegraphics[width=1em]{element-mustset.pdf}~saturn & \hfuzz=500pt  & \hfuzz=500pt \\
\hfuzz=500pt\includegraphics[width=1em]{connector.pdf}\includegraphics[width=1em]{element-mustset.pdf}~uranus & \hfuzz=500pt  & \hfuzz=500pt \\
\hfuzz=500pt\includegraphics[width=1em]{connector.pdf}\includegraphics[width=1em]{element-mustset.pdf}~neptune & \hfuzz=500pt  & \hfuzz=500pt \\
\hfuzz=500pt\includegraphics[width=1em]{connector.pdf}\includegraphics[width=1em]{element-mustset.pdf}~pluto & \hfuzz=500pt  & \hfuzz=500pt \\
\hfuzz=500pt\includegraphics[width=1em]{connector.pdf}\includegraphics[width=1em]{element-mustset.pdf}~solarBaryCenter & \hfuzz=500pt  & \hfuzz=500pt \\
\hfuzz=500pt\includegraphics[width=1em]{connector.pdf}\includegraphics[width=1em]{element-mustset.pdf}~earthMoonBaryCenter & \hfuzz=500pt  & \hfuzz=500pt \\
\hline
\end{tabularx}

\clearpage
%==================================

\section{PlatformSelector}\label{platformSelectorType}
Select a list of platforms (stations, satellites, ...).

See also \program{GnssProcessing}.


\subsection{All}\label{platformSelectorType:all}
Select all platforms.


\subsection{Wildcard}\label{platformSelectorType:wildcard}
Select all receivers/transmitters which match the
\config{name}, \config{markerName}, and \config{markerNumber}.


\keepXColumns
\begin{tabularx}{\textwidth}{N T A}
\hline
Name & Type & Annotation\\
\hline
\hfuzz=500pt\includegraphics[width=1em]{element.pdf}~name & \hfuzz=500pt string & \hfuzz=500pt wildcards: * and ?\\
\hfuzz=500pt\includegraphics[width=1em]{element.pdf}~markerName & \hfuzz=500pt string & \hfuzz=500pt wildcards: * and ?, from platform\\
\hfuzz=500pt\includegraphics[width=1em]{element.pdf}~markerNumber & \hfuzz=500pt string & \hfuzz=500pt wildcards: * and ?, from platform\\
\hline
\end{tabularx}


\subsection{File}\label{platformSelectorType:file}
Select receivers/transmitters from each row of
\configFile{inputfileStringTable}{stringTable}.
Additional columns in a row represent alternatives
if previous names are not available (e.g. without observation file).


\keepXColumns
\begin{tabularx}{\textwidth}{N T A}
\hline
Name & Type & Annotation\\
\hline
\hfuzz=500pt\includegraphics[width=1em]{element-mustset.pdf}~inputfileStringTable & \hfuzz=500pt filename & \hfuzz=500pt list of names with alternatives\\
\hline
\end{tabularx}


\subsection{Exclude}\label{platformSelectorType:exclude}
Select all receivers/transmitters except
\configClass{selector}{platformSelectorType}.


\keepXColumns
\begin{tabularx}{\textwidth}{N T A}
\hline
Name & Type & Annotation\\
\hline
\hfuzz=500pt\includegraphics[width=1em]{element-mustset-unbounded.pdf}~selector & \hfuzz=500pt \hyperref[platformSelectorType]{platformSelector} & \hfuzz=500pt \\
\hline
\end{tabularx}

\clearpage
%==================================

\section{PlotAxis}\label{plotAxisType}
Defines the style of the axes of \program{PlotGraph}.


\subsection{Standard}
General axis for arbitrary input data.


\keepXColumns
\begin{tabularx}{\textwidth}{N T A}
\hline
Name & Type & Annotation\\
\hline
\hfuzz=500pt\includegraphics[width=1em]{element.pdf}~min & \hfuzz=500pt double & \hfuzz=500pt The minimum value of the axis. If no value is given, the minimum scale value is set automatically.\\
\hfuzz=500pt\includegraphics[width=1em]{element.pdf}~max & \hfuzz=500pt double & \hfuzz=500pt The maximum value of the axis. If no value is given, the maximum scale value is set automatically.\\
\hfuzz=500pt\includegraphics[width=1em]{element.pdf}~majorTickSpacing & \hfuzz=500pt double & \hfuzz=500pt The boundary annotation.\\
\hfuzz=500pt\includegraphics[width=1em]{element.pdf}~minorTickSpacing & \hfuzz=500pt double & \hfuzz=500pt The spacing of the frame tick intervals.\\
\hfuzz=500pt\includegraphics[width=1em]{element.pdf}~gridLineSpacing & \hfuzz=500pt double & \hfuzz=500pt The spacing of the grid line intervals\\
\hfuzz=500pt\includegraphics[width=1em]{element.pdf}~gridLine & \hfuzz=500pt \hyperref[plotLineType]{plotLine} & \hfuzz=500pt The style of the grid lines.\\
\hfuzz=500pt\includegraphics[width=1em]{element.pdf}~unit & \hfuzz=500pt string & \hfuzz=500pt Naming unit to append to the axis values.\\
\hfuzz=500pt\includegraphics[width=1em]{element.pdf}~label & \hfuzz=500pt string & \hfuzz=500pt The description of the axis.\\
\hfuzz=500pt\includegraphics[width=1em]{element.pdf}~logarithmic & \hfuzz=500pt boolean & \hfuzz=500pt If set to 'yes', a logarithmic scale is used for the axis.\\
\hfuzz=500pt\includegraphics[width=1em]{element-mustset.pdf}~color & \hfuzz=500pt \hyperref[plotColorType]{plotColor} & \hfuzz=500pt Setting the color of the axis bars and labels.\\
\hfuzz=500pt\includegraphics[width=1em]{element.pdf}~changeDirection & \hfuzz=500pt boolean & \hfuzz=500pt If set to 'yes', the directions right/up are changed to left/down.\\
\hline
\end{tabularx}


\subsection{Time}
The input data are interpreted as MJD (modified Julian date).
The unit of the tick spacings should be appenend to the number and can be any of
\begin{itemize}
\item Y (year, plot with 4 digits)
\item y (year, plot with 2 digits)
\item O (month, plot using \verb|FORMAT_DATE_MAP|)
\item o (month, plot with 2 digits)
\item U (ISO week, plot using \verb|FORMAT_DATE_MAP|)
\item u (ISO week, plot using 2 digits)
\item r (Gregorian week, 7-day stride from start of week \verb|TIME_WEEK_START|)
\item K (ISO weekday, plot name of day)
\item D (date, plot using \verb|FORMAT_DATE_MAP|)
\item d (day, plot day of month 0-31 or year 1-366, via \verb|FORMAT_DATE_MAP|)
\item R (day, same as d, aligned with \verb|TIME_WEEK_START|)
\item H (hour, plot using \verb|FORMAT_CLOCK_MAP|)
\item h (hour, plot with 2 digits)
\item M (minute, plot using \verb|FORMAT_CLOCK_MAP|)
\item m (minute, plot with 2 digits)
\item S (second, plot using \verb|FORMAT_CLOCK_MAP|)
\item s (second, plot with 2 digits).
\end{itemize}

A secondary time axis can be added to specify larger intervals (e.g dates of hourly data).

Examples: Settings for Fig.~\ref{plotAxisType:plotAxisTime1}: \config{majorTickSpacing}=\verb|6H|, secondary: \config{majorTickSpacing}=\verb|1D|.
\fig{!hb}{1.0}{plotAxisTime1}{plotAxisType:plotAxisTime1}{Time axis for daily data.}

Settings for Fig.~\ref{plotAxisType:plotAxisTime2}: \config{majorTickSpacing}=\verb|2d|, secondary: \config{majorTickSpacing}=\verb|1O|, \config{options}=\verb|FORMAT_DATE_MAP="o yyyy"|.
\fig{!hb}{1.0}{plotAxisTime2}{plotAxisType:plotAxisTime2}{Time axis for monthly data.}

Settings for Fig.~\ref{plotAxisType:plotAxisTime3}: \config{majorTickSpacing}=\verb|1o|, secondary: \config{majorTickSpacing}=\verb|1Y|, \config{options}=\verb|FORMAT_DATE_MAP="mm"|.
\fig{!hb}{1.0}{plotAxisTime3}{plotAxisType:plotAxisTime3}{Time axis for yearly data.}



\keepXColumns
\begin{tabularx}{\textwidth}{N T A}
\hline
Name & Type & Annotation\\
\hline
\hfuzz=500pt\includegraphics[width=1em]{element.pdf}~min & \hfuzz=500pt time & \hfuzz=500pt The minimum value of the time axis. If no value is given, the minimum scale value is set automatically.\\
\hfuzz=500pt\includegraphics[width=1em]{element.pdf}~max & \hfuzz=500pt time & \hfuzz=500pt The maximum value of the time axis. If no value is given, the maximum scale value is set automatically.\\
\hfuzz=500pt\includegraphics[width=1em]{element.pdf}~majorTickSpacing & \hfuzz=500pt string & \hfuzz=500pt Y: year, o: month\\
\hfuzz=500pt\includegraphics[width=1em]{element.pdf}~minorTickSpacing & \hfuzz=500pt string & \hfuzz=500pt D: date, d: day\\
\hfuzz=500pt\includegraphics[width=1em]{element.pdf}~gridLineSpacing & \hfuzz=500pt string & \hfuzz=500pt H: clock, h: hour, m: minute, s: second\\
\hfuzz=500pt\includegraphics[width=1em]{element.pdf}~secondary & \hfuzz=500pt sequence & \hfuzz=500pt secondary time axis\\
\hfuzz=500pt\includegraphics[width=1em]{connector.pdf}\includegraphics[width=1em]{element.pdf}~majorTickSpacing & \hfuzz=500pt string & \hfuzz=500pt Y: year, o: month\\
\hfuzz=500pt\includegraphics[width=1em]{connector.pdf}\includegraphics[width=1em]{element.pdf}~minorTickSpacing & \hfuzz=500pt string & \hfuzz=500pt D: date, d: day\\
\hfuzz=500pt\includegraphics[width=1em]{connector.pdf}\includegraphics[width=1em]{element.pdf}~gridLineSpacing & \hfuzz=500pt string & \hfuzz=500pt H: clock, h: hour, m: minute, s: second\\
\hfuzz=500pt\includegraphics[width=1em]{element-mustset.pdf}~color & \hfuzz=500pt \hyperref[plotColorType]{plotColor} & \hfuzz=500pt color of axis bars and labels\\
\hfuzz=500pt\includegraphics[width=1em]{element.pdf}~gridLine & \hfuzz=500pt \hyperref[plotLineType]{plotLine} & \hfuzz=500pt The style of the grid lines.\\
\hfuzz=500pt\includegraphics[width=1em]{element.pdf}~changeDirection & \hfuzz=500pt boolean & \hfuzz=500pt right-\$>\$left / up-\$>\$down\\
\hfuzz=500pt\includegraphics[width=1em]{element-unbounded.pdf}~options & \hfuzz=500pt string & \hfuzz=500pt adjust date format\\
\hline
\end{tabularx}


\subsection{Labeled}
Axis with string labels. The coordinate system is based on the label indices (e.g. 0, 1, 2).


\keepXColumns
\begin{tabularx}{\textwidth}{N T A}
\hline
Name & Type & Annotation\\
\hline
\hfuzz=500pt\includegraphics[width=1em]{element-mustset-unbounded.pdf}~labels & \hfuzz=500pt string & \hfuzz=500pt tick labels (ticks are placed at their index. e.g. 0, 1, ..., 5)\\
\hfuzz=500pt\includegraphics[width=1em]{element.pdf}~min & \hfuzz=500pt expression & \hfuzz=500pt minimum value of the axis\\
\hfuzz=500pt\includegraphics[width=1em]{element.pdf}~max & \hfuzz=500pt expression & \hfuzz=500pt maximum values of the axis\\
\hfuzz=500pt\includegraphics[width=1em]{element.pdf}~orthogonalLabels & \hfuzz=500pt boolean & \hfuzz=500pt labels are oriented orthogonal to axis\\
\hfuzz=500pt\includegraphics[width=1em]{element.pdf}~gridLine & \hfuzz=500pt \hyperref[plotLineType]{plotLine} & \hfuzz=500pt The style of the grid lines.\\
\hfuzz=500pt\includegraphics[width=1em]{element-mustset.pdf}~color & \hfuzz=500pt \hyperref[plotColorType]{plotColor} & \hfuzz=500pt set the color of the axis and labels\\
\hfuzz=500pt\includegraphics[width=1em]{element.pdf}~changeDirection & \hfuzz=500pt boolean & \hfuzz=500pt If set to 'yes', the directions right/up are changed to left/down.\\
\hline
\end{tabularx}

\clearpage
%==================================

\section{PlotColor}\label{plotColorType}
Selects a color.
Used in \program{PlotDegreeAmplitudes}, \program{PlotGraph}, \program{PlotMap},
\program{PlotMatrix}, \program{PlotSphericalHarmonicsTriangle}.



\keepXColumns
\begin{tabularx}{\textwidth}{N T A}
\hline
Name & Type & Annotation\\
\hline
\hfuzz=500pt\includegraphics[width=1em]{element-mustset.pdf}~plotColorType & \hfuzz=500pt choice & \hfuzz=500pt color\\
\hfuzz=500pt\includegraphics[width=1em]{connector.pdf}\includegraphics[width=1em]{element-mustset.pdf}~black & \hfuzz=500pt  & \hfuzz=500pt \\
\hfuzz=500pt\includegraphics[width=1em]{connector.pdf}\includegraphics[width=1em]{element-mustset.pdf}~red & \hfuzz=500pt  & \hfuzz=500pt \\
\hfuzz=500pt\includegraphics[width=1em]{connector.pdf}\includegraphics[width=1em]{element-mustset.pdf}~blue & \hfuzz=500pt  & \hfuzz=500pt \\
\hfuzz=500pt\includegraphics[width=1em]{connector.pdf}\includegraphics[width=1em]{element-mustset.pdf}~green & \hfuzz=500pt  & \hfuzz=500pt \\
\hfuzz=500pt\includegraphics[width=1em]{connector.pdf}\includegraphics[width=1em]{element-mustset.pdf}~orange & \hfuzz=500pt  & \hfuzz=500pt \\
\hfuzz=500pt\includegraphics[width=1em]{connector.pdf}\includegraphics[width=1em]{element-mustset.pdf}~darkred & \hfuzz=500pt  & \hfuzz=500pt \\
\hfuzz=500pt\includegraphics[width=1em]{connector.pdf}\includegraphics[width=1em]{element-mustset.pdf}~yellow & \hfuzz=500pt  & \hfuzz=500pt \\
\hfuzz=500pt\includegraphics[width=1em]{connector.pdf}\includegraphics[width=1em]{element-mustset.pdf}~lightgreen & \hfuzz=500pt  & \hfuzz=500pt \\
\hfuzz=500pt\includegraphics[width=1em]{connector.pdf}\includegraphics[width=1em]{element-mustset.pdf}~gray & \hfuzz=500pt  & \hfuzz=500pt \\
\hfuzz=500pt\includegraphics[width=1em]{connector.pdf}\includegraphics[width=1em]{element-mustset.pdf}~rgb & \hfuzz=500pt sequence & \hfuzz=500pt \\
\hfuzz=500pt\quad\includegraphics[width=1em]{connector.pdf}\includegraphics[width=1em]{element-mustset.pdf}~red & \hfuzz=500pt uint & \hfuzz=500pt 0..255\\
\hfuzz=500pt\quad\includegraphics[width=1em]{connector.pdf}\includegraphics[width=1em]{element-mustset.pdf}~green & \hfuzz=500pt uint & \hfuzz=500pt 0..255\\
\hfuzz=500pt\quad\includegraphics[width=1em]{connector.pdf}\includegraphics[width=1em]{element-mustset.pdf}~blue & \hfuzz=500pt uint & \hfuzz=500pt 0..255\\
\hfuzz=500pt\includegraphics[width=1em]{connector.pdf}\includegraphics[width=1em]{element-mustset.pdf}~grayscale & \hfuzz=500pt sequence & \hfuzz=500pt \\
\hfuzz=500pt\quad\includegraphics[width=1em]{connector.pdf}\includegraphics[width=1em]{element-mustset.pdf}~value & \hfuzz=500pt uint & \hfuzz=500pt 0..255\\
\hfuzz=500pt\includegraphics[width=1em]{connector.pdf}\includegraphics[width=1em]{element-mustset.pdf}~namedColor & \hfuzz=500pt sequence & \hfuzz=500pt \\
\hfuzz=500pt\quad\includegraphics[width=1em]{connector.pdf}\includegraphics[width=1em]{element-mustset.pdf}~colorName & \hfuzz=500pt string & \hfuzz=500pt name after GMT definition\\
\hfuzz=500pt\includegraphics[width=1em]{connector.pdf}\includegraphics[width=1em]{element-mustset.pdf}~cycler & \hfuzz=500pt sequence & \hfuzz=500pt \\
\hfuzz=500pt\quad\includegraphics[width=1em]{connector.pdf}\includegraphics[width=1em]{element-mustset.pdf}~index & \hfuzz=500pt uint & \hfuzz=500pt pick color based on index expression\\
\hfuzz=500pt\quad\includegraphics[width=1em]{connector.pdf}\includegraphics[width=1em]{element.pdf}~inputfileColorList & \hfuzz=500pt filename & \hfuzz=500pt list of colors as defined by GMT\\
\hline
\end{tabularx}

\clearpage
%==================================

\section{PlotColorbar}\label{plotColorbarType}
A colorbar as used in \program{PlotMap}, \program{PlotMatrix}, \program{PlotSphericalHarmonicsTriangle}.


\keepXColumns
\begin{tabularx}{\textwidth}{N T A}
\hline
Name & Type & Annotation\\
\hline
\hfuzz=500pt\includegraphics[width=1em]{element-mustset.pdf}~plotColorbarType & \hfuzz=500pt sequence & \hfuzz=500pt \\
\hfuzz=500pt\includegraphics[width=1em]{connector.pdf}\includegraphics[width=1em]{element.pdf}~min & \hfuzz=500pt double & \hfuzz=500pt \\
\hfuzz=500pt\includegraphics[width=1em]{connector.pdf}\includegraphics[width=1em]{element.pdf}~max & \hfuzz=500pt double & \hfuzz=500pt \\
\hfuzz=500pt\includegraphics[width=1em]{connector.pdf}\includegraphics[width=1em]{element.pdf}~annotation & \hfuzz=500pt double & \hfuzz=500pt boundary annotation\\
\hfuzz=500pt\includegraphics[width=1em]{connector.pdf}\includegraphics[width=1em]{element.pdf}~unit & \hfuzz=500pt string & \hfuzz=500pt appended to axis values\\
\hfuzz=500pt\includegraphics[width=1em]{connector.pdf}\includegraphics[width=1em]{element.pdf}~label & \hfuzz=500pt string & \hfuzz=500pt description of the axis\\
\hfuzz=500pt\includegraphics[width=1em]{connector.pdf}\includegraphics[width=1em]{element.pdf}~logarithmic & \hfuzz=500pt boolean & \hfuzz=500pt use logarithmic scale\\
\hfuzz=500pt\includegraphics[width=1em]{connector.pdf}\includegraphics[width=1em]{element.pdf}~triangleLeft & \hfuzz=500pt boolean & \hfuzz=500pt \\
\hfuzz=500pt\includegraphics[width=1em]{connector.pdf}\includegraphics[width=1em]{element.pdf}~triangleRight & \hfuzz=500pt boolean & \hfuzz=500pt \\
\hfuzz=500pt\includegraphics[width=1em]{connector.pdf}\includegraphics[width=1em]{element.pdf}~illuminate & \hfuzz=500pt boolean & \hfuzz=500pt illuminate\\
\hfuzz=500pt\includegraphics[width=1em]{connector.pdf}\includegraphics[width=1em]{element.pdf}~vertical & \hfuzz=500pt boolean & \hfuzz=500pt plot vertical color bar on the right\\
\hfuzz=500pt\includegraphics[width=1em]{connector.pdf}\includegraphics[width=1em]{element.pdf}~length & \hfuzz=500pt double & \hfuzz=500pt length of colorbar in percent\\
\hfuzz=500pt\includegraphics[width=1em]{connector.pdf}\includegraphics[width=1em]{element.pdf}~margin & \hfuzz=500pt double & \hfuzz=500pt between colorbar and figure [cm]\\
\hfuzz=500pt\includegraphics[width=1em]{connector.pdf}\includegraphics[width=1em]{element.pdf}~colorTable & \hfuzz=500pt string & \hfuzz=500pt name of the color bar\\
\hfuzz=500pt\includegraphics[width=1em]{connector.pdf}\includegraphics[width=1em]{element.pdf}~reverse & \hfuzz=500pt boolean & \hfuzz=500pt reverse direction\\
\hfuzz=500pt\includegraphics[width=1em]{connector.pdf}\includegraphics[width=1em]{element.pdf}~showColorbar & \hfuzz=500pt boolean & \hfuzz=500pt \\
\hline
\end{tabularx}

\clearpage
%==================================

\section{PlotGraphLayer}\label{plotGraphLayerType}
Defines the content of an xy-plot of \program{PlotGraph}.
Multiple layers are are plotted sequentially. With \config{plotOnSecondAxis}
the alternative y-axis on the right hand side can be selected if provided.


\subsection{LinesAndPoints}\label{plotGraphLayerType:linesAndPoints}
Draws a \configClass{line}{plotLineType} and/or points (\configClass{symbol}{plotSymbolType})
of xy data. The standard \reference{dataVariables}{general.parser:dataVariables}
are available to select the data columns of \configFile{inputfileMatrix}{matrix}.
If no \configClass{color}{plotColorType} of the \configClass{symbol}{plotSymbolType}
is given a \configClass{colorbar}{plotColorbarType}
is required and the color is determined by \config{valueZ}.
Additionally a vertical error bar can be plotted at each data point with
size \config{valueErrorBar}.

See \program{Gravityfield2AreaMeanTimeSeries} for an example plot.


\keepXColumns
\begin{tabularx}{\textwidth}{N T A}
\hline
Name & Type & Annotation\\
\hline
\hfuzz=500pt\includegraphics[width=1em]{element-mustset.pdf}~inputfileMatrix & \hfuzz=500pt filename & \hfuzz=500pt each line contains x,y\\
\hfuzz=500pt\includegraphics[width=1em]{element.pdf}~valueX & \hfuzz=500pt expression & \hfuzz=500pt expression for x-values (input columns are named data0, data1, ...)\\
\hfuzz=500pt\includegraphics[width=1em]{element-mustset.pdf}~valueY & \hfuzz=500pt expression & \hfuzz=500pt expression for y-values (input columns are named data0, data1, ...)\\
\hfuzz=500pt\includegraphics[width=1em]{element.pdf}~valueZ & \hfuzz=500pt expression & \hfuzz=500pt expression for the colorbar\\
\hfuzz=500pt\includegraphics[width=1em]{element.pdf}~valueErrorBar & \hfuzz=500pt expression & \hfuzz=500pt expression for error bars (input columns are named data0, data1, ...)\\
\hfuzz=500pt\includegraphics[width=1em]{element.pdf}~description & \hfuzz=500pt string & \hfuzz=500pt text of the legend\\
\hfuzz=500pt\includegraphics[width=1em]{element.pdf}~line & \hfuzz=500pt \hyperref[plotLineType]{plotLine} & \hfuzz=500pt \\
\hfuzz=500pt\includegraphics[width=1em]{element.pdf}~symbol & \hfuzz=500pt \hyperref[plotSymbolType]{plotSymbol} & \hfuzz=500pt \\
\hfuzz=500pt\includegraphics[width=1em]{element.pdf}~plotOnSecondAxis & \hfuzz=500pt boolean & \hfuzz=500pt draw dataset on a second Y-axis (if available).\\
\hline
\end{tabularx}


\subsection{ErrorEnvelope}
Draws a symmetrical envelope around \config{valueY} as function of \config{valueX}
using deviations \config{valueErrors}.
The standard \reference{dataVariables}{general.parser:dataVariables}
are available to select the data columns of \configFile{inputfileMatrix}{matrix}.
The data line itself is not plotted but must be added as extra
\configClass{layer:linesAndPoints}{plotGraphLayerType:linesAndPoints}.


\keepXColumns
\begin{tabularx}{\textwidth}{N T A}
\hline
Name & Type & Annotation\\
\hline
\hfuzz=500pt\includegraphics[width=1em]{element-mustset.pdf}~inputfileMatrix & \hfuzz=500pt filename & \hfuzz=500pt each line contains x,y\\
\hfuzz=500pt\includegraphics[width=1em]{element.pdf}~valueX & \hfuzz=500pt expression & \hfuzz=500pt expression for x-values (input columns are named data0, data1, ...)\\
\hfuzz=500pt\includegraphics[width=1em]{element-mustset.pdf}~valueY & \hfuzz=500pt expression & \hfuzz=500pt expression for y-values (input columns are named data0, data1, ...)\\
\hfuzz=500pt\includegraphics[width=1em]{element-mustset.pdf}~valueErrors & \hfuzz=500pt expression & \hfuzz=500pt expression for error values\\
\hfuzz=500pt\includegraphics[width=1em]{element.pdf}~description & \hfuzz=500pt string & \hfuzz=500pt text of the legend\\
\hfuzz=500pt\includegraphics[width=1em]{element.pdf}~fillColor & \hfuzz=500pt \hyperref[plotColorType]{plotColor} & \hfuzz=500pt fill color of the envelope\\
\hfuzz=500pt\includegraphics[width=1em]{element.pdf}~edgeLine & \hfuzz=500pt \hyperref[plotLineType]{plotLine} & \hfuzz=500pt edge line style of the envelope\\
\hfuzz=500pt\includegraphics[width=1em]{element.pdf}~plotOnSecondAxis & \hfuzz=500pt boolean & \hfuzz=500pt draw dataset on a second Y-axis (if available).\\
\hline
\end{tabularx}


\subsection{Bars}
Creates a bar plot with vertical or \config{horizontal} bars out of the given
x- and y-values. The standard \reference{dataVariables}{general.parser:dataVariables}
are available to select the data columns of \configFile{inputfileMatrix}{matrix}.
The bars ranges from \config{valueBase} (can be also an expression) to the \config{valueY}.
If no \configClass{color}{plotColorType} is given a \configClass{colorbar}{plotColorbarType}
is required and the color is determined by \config{valueZ}.

See \program{Instrument2Histogram} for an example plot.


\keepXColumns
\begin{tabularx}{\textwidth}{N T A}
\hline
Name & Type & Annotation\\
\hline
\hfuzz=500pt\includegraphics[width=1em]{element-mustset.pdf}~inputfileMatrix & \hfuzz=500pt filename & \hfuzz=500pt each line contains x,y\\
\hfuzz=500pt\includegraphics[width=1em]{element.pdf}~valueX & \hfuzz=500pt expression & \hfuzz=500pt expression for x-values (input columns are named data0, data1, ...)\\
\hfuzz=500pt\includegraphics[width=1em]{element-mustset.pdf}~valueY & \hfuzz=500pt expression & \hfuzz=500pt expression for y-values (input columns are named data0, data1, ...)\\
\hfuzz=500pt\includegraphics[width=1em]{element.pdf}~valueZ & \hfuzz=500pt expression & \hfuzz=500pt expression for the colorbar\\
\hfuzz=500pt\includegraphics[width=1em]{element.pdf}~valueBase & \hfuzz=500pt expression & \hfuzz=500pt base value of bars (default: minimum y-value)\\
\hfuzz=500pt\includegraphics[width=1em]{element.pdf}~width & \hfuzz=500pt expression & \hfuzz=500pt width of bars (default: minimum x-gap)\\
\hfuzz=500pt\includegraphics[width=1em]{element.pdf}~horizontal & \hfuzz=500pt boolean & \hfuzz=500pt draw horizontal bars instead of vertical\\
\hfuzz=500pt\includegraphics[width=1em]{element.pdf}~description & \hfuzz=500pt string & \hfuzz=500pt text of the legend\\
\hfuzz=500pt\includegraphics[width=1em]{element.pdf}~color & \hfuzz=500pt \hyperref[plotColorType]{plotColor} & \hfuzz=500pt \\
\hfuzz=500pt\includegraphics[width=1em]{element.pdf}~edgeLine & \hfuzz=500pt \hyperref[plotLineType]{plotLine} & \hfuzz=500pt line\\
\hfuzz=500pt\includegraphics[width=1em]{element.pdf}~plotOnSecondAxis & \hfuzz=500pt boolean & \hfuzz=500pt draw dataset on a second Y-axis (if available).\\
\hline
\end{tabularx}


\subsection{Gridded}
Creates a regular grid of yxz values. The standard \reference{dataVariables}{general.parser:dataVariables}
are available to select the data columns of \configFile{inputfileMatrix}{matrix}.
Empty grid cells are not plotted. Cells with more than one value will be set to the mean value.
The grid spacing is determined by the median spacing of the input data or set by \config{incrementX/Y}.

See \program{Orbit2ArgumentOfLatitude} for an example plot.


\keepXColumns
\begin{tabularx}{\textwidth}{N T A}
\hline
Name & Type & Annotation\\
\hline
\hfuzz=500pt\includegraphics[width=1em]{element-mustset.pdf}~inputfileMatrix & \hfuzz=500pt filename & \hfuzz=500pt each line contains x,y,z\\
\hfuzz=500pt\includegraphics[width=1em]{element.pdf}~valueX & \hfuzz=500pt expression & \hfuzz=500pt expression for x-values (input columns are named data0, data1, ...)\\
\hfuzz=500pt\includegraphics[width=1em]{element-mustset.pdf}~valueY & \hfuzz=500pt expression & \hfuzz=500pt expression for y-values (input columns are named data0, data1, ...)\\
\hfuzz=500pt\includegraphics[width=1em]{element-mustset.pdf}~valueZ & \hfuzz=500pt expression & \hfuzz=500pt expression for the colorbar\\
\hfuzz=500pt\includegraphics[width=1em]{element.pdf}~incrementX & \hfuzz=500pt double & \hfuzz=500pt the grid spacing\\
\hfuzz=500pt\includegraphics[width=1em]{element.pdf}~incrementY & \hfuzz=500pt double & \hfuzz=500pt the grid spacing\\
\hfuzz=500pt\includegraphics[width=1em]{element.pdf}~plotOnSecondAxis & \hfuzz=500pt boolean & \hfuzz=500pt draw dataset on a second Y-axis (if available).\\
\hline
\end{tabularx}


\subsection{Rectangle}
Plots a rectangle to highlight an area.


\keepXColumns
\begin{tabularx}{\textwidth}{N T A}
\hline
Name & Type & Annotation\\
\hline
\hfuzz=500pt\includegraphics[width=1em]{element.pdf}~minX & \hfuzz=500pt double & \hfuzz=500pt empty: left\\
\hfuzz=500pt\includegraphics[width=1em]{element.pdf}~maxX & \hfuzz=500pt double & \hfuzz=500pt empty: right\\
\hfuzz=500pt\includegraphics[width=1em]{element.pdf}~minY & \hfuzz=500pt double & \hfuzz=500pt empty: bottom\\
\hfuzz=500pt\includegraphics[width=1em]{element.pdf}~maxY & \hfuzz=500pt double & \hfuzz=500pt empty: top\\
\hfuzz=500pt\includegraphics[width=1em]{element.pdf}~description & \hfuzz=500pt string & \hfuzz=500pt text of the legend\\
\hfuzz=500pt\includegraphics[width=1em]{element.pdf}~edgeLine & \hfuzz=500pt \hyperref[plotLineType]{plotLine} & \hfuzz=500pt \\
\hfuzz=500pt\includegraphics[width=1em]{element.pdf}~fillColor & \hfuzz=500pt \hyperref[plotColorType]{plotColor} & \hfuzz=500pt \\
\hfuzz=500pt\includegraphics[width=1em]{element.pdf}~plotOnSecondAxis & \hfuzz=500pt boolean & \hfuzz=500pt draw dataset on a second Y-axis (if available).\\
\hline
\end{tabularx}


\subsection{Text}
Writes a \config{text} at \config{originX} and \config{originY} position in the graph.
With \config{clip} the text is cutted at the boundaries of the plotting area.


\keepXColumns
\begin{tabularx}{\textwidth}{N T A}
\hline
Name & Type & Annotation\\
\hline
\hfuzz=500pt\includegraphics[width=1em]{element-mustset.pdf}~text & \hfuzz=500pt string & \hfuzz=500pt \\
\hfuzz=500pt\includegraphics[width=1em]{element-mustset.pdf}~originX & \hfuzz=500pt double & \hfuzz=500pt \\
\hfuzz=500pt\includegraphics[width=1em]{element-mustset.pdf}~originY & \hfuzz=500pt double & \hfuzz=500pt \\
\hfuzz=500pt\includegraphics[width=1em]{element.pdf}~offsetX & \hfuzz=500pt double & \hfuzz=500pt [cm] x-offset from origin\\
\hfuzz=500pt\includegraphics[width=1em]{element.pdf}~offsetY & \hfuzz=500pt double & \hfuzz=500pt [cm] y-offset from origin\\
\hfuzz=500pt\includegraphics[width=1em]{element.pdf}~alignment & \hfuzz=500pt string & \hfuzz=500pt L, C, R (left, center, right) and T, M, B (top, middle, bottom)\\
\hfuzz=500pt\includegraphics[width=1em]{element.pdf}~fontSize & \hfuzz=500pt double & \hfuzz=500pt [pt]\\
\hfuzz=500pt\includegraphics[width=1em]{element-mustset.pdf}~fontColor & \hfuzz=500pt \hyperref[plotColorType]{plotColor} & \hfuzz=500pt \\
\hfuzz=500pt\includegraphics[width=1em]{element.pdf}~clip & \hfuzz=500pt boolean & \hfuzz=500pt clip at boundaries\\
\hfuzz=500pt\includegraphics[width=1em]{element.pdf}~plotOnSecondAxis & \hfuzz=500pt boolean & \hfuzz=500pt draw dataset on a second Y-axis (if available).\\
\hline
\end{tabularx}


\subsection{DegreeAmplitudes}
Plot degree amplitudes of potential coefficients computed by \program{Gravityfield2DegreeAmplitudes}
or \program{PotentialCoefficients2DegreeAmplitudes}.
The standard \reference{dataVariables}{general.parser:dataVariables} are available to select
the data columns of \configFile{inputfileMatrix}{matrix}. It plots a solid line for the
\config{valueSignal} and a dotted line for the \config{valueError} per default.


\keepXColumns
\begin{tabularx}{\textwidth}{N T A}
\hline
Name & Type & Annotation\\
\hline
\hfuzz=500pt\includegraphics[width=1em]{element-mustset.pdf}~inputfileMatrix & \hfuzz=500pt filename & \hfuzz=500pt degree amplitudes\\
\hfuzz=500pt\includegraphics[width=1em]{element.pdf}~valueDegree & \hfuzz=500pt expression & \hfuzz=500pt expression for x-values (degrees) (input columns are named data0, data1, ...)\\
\hfuzz=500pt\includegraphics[width=1em]{element.pdf}~valueSignal & \hfuzz=500pt expression & \hfuzz=500pt expression for y-values (signal) (input columns are named data0, data1, ...)\\
\hfuzz=500pt\includegraphics[width=1em]{element.pdf}~valueErrors & \hfuzz=500pt expression & \hfuzz=500pt expression for y-values (formal errors)\\
\hfuzz=500pt\includegraphics[width=1em]{element.pdf}~description & \hfuzz=500pt string & \hfuzz=500pt text of the legend\\
\hfuzz=500pt\includegraphics[width=1em]{element.pdf}~lineSignal & \hfuzz=500pt \hyperref[plotLineType]{plotLine} & \hfuzz=500pt \\
\hfuzz=500pt\includegraphics[width=1em]{element.pdf}~lineErrors & \hfuzz=500pt \hyperref[plotLineType]{plotLine} & \hfuzz=500pt \\
\hfuzz=500pt\includegraphics[width=1em]{element.pdf}~plotOnSecondAxis & \hfuzz=500pt boolean & \hfuzz=500pt draw dataset on a second Y-axis (if available).\\
\hline
\end{tabularx}

\clearpage
%==================================

\section{PlotLegend}\label{plotLegendType}
Plot a legend of the descriptions provided in
\configClass{plotGraphLayer}{plotGraphLayerType} in \program{PlotGraph}.


\keepXColumns
\begin{tabularx}{\textwidth}{N T A}
\hline
Name & Type & Annotation\\
\hline
\hfuzz=500pt\includegraphics[width=1em]{element-mustset.pdf}~plotLegendType & \hfuzz=500pt sequence & \hfuzz=500pt \\
\hfuzz=500pt\includegraphics[width=1em]{connector.pdf}\includegraphics[width=1em]{element.pdf}~width & \hfuzz=500pt double & \hfuzz=500pt legend width [cm]\\
\hfuzz=500pt\includegraphics[width=1em]{connector.pdf}\includegraphics[width=1em]{element.pdf}~height & \hfuzz=500pt double & \hfuzz=500pt legend height [cm] (default: estimated)\\
\hfuzz=500pt\includegraphics[width=1em]{connector.pdf}\includegraphics[width=1em]{element.pdf}~positionX & \hfuzz=500pt double & \hfuzz=500pt legend x-position in normalized (0-1) coordinates.\\
\hfuzz=500pt\includegraphics[width=1em]{connector.pdf}\includegraphics[width=1em]{element.pdf}~positionY & \hfuzz=500pt double & \hfuzz=500pt legend y-position in normalized (0-1) coordinates.\\
\hfuzz=500pt\includegraphics[width=1em]{connector.pdf}\includegraphics[width=1em]{element.pdf}~anchorPoint & \hfuzz=500pt string & \hfuzz=500pt Two character combination of L, C, R (for left, center, or right) and T, M, B for top, middle, or bottom. e.g., TL for top left\\
\hfuzz=500pt\includegraphics[width=1em]{connector.pdf}\includegraphics[width=1em]{element.pdf}~columns & \hfuzz=500pt uint & \hfuzz=500pt number of columns in legend\\
\hfuzz=500pt\includegraphics[width=1em]{connector.pdf}\includegraphics[width=1em]{element.pdf}~textColor & \hfuzz=500pt \hyperref[plotColorType]{plotColor} & \hfuzz=500pt color of the legend text\\
\hfuzz=500pt\includegraphics[width=1em]{connector.pdf}\includegraphics[width=1em]{element.pdf}~fillColor & \hfuzz=500pt \hyperref[plotColorType]{plotColor} & \hfuzz=500pt fill color of the legend box\\
\hfuzz=500pt\includegraphics[width=1em]{connector.pdf}\includegraphics[width=1em]{element.pdf}~edgeLine & \hfuzz=500pt \hyperref[plotLineType]{plotLine} & \hfuzz=500pt style of the legend box edge\\
\hline
\end{tabularx}

\clearpage
%==================================

\section{PlotLine}\label{plotLineType}
Defines the line style to be plotted.


\subsection{Solid}
Draws a solid line.


\keepXColumns
\begin{tabularx}{\textwidth}{N T A}
\hline
Name & Type & Annotation\\
\hline
\hfuzz=500pt\includegraphics[width=1em]{element.pdf}~width & \hfuzz=500pt double & \hfuzz=500pt line width [p]\\
\hfuzz=500pt\includegraphics[width=1em]{element-mustset.pdf}~color & \hfuzz=500pt \hyperref[plotColorType]{plotColor} & \hfuzz=500pt \\
\hline
\end{tabularx}


\subsection{Dashed}
Draws a dashed line.


\keepXColumns
\begin{tabularx}{\textwidth}{N T A}
\hline
Name & Type & Annotation\\
\hline
\hfuzz=500pt\includegraphics[width=1em]{element.pdf}~width & \hfuzz=500pt double & \hfuzz=500pt line width [p]\\
\hfuzz=500pt\includegraphics[width=1em]{element-mustset.pdf}~color & \hfuzz=500pt \hyperref[plotColorType]{plotColor} & \hfuzz=500pt \\
\hline
\end{tabularx}


\subsection{Dotted}
Draws a dotted line.


\keepXColumns
\begin{tabularx}{\textwidth}{N T A}
\hline
Name & Type & Annotation\\
\hline
\hfuzz=500pt\includegraphics[width=1em]{element.pdf}~width & \hfuzz=500pt double & \hfuzz=500pt line width [p]\\
\hfuzz=500pt\includegraphics[width=1em]{element-mustset.pdf}~color & \hfuzz=500pt \hyperref[plotColorType]{plotColor} & \hfuzz=500pt   \\
\hline
\end{tabularx}


\subsection{Custom}
Draws a custom line. The line \config{style} code is described in
\url{https://docs.generic-mapping-tools.org/latest/cookbook/features.html#specifying-pen-attributes}.


\keepXColumns
\begin{tabularx}{\textwidth}{N T A}
\hline
Name & Type & Annotation\\
\hline
\hfuzz=500pt\includegraphics[width=1em]{element-mustset.pdf}~style & \hfuzz=500pt string & \hfuzz=500pt line style code\\
\hfuzz=500pt\includegraphics[width=1em]{element.pdf}~width & \hfuzz=500pt double & \hfuzz=500pt line width [p]\\
\hfuzz=500pt\includegraphics[width=1em]{element-mustset.pdf}~color & \hfuzz=500pt \hyperref[plotColorType]{plotColor} & \hfuzz=500pt \\
\hline
\end{tabularx}

\clearpage
%==================================

\section{PlotMapLayer}\label{plotMapLayerType}
Defines the content of a map of \program{PlotMap}. Multiple layers are are plotted sequentially.


\subsection{GriddedData}
Creates a regular grid of yxz values. The standard \reference{dataVariables}{general.parser:dataVariables}
are available to select the data column of \configFile{inputfileGriddedData}{griddedData}.
Empty grid cells are not plotted. Cells with more than one value will be set to the mean value.
The grid spacing can be determined automatically for regular rectangular grids otherwise
it must be set with \config{increment}. To get a better display together with some projections
the grid should be internally \config{resample}d to higher resolution.
It is assumed that the points of \configFile{inputfileGriddedData}{griddedData} represents centers of grid cells.
This assumption can be changed with \config{gridlineRegistered} (e.g if the data starts at the north pole).


\keepXColumns
\begin{tabularx}{\textwidth}{N T A}
\hline
Name & Type & Annotation\\
\hline
\hfuzz=500pt\includegraphics[width=1em]{element-mustset.pdf}~inputfileGriddedData & \hfuzz=500pt filename & \hfuzz=500pt \\
\hfuzz=500pt\includegraphics[width=1em]{element-mustset.pdf}~value & \hfuzz=500pt expression & \hfuzz=500pt expression to compute values (input columns are named data0, data1, ...)\\
\hfuzz=500pt\includegraphics[width=1em]{element.pdf}~increment & \hfuzz=500pt angle & \hfuzz=500pt the grid spacing [degrees]\\
\hfuzz=500pt\includegraphics[width=1em]{element.pdf}~illuminate & \hfuzz=500pt boolean & \hfuzz=500pt illuminate grid\\
\hfuzz=500pt\includegraphics[width=1em]{element.pdf}~resample & \hfuzz=500pt sequence & \hfuzz=500pt \\
\hfuzz=500pt\includegraphics[width=1em]{connector.pdf}\includegraphics[width=1em]{element.pdf}~intermediateDpi & \hfuzz=500pt double & \hfuzz=500pt oversample grid for a smoother visual effect\\
\hfuzz=500pt\includegraphics[width=1em]{connector.pdf}\includegraphics[width=1em]{element-mustset.pdf}~interpolationMethod & \hfuzz=500pt choice & \hfuzz=500pt interpolation method for oversampling\\
\hfuzz=500pt\quad\includegraphics[width=1em]{connector.pdf}\includegraphics[width=1em]{element-mustset.pdf}~bspline & \hfuzz=500pt  & \hfuzz=500pt B-Spline interpolation\\
\hfuzz=500pt\quad\includegraphics[width=1em]{connector.pdf}\includegraphics[width=1em]{element-mustset.pdf}~bicubic & \hfuzz=500pt  & \hfuzz=500pt bicubic interpolation\\
\hfuzz=500pt\quad\includegraphics[width=1em]{connector.pdf}\includegraphics[width=1em]{element-mustset.pdf}~bilinear & \hfuzz=500pt  & \hfuzz=500pt bilinear interpolation\\
\hfuzz=500pt\quad\includegraphics[width=1em]{connector.pdf}\includegraphics[width=1em]{element-mustset.pdf}~nearest & \hfuzz=500pt  & \hfuzz=500pt nearest neighbour interpolation\\
\hfuzz=500pt\includegraphics[width=1em]{connector.pdf}\includegraphics[width=1em]{element.pdf}~threshold & \hfuzz=500pt double & \hfuzz=500pt A threshold of 1.0 requires all (4 or 16) nodes involved in interpolation to be non-NaN. 0.5 will interpolate about half way from a non-NaN value; 0.1 will go about 90\% of the way.\\
\hfuzz=500pt\includegraphics[width=1em]{element.pdf}~gridlineRegistered & \hfuzz=500pt boolean & \hfuzz=500pt treat input as point values instead of cell means\\
\hline
\end{tabularx}


\subsection{Points}\label{plotMapLayerType:points}
Draws points (\configClass{symbol}{plotSymbolType}) and/or \configClass{line}{plotLineType}s
between the points. If no \configClass{color}{plotColorType} of the \configClass{symbol}{plotSymbolType}
is given a \configClass{colorbar}{plotColorbarType} is required and the color is determined
by the \config{value} expression. The standard \reference{dataVariables}{general.parser:dataVariables}
are available to select the data column of \configFile{inputfileGriddedData}{griddedData}.


\keepXColumns
\begin{tabularx}{\textwidth}{N T A}
\hline
Name & Type & Annotation\\
\hline
\hfuzz=500pt\includegraphics[width=1em]{element-mustset.pdf}~inputfileGriddedData & \hfuzz=500pt filename & \hfuzz=500pt \\
\hfuzz=500pt\includegraphics[width=1em]{element.pdf}~value & \hfuzz=500pt expression & \hfuzz=500pt expression to compute color (input columns are named data0, data1, ...)\\
\hfuzz=500pt\includegraphics[width=1em]{element.pdf}~symbol & \hfuzz=500pt \hyperref[plotSymbolType]{plotSymbol} & \hfuzz=500pt \\
\hfuzz=500pt\includegraphics[width=1em]{element.pdf}~line & \hfuzz=500pt \hyperref[plotLineType]{plotLine} & \hfuzz=500pt style of connecting lines\\
\hfuzz=500pt\includegraphics[width=1em]{element.pdf}~drawLineAsGreatCircle & \hfuzz=500pt boolean & \hfuzz=500pt draw connecting lines as great circles (otherwise, a straight line is drawn instead)\\
\hline
\end{tabularx}


\subsection{Arrows}
Draws an arrow for each point in \configFile{inputfileGriddedData}{griddedData}.
The arrows are defined by the expressions \config{valueNorth/East}.
The standard \reference{dataVariables}{general.parser:dataVariables}
are available to select the correspondent data columns of \configFile{inputfileGriddedData}{griddedData}.
The \config{scale} factor converts the input units to cm in the plot.
If no \configClass{color}{plotColorType} is given a \configClass{colorbar}{plotColorbarType} is required
and the color is determined by the \config{value} expression.
With \config{scaleArrow} a reference arrow as legend can be plotted inside or outside the map.


\keepXColumns
\begin{tabularx}{\textwidth}{N T A}
\hline
Name & Type & Annotation\\
\hline
\hfuzz=500pt\includegraphics[width=1em]{element-mustset.pdf}~inputfileGriddedData & \hfuzz=500pt filename & \hfuzz=500pt grid file with north and east values for arrows\\
\hfuzz=500pt\includegraphics[width=1em]{element-mustset.pdf}~valueNorth & \hfuzz=500pt expression & \hfuzz=500pt expression to compute north values (input columns are named data0, data1, ...)\\
\hfuzz=500pt\includegraphics[width=1em]{element-mustset.pdf}~valueEast & \hfuzz=500pt expression & \hfuzz=500pt expression to compute east values (input columns are named data0, data1, ...)\\
\hfuzz=500pt\includegraphics[width=1em]{element.pdf}~value & \hfuzz=500pt expression & \hfuzz=500pt expression to compute arrow color (input columns are named data0, data1, ...)\\
\hfuzz=500pt\includegraphics[width=1em]{element.pdf}~scale & \hfuzz=500pt double & \hfuzz=500pt [cm per input unit] length scale factor\\
\hfuzz=500pt\includegraphics[width=1em]{element-mustset.pdf}~penSize & \hfuzz=500pt double & \hfuzz=500pt [pt] width of arrow shaft\\
\hfuzz=500pt\includegraphics[width=1em]{element-mustset.pdf}~headSize & \hfuzz=500pt double & \hfuzz=500pt [pt] size of arrow head, 0: no head, negative: reverse head\\
\hfuzz=500pt\includegraphics[width=1em]{element.pdf}~color & \hfuzz=500pt \hyperref[plotColorType]{plotColor} & \hfuzz=500pt empty: from value\\
\hfuzz=500pt\includegraphics[width=1em]{element.pdf}~scaleArrow & \hfuzz=500pt sequence & \hfuzz=500pt draw an arrow for scale reference\\
\hfuzz=500pt\includegraphics[width=1em]{connector.pdf}\includegraphics[width=1em]{element-mustset.pdf}~originX & \hfuzz=500pt double & \hfuzz=500pt [0-1] 0: left, 1: right\\
\hfuzz=500pt\includegraphics[width=1em]{connector.pdf}\includegraphics[width=1em]{element-mustset.pdf}~originY & \hfuzz=500pt double & \hfuzz=500pt [0-1] 0: bottom, 1: top\\
\hfuzz=500pt\includegraphics[width=1em]{connector.pdf}\includegraphics[width=1em]{element-mustset.pdf}~length & \hfuzz=500pt double & \hfuzz=500pt in same unit as valueNorth and valueEast\\
\hfuzz=500pt\includegraphics[width=1em]{connector.pdf}\includegraphics[width=1em]{element-mustset.pdf}~unit & \hfuzz=500pt string & \hfuzz=500pt displayed unit text (e.g. 1 cm)\\
\hfuzz=500pt\includegraphics[width=1em]{connector.pdf}\includegraphics[width=1em]{element.pdf}~label & \hfuzz=500pt string & \hfuzz=500pt description of the arrows\\
\hline
\end{tabularx}


\subsection{Polygon}
Draws a \configFile{inputfilePolygon}{polygon}.
If \configClass{fillColor}{plotColorType} is not set and a \config{value}
is given the fill color is taken from a \configClass{colorbar}{plotColorbarType}.


\keepXColumns
\begin{tabularx}{\textwidth}{N T A}
\hline
Name & Type & Annotation\\
\hline
\hfuzz=500pt\includegraphics[width=1em]{element-mustset.pdf}~inputfilePolygon & \hfuzz=500pt filename & \hfuzz=500pt \\
\hfuzz=500pt\includegraphics[width=1em]{element.pdf}~line & \hfuzz=500pt \hyperref[plotLineType]{plotLine} & \hfuzz=500pt style of border lines\\
\hfuzz=500pt\includegraphics[width=1em]{element.pdf}~fillColor & \hfuzz=500pt \hyperref[plotColorType]{plotColor} & \hfuzz=500pt polygon fill color (no fill color: determine from value if given, else: no fill)\\
\hfuzz=500pt\includegraphics[width=1em]{element.pdf}~value & \hfuzz=500pt double & \hfuzz=500pt value to compute fill color from a colorbar (ignored if a fillColor is given)\\
\hfuzz=500pt\includegraphics[width=1em]{element.pdf}~drawLineAsGreatCircle & \hfuzz=500pt boolean & \hfuzz=500pt draw connecting lines as great circles (otherwise, a straight line is drawn instead)\\
\hline
\end{tabularx}


\subsection{Coast}
Plots coastlines. GMT provides them in different \config{resolution}s.
Features with an area smaller than \config{minArea} in $km^2$ will not be plotted.


\keepXColumns
\begin{tabularx}{\textwidth}{N T A}
\hline
Name & Type & Annotation\\
\hline
\hfuzz=500pt\includegraphics[width=1em]{element-mustset.pdf}~resolution & \hfuzz=500pt choice & \hfuzz=500pt \\
\hfuzz=500pt\includegraphics[width=1em]{connector.pdf}\includegraphics[width=1em]{element-mustset.pdf}~crude & \hfuzz=500pt  & \hfuzz=500pt \\
\hfuzz=500pt\includegraphics[width=1em]{connector.pdf}\includegraphics[width=1em]{element-mustset.pdf}~low & \hfuzz=500pt  & \hfuzz=500pt \\
\hfuzz=500pt\includegraphics[width=1em]{connector.pdf}\includegraphics[width=1em]{element-mustset.pdf}~medium & \hfuzz=500pt  & \hfuzz=500pt \\
\hfuzz=500pt\includegraphics[width=1em]{connector.pdf}\includegraphics[width=1em]{element-mustset.pdf}~high & \hfuzz=500pt  & \hfuzz=500pt \\
\hfuzz=500pt\includegraphics[width=1em]{connector.pdf}\includegraphics[width=1em]{element-mustset.pdf}~full & \hfuzz=500pt  & \hfuzz=500pt \\
\hfuzz=500pt\includegraphics[width=1em]{element.pdf}~line & \hfuzz=500pt \hyperref[plotLineType]{plotLine} & \hfuzz=500pt line style for coastlines\\
\hfuzz=500pt\includegraphics[width=1em]{element.pdf}~landColor & \hfuzz=500pt \hyperref[plotColorType]{plotColor} & \hfuzz=500pt fill land area\\
\hfuzz=500pt\includegraphics[width=1em]{element.pdf}~oceanColor & \hfuzz=500pt \hyperref[plotColorType]{plotColor} & \hfuzz=500pt fill ocean area\\
\hfuzz=500pt\includegraphics[width=1em]{element.pdf}~minArea & \hfuzz=500pt uint & \hfuzz=500pt [km\textasciicircum{}2] features with a smaller area than this are dropped\\
\hline
\end{tabularx}


\subsection{Rivers}
Plots rivers and lakes. GMT provides different classes
(\url{https://docs.generic-mapping-tools.org/latest/coast.html}).


\keepXColumns
\begin{tabularx}{\textwidth}{N T A}
\hline
Name & Type & Annotation\\
\hline
\hfuzz=500pt\includegraphics[width=1em]{element-mustset.pdf}~class & \hfuzz=500pt choice & \hfuzz=500pt \\
\hfuzz=500pt\includegraphics[width=1em]{connector.pdf}\includegraphics[width=1em]{element-mustset.pdf}~riversCanalsLakes & \hfuzz=500pt  & \hfuzz=500pt \\
\hfuzz=500pt\includegraphics[width=1em]{connector.pdf}\includegraphics[width=1em]{element-mustset.pdf}~riversCanals & \hfuzz=500pt  & \hfuzz=500pt \\
\hfuzz=500pt\includegraphics[width=1em]{connector.pdf}\includegraphics[width=1em]{element-mustset.pdf}~permanentRiversLakes & \hfuzz=500pt  & \hfuzz=500pt \\
\hfuzz=500pt\includegraphics[width=1em]{connector.pdf}\includegraphics[width=1em]{element-mustset.pdf}~permanentRivers & \hfuzz=500pt  & \hfuzz=500pt \\
\hfuzz=500pt\includegraphics[width=1em]{connector.pdf}\includegraphics[width=1em]{element-mustset.pdf}~intermittentRivers & \hfuzz=500pt  & \hfuzz=500pt \\
\hfuzz=500pt\includegraphics[width=1em]{connector.pdf}\includegraphics[width=1em]{element-mustset.pdf}~canals & \hfuzz=500pt  & \hfuzz=500pt \\
\hfuzz=500pt\includegraphics[width=1em]{connector.pdf}\includegraphics[width=1em]{element-mustset.pdf}~singleClass & \hfuzz=500pt sequence & \hfuzz=500pt \\
\hfuzz=500pt\quad\includegraphics[width=1em]{connector.pdf}\includegraphics[width=1em]{element-mustset-unbounded.pdf}~class & \hfuzz=500pt uint & \hfuzz=500pt 0-10. See GMT documentation\\
\hfuzz=500pt\includegraphics[width=1em]{element-mustset.pdf}~line & \hfuzz=500pt \hyperref[plotLineType]{plotLine} & \hfuzz=500pt \\
\hline
\end{tabularx}


\subsection{PoliticalBoundary}
Plots national boundaries. GMT provides them in different \config{resolution}s.


\keepXColumns
\begin{tabularx}{\textwidth}{N T A}
\hline
Name & Type & Annotation\\
\hline
\hfuzz=500pt\includegraphics[width=1em]{element-mustset.pdf}~resolution & \hfuzz=500pt choice & \hfuzz=500pt \\
\hfuzz=500pt\includegraphics[width=1em]{connector.pdf}\includegraphics[width=1em]{element-mustset.pdf}~crude & \hfuzz=500pt  & \hfuzz=500pt \\
\hfuzz=500pt\includegraphics[width=1em]{connector.pdf}\includegraphics[width=1em]{element-mustset.pdf}~low & \hfuzz=500pt  & \hfuzz=500pt \\
\hfuzz=500pt\includegraphics[width=1em]{connector.pdf}\includegraphics[width=1em]{element-mustset.pdf}~medium & \hfuzz=500pt  & \hfuzz=500pt \\
\hfuzz=500pt\includegraphics[width=1em]{connector.pdf}\includegraphics[width=1em]{element-mustset.pdf}~high & \hfuzz=500pt  & \hfuzz=500pt \\
\hfuzz=500pt\includegraphics[width=1em]{connector.pdf}\includegraphics[width=1em]{element-mustset.pdf}~full & \hfuzz=500pt  & \hfuzz=500pt \\
\hfuzz=500pt\includegraphics[width=1em]{element-mustset.pdf}~line & \hfuzz=500pt \hyperref[plotLineType]{plotLine} & \hfuzz=500pt \\
\hline
\end{tabularx}


\subsection{BlueMarble}
An image of the Earth's surface as seen from outer space -
the image is known as \emph{blue marble}. The directory of \config{inputfileChannels}
contains several files in different resolutions representing the Earth's surface each
month throughout a year.

\fig{!hb}{0.8}{blueMarble}{fig:blueMarbleMap}{The blue marble.}


\keepXColumns
\begin{tabularx}{\textwidth}{N T A}
\hline
Name & Type & Annotation\\
\hline
\hfuzz=500pt\includegraphics[width=1em]{element-mustset.pdf}~inputfileImage & \hfuzz=500pt filename & \hfuzz=500pt Blue Marble image file\\
\hfuzz=500pt\includegraphics[width=1em]{element.pdf}~brightness & \hfuzz=500pt double & \hfuzz=500pt brightness of bitmap [-1, 1]\\
\hfuzz=500pt\includegraphics[width=1em]{element.pdf}~illuminate & \hfuzz=500pt sequence & \hfuzz=500pt add hillshade based on topography\\
\hfuzz=500pt\includegraphics[width=1em]{connector.pdf}\includegraphics[width=1em]{element-mustset.pdf}~inputfileTopography & \hfuzz=500pt filename & \hfuzz=500pt GMT grid file containing topography.\\
\hfuzz=500pt\includegraphics[width=1em]{connector.pdf}\includegraphics[width=1em]{element.pdf}~azimuth & \hfuzz=500pt angle & \hfuzz=500pt direction of lighting source [deg]\\
\hfuzz=500pt\includegraphics[width=1em]{connector.pdf}\includegraphics[width=1em]{element.pdf}~elevation & \hfuzz=500pt angle & \hfuzz=500pt direction of lighting source [deg]\\
\hfuzz=500pt\includegraphics[width=1em]{connector.pdf}\includegraphics[width=1em]{element.pdf}~ambient & \hfuzz=500pt double & \hfuzz=500pt ambient lighting\\
\hfuzz=500pt\includegraphics[width=1em]{connector.pdf}\includegraphics[width=1em]{element.pdf}~diffuse & \hfuzz=500pt double & \hfuzz=500pt diffuse lighting\\
\hfuzz=500pt\includegraphics[width=1em]{connector.pdf}\includegraphics[width=1em]{element.pdf}~specular & \hfuzz=500pt double & \hfuzz=500pt specular reflection\\
\hfuzz=500pt\includegraphics[width=1em]{connector.pdf}\includegraphics[width=1em]{element.pdf}~shine & \hfuzz=500pt double & \hfuzz=500pt surface shine\\
\hfuzz=500pt\includegraphics[width=1em]{connector.pdf}\includegraphics[width=1em]{element.pdf}~amplitude & \hfuzz=500pt double & \hfuzz=500pt scale gradient by factor\\
\hline
\end{tabularx}


\subsection{Text}
Writes a \config{text} at \config{originLongitude} and \config{originLatitude} position in the map.
With \config{clip} the text is cutted at the boundaries of the plotting area.


\keepXColumns
\begin{tabularx}{\textwidth}{N T A}
\hline
Name & Type & Annotation\\
\hline
\hfuzz=500pt\includegraphics[width=1em]{element-mustset.pdf}~text & \hfuzz=500pt string & \hfuzz=500pt \\
\hfuzz=500pt\includegraphics[width=1em]{element-mustset.pdf}~originLongitude & \hfuzz=500pt angle & \hfuzz=500pt [deg]\\
\hfuzz=500pt\includegraphics[width=1em]{element-mustset.pdf}~originLatitude & \hfuzz=500pt angle & \hfuzz=500pt [deg]\\
\hfuzz=500pt\includegraphics[width=1em]{element.pdf}~offsetX & \hfuzz=500pt double & \hfuzz=500pt [cm] x-offset from origin\\
\hfuzz=500pt\includegraphics[width=1em]{element.pdf}~offsetY & \hfuzz=500pt double & \hfuzz=500pt [cm] y-offset from origin\\
\hfuzz=500pt\includegraphics[width=1em]{element.pdf}~alignment & \hfuzz=500pt string & \hfuzz=500pt L, C, R (left, center, right) and T, M, B (top, middle, bottom)\\
\hfuzz=500pt\includegraphics[width=1em]{element.pdf}~fontSize & \hfuzz=500pt double & \hfuzz=500pt \\
\hfuzz=500pt\includegraphics[width=1em]{element-mustset.pdf}~fontColor & \hfuzz=500pt \hyperref[plotColorType]{plotColor} & \hfuzz=500pt \\
\hfuzz=500pt\includegraphics[width=1em]{element.pdf}~clip & \hfuzz=500pt boolean & \hfuzz=500pt clip at boundaries\\
\hline
\end{tabularx}

\clearpage
%==================================

\section{PlotMapProjection}\label{plotMapProjectionType}
Selects the underlying projection of \program{PlotMap}.


\subsection{Robinson}
The Robinson projection, presented by Arthur H. Robinson in 1963,
is a modified cylindrical projection that is neither conformal nor equal-area.
Central meridian and all parallels are straight lines; other meridians are curved.
It uses lookup tables rather than analytic expressions to make the world map look right.


\keepXColumns
\begin{tabularx}{\textwidth}{N T A}
\hline
Name & Type & Annotation\\
\hline
\hfuzz=500pt\includegraphics[width=1em]{element.pdf}~centralMeridian & \hfuzz=500pt angle & \hfuzz=500pt central meridian [degree]\\
\hline
\end{tabularx}


\subsection{Orthographic}
The orthographic azimuthal projection is a perspective projection from infinite distance.
It is therefore often used to give the appearance of a globe viewed from space.


\keepXColumns
\begin{tabularx}{\textwidth}{N T A}
\hline
Name & Type & Annotation\\
\hline
\hfuzz=500pt\includegraphics[width=1em]{element.pdf}~lambdaCenter & \hfuzz=500pt angle & \hfuzz=500pt central point [degree]\\
\hfuzz=500pt\includegraphics[width=1em]{element.pdf}~phiCenter & \hfuzz=500pt angle & \hfuzz=500pt central point [degree]\\
\hline
\end{tabularx}


\subsection{Perspective sphere}
The orthographic azimuthal projection is a perspective projection from infinite distance.
It is therefore often used to give the appearance of a globe viewed from space.


\keepXColumns
\begin{tabularx}{\textwidth}{N T A}
\hline
Name & Type & Annotation\\
\hline
\hfuzz=500pt\includegraphics[width=1em]{element.pdf}~lambdaCenter & \hfuzz=500pt angle & \hfuzz=500pt longitude of central point in degrees\\
\hfuzz=500pt\includegraphics[width=1em]{element.pdf}~phiCenter & \hfuzz=500pt angle & \hfuzz=500pt latitude of central point in degrees\\
\hfuzz=500pt\includegraphics[width=1em]{element.pdf}~altitude & \hfuzz=500pt double & \hfuzz=500pt [km]\\
\hfuzz=500pt\includegraphics[width=1em]{element.pdf}~azimuth & \hfuzz=500pt angle & \hfuzz=500pt to the east of north of view [degrees]\\
\hfuzz=500pt\includegraphics[width=1em]{element.pdf}~tilt & \hfuzz=500pt angle & \hfuzz=500pt upward tilt of the plane of projection, if negative, then the view is centered on the horizon [degrees]\\
\hfuzz=500pt\includegraphics[width=1em]{element.pdf}~viewpointTwist & \hfuzz=500pt angle & \hfuzz=500pt clockwise twist of the viewpoint [degrees]\\
\hfuzz=500pt\includegraphics[width=1em]{element.pdf}~viewpointWidth & \hfuzz=500pt angle & \hfuzz=500pt width of the viewpoint [degrees]\\
\hfuzz=500pt\includegraphics[width=1em]{element.pdf}~viewpointHeight & \hfuzz=500pt angle & \hfuzz=500pt height of the viewpoint [degrees]\\
\hline
\end{tabularx}


\subsection{Polar}
Stereographic projection around given central point.


\keepXColumns
\begin{tabularx}{\textwidth}{N T A}
\hline
Name & Type & Annotation\\
\hline
\hfuzz=500pt\includegraphics[width=1em]{element.pdf}~lambdaCenter & \hfuzz=500pt angle & \hfuzz=500pt longitude of central point in degrees\\
\hfuzz=500pt\includegraphics[width=1em]{element.pdf}~phiCenter & \hfuzz=500pt angle & \hfuzz=500pt latitude of central point in degrees\\
\hline
\end{tabularx}


\subsection{Skyplot}
Skyplot used to plot azimuth/elevation data as generated by
\program{GnssAntennaDefinition2Skyplot} or \program{GnssResiduals2Skyplot}.


\subsection{UTM}
A particular subset of the transverse Mercator is the Universal Transverse Mercator (UTM)
which was adopted by the US Army for large-scale military maps.
Here, the globe is divided into 60 zones between 84$^{o}$S and 84$^{o}$N, most of which are 6$^{o}$ wide.
Each of these UTM zones have their unique central meridian.


\keepXColumns
\begin{tabularx}{\textwidth}{N T A}
\hline
Name & Type & Annotation\\
\hline
\hfuzz=500pt\includegraphics[width=1em]{element-mustset.pdf}~zone & \hfuzz=500pt string & \hfuzz=500pt UTM zone code (e.g. 33N)\\
\hline
\end{tabularx}


\subsection{Lambert}
This conic projection was designed by Lambert (1772) and has been used extensively for mapping of regions with predominantly east-west orientation.


\keepXColumns
\begin{tabularx}{\textwidth}{N T A}
\hline
Name & Type & Annotation\\
\hline
\hfuzz=500pt\includegraphics[width=1em]{element-mustset.pdf}~lambda0 & \hfuzz=500pt angle & \hfuzz=500pt longitude of projection center [deg]\\
\hfuzz=500pt\includegraphics[width=1em]{element-mustset.pdf}~phi0 & \hfuzz=500pt angle & \hfuzz=500pt latitude of projection centert [deg]\\
\hfuzz=500pt\includegraphics[width=1em]{element-mustset.pdf}~phi1 & \hfuzz=500pt angle & \hfuzz=500pt latitude of first standard parallel [deg]\\
\hfuzz=500pt\includegraphics[width=1em]{element-mustset.pdf}~phi2 & \hfuzz=500pt angle & \hfuzz=500pt latitude of first standard parallel [deg]\\
\hline
\end{tabularx}


\subsection{Linear}
Linear mapping of longitude/latitude to x/y (Plate Caree).


\subsection{Mollweide}
This pseudo-cylindrical, equal-area projection was developed by Mollweide in 1805. Parallels are unequally spaced straight
lines with the meridians being equally spaced elliptical arcs. The scale is only true along latitudes 40$^{o}$44' north and south.
The projection is used mainly for global maps showing data distributions.


\keepXColumns
\begin{tabularx}{\textwidth}{N T A}
\hline
Name & Type & Annotation\\
\hline
\hfuzz=500pt\includegraphics[width=1em]{element.pdf}~centralMeridian & \hfuzz=500pt angle & \hfuzz=500pt central meridian [degree]\\
\hline
\end{tabularx}

\clearpage
%==================================

\section{PlotSymbol}\label{plotSymbolType}
Plots a symbol as used e.g. in \configClass{plotGraphLayer:linesAndPoints}{plotGraphLayerType:linesAndPoints}
or \configClass{plotMapLayer:points}{plotMapLayerType:points}.


\keepXColumns
\begin{tabularx}{\textwidth}{N T A}
\hline
Name & Type & Annotation\\
\hline
\hfuzz=500pt\includegraphics[width=1em]{element-mustset.pdf}~plotSymbolType & \hfuzz=500pt choice & \hfuzz=500pt symbol\\
\hfuzz=500pt\includegraphics[width=1em]{connector.pdf}\includegraphics[width=1em]{element-mustset.pdf}~circle & \hfuzz=500pt sequence & \hfuzz=500pt \\
\hfuzz=500pt\quad\includegraphics[width=1em]{connector.pdf}\includegraphics[width=1em]{element.pdf}~color & \hfuzz=500pt \hyperref[plotColorType]{plotColor} & \hfuzz=500pt empty: determined from value\\
\hfuzz=500pt\quad\includegraphics[width=1em]{connector.pdf}\includegraphics[width=1em]{element.pdf}~size & \hfuzz=500pt double & \hfuzz=500pt size of symbol [point]\\
\hfuzz=500pt\quad\includegraphics[width=1em]{connector.pdf}\includegraphics[width=1em]{element.pdf}~blackContour & \hfuzz=500pt boolean & \hfuzz=500pt \\
\hfuzz=500pt\includegraphics[width=1em]{connector.pdf}\includegraphics[width=1em]{element-mustset.pdf}~star & \hfuzz=500pt sequence & \hfuzz=500pt \\
\hfuzz=500pt\quad\includegraphics[width=1em]{connector.pdf}\includegraphics[width=1em]{element.pdf}~color & \hfuzz=500pt \hyperref[plotColorType]{plotColor} & \hfuzz=500pt empty: determined from value\\
\hfuzz=500pt\quad\includegraphics[width=1em]{connector.pdf}\includegraphics[width=1em]{element.pdf}~size & \hfuzz=500pt double & \hfuzz=500pt size of symbol [point]\\
\hfuzz=500pt\quad\includegraphics[width=1em]{connector.pdf}\includegraphics[width=1em]{element.pdf}~blackContour & \hfuzz=500pt boolean & \hfuzz=500pt \\
\hfuzz=500pt\includegraphics[width=1em]{connector.pdf}\includegraphics[width=1em]{element-mustset.pdf}~cross & \hfuzz=500pt sequence & \hfuzz=500pt \\
\hfuzz=500pt\quad\includegraphics[width=1em]{connector.pdf}\includegraphics[width=1em]{element.pdf}~color & \hfuzz=500pt \hyperref[plotColorType]{plotColor} & \hfuzz=500pt empty: determined from value\\
\hfuzz=500pt\quad\includegraphics[width=1em]{connector.pdf}\includegraphics[width=1em]{element.pdf}~size & \hfuzz=500pt double & \hfuzz=500pt size of symbol [point]\\
\hfuzz=500pt\quad\includegraphics[width=1em]{connector.pdf}\includegraphics[width=1em]{element.pdf}~blackContour & \hfuzz=500pt boolean & \hfuzz=500pt \\
\hfuzz=500pt\includegraphics[width=1em]{connector.pdf}\includegraphics[width=1em]{element-mustset.pdf}~square & \hfuzz=500pt sequence & \hfuzz=500pt \\
\hfuzz=500pt\quad\includegraphics[width=1em]{connector.pdf}\includegraphics[width=1em]{element.pdf}~color & \hfuzz=500pt \hyperref[plotColorType]{plotColor} & \hfuzz=500pt empty: determined from value\\
\hfuzz=500pt\quad\includegraphics[width=1em]{connector.pdf}\includegraphics[width=1em]{element.pdf}~size & \hfuzz=500pt double & \hfuzz=500pt size of symbol [point]\\
\hfuzz=500pt\quad\includegraphics[width=1em]{connector.pdf}\includegraphics[width=1em]{element.pdf}~blackContour & \hfuzz=500pt boolean & \hfuzz=500pt \\
\hfuzz=500pt\includegraphics[width=1em]{connector.pdf}\includegraphics[width=1em]{element-mustset.pdf}~triangle & \hfuzz=500pt sequence & \hfuzz=500pt \\
\hfuzz=500pt\quad\includegraphics[width=1em]{connector.pdf}\includegraphics[width=1em]{element.pdf}~color & \hfuzz=500pt \hyperref[plotColorType]{plotColor} & \hfuzz=500pt empty: determined from value\\
\hfuzz=500pt\quad\includegraphics[width=1em]{connector.pdf}\includegraphics[width=1em]{element.pdf}~size & \hfuzz=500pt double & \hfuzz=500pt size of symbol [point]\\
\hfuzz=500pt\quad\includegraphics[width=1em]{connector.pdf}\includegraphics[width=1em]{element.pdf}~blackContour & \hfuzz=500pt boolean & \hfuzz=500pt \\
\hfuzz=500pt\includegraphics[width=1em]{connector.pdf}\includegraphics[width=1em]{element-mustset.pdf}~diamond & \hfuzz=500pt sequence & \hfuzz=500pt \\
\hfuzz=500pt\quad\includegraphics[width=1em]{connector.pdf}\includegraphics[width=1em]{element.pdf}~color & \hfuzz=500pt \hyperref[plotColorType]{plotColor} & \hfuzz=500pt empty: determined from value\\
\hfuzz=500pt\quad\includegraphics[width=1em]{connector.pdf}\includegraphics[width=1em]{element.pdf}~size & \hfuzz=500pt double & \hfuzz=500pt size of symbol [point]\\
\hfuzz=500pt\quad\includegraphics[width=1em]{connector.pdf}\includegraphics[width=1em]{element.pdf}~blackContour & \hfuzz=500pt boolean & \hfuzz=500pt \\
\hfuzz=500pt\includegraphics[width=1em]{connector.pdf}\includegraphics[width=1em]{element-mustset.pdf}~dash & \hfuzz=500pt sequence & \hfuzz=500pt \\
\hfuzz=500pt\quad\includegraphics[width=1em]{connector.pdf}\includegraphics[width=1em]{element.pdf}~color & \hfuzz=500pt \hyperref[plotColorType]{plotColor} & \hfuzz=500pt empty: determined from value\\
\hfuzz=500pt\quad\includegraphics[width=1em]{connector.pdf}\includegraphics[width=1em]{element.pdf}~size & \hfuzz=500pt double & \hfuzz=500pt size of symbol [point]\\
\hfuzz=500pt\quad\includegraphics[width=1em]{connector.pdf}\includegraphics[width=1em]{element.pdf}~blackContour & \hfuzz=500pt boolean & \hfuzz=500pt \\
\hline
\end{tabularx}

\clearpage
%==================================

\section{PodRightSide}\label{podRightSideType}
Observation vector for precise orbit data (POD) of \configClass{observation}{observationType}
equations in a least squares adjustment. The observations are reduced by the effect of
\configFile{inputfileAccelerometer}{instrument} and \configClass{forces}{forcesType}
(observed minus computed).


\keepXColumns
\begin{tabularx}{\textwidth}{N T A}
\hline
Name & Type & Annotation\\
\hline
\hfuzz=500pt\includegraphics[width=1em]{element-mustset.pdf}~podRightSideType & \hfuzz=500pt sequence & \hfuzz=500pt \\
\hfuzz=500pt\includegraphics[width=1em]{connector.pdf}\includegraphics[width=1em]{element-mustset.pdf}~inputfileOrbit & \hfuzz=500pt filename & \hfuzz=500pt kinematic positions of satellite as observations\\
\hfuzz=500pt\includegraphics[width=1em]{connector.pdf}\includegraphics[width=1em]{element.pdf}~inputfileAccelerometer & \hfuzz=500pt filename & \hfuzz=500pt non-gravitational forces in satellite reference frame\\
\hfuzz=500pt\includegraphics[width=1em]{connector.pdf}\includegraphics[width=1em]{element-mustset.pdf}~forces & \hfuzz=500pt \hyperref[forcesType]{forces} & \hfuzz=500pt \\
\hline
\end{tabularx}

\clearpage
%==================================

\section{SggRightSide}\label{sggRightSideType}
Observation vector for gradiometer data (satellite gravity gradiometry, SGG)
of \configClass{observation}{observationType} equations in a least squares adjustment.
The observations are reduced by an \configFile{inputfileReferenceGradiometer}{instrument},
the effect of \configClass{referencefield}{gravityfieldType}, and \configClass{tides}{tidesType}
(observed minus computed).

The reference gradiometer data can be precomputed with \program{SimulateGradiometer}.


\keepXColumns
\begin{tabularx}{\textwidth}{N T A}
\hline
Name & Type & Annotation\\
\hline
\hfuzz=500pt\includegraphics[width=1em]{element-mustset.pdf}~sggRightSideType & \hfuzz=500pt sequence & \hfuzz=500pt \\
\hfuzz=500pt\includegraphics[width=1em]{connector.pdf}\includegraphics[width=1em]{element-mustset.pdf}~inputfileGradiometer & \hfuzz=500pt filename & \hfuzz=500pt observed gravity gradients\\
\hfuzz=500pt\includegraphics[width=1em]{connector.pdf}\includegraphics[width=1em]{element-unbounded.pdf}~inputfileReferenceGradiometer & \hfuzz=500pt filename & \hfuzz=500pt precomputed gradients at orbit positions\\
\hfuzz=500pt\includegraphics[width=1em]{connector.pdf}\includegraphics[width=1em]{element-unbounded.pdf}~referencefield & \hfuzz=500pt \hyperref[gravityfieldType]{gravityfield} & \hfuzz=500pt \\
\hfuzz=500pt\includegraphics[width=1em]{connector.pdf}\includegraphics[width=1em]{element-unbounded.pdf}~tides & \hfuzz=500pt \hyperref[tidesType]{tides} & \hfuzz=500pt \\
\hline
\end{tabularx}

\clearpage
%==================================

\section{SphericalHarmonicsFilter}\label{sphericalHarmonicsFilterType}
Filtering of a spherical harmonics expansion.


\subsection{DDK}
Orderwise filtering with the DDK filter by Kusche et al. 2009.


\keepXColumns
\begin{tabularx}{\textwidth}{N T A}
\hline
Name & Type & Annotation\\
\hline
\hfuzz=500pt\includegraphics[width=1em]{element-mustset.pdf}~level & \hfuzz=500pt uint & \hfuzz=500pt DDK filter level (1, 2, 3, ...)\\
\hfuzz=500pt\includegraphics[width=1em]{element-mustset.pdf}~inputfileNormalEquation & \hfuzz=500pt filename & \hfuzz=500pt \\
\hline
\end{tabularx}


\subsection{Gauss}
Filtering the spherical harmonics expansion with a Gaussian filter.
\config{radius} gives the filter radius on the Earth surface in km.


\keepXColumns
\begin{tabularx}{\textwidth}{N T A}
\hline
Name & Type & Annotation\\
\hline
\hfuzz=500pt\includegraphics[width=1em]{element-mustset.pdf}~radius & \hfuzz=500pt double & \hfuzz=500pt filter radius [km]\\
\hline
\end{tabularx}


\subsection{Matrix}
Filtering the spherical harmonics expansion with a matrix filter.


\keepXColumns
\begin{tabularx}{\textwidth}{N T A}
\hline
Name & Type & Annotation\\
\hline
\hfuzz=500pt\includegraphics[width=1em]{element-mustset.pdf}~inputfileMatrix & \hfuzz=500pt filename & \hfuzz=500pt \\
\hfuzz=500pt\includegraphics[width=1em]{element-mustset.pdf}~minDegree & \hfuzz=500pt uint & \hfuzz=500pt of matrix\\
\hfuzz=500pt\includegraphics[width=1em]{element-mustset.pdf}~maxDegree & \hfuzz=500pt uint & \hfuzz=500pt of matrix\\
\hfuzz=500pt\includegraphics[width=1em]{element-mustset.pdf}~numbering & \hfuzz=500pt \hyperref[sphericalHarmonicsNumberingType]{sphericalHarmonicsNumbering} & \hfuzz=500pt numbering scheme of the matrix\\
\hline
\end{tabularx}

\clearpage
%==================================

\section{SphericalHarmonicsNumbering}\label{sphericalHarmonicsNumberingType}
This class organizes the numbering scheme of spherical harmonics coefficients
in a parameter vector (e.g \program{Gravityfield2SphericalHarmonicsVector} and the design matrix of
\configClass{parametrizationGravity:sphericalHarmoncis}{parametrizationGravityType:sphericalHarmonics}.


\subsection{Degree}
Numbering degree by degree:
\[ c20, c21, s21, c22, s22, c30, c31, s31, c32, s32,\ldots \]


\subsection{Order}
Numbering order by order:
\[ c20, c30, c40, \ldots, c21, s21, c31, s31, \ldots, c22, s22 \]


\subsection{OrderNonAlternating}
Numbering order by order with cnm, snm non-alternating:
\[ c20, c30, c40, \ldots, c21, c31, c41, \ldots, s21, s31, s41, \]


\subsection{File}
Numbering as specified in the chosen file.
The \configFile{inputfile}{matrix} is a matrix with the first column indicating cnm/snm with 0 or 1.
The second and third column specify degree and order.


\keepXColumns
\begin{tabularx}{\textwidth}{N T A}
\hline
Name & Type & Annotation\\
\hline
\hfuzz=500pt\includegraphics[width=1em]{element-mustset.pdf}~inputfile & \hfuzz=500pt filename & \hfuzz=500pt \\
\hline
\end{tabularx}

\clearpage
%==================================

\section{SstRightSide}\label{sstRightSideType}
Observation vector for GRACE like data (satellite-tracking and precise orbit data (POD))
of \configClass{observation}{observationType} equations in a least squares adjustment.
The observations are reduced by the effect of \configFile{inputfileAccelerometer}{instrument}
and \configClass{forces}{forcesType} (observed minus computed).


\keepXColumns
\begin{tabularx}{\textwidth}{N T A}
\hline
Name & Type & Annotation\\
\hline
\hfuzz=500pt\includegraphics[width=1em]{element-mustset.pdf}~sstRightSideType & \hfuzz=500pt sequence & \hfuzz=500pt \\
\hfuzz=500pt\includegraphics[width=1em]{connector.pdf}\includegraphics[width=1em]{element-unbounded.pdf}~inputfileSatelliteTracking & \hfuzz=500pt filename & \hfuzz=500pt ranging observations and corrections\\
\hfuzz=500pt\includegraphics[width=1em]{connector.pdf}\includegraphics[width=1em]{element.pdf}~inputfileOrbit1 & \hfuzz=500pt filename & \hfuzz=500pt kinematic positions of satellite A as observations\\
\hfuzz=500pt\includegraphics[width=1em]{connector.pdf}\includegraphics[width=1em]{element.pdf}~inputfileOrbit2 & \hfuzz=500pt filename & \hfuzz=500pt kinematic positions of satellite B as observations\\
\hfuzz=500pt\includegraphics[width=1em]{connector.pdf}\includegraphics[width=1em]{element.pdf}~inputfileAccelerometer1 & \hfuzz=500pt filename & \hfuzz=500pt non-gravitational forces in satellite reference frame A\\
\hfuzz=500pt\includegraphics[width=1em]{connector.pdf}\includegraphics[width=1em]{element.pdf}~inputfileAccelerometer2 & \hfuzz=500pt filename & \hfuzz=500pt non-gravitational forces in satellite reference frame B\\
\hfuzz=500pt\includegraphics[width=1em]{connector.pdf}\includegraphics[width=1em]{element-mustset.pdf}~forces & \hfuzz=500pt \hyperref[forcesType]{forces} & \hfuzz=500pt \\
\hline
\end{tabularx}

\clearpage
%==================================

\section{Thermosphere}\label{thermosphereType}
This class provides functions for calculating the density, temperature and velocity
in the thermosphere.
The wind is computed by HWM14 model if \config{hwm14DataDirectory} is provided.
A quiet thermosphere is assumed if \config{inputfileMagnetic3hAp} is not given.


\subsection{JB2008}
Thermosphere parameters from the JB2008 model:

Bowman, B. R., Tobiska, W. K., Marcos, F. A., Huang, C. Y., Lin, C. S., Burke, W. J. (2008).
A new empirical thermospheric density model JB2008 using new solar and geomagnetic indices.
 In AIAA/AAS Astrodynamics Specialist Conference and Exhibit. \url{https://doi.org/10.2514/6.2008-6438}


\keepXColumns
\begin{tabularx}{\textwidth}{N T A}
\hline
Name & Type & Annotation\\
\hline
\hfuzz=500pt\includegraphics[width=1em]{element-mustset.pdf}~inputfileSolfsmy & \hfuzz=500pt filename & \hfuzz=500pt solar indices\\
\hfuzz=500pt\includegraphics[width=1em]{element-mustset.pdf}~inputfileDtc & \hfuzz=500pt filename & \hfuzz=500pt \\
\hfuzz=500pt\includegraphics[width=1em]{element.pdf}~inputfileMagnetic3hAp & \hfuzz=500pt filename & \hfuzz=500pt indicies for wind model\\
\hfuzz=500pt\includegraphics[width=1em]{element.pdf}~hwm14DataDirectory & \hfuzz=500pt filename & \hfuzz=500pt directory containing dwm07b104i.dat, gd2qd.dat, hwm123114.bin\\
\hline
\end{tabularx}


\subsection{NRLMSIS2}
Thermosphere parameters from the NRLMSIS2 model:

Emmert J.D, D.P.Drob, J.M. Picone, et al. (2020), NRLMSIS 2.0: A whole-atmosphere empirical
model of temperature and neutral species densities. Earth and Space Science, Volume 8, 3
\url{https://doi.org/10.1029/2020EA001321}


\keepXColumns
\begin{tabularx}{\textwidth}{N T A}
\hline
Name & Type & Annotation\\
\hline
\hfuzz=500pt\includegraphics[width=1em]{element-mustset.pdf}~inputfileMsis & \hfuzz=500pt filename & \hfuzz=500pt input NRLMSIS 2.0\\
\hfuzz=500pt\includegraphics[width=1em]{element-mustset.pdf}~inputfileModelParameters & \hfuzz=500pt filename & \hfuzz=500pt path to msis20.parm file\\
\hfuzz=500pt\includegraphics[width=1em]{element.pdf}~inputfileMagnetic3hAp & \hfuzz=500pt filename & \hfuzz=500pt indicies for wind model\\
\hfuzz=500pt\includegraphics[width=1em]{element.pdf}~hwm14DataDirectory & \hfuzz=500pt filename & \hfuzz=500pt directory containing dwm07b104i.dat, gd2qd.dat, hwm123114.bin\\
\hline
\end{tabularx}

\clearpage
%==================================

\section{Tides}\label{tidesType}
This class computes functionals of the time depending tide potential,
e.g potential, acceleration or gravity gradients.

If several instances of the class are given the results are summed up.
Before summation every single result is multiplicated by a \config{factor}.
To get the difference between two ocean tide models you must choose one factor by 1
and the other by -1. To get the mean of two models just set each factor to 0.5.


\subsection{AstronomicalTide}\label{tidesType:astronomicalTide}
This class computes the tide generating potential (TGP) of sun, moon
and planets (Mercury, Venus, Mars, Jupiter, Saturn).
It takes into account the flattening of the Earth (At the moment only at the acceleration level).

The computed result is multiplied with \config{factor}.


\keepXColumns
\begin{tabularx}{\textwidth}{N T A}
\hline
Name & Type & Annotation\\
\hline
\hfuzz=500pt\includegraphics[width=1em]{element.pdf}~useMoon & \hfuzz=500pt boolean & \hfuzz=500pt TGP of moon\\
\hfuzz=500pt\includegraphics[width=1em]{element.pdf}~useSun & \hfuzz=500pt boolean & \hfuzz=500pt TGP of sun\\
\hfuzz=500pt\includegraphics[width=1em]{element.pdf}~usePlanets & \hfuzz=500pt boolean & \hfuzz=500pt TGP of planets\\
\hfuzz=500pt\includegraphics[width=1em]{element.pdf}~useEarth & \hfuzz=500pt boolean & \hfuzz=500pt TGP of Earth\\
\hfuzz=500pt\includegraphics[width=1em]{element.pdf}~c20Earth & \hfuzz=500pt double & \hfuzz=500pt J2 flattening of the Earth\\
\hfuzz=500pt\includegraphics[width=1em]{element.pdf}~factor & \hfuzz=500pt double & \hfuzz=500pt the result is multiplied by this factor, set -1 to subtract the field\\
\hline
\end{tabularx}


\subsection{EarthTide}\label{tidesType:earthTide}
This class computes the earth tide according to the IERS2003 conventions.
The values of solid Earth tide external potential Love numbers and
the frequency dependent corrections of these values are given in the file
\configFile{inputfileEarthtide}{earthTide}. The effect of the permanent tide is removed if
\config{includePermanentTide} is set to false.

The computed result is multiplied with \config{factor}.


\keepXColumns
\begin{tabularx}{\textwidth}{N T A}
\hline
Name & Type & Annotation\\
\hline
\hfuzz=500pt\includegraphics[width=1em]{element-mustset.pdf}~inputfileEarthtide & \hfuzz=500pt filename & \hfuzz=500pt \\
\hfuzz=500pt\includegraphics[width=1em]{element.pdf}~includePermanentTide & \hfuzz=500pt boolean & \hfuzz=500pt results in FALSE: zero tide, TRUE: tide free gravity field\\
\hfuzz=500pt\includegraphics[width=1em]{element.pdf}~factor & \hfuzz=500pt double & \hfuzz=500pt the result is multiplied by this factor, set -1 to subtract the field\\
\hline
\end{tabularx}


\subsection{PoleTide}\label{tidesType:poleTide}
The potential coefficients of the solid Earth pole tide according to the
IERS2003 conventions are given by
\begin{equation}
\begin{split}
\Delta c_{21} &= s\cdot(m_1 + o\cdot m_2), \\
\Delta s_{21} &= s\cdot(m_2 - o\cdot m_1),
\end{split}
\end{equation}
with $s$ is the \config{scale}, $o$ is the \config{outPhase} and
$(m_1,m_2)$ are the wobble variables in seconds of arc.
They are related to the polar motion variables $(x_p,y_p)$ according to
\begin{equation}
\begin{split}
m_1 &=  (x_p - \bar{x}_p), \\
m_2 &= -(y_p - \bar{y}_p),
\end{split}
\end{equation}
The mean pole $(\bar{x}_p, \bar{y}_p)$ is approximated by a polynomial
read from \configFile{inputfileMeanPole}{meanPolarMotion}.

The displacment is calculated with
\begin{equation}
\begin{split}
S_r          &= -v\sin2\vartheta(m_1\cos\lambda+m_2\sin\lambda),\\
S_\vartheta &= -h\cos2\vartheta(m_1\cos\lambda+m_2\sin\lambda),\\
S_\lambda   &=  h\cos\vartheta(m_1\sin\lambda-m_2\cos\lambda),
\end{split}
\end{equation}
where $h$ is the \config{horizontalDisplacement}
and $v$ is the \config{verticalDisplacement}.

The computed result is multiplied with \config{factor}.


\keepXColumns
\begin{tabularx}{\textwidth}{N T A}
\hline
Name & Type & Annotation\\
\hline
\hfuzz=500pt\includegraphics[width=1em]{element.pdf}~scale & \hfuzz=500pt double & \hfuzz=500pt \\
\hfuzz=500pt\includegraphics[width=1em]{element.pdf}~outPhase & \hfuzz=500pt double & \hfuzz=500pt \\
\hfuzz=500pt\includegraphics[width=1em]{element-mustset.pdf}~inputfileMeanPole & \hfuzz=500pt filename & \hfuzz=500pt \\
\hfuzz=500pt\includegraphics[width=1em]{element.pdf}~horizontalDisplacement & \hfuzz=500pt double & \hfuzz=500pt [m]\\
\hfuzz=500pt\includegraphics[width=1em]{element.pdf}~verticalDisplacement & \hfuzz=500pt double & \hfuzz=500pt [m]\\
\hfuzz=500pt\includegraphics[width=1em]{element.pdf}~factor & \hfuzz=500pt double & \hfuzz=500pt the result is multiplied by this factor, set -1 to subtract the field\\
\hline
\end{tabularx}


\subsection{OceanPoleTide}\label{tidesType:oceanPoleTide}
The ocean pole tide is generated by the centrifugal effect of polar motion on the oceans.
The potential coefficients of this effect is given by
IERS2003 conventions are given by
\begin{equation}
\begin{Bmatrix}
\Delta c_{nm}  \\
\Delta s_{nm}
\end{Bmatrix}=
\begin{Bmatrix}
c_{nm}^R  \\
s_{nm}^R
\end{Bmatrix}
(m_1\gamma^R+m_2\gamma^I)+
\begin{Bmatrix}
c_{nm}^I  \\
s_{nm}^I
\end{Bmatrix}
(m_2\gamma^R-m_1\gamma^I)
\end{equation}
where the coefficients are read from file \configFile{inputfileOceanPole}{oceanPoleTide},
$\gamma=\gamma^R+i\gamma^I$ is given by \config{gammaReal} and
\config{gammaImaginary} and $(m_1,m_2)$ are the wobble variables in radians.
They are related to the polar motion variables $(x_p,y_p)$ according to
\begin{equation}
\begin{split}
m_1 &=  (x_p - \bar{x}_p), \\
m_2 &= -(y_p - \bar{y}_p),
\end{split}
\end{equation}
The mean pole $(\bar{x}_p, \bar{y}_p)$ is approximated by a polynomial
read from \configFile{inputfileMeanPole}{meanPolarMotion}.

The computed result is multiplied with \config{factor}.


\keepXColumns
\begin{tabularx}{\textwidth}{N T A}
\hline
Name & Type & Annotation\\
\hline
\hfuzz=500pt\includegraphics[width=1em]{element-mustset.pdf}~inputfileOceanPole & \hfuzz=500pt filename & \hfuzz=500pt \\
\hfuzz=500pt\includegraphics[width=1em]{element.pdf}~minDegree & \hfuzz=500pt uint & \hfuzz=500pt \\
\hfuzz=500pt\includegraphics[width=1em]{element.pdf}~maxDegree & \hfuzz=500pt uint & \hfuzz=500pt \\
\hfuzz=500pt\includegraphics[width=1em]{element.pdf}~gammaReal & \hfuzz=500pt double & \hfuzz=500pt \\
\hfuzz=500pt\includegraphics[width=1em]{element.pdf}~gammaImaginary & \hfuzz=500pt double & \hfuzz=500pt \\
\hfuzz=500pt\includegraphics[width=1em]{element-mustset.pdf}~inputfileMeanPole & \hfuzz=500pt filename & \hfuzz=500pt \\
\hfuzz=500pt\includegraphics[width=1em]{element.pdf}~factor & \hfuzz=500pt double & \hfuzz=500pt the result is multiplied by this factor, set -1 to subtract the field\\
\hline
\end{tabularx}


\subsection{DoodsonHarmonicTide}\label{tidesType:doodsonHarmonicTide}
The time variable potential of ocean tides is given by a fourier expansion
\begin{equation}
V(\M x,t) = \sum_{f} V_f^c(\M x)\cos(\Theta_f(t)) + V_f^s(\M x)\sin(\Theta_f(t)),
\end{equation}
where $V_f^c(\M x)$ and $V_f^s(\M x)$ are spherical harmonics expansions and are
read from the file \configFile{inputfileDoodsonHarmonic}{doodsonHarmonic}.
If set the expansion is limited in the range between \config{minDegree}
and \config{maxDegree} inclusivly.
$\Theta_f(t)$ are the arguments of the tide constituents~$f$:
\begin{equation}
\Theta_f(t) = \sum_{i=1}^6 n_f^i\beta_i(t),
\end{equation}
where $\beta_i(t)$ are the Doodson's fundamental arguments ($\tau,s,h,p,N',p_s$)
and $n_f^i$ are the Doodson multipliers for the term at frequency~$f$.

The major constituents given by \configFile{inputfileDoodsonHarmonic}{doodsonHarmonic} can be used to
interpolate minor tidal constituents using the file \configFile{inputfileAdmittance}{admittance}.
This file can be created with \program{DoodsonHarmonicsCalculateAdmittance}.

After the interpolation step a selection of the computed constituents can be
choosen by \configClass{selectDoodson}{doodson}. Only these constiuents are considered for the results.
If no \configClass{selectDoodson}{doodson} is set all constituents will be used. The constituents can
be coded as Doodson number (e.g. 255.555) or as names intoduced by Darwin (e.g. M2).

The computed result is multiplied with \config{factor}.


\keepXColumns
\begin{tabularx}{\textwidth}{N T A}
\hline
Name & Type & Annotation\\
\hline
\hfuzz=500pt\includegraphics[width=1em]{element-mustset.pdf}~inputfileTides & \hfuzz=500pt filename & \hfuzz=500pt \\
\hfuzz=500pt\includegraphics[width=1em]{element.pdf}~inputfileAdmittance & \hfuzz=500pt filename & \hfuzz=500pt interpolation of minor constituents\\
\hfuzz=500pt\includegraphics[width=1em]{element-unbounded.pdf}~selectDoodson & \hfuzz=500pt \hyperref[doodson]{doodson} & \hfuzz=500pt consider only these constituents, code number (e.g. 255.555) or darwin name (e.g. M2)\\
\hfuzz=500pt\includegraphics[width=1em]{element.pdf}~minDegree & \hfuzz=500pt uint & \hfuzz=500pt \\
\hfuzz=500pt\includegraphics[width=1em]{element.pdf}~maxDegree & \hfuzz=500pt uint & \hfuzz=500pt \\
\hfuzz=500pt\includegraphics[width=1em]{element.pdf}~nodeCorr & \hfuzz=500pt uint & \hfuzz=500pt nodal corrections: 0-no corr, 1-IHO, 2-Schureman\\
\hfuzz=500pt\includegraphics[width=1em]{element.pdf}~factor & \hfuzz=500pt double & \hfuzz=500pt the result is multiplied by this factor, set -1 to subtract the field\\
\hline
\end{tabularx}


\subsection{Centrifugal}\label{tidesType:centrifugal}
Computes the centrifugal potential in a rotating system
\begin{equation}
V(\M r, t) = \frac{1}{2} (\M\omega(t)\times\M r)^2.
\end{equation}
The current rotation vector $\M\omega(t)$ is computed from the
\configClass{earthRotation}{earthRotationType}
provided by the calling program.
The computed result is multiplied with \config{factor}.

Be careful, the centrifugal potential is not harmonic.
Convolution with a harmonic kernel (e.g. to compute gravity
anomalies) is not meaningful.


\keepXColumns
\begin{tabularx}{\textwidth}{N T A}
\hline
Name & Type & Annotation\\
\hline
\hfuzz=500pt\includegraphics[width=1em]{element.pdf}~factor & \hfuzz=500pt double & \hfuzz=500pt the result is multiplied by this factor, set -1 to subtract the field\\
\hline
\end{tabularx}


\subsection{SolidMoonTide}
This class computes the solid moon tide according to the IERS2010 conventions.
The values of solid Moon tide external potential Love numbers are given and
there are no frequency dependent corrections of these values.
The computed result is multiplied with \config{factor}.


\keepXColumns
\begin{tabularx}{\textwidth}{N T A}
\hline
Name & Type & Annotation\\
\hline
\hfuzz=500pt\includegraphics[width=1em]{element.pdf}~k20 & \hfuzz=500pt double & \hfuzz=500pt \\
\hfuzz=500pt\includegraphics[width=1em]{element.pdf}~k30 & \hfuzz=500pt double & \hfuzz=500pt \\
\hfuzz=500pt\includegraphics[width=1em]{element.pdf}~factor & \hfuzz=500pt double & \hfuzz=500pt the result is multiplied by this factor, set -1 to subtract the field\\
\hline
\end{tabularx}

\clearpage
%==================================

\section{TimeSeries}\label{timeSeriesType}
This class generates a series of points in time. The series is always sorted in ascending order.
Depending of the application the series is interpreted as list of points or as intervals between the points.

\fig{!hb}{0.4}{timeSeriesIntervals}{fig:timeSeriesIntervals}{List of points $t_i$ vs. intervals $T_i$.}


\subsection{UniformSampling}\label{timeSeriesType:uniformSampling}
Generates a time series with uniform sampling. The first point in time will be \config{timeStart}.
The last generated point in time will be less or equal \config{timeEnd}.
The time step between generated points in time is given by \config{sampling}.


\keepXColumns
\begin{tabularx}{\textwidth}{N T A}
\hline
Name & Type & Annotation\\
\hline
\hfuzz=500pt\includegraphics[width=1em]{element-mustset.pdf}~timeStart & \hfuzz=500pt time & \hfuzz=500pt first point in time\\
\hfuzz=500pt\includegraphics[width=1em]{element-mustset.pdf}~timeEnd & \hfuzz=500pt time & \hfuzz=500pt last point in time will be less or equal timeEnd\\
\hfuzz=500pt\includegraphics[width=1em]{element-mustset.pdf}~sampling & \hfuzz=500pt time & \hfuzz=500pt time step between points in time\\
\hline
\end{tabularx}


\subsection{UniformInterval}
Generates a time series with uniform sampling between \config{timeStart} and \config{timeEnd}.
\config{intervallCount} gives the count of intervals. This class generates count+1 points in time
inclusive \config{timeStart} and \config{timeEnd}.


\keepXColumns
\begin{tabularx}{\textwidth}{N T A}
\hline
Name & Type & Annotation\\
\hline
\hfuzz=500pt\includegraphics[width=1em]{element-mustset.pdf}~timeStart & \hfuzz=500pt time & \hfuzz=500pt 1st point of the time series\\
\hfuzz=500pt\includegraphics[width=1em]{element-mustset.pdf}~timeEnd & \hfuzz=500pt time & \hfuzz=500pt last point of the time series\\
\hfuzz=500pt\includegraphics[width=1em]{element-mustset.pdf}~intervalCount & \hfuzz=500pt uint & \hfuzz=500pt count of intervals, count+1 points in time will generated\\
\hline
\end{tabularx}


\subsection{Irregular}\label{timeSeriesType:irregular}
The points of the time series are given explicitly with \config{time}.


\keepXColumns
\begin{tabularx}{\textwidth}{N T A}
\hline
Name & Type & Annotation\\
\hline
\hfuzz=500pt\includegraphics[width=1em]{element-mustset-unbounded.pdf}~time & \hfuzz=500pt time & \hfuzz=500pt explicit list of points in time\\
\hline
\end{tabularx}


\subsection{Monthly}
If \config{monthMiddle} is set, time points are generated at mid of each month inclusively
the \config{monthStart} in \config{yearStart} and \config{monthEnd} in \config{yearEnd}.
Otherwise times are given at the first of each month and a time point after the last month.


\keepXColumns
\begin{tabularx}{\textwidth}{N T A}
\hline
Name & Type & Annotation\\
\hline
\hfuzz=500pt\includegraphics[width=1em]{element-mustset.pdf}~monthStart & \hfuzz=500pt uint & \hfuzz=500pt \\
\hfuzz=500pt\includegraphics[width=1em]{element-mustset.pdf}~yearStart & \hfuzz=500pt uint & \hfuzz=500pt \\
\hfuzz=500pt\includegraphics[width=1em]{element-mustset.pdf}~monthEnd & \hfuzz=500pt uint & \hfuzz=500pt \\
\hfuzz=500pt\includegraphics[width=1em]{element-mustset.pdf}~yearEnd & \hfuzz=500pt uint & \hfuzz=500pt \\
\hfuzz=500pt\includegraphics[width=1em]{element.pdf}~useMonthMiddle & \hfuzz=500pt boolean & \hfuzz=500pt time points are mid of months, otherwise the 1st of each month + a time point behind the last month\\
\hline
\end{tabularx}


\subsection{Yearly}
If \config{yearMiddle} is set, time points are generated at mid of each year inclusively \config{yearStart}
and \config{yearEnd}. Otherwise times are given at the first of each year and a time point after the last year.


\keepXColumns
\begin{tabularx}{\textwidth}{N T A}
\hline
Name & Type & Annotation\\
\hline
\hfuzz=500pt\includegraphics[width=1em]{element-mustset.pdf}~yearStart & \hfuzz=500pt uint & \hfuzz=500pt \\
\hfuzz=500pt\includegraphics[width=1em]{element-mustset.pdf}~yearEnd & \hfuzz=500pt uint & \hfuzz=500pt \\
\hfuzz=500pt\includegraphics[width=1em]{element.pdf}~useYearMiddle & \hfuzz=500pt boolean & \hfuzz=500pt time points are mid of years, otherwise the 1st of each year + a time point behind the last year\\
\hline
\end{tabularx}


\subsection{EveryMonth}
Generates a time series with monthly sampling. The first point in time will be \config{timeStart} and the following
points are generated for each month at the same day and time in month.
The last generated point in time will be less or equal \config{timeEnd}.


\keepXColumns
\begin{tabularx}{\textwidth}{N T A}
\hline
Name & Type & Annotation\\
\hline
\hfuzz=500pt\includegraphics[width=1em]{element-mustset.pdf}~timeStart & \hfuzz=500pt time & \hfuzz=500pt first point in time\\
\hfuzz=500pt\includegraphics[width=1em]{element-mustset.pdf}~timeEnd & \hfuzz=500pt time & \hfuzz=500pt last point in time will be less or equal timeEnd\\
\hline
\end{tabularx}


\subsection{EveryYear}
Generates a time series with yearly sampling. The first point in time will be \config{timeStart} and the following
points are generated for each year at the same day and time in year.
The last generated point in time will be less or equal \config{timeEnd}.


\keepXColumns
\begin{tabularx}{\textwidth}{N T A}
\hline
Name & Type & Annotation\\
\hline
\hfuzz=500pt\includegraphics[width=1em]{element-mustset.pdf}~timeStart & \hfuzz=500pt time & \hfuzz=500pt first point in time\\
\hfuzz=500pt\includegraphics[width=1em]{element-mustset.pdf}~timeEnd & \hfuzz=500pt time & \hfuzz=500pt last point in time will be less or equal timeEnd\\
\hline
\end{tabularx}


\subsection{Instrument}\label{timeSeriesType:instrument}
Read a time series (epochs) from an \file{instrument file}{instrument}.
The time series can be restricted to the interval
starting from \config{timeStart} and before \config{timeEnd}.


\keepXColumns
\begin{tabularx}{\textwidth}{N T A}
\hline
Name & Type & Annotation\\
\hline
\hfuzz=500pt\includegraphics[width=1em]{element-mustset.pdf}~inputfileInstrument & \hfuzz=500pt filename & \hfuzz=500pt \\
\hfuzz=500pt\includegraphics[width=1em]{element.pdf}~timeStart & \hfuzz=500pt time & \hfuzz=500pt exclude peochs before this epoch\\
\hfuzz=500pt\includegraphics[width=1em]{element.pdf}~timeEnd & \hfuzz=500pt time & \hfuzz=500pt only epochs before this time are used\\
\hline
\end{tabularx}


\subsection{InstrumentArcIntervals}
Reconstruct a time series from an \file{instrument file}{instrument}.
The time series is the first epoch of each arc plus one time step beyond the last
epoch of the last arc (using median sampling).


\keepXColumns
\begin{tabularx}{\textwidth}{N T A}
\hline
Name & Type & Annotation\\
\hline
\hfuzz=500pt\includegraphics[width=1em]{element-mustset.pdf}~inputfileInstrument & \hfuzz=500pt filename & \hfuzz=500pt Must be regular. Time series is first epoch of each arc plus one time step extrapolated from last epoch of last arc.\\
\hline
\end{tabularx}


\subsection{Revolution}
Reads an \file{orbit file}{instrument} and create a time stamp for each ascending equator crossing.
The time series can be restricted to the interval
starting from \config{timeStart} and before \config{timeEnd}.


\keepXColumns
\begin{tabularx}{\textwidth}{N T A}
\hline
Name & Type & Annotation\\
\hline
\hfuzz=500pt\includegraphics[width=1em]{element-mustset.pdf}~inputfileOrbit & \hfuzz=500pt filename & \hfuzz=500pt \\
\hfuzz=500pt\includegraphics[width=1em]{element.pdf}~timeStart & \hfuzz=500pt time & \hfuzz=500pt exclude peochs before this epoch\\
\hfuzz=500pt\includegraphics[width=1em]{element.pdf}~timeEnd & \hfuzz=500pt time & \hfuzz=500pt only epochs before this time are used\\
\hline
\end{tabularx}


\subsection{Exclude}
In a first step a \configClass{timeSeries}{timeSeriesType} is generated.
In a second step all times are removed which are in range before or after \config{excludeMargin} seconds
of the times given by \configClass{timeSeriesExclude}{timeSeriesType}.


\keepXColumns
\begin{tabularx}{\textwidth}{N T A}
\hline
Name & Type & Annotation\\
\hline
\hfuzz=500pt\includegraphics[width=1em]{element-mustset-unbounded.pdf}~timeSeries & \hfuzz=500pt \hyperref[timeSeriesType]{timeSeries} & \hfuzz=500pt time series to be created\\
\hfuzz=500pt\includegraphics[width=1em]{element-mustset-unbounded.pdf}~timeSeriesExclude & \hfuzz=500pt \hyperref[timeSeriesType]{timeSeries} & \hfuzz=500pt exclude this time points from time series (within margin)\\
\hfuzz=500pt\includegraphics[width=1em]{element.pdf}~excludeMargin & \hfuzz=500pt double & \hfuzz=500pt on both sides [seconds]\\
\hline
\end{tabularx}


\subsection{Conditional}
Only times for which the \configClass{condition}{conditionType} is met are included in the time series.
The \config{variableLoopTime} is set to every time and the \configClass{condition}{conditionType} is evaluated.


\keepXColumns
\begin{tabularx}{\textwidth}{N T A}
\hline
Name & Type & Annotation\\
\hline
\hfuzz=500pt\includegraphics[width=1em]{element-mustset-unbounded.pdf}~timeSeries & \hfuzz=500pt \hyperref[timeSeriesType]{timeSeries} & \hfuzz=500pt only times for which condition is met will be included\\
\hfuzz=500pt\includegraphics[width=1em]{element.pdf}~variableLoopTime & \hfuzz=500pt string & \hfuzz=500pt variable with time of each loop\\
\hfuzz=500pt\includegraphics[width=1em]{element-mustset.pdf}~condition & \hfuzz=500pt \hyperref[conditionType]{condition} & \hfuzz=500pt test for each time\\
\hline
\end{tabularx}


\subsection{Interpolate}
Interpolates \config{nodeInterpolation} count points between
the given \configClass{timeSeries}{timeSeriesType} uniformly.


\keepXColumns
\begin{tabularx}{\textwidth}{N T A}
\hline
Name & Type & Annotation\\
\hline
\hfuzz=500pt\includegraphics[width=1em]{element-mustset-unbounded.pdf}~timeSeries & \hfuzz=500pt \hyperref[timeSeriesType]{timeSeries} & \hfuzz=500pt time series to be created\\
\hfuzz=500pt\includegraphics[width=1em]{element-mustset.pdf}~nodeInterpolation & \hfuzz=500pt uint & \hfuzz=500pt interpolates count points in each time interval given by the time series\\
\hline
\end{tabularx}

\clearpage
%==================================

\section{Troposphere}\label{troposphereType}
This class provides functions for calculating and estimating
the signal delay in the dry and wet atmosphere.


\subsection{ViennaMapping}\label{troposphereType:viennaMapping}

Tropospheric delays based on the Vienna Mapping Functions 3 (VMF3) model
(Landskron and Boehm 2017, DOI: \href{https://doi.org/10.1007/s00190-017-1066-2}{10.1007/s00190-017-1066-2}).

Hydrostatic and wet mapping function coefficients ($a_h$, $a_w$) and zenith delays (ZHD, ZWD) have to be provided
via \configFile{inputfileVmfCoefficients}{griddedDataTimeSeries}. This file can contain either station-specific data
(see \program{ViennaMappingFunctionStation2File}) or data on a regular global grid
(see \program{ViennaMappingFunctionGrid2File}). In the second case mapping coefficients and zenith delays are
interpolated to the requested coordinates. This includes a height correction that requires approximate meteorological
data provided via \configFile{inputfileGpt}{griddedData}.


\keepXColumns
\begin{tabularx}{\textwidth}{N T A}
\hline
Name & Type & Annotation\\
\hline
\hfuzz=500pt\includegraphics[width=1em]{element-mustset-unbounded.pdf}~inputfileVmfCoefficients & \hfuzz=500pt filename & \hfuzz=500pt ah, aw, zhd, zwd coefficients\\
\hfuzz=500pt\includegraphics[width=1em]{element-mustset.pdf}~inputfileGpt & \hfuzz=500pt filename & \hfuzz=500pt gridded GPT data\\
\hfuzz=500pt\includegraphics[width=1em]{element.pdf}~aHeight & \hfuzz=500pt double & \hfuzz=500pt parameter a (height correction)\\
\hfuzz=500pt\includegraphics[width=1em]{element.pdf}~bHeight & \hfuzz=500pt double & \hfuzz=500pt parameter b (height correction)\\
\hfuzz=500pt\includegraphics[width=1em]{element.pdf}~cHeight & \hfuzz=500pt double & \hfuzz=500pt parameter c (height correction)\\
\hline
\end{tabularx}


\subsection{GPT}\label{troposphereType:gpt}

Tropospheric delays based on the Global Pressure and Temperature 3 (GPT3) model
(Landskron and Boehm 2017, DOI: \href{https://doi.org/10.1007/s00190-017-1066-2}{10.1007/s00190-017-1066-2}).

It is an empirical model derived from the Vienna Mapping Functions 3
(VMF3, see \configClass{viennaMapping}{troposphereType:viennaMapping}) and thus does not require
additional mapping coefficients and zenith delay values.


\keepXColumns
\begin{tabularx}{\textwidth}{N T A}
\hline
Name & Type & Annotation\\
\hline
\hfuzz=500pt\includegraphics[width=1em]{element-mustset.pdf}~inputfileGpt & \hfuzz=500pt filename & \hfuzz=500pt gridded GPT data\\
\hfuzz=500pt\includegraphics[width=1em]{element.pdf}~aHeight & \hfuzz=500pt double & \hfuzz=500pt parameter a (height correction)\\
\hfuzz=500pt\includegraphics[width=1em]{element.pdf}~bHeight & \hfuzz=500pt double & \hfuzz=500pt parameter b (height correction)\\
\hfuzz=500pt\includegraphics[width=1em]{element.pdf}~cHeight & \hfuzz=500pt double & \hfuzz=500pt parameter c (height correction)\\
\hline
\end{tabularx}

\clearpage
%==================================
